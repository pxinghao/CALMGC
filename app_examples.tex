%!TEX root = bloom_logical.tex

\section{Additional Examples}
\label{sec:examples}

\subsection{Examples of Logical GC Rules}
\label{sec:examples:logical}
\begin{example}[Trivial GC]
$\tt{GCNone}(\tt{A}_{1,TS},$ $\dots,$ $\tt{A}_{n,TS}; \tt{B}_{1,TS}, \dots, \tt{B}_{m,TS}) = \langle \tt{A}_{1,TS}, \dots, \tt{A}_{n,TS} \rangle$.
\end{example}
The trivial GC returns the tombstone lattices unchanged.
If $\tt{A}_1, \dots, \tt{A}_n$ are all output relations, then $\tt{GCNone}$ is the only garbage collection rule that respects output relations.

\begin{example}[Union GC]
\label{ex:union_gc}
Consider the union $R: \tt{B} ~\tt{<op>}~ \tt{A} \cup \tt{C}$, where $\tt{<op>}$ is a merge or deferred merge.
Then $\tt{GCUnion}(\tt{A}_{TS}, \tt{C}_{TS}; \tt{B}_{TS}) = \langle (\tt{A}_\exists - \tt{A}^*, \tt{A}_\top \cup \tt{A}^*), (\tt{C}_\exists - \tt{C}^*, \tt{C}_\top \cup \tt{C}^*) \rangle$ where $\tt{A}^* = \tt{A}_\exists \cap  (\tt{B}_\exists \cup \tt{B}_\top)$ and $\tt{C}^* = \tt{C}_\exists \cap  (\tt{B}_\exists \cup \tt{B}_\top)$ is safe for $R$.
\end{example}
Invariant \ref{inv:merge_gc} for the union rule reduces to $\tt{A}_\top \cup \tt{C}_\top \subseteq \tt{A}_\exists \cup \tt{C}_\exists \cup \tt{B}_\exists \cup \tt{B}_\top$.
Then,
\begin{align*}
&\tt{A}_\top \cup \tt{A}^* \cup \tt{C}_\top \cup \tt{A}^*\\
&\quad\subseteq\quad \tt{A}_\exists \cup \tt{C}_\exists \cup \tt{B}_\exists \cup \tt{B}_\top\\
&\quad=\quad (\tt{A}_\exists \cap (\tt{B}_\exists \cup \tt{B}_\top)^c) \cup (\tt{C}_\exists \cap (\tt{B}_\exists \cup \tt{B}_\top)^c) \cup \tt{B}_\exists \cup \tt{B}_\top\\
&\quad=\quad (\tt{A}_\exists - \tt{A}^*) \cup (\tt{C}_\exists - \tt{A}^*) \cup \tt{B}_\exists \cup \tt{B}_\top\\
\end{align*}
so $\tt{GCUnion}$ is safe for $R$.

\begin{example}[Intersection GC]
\label{ex:intersection_gc}
Consider the union $R: \tt{B} ~\tt{<op>}~ \tt{A} \cap \tt{C}$, where $\tt{<op>}$ is a merge or deferred merge.
Then $\tt{GCIntersect}(\tt{A}_{TS}, \tt{C}_{TS}; \tt{B}_{TS}) = \langle (\tt{A}_\exists - \tt{A}^*, \tt{A}_\top \cup \tt{A}^*), (\tt{C}_\exists - \tt{C}^*, \tt{C}_\top \cup \tt{C}^*) \rangle$ where $\tt{A}^* = \tt{A}_\exists \cap  (\tt{B}_\exists \cup \tt{B}_\top)$ and $\tt{C}^* = \tt{C}_\exists \cap  (\tt{B}_\exists \cup \tt{B}_\top)$ is safe for $R$.
\end{example}
Note that $\tt{GCIntersect} = \tt{GCUnion}$, both of which tombstones tuples from $\tt{A}_\exists$ and $\tt{C}_\exists$ after they have appeared in $\tt{B}_\exists \cup \tt{B}_\top$.
In this case, Invariant \ref{inv:merge_gc} reduces to $\tt{A}_\top \in \tt{B}_\exists \cup \tt{B}_\top$ and $\tt{C}_\top \in \tt{B}_\exists \cup \tt{B}_\top$.
Then $\tt{A}_\top \cup \tt{A}^* \subseteq \tt{B}_\exists \cup \tt{B}_\top$ and $\tt{C}_\top \cup \tt{C}^* \subseteq \tt{B}_\exists \cup \tt{B}_\top$, so $\tt{GCIntersect}$ is safe for $R$.

\begin{example}[Select GC]
Consider the selection $R: \tt{B} ~\tt{<op>}~ \sigma_P(\tt{A})$, where $\tt{<op>}$ is either a merge or deferred merge.
Then $\tt{GCSelect}(\tt{A}_{TS}; \tt{B}_{TS}) = (\tt{A}_\exists - \tt{A}^*, \tt{A}_\top \cup \tt{A}^*)$ where $\tt{A}^* = \sigma_{\neg P}(\tt{A}_\exists \cup \tt{A}_\top) \cup (\sigma_P(\tt{A}_\exists \cup \tt{A}_\top) \cap (\tt{B}_\exists \cup \tt{B}_\top))$ is safe for $R$.
\end{example}
Similar to Example \ref{ex:project_gc}, the Merge GC Invariant \ref{inv:merge_gc} for selection reduces to $\sigma_P(\tt{A}_\top) \subseteq \tt{B}_\exists \cup \tt{B}_\top$.
If $\tt{A}_{TS}$ and $\tt{B}_{TS}$ satisfy the invariant, then
\begin{align*}
&\sigma_P(\tt{A}_\top \cup \tt{A}^*)\\
&= \sigma_P(\tt{A}_\top) \cup \sigma_P(\sigma_{\neg P}(\tt{A}_\exists \cup \tt{A}_\top) \cup (\sigma_P(\tt{A}_\exists \cup \tt{A}_\top) \cap (\tt{B}_\exists \cup \tt{B}_\top)))\\
&= \sigma_P(\tt{A}_\top) \cup (\sigma_P(\tt{A}_\exists \cup \tt{A}_\top) \cap (\tt{B}_\exists \cup \tt{B}_\top))\\
&= \sigma_P(\tt{A}_\top) \cup (\sigma_P(\tt{A}_\exists) \cap (\tt{B}_\exists \cup \tt{B}_\top))\\
&\subseteq \tt{B}_\exists \cup \tt{B}_\top,
\end{align*}
so $\tt{GCSelect}$ is safe for $R$.

\begin{example}[Linear GC]
Consider the rule $R: \tt{B} ~\tt{<op>}~ \tt{f}(\tt{A})$, where $\tt{f}$ is a linear function, i.e., $\tt{f}(\tt{A}) = \bigcup_{a\in \tt{A}} \tt{f}(\{a\})$.
Then $\tt{GCAll}(\tt{A}_{TS}; \tt{B}_{1,TS}, \dots, \tt{B}_{m,TS})$ $=$ $(\emptyset,$ $\tt{A}_\exists \cup \tt{A}_\top)$ is safe for $R$.
\end{example}
Since $\tt{f}$ is linear, Invariant \ref{inv:merge_gc} reduces to $\tt{f}\tt{(}\tt{A}_\top\tt{}) \subseteq \tt{B}_\exists \cup \tt{B}_\top$, but this is true under the condition $\tt{B}_\exists \cup \tt{B}_\top \supseteq \tt{f}\tt{(}\tt{A}_{\exists} \cup \tt{A}_{\top}\tt{)} = \tt{f}\tt{(}\tt{A}_\exists\tt{}) \cup \tt{f}\tt{(}\tt{A}_\top\tt{})$.
Thus, Invariant \ref{inv:merge_gc} is trivially true for linear $\tt{f}$.
Intuitively, a tuple $a \in \tt{A}_\exists$ is independent of all other tuples when $\tt{f}$ is linear, and hence, its complete effects will have been realized in $\tt{B}$ (either through a merge or deferred merge) by the point at which we perform garbage collection.
Therefore, we can always reclaim the entire set without impacting our results.

In particular, observe that the copy, union, select, and project rules of previous examples are all linear, and thus can be fully reclaimed at the end of each timestep.

\begin{example}[Join GC]
\label{ex:join_gc_no_manifests}
Consider the join $R: \tt{B} ~\tt{<op>}~ \tt{A} \bowtie_{\tt{A}.k_a = \tt{C}.k_c} \tt{C}$, where $\tt{<op>}$ is either a merge or deferred merge.
The only garbage collection rule that is safe for $R$ is the trivial $\tt{GCTrivial}(\tt{A}_{TS}, \tt{C}_{TS}; \tt{B}_{TS}) = \langle \tt{A}_{TS}, \tt{C}_{TS} \rangle$.
\end{example}
Without additional information, one cannot guarantee that a tuple in $\tt{A}$ will not match any future tuple in $\tt{C}$ and vice versa.
In some cases, however, one may be provided with the specific information that no more tuples matching the attribute will ever appear.

Formally, let $\mathcal{X}$ be the domain of possible tuples for a table $\tt{X}$, and denote the powerset of $\mathcal{X}$ as $\mathbb{P}(\mathcal{X})$.
Also let $\mathcal{K}$ be the domain of possible values of an attribute $\tt{X}.k$.
A manifest $p = (p.k, p.\tt{X}) \in \mathcal{K} \times \mathbb{P}(\mathcal{X})$ maps a key in $\mathcal{K}$ to a subset of $\mathbb{P}(\mathcal{X})$.
We denote by $\tt{X}$ by $\tt{P}_\tt{X}$ a table of manifests (with $k$ as its primary key) for $\tt{X}$, and say it is consistent with $\tt{X}$ if $\forall p \in \tt{P}_\tt{X}$, $p.\tt{X} \supseteq \{x \in \tt{X} ~:~ x.k = p.k\}$.
We will always assume that manifest tables are always consistent.

In the join rule above, we can reclaim from $\tt{A}$ given manifests on $\tt{C}$ and vice versa:

\begin{example}[Join with manifests]
Consider the join in Example \ref{ex:join_gc_no_manifests}, now augmented with manifest tables $\tt{P}_\tt{A}$ and $\tt{P}_\tt{C}$.
(If manifests are not available, we can always let $\tt{P} = \emptyset$ with no loss of generality.)
The garbage collection rule $\tt{GCJoin}(\tt{A}_{TS},$ $\tt{P}_{\tt{A},TS},$ $\tt{C}_{TS},$ $\tt{P}_{\tt{C},TS};$ $\tt{B}_{TS})$ $=$ $\langle (\tt{A}_\exists - \tt{A}^*, \tt{A}_\top \cup \tt{A}^*), (\tt{P}_{\tt{A},\exists} - \tt{P}_\tt{A}^*, \tt{P}_{\tt{A},\top} \cup \tt{P}_\tt{A}^*), (\tt{C}_\exists - \tt{C}^*, \tt{C}_\top \cup \tt{C}^*), (\tt{P}_{\tt{C},\exists} - \tt{P}_\tt{C}^*, \tt{P}_{\tt{C},\top} \cup \tt{P}_\tt{C}^*) \rangle$, with
\begin{align*}
\tt{A}^* &= \{a \in \tt{A}_\exists \cup \tt{A}_\top ~:~ \exists p_c \in \tt{P}_{\tt{C},\exists} \cup \tt{P}_{\tt{C},\top},
% \\&
\text{ such that } (a.k_a = p_c.k_c) \wedge \left(\{a\} \times p_c.\tt{C} \subseteq \tt{B}_\exists \cup \tt{B}_\top\right)\}\\
\tt{P}_\tt{A}^* &= \{p_a \in \tt{P}_{\tt{A},\exists} \cup \tt{P}_{\tt{A},\top} ~:~ \exists p_c \in \tt{P}_{\tt{C},\exists} \cup \tt{P}_{\tt{C},\top},
% \\&
\text{ such that } (p_a.k_a = p_c.k_c) \wedge \left(p_a.\tt{A} \times p_c.\tt{C} \subseteq \tt{B}_\exists \cup \tt{B}_\top\right)\}\\
\tt{C}^* &= \{c \in \tt{C}_\exists \cup \tt{C}_\top ~:~ \exists p_a \in \tt{P}_{\tt{A},\exists} \cup \tt{P}_{\tt{A},\top},
% \\&
\text{ such that } (p_a.k_a = c.k_c) \wedge \left(p_a.\tt{A} \times \{c\} \subseteq \tt{B}_\exists \cup \tt{B}_\top\right)\}\\
\tt{P}_\tt{A}^* &= \{p_a \in \tt{P}_{\tt{A},\exists} \cup \tt{P}_{\tt{A},\top} ~:~ \exists p_c \in \tt{P}_{\tt{C},\exists} \cup \tt{P}_{\tt{C},\top}
% \\&
\text{ such that } (p_a.k_a = p_c.k_c) \wedge \left(p_a.\tt{A} \times p_c.\tt{C} \subseteq \tt{B}_\exists \cup \tt{B}_\top\right)\}
\end{align*}
is safe for the join.
\end{example}
In words, a tuple $a$ can be tombstoned if it matches a punctuation $p_c$ in $\tt{P}_\tt{C}$ \emph{and} the result of joining $\{a\}$ with $\tt{C}$ is already effected in $\tt{B}_{TS}$.
Hence, any $c \in \tt{C}$ either does not match $a$, or is already joined with $a$ in $\tt{B}$ and so will not affect any future join computation.
In addition, we tombstone a punctuation $p_a$ if it matches a punctuation $p_c$ and the join of their sets $p_a.\tt{A} \times p_c.\tt{C}$ is already effected in $\tt{B}_{TS}$.
It is a simple exercise to show that $\tt{GCJoin}$ tombstones a punctuation $p_a$ only when its set $p_a.\tt{A}$ is tombstoned (or does not exist yet), i.e., $(p_a \in \tt{P}_{\tt{A},\top}) \implies \tt{A}_\exists \cap p_a.\tt{A} = \emptyset \implies \sigma_{\tt{A}.k_a = p_a.k_a}(\tt{A}_\exists) = \emptyset$.

\begin{example}[DR+: Positive Difference]
Consider the rule $R: \tt{B} ~\tt{<op>}~ \tt{A} - \tt{C}$, where $\tt{<op>}$ is a merge or deferred merge.
Then the rule $\tt{GCDR+}(\tt{A}_{TS}, \tt{C}_{TS};$ $\tt{B}_{TS}) = \langle (\tt{A}_\exists - \tt{A}^*, \tt{A}_\top \cup \tt{A}^*), \tt{C}_{TS}\rangle$
where $\tt{A}^* = (\tt{A}_\exists \cup \tt{A}_\top) \cap (\tt{C}_\exists \cup \tt{C}_\top)$
is safe for $R$.
\end{example}
For set differences, the Merge GC Invariant \ref{inv:merge_gc} reduces to
$\forall \widehat{\tt{A}} \supseteq \tt{A}_\exists, \forall \widehat{\tt{C}} \supseteq \tt{C}_\exists$: $\widehat{\tt{A}} - \widehat{\tt{C}} = (\widehat{\tt{A}} \cup \tt{A}_\top) - (\widehat{\tt{C}} \cup \tt{C}_\top)$.
In the case where $\widehat{\tt{A}} = \tt{A}_\exists$ and $\widehat{\tt{C}} = \tt{C}_\exists$, it implies that $\tt{A}_\top \subseteq \tt{C}_\exists \cup \tt{C}_\top$;
in the case where $\widehat{\tt{A}} = \tt{A}_\exists \cup \tt{C}_\top$ and $\widehat{\tt{C}} = \tt{A}_\exists \cup \tt{C}_\exists$, it implies that $\tt{C}_\top = \emptyset$.
Conversely $\tt{C}_\top = \emptyset$ and $\tt{A}_\top \subseteq \tt{C}_\exists \cup \tt{C}_\top$ suffices to satisfy the invariant.
One can now easily verify that $\tt{GCDR+}$ maintains the invariant.


\subsection{Examples of Instantiated GC Rules}
\label{sec:examples:instantiated}

\begin{example}[Select GC]
$\tt{iGCSelect}(\tt{A}_I; \tt{B}_I) = \sigma_{\neg P}(\tt{A}_I) \cup (\sigma_P(\tt{A}_I) \cap \tt{B}_I)$ with $\mathbb{W}(\tt{iGCProject}) = \{\tt{A}\}$ is weakly consistent with $\tt{GCSelect}$ on $\tt{A}$.
\end{example}
If $\tt{A}_I \subseteq \tt{A}_\exists \cup \tt{A}_\top$ and $\tt{B}_I \subseteq \tt{B}_\exists \cup \tt{B}_\top$, then
simple relational algebra demonstrates that
$\tt{iGCSelect}(\tt{A}_I; \tt{B}_I) = \sigma_{\neg P}(\tt{A}_I) \cup (\sigma_P(\tt{A}_I) \cap \tt{B}_I) \subseteq \sigma_{\neg P}(\tt{A}_\exists \cup \tt{A}_\top) \cup (\sigma_P(\tt{A}_\exists \cup \tt{A}_\top) \cap (\tt{B}_\exists \cup \tt{B}_\top)) = \tt{GCSelect}(\tt{A}_{TS}; \tt{B}_{TS})$.

\begin{example}[Linear GC]
$\tt{iGCAll}(\tt{A}_I;$ $\tt{B}_{1,E},$ $\dots,$ $\tt{B}_{m,E})$ $=$ $\tt{A}_I$ with $\mathbb{W}(\tt{iGCProject}) = \{\tt{A}\}$ is weakly consistent with $\tt{GCAll}$ on $\tt{A}$.
\end{example}

\begin{example}[Join with punctuations]
$\tt{iGCJoin}(\tt{A}_I,$ $\tt{P}_{\tt{A},I},$ $\tt{C}_I,$ $\tt{P}_{\tt{C},E};$ $\tt{B}_I)$ $=$ $\langle \tt{A}_I^-, \tt{P}_{\tt{A},I}^-, \tt{C}_I^-, \tt{P}_{\tt{C},I}^-\rangle$, where
\begin{align*}
\tt{A}_I^- &= \{a \in \tt{A}_I ~:~ \exists p_c \in \tt{P}_{\tt{C},I},
% \\&
\text{ such that } (a.k_a = p_c.k_c) \wedge \left(\{a\} \times p_c.\tt{C} \subseteq \tt{B}_I\right)\}\\
\tt{P}_\tt{A}^- &= \{p_a \in \tt{P}_{\tt{A},I} ~:~ \exists p_c \in \tt{P}_{\tt{C},I},
% \\&
\text{ such that } (p_a.k_a = p_c.k_c) \wedge \left(p_a.\tt{A} \times p_c.\tt{C} \subseteq \tt{B}_I\right)\}\\
\tt{C}_I^- &= \{c \in \tt{C}_I ~:~ \exists p_a \in \tt{P}_{\tt{A},I},
% \\&
\text{ such that } (p_a.k_a = c.k_c) \wedge \left(\{c\} \times p_a.\tt{A} \subseteq \tt{B}_I\right)\}\\
\tt{P}_\tt{C}^- &= \{p_c \in \tt{P}_{\tt{C},I} ~:~ \exists p_a \in \tt{P}_{\tt{A},I},
% \\&
\text{ such that } (p_a.k_a = p_c.k_c) \wedge \left(p_a.\tt{A} \times p_c.\tt{C} \subseteq \tt{B}_I\right)\}
\end{align*}
with $\mathbb{W}(\tt{iGCJoin}) = \{\tt{A}, \tt{P}_\tt{A}, \tt{C}, \tt{P}_\tt{A}\}$ is weakly consistent with $\tt{GCJoin}$ on $\tt{A}$, $\tt{P}_\tt{A}$, $\tt{C}$ and $\tt{P}_\tt{C}$.
\end{example}
Proof of consistency of $\tt{iGCJoin}$ with $\tt{GCJoin}$ is straightforward and thus omitted.
As noted before, $p_a \in \tt{P}_{\tt{A},!} - \tt{P}_{\tt{A},I} \implies p_a \in \tt{P}_{\tt{A},\top} \implies \sigma_{\tt{A}.k_a = p_a.k_a}(\tt{A}_\exists) = \emptyset$, so we can also tombstone tuples whose keys match $\tt{P}_{\tt{A},!} - \tt{P}_{\tt{A},I}$:

\begin{example}[Join with punctuations, version 2]
$\tt{iGCJoinV2}(\tt{A}_I, \tt{P}_{\tt{A},I}, \tt{C}_I, \tt{P}_{\tt{C},E}; \tt{B}_I) = \langle \tt{A}_I^-, \tt{P}_{\tt{A},E}^-, \tt{C}_I^-, \tt{P}_{\tt{C},E}^-\rangle$, where
\begin{align*}
\tt{A}_I^- &= \{a \in \tt{A}_I ~:~ \exists p_c \in \tt{P}_{\tt{C},I}, \text{ such that } (a.k_a = p_c.k_c)
% \\&
\wedge \left(\{a\} \times p_c.\tt{C} \subseteq \tt{B}_I\right)\} \cup \{a \in \tt{A}_I ~:~ a.k_a \in \tt{P}_{\tt{A},!} - \pi_!(\tt{P}_{\tt{A},I})\}\\
\tt{P}_\tt{A}^- &= \{p_a \in \tt{P}_{\tt{A},I} ~:~ \exists p_c \in \tt{P}_{\tt{C},I}, \text{ such that } (p_a.k_a = p_c.k_c)
% \\&
\wedge \left(p_a.\tt{A} \times p_c.\tt{C} \subseteq \tt{B}_I\right)\}\\
\tt{C}_I^- &= \{c \in \tt{C}_I ~:~ \exists p_a \in \tt{P}_{\tt{A},I}, \text{ such that } (p_a.k_a = c.k_c)
% \\&
\wedge \left(\{c\} \times p_a.\tt{A} \subseteq \tt{B}_I\right)\} \cup \{c \in \tt{C}_I ~:~ c.k_c \in \tt{P}_{\tt{C},!} - \pi_!(\tt{P}_{\tt{C},I})\}\\
\tt{P}_\tt{C}^- &= \{p_c \in \tt{P}_{\tt{C},I} ~:~ \exists p_a \in \tt{P}_{\tt{A},I}, \text{ such that } (p_a.k_a = p_c.k_c)
% \\&
\wedge \left(p_a.\tt{A} \times p_c.\tt{C} \subseteq \tt{B}_I\right)\}
\end{align*}
with $\mathbb{W}(\tt{iGCJoin}) = \{\tt{A}, \tt{P}_\tt{A}, \tt{C}, \tt{P}_\tt{A}\}$ is weakly consistent with $\tt{GCJoin}$ on $\tt{A}$, $\tt{P}_\tt{A}$, $\tt{C}$ and $\tt{P}_\tt{C}$.
\end{example}


\begin{example}[DR+: Positive Difference]
$\tt{iGCDR+}(\tt{A}_I, \tt{C}_I; \tt{B}_I) = \langle \tt{A}_I \cap \tt{C}, \tt{C}_I\rangle$ with $\mathbb{W}(\tt{iGCDR+}) = \{\tt{A}, \tt{C}\}$ is weakly consistent with $\tt{GCDR+}$ on both $\tt{A}$ and $\tt{C}$.
\end{example}


