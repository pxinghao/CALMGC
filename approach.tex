%!TEX root = bloom_logical.tex

\section{Approach}
\label{sec:approach}

Our approach in this paper proceeds via two program rewrites.
The first rewrite takes an Edelweiss program $\mathfrak{P}$ and produces a \emph{logical garbage collection} program $\mathfrak{P}_{GC}$, where tuples are marked for reclamation without actually being reclaimed.
A second rewrite then turns $\mathfrak{P}_{GC}$ into an \emph{instantiated garbage collection} program $\mathfrak{P}_{iGC}$ which actually deletes tuples and reclaims storage.
Only $\mathfrak{P}_{iGC}$ is eventually executed;
however, we will argue that if all three programs had been executed together, the sets in $\mathfrak{P}$, $\mathfrak{P}_{GC}$, and $\mathfrak{P}_{iGC}$ would evolve together and maintain relationships to one another.
These relationships will demonstrate that $\mathfrak{P}_{iGC}$ produces the same output as $\mathfrak{P}$ without the need for additional coordination.

\textbf{Logical Rewrite.}
Key to the logical rewrite are `tombstone sets' which contain tuples that have been marked as `tombstoned', or safe for deletion.
Every set $\tt{A}$ in $\mathfrak{P}$ has a corresponding tombstone set $\tt{A}_{TS} = (\tt{A}_\exists, \tt{A}_\top)$ in $\mathfrak{P}_{GC}$.
Logical garbage collection rules are also added to $\mathfrak{P}_{GC}$ to mark tuples as being safe for deletion, by moving them from $\tt{A}_\exists$ to $\tt{A}_\top$.

A simple induction on the execution of $\mathfrak{P}$ and $\mathfrak{P}_{GC}$ demonstrates that we always have $\tt{A} = \tt{A}_\exists \cup \tt{A}_\top$.
Since we also never tombstone any tuples from sets that have been identified as output sets, $\mathfrak{P}_{GC}$ produces the correct output $\tt{A}_\exists = \tt{A}$.
In addition, we ensure that tombstoned tuples can in fact be safely deleted, i.e., reclaiming the tuples will not affect the execution and output of the program.
We formalize this idea in Invariant \ref{inv:merge_gc}, and prove that the invariant is maintained by $\mathfrak{P}_{GC}$.

The monotonicity of logical garbage collection ensures, by the CALM theorem, that no additional coordination is required.
In particular, if $\mathfrak{P}$ is monotone and coordination-free, then $\mathfrak{P}_{GC}$ is also monotone and coordination-free.


\textbf{Instantiated Rewrite.}
The logical rewrite $\mathfrak{P}_{GC}$ does not in fact provide any storage reclamation.
To do that, we perform a second write of $\mathfrak{P}_{GC}$ into an \emph{instantiated garbage collection} program $\mathfrak{P}_{iGC}$.
Unlike $\mathfrak{P}_{GC}$ which operates on tombstone sets $\tt{A}_{TS}$'s, $\mathfrak{P}_{iGC}$ operates on ordinary sets, denoted $\tt{A}_I$ to indicate that deletions are instantiated.
Some of the sets $\tt{A}_I$ in $\mathfrak{P}_{iGC}$ are also augmented with primary keys $\tt{A}_!$, if primary keys are available and provide useful information for more efficient garbage collection.
Instantiated garbage collection rules are added to $\mathfrak{P}_{iGC}$ corresponding to logical garbage collection rules of $\mathfrak{P}$.

On every set $\tt{A}$, the instantiated program $\mathfrak{P}_{iGC}$ either tries to maintain \emph{weak consistency} with $\mathfrak{P}_{GC}$ on $\tt{A}$, that is,
\[ \tt{A}_\exists \subseteq \tt{A}_I \subseteq \tt{A}_\exists \cup \tt{A}_\top = \tt{A},\]
or it maintains \emph{strong consistency} on $\tt{A}$, that is,
\[ \tt{A}_\exists = \tt{A}_I \subseteq \tt{A}_\exists \cup \tt{A}_\top = \tt{A}.\]
Intuitively, tuples in $\tt{A}_\exists$ are important for future derivations and must be kept in $\tt{A}_I$.
Tombstones $\tt{A}_\top$ of a weakly consistent $\tt{A}$ can be partially deleted; for a strongly consistent $\tt{A}$, the tombstones $\tt{A}_\top$ must be completely removed from $\tt{A}_I$, or else could interfere in future derivations.
When the above relationship holds, Invariant \ref{inv:merge_gc} is a statement that executing a Bloom operator in $\mathfrak{P}_{iGC}$ using instantiated sets $\tt{A}_I$'s produces the same outcome as the equivalent execution in $\mathfrak{P}$ using the complete sets $\tt{A}$.

In order to maintain the consistency of $\mathfrak{P}_{iGC}$ with $\mathfrak{P}_{GC}$ on $\tt{A}$, we will also require that the instantiated garbage collection rule $\tt{iGC}$ for $\tt{A}_I$ agrees with its corresponding logical garbage collection rule $\tt{GC}$ for $\tt{A}_{TS}$.
Specifically, if $\tt{A}_I$ is weakly consistent with $\tt{A}_{TS}$, then $\tt{iGC}$ must delete a subset of tuples that are tombstoned by $\tt{GC}$; and if $\tt{A}_I$ is strongly consistent with $\tt{A}_{TS}$, then $\tt{iGC}$ must delete exactly the tuples that are tombstoned by $\tt{GC}$.
The decision of whether to maintain strong or weak consistency for the set $\tt{A}$ is determined by its associated instantiated garbage collection rule.

\subsection{Coordination-Freeness}
By construction, garbage collection rules in $\mathfrak{P}_{GC}$ are monotone, and only promote tuples from $\tt{A}_\exists$ to $\tt{A}_\top$.
Hence, the CALM theorem implies that $\mathfrak{P}_{GC}$ requires no additional coordination over $\mathfrak{P}$ to achieve confluence.
In particular, if $\mathfrak{P}$ is already monotone and coordination-free, then $\mathfrak{P}_{GC}$ is also monotone and coordination-free.
This is the case if $\mathfrak{P}$ is written in monotone Edelweiss (i.e., the Bloom equivalent of Dedalus+), in which case $\mathfrak{P}_{iGC}$ is expressed in monotone Bloom$^L$.

The coordination-freeness of $\mathfrak{P}_{iGC}$ is a more complex issue.
Instantiated garbage collection rules do delete tuples but shrink the instantiated sets, and are not guaranteed to be monotone.
Hence, $\mathfrak{P}_{iGC}$ is not expressible in monotone Bloom$^L$, and we cannot appeal to analysis of \cite{marczak2012confluence} to demonstrate confluence.
Nevertheless, our correctness proofs show that $\tt{P}_{iGC}$ produces the same output as $\mathfrak{P}$.
Indeed, one can show that $\mathfrak{P}_{iGC}$ is not necessarily confluent on non-output sets, in particular, different runs of $\mathfrak{P}_{iGC}$ could disagree on the tuples which are tombstoned in $\mathfrak{P}_{GC}$.
Coordination is needed only to achieve agreement on the non-output sets.
On the other hand, we will argue that $\mathfrak{P}_{iGC}$ is confluent if viewed as a query on the output sets, for which no additional coordination is required.
This argument is along the lines of \cite{ameloot2013relational}, which formalizes the CALM theory as a statement that the output query is monotone if and only if it is coordination free.
One may therefore choose to execute $\mathfrak{P}_{iGC}$ coordination-free, and still get confluence in the sets that one cares about.








