% This is "sig-alternate.tex" V2.1 April 2013
% This file should be compiled with V2.5 of "sig-alternate.cls" May 2012
%
% This example file demonstrates the use of the 'sig-alternate.cls'
% V2.5 LaTeX2e document class file. It is for those submitting
% articles to ACM Conference Proceedings WHO DO NOT WISH TO
% STRICTLY ADHERE TO THE SIGS (PUBS-BOARD-ENDORSED) STYLE.
% The 'sig-alternate.cls' file will produce a similar-looking,
% albeit, 'tighter' paper resulting in, invariably, fewer pages.
%
% ----------------------------------------------------------------------------------------------------------------
% This .tex file (and associated .cls V2.5) produces:
%       1) The Permission Statement
%       2) The Conference (location) Info information
%       3) The Copyright Line with ACM data
%       4) NO page numbers
%
% as against the acm_proc_article-sp.cls file which
% DOES NOT produce 1) thru' 3) above.
%
% Using 'sig-alternate.cls' you have control, however, from within
% the source .tex file, over both the CopyrightYear
% (defaulted to 200X) and the ACM Copyright Data
% (defaulted to X-XXXXX-XX-X/XX/XX).
% e.g.
% \CopyrightYear{2007} will cause 2007 to appear in the copyright line.
% \crdata{0-12345-67-8/90/12} will cause 0-12345-67-8/90/12 to appear in the copyright line.
%
% ---------------------------------------------------------------------------------------------------------------
% This .tex source is an example which *does* use
% the .bib file (from which the .bbl file % is produced).
% REMEMBER HOWEVER: After having produced the .bbl file,
% and prior to final submission, you *NEED* to 'insert'
% your .bbl file into your source .tex file so as to provide
% ONE 'self-contained' source file.
%
% ================= IF YOU HAVE QUESTIONS =======================
% Questions regarding the SIGS styles, SIGS policies and
% procedures, Conferences etc. should be sent to
% Adrienne Griscti (griscti@acm.org)
%
% Technical questions _only_ to
% Gerald Murray (murray@hq.acm.org)
% ===============================================================
%
% For tracking purposes - this is V2.0 - May 2012

\documentclass{sig-alternate-05-2015}
%% Language and font encodings
\usepackage[english]{babel}
\usepackage[utf8x]{inputenc}
\usepackage[T1]{fontenc}

% %% Sets page size and margins
% \usepackage[top=3cm,bottom=2cm,left=3cm,right=3cm,marginparwidth=1.75cm]{geometry}

%% Useful packages
\usepackage{amsmath}
\usepackage{amssymb}
\usepackage{amsfonts}
\usepackage{graphicx}
\usepackage[colorinlistoftodos]{todonotes}
% \usepackage[colorlinks=true, allcolors=blue]{hyperref}
% \usepackage{amsthm}
\usepackage{listings}
\usepackage{xcolor}


\lstset{
  basicstyle=\ttfamily\footnotesize,
  mathescape
}

\numberwithin{equation}{section}


\newtheorem{theorem}{Theorem}[section]
\newtheorem{thm}[theorem]{Theorem}
% \newtheorem*{thmnon}[theorem]{Theorem}
\newtheorem{lemma}[theorem]{Lemma}
% \newtheorem*{lemma*}[theorem]{Lemma}
\newtheorem{proposition}[theorem]{Proposition}
\newtheorem{cor}[theorem]{Corollary}
\newtheorem{axiom}[theorem]{Axiom}
\newtheorem{conj}[theorem]{Conjecture}

\newdef{defn}{Definition}%[section]

\newtheorem{rmk}{Remark}[section]
\newtheorem{example}{Example}[section]

\newtheorem{property}{Property}[section]
\newtheorem{invariant}[property]{Invariant}
\newtheorem{assumption}[property]{Assumption}

\newenvironment{claim}[1]{\par\noindent\underline{Claim:}\space#1}{}
\newenvironment{claimproof}[1]{\par\noindent\underline{Proof:}\space#1}{\hfill $\blacksquare$}

\renewcommand{\tt}[1]{\ifmmode\text{\texttt{#1}}\else\texttt{#1}\fi}

\newcommand{\jmh}[1]{{\textcolor{red}{[[#1 -- jmh]]}}}


\begin{document}

% Copyright
% \setcopyright{acmcopyright}
%\setcopyright{acmlicensed}
\setcopyright{rightsretained}
%\setcopyright{usgov}
%\setcopyright{usgovmixed}
%\setcopyright{cagov}
%\setcopyright{cagovmixed}


\setcopyright{rightsretained}
\conferenceinfo{PODS '18,}{June 3--8, 2018, Houston, TX, USA.}
\isbn{978-1-4503-4198-1/17/05.}
\doi{http://dx.doi.org/XXXX.XXXX}
%Authors, replace the red X's with your assigned DOI string.

\clubpenalty=10000 
\widowpenalty = 10000

\title{Keep CALM and Garbage Collect}
% \subtitle{[Extended Abstract]
% \titlenote{A full version of this paper is available as
% \textit{Author's Guide to Preparing ACM SIG Proceedings Using
% \LaTeX$2_\epsilon$\ and BibTeX} at
% \texttt{www.acm.org/eaddress.htm}}}
%
% You need the command \numberofauthors to handle the 'placement
% and alignment' of the authors beneath the title.
%
% For aesthetic reasons, we recommend 'three authors at a time'
% i.e. three 'name/affiliation blocks' be placed beneath the title.
%
% NOTE: You are NOT restricted in how many 'rows' of
% "name/affiliations" may appear. We just ask that you restrict
% the number of 'columns' to three.
%
% Because of the available 'opening page real-estate'
% we ask you to refrain from putting more than six authors
% (two rows with three columns) beneath the article title.
% More than six makes the first-page appear very cluttered indeed.
%
% Use the \alignauthor commands to handle the names
% and affiliations for an 'aesthetic maximum' of six authors.
% Add names, affiliations, addresses for
% the seventh etc. author(s) as the argument for the
% \additionalauthors command.
% These 'additional authors' will be output/set for you
% without further effort on your part as the last section in
% the body of your article BEFORE References or any Appendices.

\numberofauthors{2} %  in this sample file, there are a *total*
% of EIGHT authors. SIX appear on the 'first-page' (for formatting
% reasons) and the remaining two appear in the \additionalauthors section.
%
\author{
% You can go ahead and credit any number of authors here,
% e.g. one 'row of three' or two rows (consisting of one row of three
% and a second row of one, two or three).
%
% The command \alignauthor (no curly braces needed) should
% precede each author name, affiliation/snail-mail address and
% e-mail address. Additionally, tag each line of
% affiliation/address with \affaddr, and tag the
% e-mail address with \email.
%
% 1st. author
\alignauthor
Xinghao Pan\\
       \affaddr{DSO National Laboratories}\\
       \email{pxinghao@dso.org.sg}
% 2nd. author
\alignauthor
Joseph M. Hellerstein\\
       \affaddr{University of California, Berkeley}\\
       \email{hellerstein@cs.berkeley.edu}
}

\maketitle

%!TEX root = bloom_logical.tex

\begin{abstract}
Programming with immutable state has been a popular approach due to its simplicity of reasoning about distributed consistency.
This has been partly motivated by the CALM theorem \cite{hellerstein2010declarative,ameloot2013relational} which states that monotone programs are those that may be executed without coordination.
However, programs with immutable state suffer from unbounded growth of storage.
Edelweiss \cite{conway2014edelweiss} was a first step towards addressing this problem by introducing automated garbage collection for some common operations.
Unfortunately, Edelweiss depended on non-monotone negations and also used non-monotone deletions to reclaim storage, which appears to diverge from the original motivation of immutability, namely, to write monotone and thus coordination-free programs.


In this work, we harmonize the notion of CALM consistency with garbage collection.
The intuitions behind our approach are (i) data that does not affect future computation can be safely reclaimed, and (ii) garbage collection does not affect the monotonicity of the output and coordination-freeness of the program.
We formalize these intuitions in a general framework defining a class of garbage collection rules, and provide an analysis of the correctness of our rules.
We prove that our automated garbage collection requires no additional coordination, and in particular is coordination-free whenever the original program is monotone.




% The CALM theorem \cite{hellerstein2010declarative, ameloot2013relational} states that coordination-free programs are exactly monotonic queries.
% Informally, a ``coordination'' is communication whose purpose is to detect or discover the completeness of input.
% A coordination-free program may require other forms of communications, but can always generate partial output corresponding to any partial input, without the need for future retraction.
% Conversely, a program that is not coordination-free requires the knowledge of input completeness in order to generate output that it will not later retract, and therefore cannot generate the correct answer until the full input is given.

% Bloom \cite{alvaro2011consistency} is a programming model that encapsulates this philosophy by enabling high-level consistency analysis, which identify rules and cycles of non-monotonicity as points of order necessitating coordination.

% To sidestep challenges in distributed consistency, practitioners and academics have resorted to writing systems with immutable states, a model which has been termed Event Log Exchange (ELE).
% Edelweiss \cite{conway2014edelweiss} specializes Bloom to a sublanguage that omits primitives for mutating or deleting data, with the specific application of event log exchanges (ELE) programs in mind.
% ELE programs written with immutable data are immediately monotone, and thus coordination-free.

% While the ELE programming paradigm simplifies distributed programming, it results in unbounded storage growth.
% Edelweiss combats this through automatic garbage that discards useless data and reclaims storage.
% Unfortunately, the opportunities for storage reclamation identified in \cite{conway2014edelweiss} largely exploit non-monotonic operations such as set negation.
% This runs counter to the original motivation for immutability, namely, to write monotone and therefore coordination-free programs.

% In this work, we harmonize the notion of CALM consistency with ELE.
% The intuitions behind our approach are (i) data that does not affect future computation can be safely reclaimed, and (ii) garbage collection does not affect the monotonicity of the output and coordination-freeness of the program.
% We define properties of valid GC rules according to these intuitions, and provide an analysis of garbage collection.
% We prove that our automated garbage collection is monotone (output always increases) and coordination-free (no additional inter-node points of order) whenever the original program is monotone.
\end{abstract}




%
% The code below should be generated by the tool at
% http://dl.acm.org/ccs.cfm
% Please copy and paste the code instead of the example below. 
%
\begin{CCSXML}
<ccs2012>
<concept>
<concept_id>10003752.10010070.10010111.10011734</concept_id>
<concept_desc>Theory of computation~Logic and databases</concept_desc>
<concept_significance>500</concept_significance>
</concept>
<concept>
<concept_id>10003752.10003753</concept_id>
<concept_desc>Theory of computation~Models of computation</concept_desc>
<concept_significance>300</concept_significance>
</concept>
<concept>
<concept_id>10002951.10002952.10003190.10003195</concept_id>
<concept_desc>Information systems~Parallel and distributed DBMSs</concept_desc>
<concept_significance>300</concept_significance>
</concept>
<concept>
<concept_id>10011007.10010940.10010941.10010949.10010950.10010954</concept_id>
<concept_desc>Software and its engineering~Garbage collection</concept_desc>
<concept_significance>300</concept_significance>
</concept>
</ccs2012>
\end{CCSXML}

\ccsdesc[500]{Theory of computation~Logic and databases}
\ccsdesc[300]{Theory of computation~Models of computation}
\ccsdesc[300]{Information systems~Parallel and distributed DBMSs}
\ccsdesc[300]{Software and its engineering~Garbage collection}


%
% End generated code
%

%
%  Use this command to print the description
%
\printccsdesc

% We no longer use \terms command
%\terms{Theory}

\keywords{CALM; Datalog; Garbage collection; Relational transducers}

%!TEX root = bloom_logical.tex

\section{Introduction}
\label{sec:intro}

Distributed and parallel systems are becoming increasingly commonplace.
However, ensuring consistency is difficult and expensive as it requires distributed nodes to coordinate and wait for input and / or computation from other nodes before producing an output.
The CALM theorem \cite{hellerstein2010declarative,ameloot2013relational} provides a sharp characterization of coordination-freeness: monotone queries are exactly those that can be computed without coordination.
Programmers motivated to write coordination-free programs thus have a clear task to express their queries in a monotone fashion.

This approach is taken in \cite{marczak2012confluence}, which introduced the spatio-temporal Datalog-like language of Dedalus, and specialized it to Dedalus$^+$, a sublanguage that disallows negation, mutation, and deletion\footnote{
	More precisely, all facts in Dedalus$^+$ are persisted over all time, so nothing derived is ever `mutated' or `deleted'.
}.
The analysis of \cite{marczak2012confluence} proves that programs expressed in Dedalus$^+$ always produce the same output regardless of the order in which messages are sent and received; coordination is therefore unnecessary for such monotone programs.

A related though different programming model for simplifying distributed consistency issues is to write systems avoiding mutable state, where processes accumulate and exchange immutable logs of messages or events, a model that has been termed Event Log Exchange (ELE).
Edelweiss \cite{conway2014edelweiss} takes this approach, and specializes Bloom \cite{alvaro2011consistency}, an instantiation of Dedalus, to a sublanguage that omits primitives for mutating or deleting data.
However, using only immutable state creates a separate problem of unbounded storage growth, since data is never deleted.
To combat this problem, \cite{conway2014edelweiss} proposed garbage collection rules for some common operations.
These rules are automatically added to the Edelweiss program, discarding data that will never be useful in the future.

Unfortunately, \cite{conway2014edelweiss} suffers from a number of shortcomings.
\begin{enumerate}
\item First, Edelweiss is often dependent on set negation for identifying opportunities for garbage collection.
The programmer is thus forced to give up coordination-freeness (through the use of negation) for the chance to reclaim storage, or vice versa.
\item Secondly, garbage collection rules are by their very nature deletion operations, even if the original program is monotone.
This makes it impossible to appeal to the analysis of \cite{marczak2012confluence} to prove coordination-freeness.
In fact, since the CALM theorem states an equivalency of monotone queries and coordination-free programs, one should suspect that adding non-monotone deletion operations would introduce the need for coordination.
\item Lastly, \cite{conway2014edelweiss} only provides intuitive arguments without formal proofs for why each individual garbage collection rule is correct.
This ad-hoc approach does not provide a procedure for identifying new garbage collection rules and for proving their correctness.
\end{enumerate}

In this paper, we address the above shortcomings and move beyond the approach of \cite{conway2014edelweiss}.
The intuitions behind our approach are (i) data that does not affect future computation can be safely reclaimed, and (ii) garbage collection does not affect the monotonicity of the output and coordination-freeness of the program.
Our contributions are:
\begin{enumerate}
\item We formalize a general framework for garbage collection, and identify a class of garbage collection rules that may be safely added to the Bloom program without compromising correctness.
\item We provide specific examples of garbage collection rules corresponding to monotone operations, thereby demonstrating that opportunities for storage reclamation are not dependent on non-monotone set negations.
\item We explain why our garbage collection does not require additional coordination, despite using non-monotone deletions.
\end{enumerate}

In particular, our results will show that a monotone program can be automatically garbage collected while remaining coordination-free.
We will also show that the garbage collection rules of \cite{conway2014edelweiss} are examples of the class of rules we identifies, and so our approach is a generalization of \cite{conway2014edelweiss}.





The rest of this paper is organized as follows.
We begin with some preliminaries and notations in Section \ref{sec:prelims}.
An overview of our approach is presented in Section \ref{sec:approach}.
Section \ref{sec:rewrites} then describes the rewrite of an ELE program expressed in Edelweiss into a logical garbage collection program and an instantiated garbage collection program.
We then analyze the logical program in Section \ref{sec:logical} and the instantiated program in Section \ref{sec:instantiated}.
For clarity of exposition, we defer the discussion of garbage collection for asynchronous rules (which send messages between nodes) to Section \ref{sec:async_gc}.
Finally, we conclude in Section \ref{sec:conclude}.



%!TEX root = bloom_logical.tex

\section{Preliminaries}
\label{sec:prelims}
\subsection{CALM}

\subsection{Dedalus, Bloom, Edelweiss}
Dedalus is a spatio-temporal Datalog language, with the following types of rules.
\begin{enumerate}
\item \emph{Deductive} rules are rules which are local in space and time:
  \[\tt{p(}\overline{\tt{W}}\tt{)} ~\leftarrow~ \tt{b}_1\tt{(}\overline{\tt{X}}_1\tt{)}, \dots, \tt{b}_m\tt{(}\overline{\tt{X}}_m\tt{)}, \neg\tt{c}_1\tt{(}\overline{\tt{Y}}_1\tt{)}, \dots, \neg\tt{c}_n\tt{(}\overline{\tt{Y}}_n\tt{)}.\]
\item \emph{Inductive} rules are local in space and but take place at the next time step:
  \[\tt{p(}\overline{\tt{W}}\tt{)@next} ~\leftarrow~ \tt{b}_1\tt{(}\overline{\tt{X}}_1\tt{)}, \dots, \tt{b}_m\tt{(}\overline{\tt{X}}_m\tt{)}, \neg\tt{c}_1\tt{(}\overline{\tt{Y}}_1\tt{)}, \dots, \neg\tt{c}_n\tt{(}\overline{\tt{Y}}_n\tt{)}.\]
\item \emph{Asynchronous} rules are derivations occurring at an unspecified future time, at possibly different nodes:
  \[\tt{p(\#D,}\overline{\tt{W}}\tt{)@async} ~\leftarrow~ \tt{b}_1\tt{(\#S,}\overline{\tt{X}}_1\tt{)}, \dots, \tt{b}_m\tt{(\#S,}\overline{\tt{X}}_m\tt{)}, \neg\tt{c}_1\tt{(\#S,}\overline{\tt{Y}}_1\tt{)}, \dots, \neg\tt{c}_n\tt{(\#S,}\overline{\tt{Y}}_n\tt{)},\]
indicates a derivation at node $\tt{D}$ caused by node $\tt{S}$.
\end{enumerate}

Bloom statements are of the form \textit{<collection-variable> <op> <collection-expression>}, where $op$ is one of the following:
\begin{enumerate}
\item $\tt{<=}$ merge: LHS includes content of RHS in the current timestep
\item $\tt{<+}$ deferred merge: LHS includes content of RHS in the next timestep
\item $\tt{<-}$ deferred delete: LHS will not include content of RHS in the next timestep
\item $\tt{<}\sim$ async merge: (Remote) LHS will include content of RHS in some non-deterministic future timestep
\end{enumerate}
The Bloom rules correspond to Dedalus rules: merges are deductive, deferred merges and deletes are inductive, and asynchronous merges are asynchronous Dedalus rules.
For example, the Bloom statement $\tt{P <= join [B, C], [B.key1, C.key2]}$ corresponds to the Dedalus rule $\tt{p(key,}\overline{\tt{X}}\tt{,}\overline{\tt{Y}}\tt{)}$ $~\leftarrow~$ $\tt{b(key,}\overline{\tt{X}}\tt{),$ $c(key,}\overline{\tt{Y}}\tt{)}$\footnote{Is this correct Dedalus?}.

We observe that the expression $\tt{join [B, C], [B.key1, C.key2]}$ is really a function of $\tt{B}$ and $\tt{C}$, and we could replace it with some $\tt{f(B,C)}$.
In the sequel, we will use this functional notation for ease of exposition; however, the reader should keep in mind that underlying the notation is a Bloom / Dedalus / Datalog rule defining derivation of tuples.

(Also need to talk about tables, scratches, and channels.)

Edelweiss restricted Bloom to a sublanguage that omits primitives for mutating or deleting data:
\begin{enumerate}
\item Deletion rules ($\tt{<-}$) cannot be used.
\item Channel messages are stored persistently. That is, the LHS of a rule that reads messages from a channel must be persistent.
\item Channels are derived from persistent collections. That is, if a channel appears on the LHS of a rule, the rule's RHS must consist of monotone operators over persistent collections.
\end{enumerate}

Bloom$^L$ extends Bloom to support lattices and merge ($\tt{<=}$) and deferred merge ($\tt{<+}$) operations on lattices.
While our original program will be in Bloom, our rewritten GC program will be in  Bloom$^L$.


\subsection{Lattices: Tombstones and Products}
% Promote everything to tombstone sets, and make state a product lattice.
% Any sub-vector of the product lattice is itself a lattice; we can define join and meet on the sub-lattice to mean operations on full product lattice.
A bounded join semilattice $\mathcal{L}$ is a triple $(S, \cup, \bot)$ where $S$ is a set, $\cup$ is a binary, associative, commutative, and idempotent operator, and $\bot \in S$ is a least element such that $\forall s \in S: s \cup \bot = s$.
The operator $\cup$ produces a partial order on $S$; we say that $s \leq t$ if $s \cup t = t$.
For convenience, we will often refer to the elements of $\mathcal{L}$ themselves as the `lattice'.

For our purposes of garbage collection, we will sometimes define a binary difference operator $-$, with the requirement that $s - \bot = s$ for all $s \in \mathcal{S}$.


\subsubsection{Product lattices}
Given $L$ lattices $\mathcal{L}_i = (S_i, \cup_i, \bot_i)$ for $i = 1, \dots, L$, we define the product lattice $\mathcal{L}_\times = \left(S_1 \times \dots \times S_L, \cup_\times, \bot_\times\right)$, where $\langle s_1, \dots, s_L \rangle \cup_\times \langle t_1, \dots, t_L \rangle = \langle s_1 \cup_1 t_1, \dots, s_L \cup_L t_L \rangle$ and $\bot_\times = \langle \bot_1, \dots, \bot_L \rangle$.
Conversely, given a product lattice $\mathcal{L}_\times$ and a subset $V =\{v_1, \dots, v_n\} \subseteq \{1, \dots, L\}$, we can define a sub-product lattice $\mathcal{L}_V = (S_{v_1} \times \dots S_{v_n}, \cup_V, \bot_V)$ with the obvious join $\cup_V$ and least element $\bot_V$.
Given two elements $\langle s_{v_1}, \dots, s_{v_n}\rangle \in \mathcal{L}_V$ and $\langle t_{u_1}, \dots, t_{u_m}\rangle \in \mathcal{L}_U$ in two lattices $\mathcal{L}_V$ and $\mathcal{L}_U$ such that $V \subseteq \{1,\dots,L\}$ and $U \subseteq \{1,\dots,L\}$, we define their merge as
$\langle s_{v_1}, \dots, s_{v_n}\rangle \cup \langle t_{u_1}, \dots, t_{u_m}\rangle = \langle x_{v,1}, \dots, x_{v,L} \rangle \cup_\times \langle y_{u,1}, \dots, y_{u,L} \rangle$,
where $x_{v,i} = s_i$ if $i \in V$ and $x_{v,i} = \bot_i$ otherwise, and analogously for $t$ and $y$.

If differences are defined for each lattice $\mathcal{L}_i$, then we can also define a difference operator for $\mathcal{L}_\times$ and on the lattices $\mathcal{L}_U$, $\mathcal{L}_V$ in the same way as we had for $\cup$.

Product lattices gives us a convenient way of composing multiple lattices so that we may operate on them together.
This will be particular handy when we garbage collect from across multiple sets simultaneously.


\subsubsection{Tombstone lattices}
\textbf{Lattices for logical rewrite $\mathfrak{P}_{GC}$:}
Our main algebraic tool for the logical rewrite is the \emph{tombstone lattice}:
given a table $\tt{A}$ in our program, we lift it to a 3-phase lattice $\tt{A}_{TS} = (\tt{A}_\exists, \tt{A}_\top)$ endowed with an additional `tombstone' state $\top$.
Intuitively, tuples marked as tombstoned are those that can be safely reclaimed without affecting computation or confluence.
We will maintain $\tt{A}_\exists \cap \tt{A}_\top = \emptyset$.
The join operator for tombstone lattices is defined as:
\begin{align*}
\tt{A}_{TS} \cup \tt{B}_{TS} = (\tt{A}_\exists, \tt{A}_\top) \cup (\tt{B}_\exists, \tt{B}_\top) = ((\tt{A}_\exists \cup \tt{B}_\exists) - (\tt{A}_\top \cup \tt{B}_\top), \tt{A}_\top \cup \tt{B}_\top).
\end{align*}
It is easy to verify that the merge is associative, commutative and idempotent.
The least element is simply the empty set $(\emptyset, \emptyset)$.

We will also have unique keys for some tables, which we denote as $\tt{A}_!$ or $\pi_!(\tt{A})$, the projection of $\tt{A}$ onto the unique keys.
The keys $\tt{A}_!$ can be kept as part of the lattice, which is now augmented to $(\tt{A}_\exists, \tt{A}_\top, \tt{A}_!)$, with the requirement that $\tt{A}_! = \pi_!(\tt{A}_\exists \cup \tt{A}_\top)$.
In an abuse of notation, we will sometimes use $\tt{A}_!$ to refer to the tuples $(\tt{A}_\exists \cup \tt{A}_\top)$ instead of keys $\pi_!(\tt{A}_\exists \cup \tt{A}_\top)$ of the table.
The join on this lattice is defined by
\begin{align*}
(\tt{A}_\exists, \tt{A}_\top, \tt{A}_!) \cup (\tt{B}_\exists, \tt{B}_\top, \tt{B}_!) \quad=\quad ((\tt{A}_\exists \cup \tt{B}_\exists) - (\tt{A}_! \cup \tt{B}_!), \tt{A}_\top \cup \tt{B}_\top, \tt{A}_! \cup \tt{B}_!).
\end{align*}
Note that $\tt{A}_!$ is a redundant representation if $\tt{A}_\exists$ and $\tt{A}_\top$ are provided.
Hence, in the logical garbage collection program $\mathfrak{P}_{GC}$ defined in Section \ref{sec:logical}, we will omit the use of $\tt{A}_!$.

\textbf{Lattices for logical rewrite $\mathfrak{P}_{GC}$:}
The above representation requires storage for tombstoned tuples in $\tt{A}_\top$, and does not enable any real storage reclamation.
In our instantiated program $\mathfrak{P}_{iGC}$ defined in Section \ref{sec:instantiated}, we will use compressed representations where we drop some or all of $\tt{A}_\top$.
We keep instantiated tuples $\tt{A}_I$ instead of $(\tt{A}_\exists, \tt{A}_\top)$; in Section \ref{sec:instantiated} we will show $\tt{A}_\exists \subseteq \tt{A}_I \subseteq \tt{A}_\exists \cup \tt{A}_\top$.

If primary keys are not available, or if we choose not to instantiate and keep them, we will call $\tt{A}_I$ the \emph{non-augmented} representation.
Otherwise, if primary keys are available and we choose to keep them, we will call $\tt{A}_E = (\tt{A}_I, \tt{A}_!)$ the \emph{key-augmented} representation.
The obvious join for this lattice is
\begin{align*}
\tt{A}_E \cup \tt{B}_E \quad=\quad (\tt{A}_I, \tt{A}_!) \cup (\tt{B}_I, \tt{B}_!) \quad=\quad ((\tt{A}_I \cup \tt{B}_I) - (\tt{A}_! \cup \tt{B}_!), \tt{A}_! \cup \tt{B}_!).
\end{align*}
We also define the difference operator
% \footnote{In the sequel, we will only use differences on $\tt{A}$ and $\tt{A}_E$}
as
\begin{align*}
% \tt{A}_{TS} - \tt{B}_{TS} &\quad=\quad (\tt{A}_\exists - (\tt{B}_\exists \cup \tt{B}_\top), \tt{A}_\top \cup \tt{B}_\exists \cup \tt{B}_\top) &&\quad=\quad \tt{A}_{TS} \cup (\emptyset, \tt{B}_\exists \cup \tt{B}_\top),\\
% (\tt{A}_\exists, \tt{A}_\top, \tt{A}_!) - (\tt{B}_\exists, \tt{B}_\top, \tt{B}_!) &\quad=\quad (\tt{A}_\exists - \tt{B}_!, \tt{A}_\top \cup \tt{B}_\exists \cup \tt{B}_\top, \tt{A}_! \cup \tt{B}_!) &&\quad=\quad (\tt{A}_\exists, \tt{A}_\top, \tt{A}_!) \cup (\emptyset, \tt{B}_\exists \cup \tt{B}_\top, \tt{B}_!),\\
\tt{A}_E - \tt{B}_E &\quad=\quad (\tt{A}_\exists - \tt{B}_!, \tt{A}_! \cup \tt{B}_!) &&\quad=\quad \tt{A}_E \cup (\emptyset, \tt{B}_!).
\end{align*}
% In all cases, the difference promotes all tuples in the second operand to the tombstone state.
That is, the difference deletes all tuples in the second operand $\tt{B}_E$ from the first operand $\tt{A}_E$.


\subsection{Bloom$^L$ Execution}
\label{sec:prelims:exec}
(Using formalism and definitions of Bloom$^L$ writeup..)

A Bloom$^L$ \emph{program} is a set $\{\mathfrak{P}_v ~:~ v \in V\}$ for some finite $V$, where each $\mathfrak{P}_v$ is itself a finite collection of rules.
A message-time assignment $\tau$ is a function taking $(R_i, v, t, v')$, with $R_i \in \mathfrak{P}_v$ asynchronous merge rule, and returning $\tau(R_i, v, t, v') = t'$.
Intuitively, this indicates that the message sent from node $v$ at time $t$ due to asynchronous merge rule $R_i$ will arrive at the destination node $v'$ at time $t'$.
The message-time assignment $\tau$ is \emph{causally plausible} if it obeys causality, i.e., there is no sequence $(R_1, v_1, t_1, v'_1), \dots, (R_n, v_n, t_n, v'_n)$ such that $v_k = v'_{k-1}$, and $\tau(R_k, v_k, t_k, v'_k) \leq t_{k+1}$, but also $v'_n = v_1$ and $t_1 \geq \tau(R_n, v_n, t_n, v_n)$.
Note that a causally plausible $\tau$ induces a (partial) ordering on the node-timestamp space: $(v,t) <_\tau (v',t')$ if either $\exists R \in \mathfrak{P}_v$ such that $\tau(R,v,t,v') \leq t'$, or $\exists v'', t''$ such that $(v,t) \leq_\tau (v'', t'')$ and $(v'', t'') <_\tau (v',t')$.
Intuitively, $(v,t) <_\tau (v',t')$ if there is a sequence of messages that allows us to go from $(v,t)$ to $(v',t')$.
For a causally plausible $\tau$, we cannot have $(v,t) <_\tau (v',t')$ and also $(v',t') <_\tau (v,t)$, as that would violate the causal plausibility of $\tau$.

Operationally, for each rule $R_i$, node $v$, and time $t$, we associate an operator $T_{i,v,t}(\mathcal{E}, I) := (\mathcal{E}, I')$:
\begin{enumerate}
\item\emph{Merge:} $I'(\tt{B}@(v,t)) ~=~ I(\tt{B}@(v,t)) ~\sqcup~ \tt{f(}I(\tt{A}_1@(v,t)),\dots,I(\tt{A}_k@(v,t))\tt{)}$.
\item\emph{Deferred Delete:} Deferred merges and deletes are treated together.
Suppose $\tt{B}$ appears on the LHS for statements
\begin{align*}
\tt{B} &\quad\tt{<+}\quad \tt{f(}\tt{A}_{11},\dots,\tt{A}_{1k_1}\tt{)}\\
&\quad\vdots\\
\tt{B} &\quad\tt{<+}\quad \tt{f(}\tt{A}_{n1},\dots,\tt{A}_{nk_n}\tt{)}\\
\tt{B} &\quad\tt{<-}\quad \tt{f(}\tt{C}_{11},\dots,\tt{C}_{1l_1}\tt{)}\\
&\quad\vdots\\
\tt{B} &\quad\tt{<-}\quad \tt{f(}\tt{C}_{m1},\dots,\tt{C}_{ml_m}\tt{)}
\end{align*}
Then the corresponding operator is
\begin{align*}
I'(\tt{B}@(v,t+1)) ~=~ I(\tt{B}@(v,t+1)) ~\sqcup~ \Bigg(
&\bigsqcup_{i=1}^n \tt{f(}I(\tt{A}_{i1}@(v,t)),\dots,I(\tt{A}_{ik_i}@(v,t))\tt{)}\\
&~-~
\bigsqcup_{i=1}^n \tt{f(}I(\tt{C}_{i1}@(v,t)),\dots,I(\tt{C}_{il_i}@(v,t))\tt{)}
\Bigg)
\end{align*}
\item\emph{Asynchronous Merge:} $I'(\tt{chn}@(v',t')) ~=~ I(\tt{chn}@(v',t')) ~\sqcup~ \tt{f}_{v'}\tt{(}I(\tt{A}_1@(v,t)),\dots,I(\tt{A}_k@(v,t))\tt{)}$, where $t' = \tau(R_i, v, t, v')$.
\end{enumerate}
For simplicity, we assume that each relation $\tt{A}$ is spatially unique, i.e., there exists only one $v$ for which $\tt{A}@v$ is instantiated, and thus drop the node specifier when it is obvious from context.
We similarly assume that channels are one-to-one, and omit the destination node specifier.

An instantaneous run $\phi$ is a successive application of merge ($\tt{<=}$) operators $T_{\phi(1),v,t}, T_{\phi(2),v,t}, \dots$ to some initial condition $(\mathcal{E}, I)$.
A between-instants run $\psi$ is a successive application of asynchronous merge ($\tt{<}\sim$), and deferred merge and delete ($\tt{<+}$ and $\tt{<-}$) operators $T_{\psi(1),v,t}, T_{\psi(2),v,t}, \dots$ to some initial condition $(\mathcal{E}, I)$.
Observe that a between-instants run always reaches the same fixed point for a given initial condition $(\mathcal{E}, I)$, since its operators are dependent only on relations at $(v,t)$, but make modifications to relations at $(v,t+1)$ or some $(v',t') >_\tau (v,t)$.

A run $\Phi$ is a sequence of runs $\phi_{v_1,t_1}, \psi_{v_1,t_1},, \phi_{v_2,t_2}, \psi_{v_2,t_2}, \dots$ such that $\forall (v_i, t_i), \forall (v, t) <_\tau (v_i, t_i), \exists j < i: (v, t) = (v_j, t_j)$, that is, every $(v,t)$ that is $\tau$-less than $(v_i, t_i)$ must be run before $(v,t)$.
Furthermore, $\phi_{v_1, t_1}$ is applied with the initial conditions $(\mathcal{E}, I_\emptyset)$, and $\psi_{v_k,t_k}$ is applied to the fixed point of $\phi_{v_k, t_k}$, and $\phi(v_{k+1}, t_{k+1})$ is applied to the fixed point of $\psi_{v_k, t_k}$ as the initial conditions.
This is only well defined for $\psi_{v_k, t_k}$ and $\phi_{v_{k+1}, t_{k+1}}$ when $\phi_{v_{k'}, t_{k'}}$ has a fixed point for all $k' \leq k$.
We also point out that by definition of the operators, an instantaneous run $\psi_{v_k,t_k}$ and $\phi_{v_{k+1},t_{k+1}}$ cannot affect any relation $\tt{B}@(v',t')$ if $(v',t') \leq_\tau (v_k,t_k)$;
hence, the relation $\tt{B}@(v',t')$ remains unchanged after we terminate $\phi_{v',t'}$.
% We denote the assignment obtained after applying $T_{\pi_{v,t}(i),v,t}$ as $I^{\Pi_\tau}_{v,t,i}$;
% if $\pi_{v,t}$ terminates, we denote the fixed point as $I^{\Pi_\tau}_{v,t}$.

We assume that every instantaneous run $\phi_{v,t}$ terminates, for otherwise our Bloom$^L$ program does not halt.

Let $I^\Phi_{v,t,i}$ be the assignment obtained after applying $T_{\phi_{v,t}(i),v,t}$, and $I^\Phi_{v,t+1,0}$ be the assignment obtained after applying $\psi_{v,t}$.
We will most often be interested in the assignment to lattices at $(v,t)$ after partially applying $\phi_{v,t}$ and $\psi_{v,t}$;
for simplicity, we will write $\tt{A}_{t,i} = I^\Phi_{v,t,i}(\tt{A}@(v,t))$ and $\tt{A}_{t} = \tt{A}_{t,0} = I^\Phi_{v,t,0}(\tt{A}@(v,t))$.



%!TEX root = bloom_logical.tex

\section{Approach}
\label{sec:approach}

Our approach in this paper proceeds via two program rewrites.
The first rewrite takes an Edelweiss program $\mathfrak{P}$ and produces a \emph{logical garbage collection} program $\mathfrak{P}_{GC}$, where tuples are marked for reclamation without actually being reclaimed.
A second rewrite then turns $\mathfrak{P}_{GC}$ into an \emph{instantiated garbage collection} program $\mathfrak{P}_{iGC}$ which actually deletes tuples and reclaims storage.
Only $\mathfrak{P}_{iGC}$ is eventually executed;
however, we will argue that if all three programs had been executed together, the sets in $\mathfrak{P}$, $\mathfrak{P}_{GC}$, and $\mathfrak{P}_{iGC}$ would evolve together and maintain relationships to one another.
These relationships will demonstrate that $\mathfrak{P}_{iGC}$ produces the same output as $\mathfrak{P}$ without the need for additional coordination.

\textbf{Logical Rewrite.}
Key to the logical rewrite are `tombstone sets' which contain tuples that have been marked as `tombstoned', or safe for deletion.
Every set $\tt{A}$ in $\mathfrak{P}$ has a corresponding tombstone set $\tt{A}_{TS} = (\tt{A}_\exists, \tt{A}_\top)$ in $\mathfrak{P}_{GC}$.
Logical garbage collection rules are also added to $\mathfrak{P}_{GC}$ to mark tuples as being safe for deletion, by moving them from $\tt{A}_\exists$ to $\tt{A}_\top$.

A simple induction on the execution of $\mathfrak{P}$ and $\mathfrak{P}_{GC}$ demonstrates that we always have $\tt{A} = \tt{A}_\exists \cup \tt{A}_\top$.
Since we also never tombstone any tuples from sets that have been identified as output sets, $\mathfrak{P}_{GC}$ produces the correct output $\tt{A}_\exists = \tt{A}$.
In addition, we ensure that tombstoned tuples can in fact be safely deleted, i.e., reclaiming the tuples will not affect the execution and output of the program.
We formalize this idea in Invariant \ref{inv:merge_gc}, and prove that the invariant is maintained by $\mathfrak{P}_{GC}$.

The monotonicity of logical garbage collection ensures, by the CALM theorem, that no additional coordination is required.
In particular, if $\mathfrak{P}$ is monotone and coordination-free, then $\mathfrak{P}_{GC}$ is also monotone and coordination-free.


\textbf{Instantiated Rewrite.}
The logical rewrite $\mathfrak{P}_{GC}$ does not in fact provide any storage reclamation.
To do that, we perform a second write of $\mathfrak{P}_{GC}$ into an \emph{instantiated garbage collection} program $\mathfrak{P}_{iGC}$.
Unlike $\mathfrak{P}_{GC}$ which operates on tombstone sets $\tt{A}_{TS}$'s, $\mathfrak{P}_{iGC}$ operates on ordinary sets, denoted $\tt{A}_I$ to indicate that deletions are instantiated.
Some of the sets $\tt{A}_I$ in $\mathfrak{P}_{iGC}$ are also augmented with primary keys $\tt{A}_!$, if primary keys are available and provide useful information for more efficient garbage collection.
Instantiated garbage collection rules are added to $\mathfrak{P}_{iGC}$ corresponding to logical garbage collection rules of $\mathfrak{P}$.

On every set $\tt{A}$, the instantiated program $\mathfrak{P}_{iGC}$ either tries to maintain \emph{weak consistency} with $\mathfrak{P}_{GC}$ on $\tt{A}$, that is,
\[ \tt{A}_\exists \subseteq \tt{A}_I \subseteq \tt{A}_\exists \cup \tt{A}_\top = \tt{A},\]
or it maintains \emph{strong consistency} on $\tt{A}$, that is,
\[ \tt{A}_\exists = \tt{A}_I \subseteq \tt{A}_\exists \cup \tt{A}_\top = \tt{A}.\]
Intuitively, tuples in $\tt{A}_\exists$ are important for future derivations and must be kept in $\tt{A}_I$.
Tombstones $\tt{A}_\top$ of a weakly consistent $\tt{A}$ can be partially deleted; for a strongly consistent $\tt{A}$, the tombstones $\tt{A}_\top$ must be completely removed from $\tt{A}_I$, or else could interfere in future derivations.
When the above relationship holds, Invariant \ref{inv:merge_gc} is a statement that executing a Bloom operator in $\mathfrak{P}_{iGC}$ using instantiated sets $\tt{A}_I$'s produces the same outcome as the equivalent execution in $\mathfrak{P}$ using the complete sets $\tt{A}$.

In order to maintain the consistency of $\mathfrak{P}_{iGC}$ with $\mathfrak{P}_{GC}$ on $\tt{A}$, we will also require that the instantiated garbage collection rule $\tt{iGC}$ for $\tt{A}_I$ agrees with its corresponding logical garbage collection rule $\tt{GC}$ for $\tt{A}_{TS}$.
Specifically, if $\tt{A}_I$ is weakly consistent with $\tt{A}_{TS}$, then $\tt{iGC}$ must delete a subset of tuples that are tombstoned by $\tt{GC}$; and if $\tt{A}_I$ is strongly consistent with $\tt{A}_{TS}$, then $\tt{iGC}$ must delete exactly the tuples that are tombstoned by $\tt{GC}$.
The decision of whether to maintain strong or weak consistency for the set $\tt{A}$ is determined by its associated instantiated garbage collection rule.

\subsection{Coordination-Freeness}
By construction, garbage collection rules in $\mathfrak{P}_{GC}$ are monotone, and only promote tuples from $\tt{A}_\exists$ to $\tt{A}_\top$.
Hence, the CALM theorem implies that $\mathfrak{P}_{GC}$ requires no additional coordination over $\mathfrak{P}$ to achieve confluence.
In particular, if $\mathfrak{P}$ is already monotone and coordination-free, then $\mathfrak{P}_{GC}$ is also monotone and coordination-free.
This is the case if $\mathfrak{P}$ is written in monotone Edelweiss (i.e., the Bloom equivalent of Dedalus+), in which case $\mathfrak{P}_{iGC}$ is expressed in monotone Bloom$^L$.

The coordination-freeness of $\mathfrak{P}_{iGC}$ is a more complex issue.
Instantiated garbage collection rules do delete tuples but shrink the instantiated sets, and are not guaranteed to be monotone.
Hence, $\mathfrak{P}_{iGC}$ is not expressible in monotone Bloom$^L$, and we cannot appeal to analysis of \cite{marczak2012confluence} to demonstrate confluence.
Nevertheless, our correctness proofs show that $\tt{P}_{iGC}$ produces the same output as $\mathfrak{P}$.
Indeed, one can show that $\mathfrak{P}_{iGC}$ is not necessarily confluent on non-output sets, in particular, different runs of $\mathfrak{P}_{iGC}$ could disagree on the tuples which are tombstoned in $\mathfrak{P}_{GC}$.
Coordination is needed only to achieve agreement on the non-output sets.
On the other hand, we will argue that $\mathfrak{P}_{iGC}$ is confluent if viewed as a query on the output sets, for which no additional coordination is required.
This argument is along the lines of \cite{ameloot2013relational}, which formalizes the CALM theory as a statement that the output query is monotone if and only if it is coordination free.
One may therefore choose to execute $\mathfrak{P}_{iGC}$ coordination-free, and still get confluence in the sets that one cares about.











%!TEX root = bloom_logical.tex

\section{Rewrites}
\label{sec:rewrites}

We present rewrites of an Edelweiss program $\mathfrak{P}$ into a logical garbage collection program $\mathfrak{P}_{GC}$ and an instantiated garbage collection program $\mathfrak{P}_{iGC}$.

\setcounter{subsection}{-1}
\subsection{Simplifying copy rewrite}
Before presenting the rewrites $\mathfrak{P}_{GC}$ and $\mathfrak{P}_{iGC}$, We first make the following simplifying assumption about $\mathfrak{P}$.
\begin{assumption}
\label{ass:copy}
Let $\tt{A}$ be a table or channel in $\mathfrak{P}$ appearing in multiple rules
\begin{align*}
R_1: && \tt{B}_1 &\quad\tt{<op>}\quad \tt{f}_1\tt{(}\tt{A},\tt{C}_{11},\dots,\tt{C}_{1k_1}\tt{)}\\
&&&\quad\vdots\\
R_n: && \tt{B}_n &\quad\tt{<op>}\quad \tt{f}_n\tt{(}\tt{A},\tt{C}_{n1},\dots,\tt{C}_{nk_n}\tt{)}
\end{align*}
with $n \geq 2$.
Then it must be the case that $R_i: \tt{B}_i ~\tt{<=}~ \tt{A}$ is an identity copy.
\end{assumption}
Assumption \ref{ass:copy} states that any table $\tt{A}$ can either appear as input on the RHS of a single rule, or otherwise each of the rules it appears in must be a copy rule.

The assumption can be made without loss of generality, as any $\mathfrak{P}$ can always be rewritten into a program $\mathfrak{P}_0$ that satisfies Assumption \ref{ass:copy}.
Suppose $\tt{A}$ is a table in $\mathfrak{P}$ appearing on the RHS of multiple rules $R_1,\dots,R_n$.
We can define $\tt{A}_i$ in $\mathfrak{P}_0$ for each $R_i$, and replace $\tt{A}$ with $\tt{A}_i$ in $R_i$, and add the rules $\tt{A}_i ~\tt{<=}~ \tt{A}$.
The resultant $\mathfrak{P}_0$ is essentially equivalent to $\mathfrak{P}$, in the sense that any run of $\mathfrak{P}$ corresponds to a run of $\mathfrak{P}_0$, and vice versa.

In practice, the above rewrite should not increase storage significantly.
One only needs to keep a single additional bit for each tuple and $R_i$ to record whether the tuple is in $\tt{A}_i$.

This assumption is really only required so that we can specify garbage collection for each individual rule $R_i$ without having to reason about the logic of composing the garbage collection together.

We also point out that the rules in Assumption \ref{ass:copy} can be combined into a single rule using the product lattice notation:
\begin{align*}
R: && \langle \tt{B}_1, \dots, \tt{B}_n \rangle &\quad\tt{<=}\quad \langle \underbrace{\tt{A}, \dots, \tt{A}}_{\text{$n$ times}} \rangle
\end{align*}
so we can make the following assumption instead:
\begin{assumption}
\label{ass:single_rule}
Every table or channel in $\mathfrak{P}$ appears in the RHS of at most one rule.
\end{assumption}


\subsection{Logical Garbage Collection Rewrite}
The idea of the GC rewrite is to lift the program's state into tombstone lattices.
Within the timestep, we perform operations on the lifted representation instead of the original state.
Garbage collection occurs between timesteps, marking tuples for reclamation.
In this section, our storage reclamation is only logical, i.e., tuples are never actually deleted.
This leads to a program with no additional coordination required.
We then present a second rewrite where the reclamation is instantiated; in the second rewrite, we guarantee confluence of output sets, but not of intermediate sets from which storage is reclaimed.

For clarity, we also omit garbage collection for asynchronous rules.
Doing so will allow us to develop simpler analyses that illuminate the overall approach.
We will then revisit garbage collection for asynchronous rules in later sections, building on the approach in this section.

We begin with a program $\mathfrak{P}$ with rules $R_1, \dots, R_n$, and rewrite it into a GC program $\mathfrak{P}_{GC}$.
We assume that all tables in $\mathfrak{P}$ are persistent (with implicit rules $\tt{A} ~\tt{<+}~ \tt{A}$), but will make any persistence in $\mathfrak{P}$ explicit.



\subsubsection{State}
Let $\tt{A}$ be a table in $\mathfrak{P}$.
We introduce the table $\tt{A}_\exists$ and tombstone set $\tt{A}_{TS} = (\tt{A}_\exists, \tt{A}_\top)$ in $\mathfrak{P}_{GC}$.
Intuitively, $\tt{A}_\exists$ should reflect facts in $\tt{A}$ which are important for future derivations and thus not garbage collected.
$\tt{A}_{TS}$ contains facts of the current timestep, but with tombstones marking tuples that could be deleted without affecting future derivations.
We also add rule
\begin{align}
(\tt{A}_\exists, \tt{A}_\top) &\quad\tt{<+}\quad (\tt{A}_\exists, \tt{A}_\top)\label{eq:aexistmerge}
\end{align}
All non-GC rules in $\mathfrak{P}_{GC}$ will be of the form $(\tt{A}_\exists, \tt{A}_\top) ~\tt{<op>}~ (\tt{X}, \emptyset)$.


\subsubsection{Merges, Deferred Merges}
We next consider rules in the original program $\mathfrak{P}$.
For every rule
\begin{align}
\tt{B} \quad\tt{<op>}\quad \tt{f(}\tt{A}_1,\dots,\tt{A}_n\tt{)}
\end{align}
in $\mathfrak{P}$, where $\tt{<op>}$ is a merge $\tt{<=}$ or deferred merge $\tt{<+}$, we add the below rule to $\mathfrak{P}_{GC}$:
\begin{align*}
(\tt{B}_\exists, \tt{B}_\top) &\quad\tt{<op>}\quad (\tt{f(}\tt{A}_{1,\exists} \cup \tt{A}_{1,\top},\dots,\tt{A}_{n,\exists} \cup \tt{A}_{n,\top}\tt{)}, \emptyset)
\end{align*}

\subsubsection{Asynchronous merges}
Asynchronous merges are messaging rules between nodes.
On the sender side, in $\mathfrak{P}$, we have the rule 
\begin{align}
\overline{\tt{B}} \quad\tt{<}\sim\quad \tt{f(}\tt{A}_1,\dots,\tt{A}_n\tt{)}
\end{align}
and on the receiver side we have
\begin{align}
\tt{B} \quad\tt{<=}\quad \overline{\tt{B}}
\end{align}
These are respectively replaced in $\mathfrak{P}_{GC}$ by
\begin{align}
(\overline{\tt{B}}, \emptyset) &\quad\tt{<}\sim\quad (\tt{f(}\tt{A}_{1,\exists} \cup \tt{A}_{1,\top},\dots,\tt{A}_{n,\exists} \cup \tt{A}_{n,\top}\tt{)}, \emptyset)
\end{align}
and
\begin{align}
(\tt{B}_\exists, \tt{B}_\exists) &\quad\tt{<=}\quad (\overline{\tt{B}}, \emptyset)
\end{align}
For now, we do not perform garbage collection for asynchronous merge rules.
That is, we treat all sets $\tt{A}_1, \dots, \tt{A}_n$ that appear as input on the RHS of asynchronous rules as `output' sets from which we never tombstone any tuple.
We defer the garbage collection for asynchronous rules to Section \ref{sec:async_gc}.

\subsubsection{Garbage collection}
The above rewrite rules do not perform any garbage collection; in fact, it is easy to see that they are equivalent to $\mathfrak{P}$.
We will now add rules that mark tuples for reclamation.

We say that a GC rule is safe for a rule $R_i$ if it only tombstones tuples that are guaranteed to have no effect on the future derivations of $R_i$.
This will be made formal in the section below.

Suppose that a table $\tt{A}$ is the input to a single rule $R: \tt{B} ~\tt{<op>}~ \tt{f}\tt{(}\tt{A},\tt{C}_1,\dots,\tt{C}_m\tt{)}$, where $\tt{<op>}$ is either a merge $\tt{<=}$ or deferred merge $\tt{<+}$, and $R$ is not the implicit persistence rule $\tt{A} ~\tt{<+} \tt{A}$.
Suppose also that $\tt{GC}$ is a garbage collection rule that is safe for $R$.
We will then add the following GC rule to $\mathfrak{P}_{GC}$:
\begin{align}
\langle \tt{A}_{TS}, \tt{C}_{1,TS}, \dots, \tt{C}_{m,TS} \rangle
\quad\tt{<+}\quad
\tt{GC}(\tt{A}_{TS}, \tt{C}_{1,TS}, \dots, \tt{C}_{m,TS}; \tt{B}_{TS}).
\end{align}
By Assumption \ref{ass:copy}, $\tt{C}_1,\dots,\tt{C}_m$ also do not appear in the input to any other rule of $\mathfrak{P}$.

If the table $\tt{A}$ is the input to multiple rules $R_1, \dots, R_m$, then it must be the case that $R_i: \tt{B}_i ~\tt{<=}~ \tt{A}$ is an identity copy.
In that case, we add the garbage collection rule
\begin{align}
(\tt{A}_\exists, \tt{A}_\top) \quad\tt{<+}\quad \tt{GCCopy}(\tt{A}_{TS}; \tt{B}_{1,TS}, \dots, \tt{B}_{m,TS})
\end{align}
where $\tt{GCCopy}(\tt{A}_{TS}; \tt{B}_{1,TS}, \dots, \tt{B}_{m,TS}) = \left(\tt{A}_\exists - \bigcap_{i=1}^m (\tt{B}_{i,\exists} \cup \tt{B}_{i,\top}), \tt{A}_\top \cup \left(\tt{A}_\exists \cap \bigcap_{i=1}^m (\tt{B}_{i,\exists} \cup \tt{B}_{i,\top})\right)\right)$
We will show $\tt{GCCopy}$ is safe for each $R_i$.

We highlight that we only add at most one GC rule for each $\tt{A}$ in $\mathfrak{P}_{GC}$\footnote{
While it is possible to have multiple GC rules performing reclamation, we would require an additional restriction that the GC rules maintain the GC Invariant \ref{inv:merge_gc} under merges.
(Maybe we can present this in an appendix?)
}, i.e. each $\tt{A}$ only appears on the LHS of one GC rule.


% \subsubsection{Example}
% Here we show an example Bloom program $\mathfrak{P}$ which we rewrite to a GC-enabled program $\mathfrak{P}_{GC}$.
% \begin{align}
% \tt{A} &\quad\tt{<=}\quad \tt{chn\_in}\\
% \tt{B} &\quad\tt{<=}\quad \pi_{x,y}\tt{(A)}\\
% \tt{C} &\quad\tt{<+}\quad \pi_z\tt{(A)}
% % \tt{chn\_out} &\quad\tt{<}\sim\quad \sigma_{x=1}\tt{(B)}\\
% \end{align}
% is rewritten as
% \begin{align}
% (\tt{A}_\exists, \tt{A}_\top) &\quad\tt{<+}\quad (\tt{A}_\exists, \tt{A}_\top)\\
% (\tt{B}_\exists, \tt{B}_\top) &\quad\tt{<+}\quad (\tt{B}_\exists, \tt{A}_\top)\\
% (\tt{C}_\exists, \tt{C}_\top) &\quad\tt{<+}\quad (\tt{C}_\exists, \tt{A}_\top)\\
% \nonumber\\
% (\tt{A}_\exists, \tt{A}_\top) &\quad\tt{<=}\quad (\tt{chn\_in},\emptyset)\\
% % \tt{chn\_ack\_in} &\quad\tt{<}\sim\quad \tt{chn\_in}\\
% \nonumber\\
% (\tt{B}_\exists, \tt{B}_\top) &\quad\tt{<=}\quad (\pi_{x,y}\tt{(}\tt{A}_\exists \cup \tt{A}_\top\tt{)}, \emptyset)\\
% \nonumber\\
% (\tt{C}_\exists, \tt{C}_\top) &\quad\tt{<+}\quad (\pi_z\tt{(}\tt{A}_\exists \cup \tt{A}_\top\tt{)}, \emptyset)\\
% \nonumber\\
% % \tt{chn\_out} &\quad\tt{<}\sim\quad \sigma_{x=1}\tt{(}\tt{B}_\exists^{now} \cup \tt{B}_\top^{now}\tt{)}\\
% % \tt{chn\_approx\_out} &\quad\tt{<=}\quad \tt{chn\_ack\_out}\\
% % \nonumber\\
% (\tt{A}_\exists, \tt{A}_\top)
% &\quad\tt{<=}\quad
% (\emptyset, (\tt{A}_\exists \cup \tt{A}_\top) \bowtie_{x,y} (\tt{B}_\exists \cup \tt{B}_\top)
% \cap
% ((\tt{A}_\exists \cup \tt{A}_\top) \bowtie_{z} (\tt{C}_\exists \cup \tt{C}_\top))\\
% % (\tt{B}_\exists^{now}, \tt{B}_\top^{now})
% % &\quad\tt{<=}\quad (\emptyset, (\tt{B}_\exists^{now} \cup \tt{B}_\top^{now}) \bowtie \tt{chn\_approx\_out})\\
% \end{align}


\subsection{Instantiated Garbage Collection Rewrite}

\subsubsection{State}
For every set $\tt{A}$ in $\mathfrak{P}$, we introduce a corresponding $\tt{A}_I$ in $\mathfrak{P}_{iGC}$ holding the instantiated tuples.
If $\tt{A}$ has primary keys $\tt{A}_!$, we may choose to keep $\tt{A}_!$ together with $\tt{A}_I$ in a key-augmented lattice $(\tt{A}_I, \tt{A}_!)$ in $\mathfrak{P}_{iGC}$;
otherwise we simply use the non-augmented representation $\tt{A}_I$ without keys.
This choice is dependent on the instantiated garbage collection rules, which we describe in the following subsection.
For convenience, we will write $\tt{A}_E$ to mean $(\tt{A}_I, \tt{A}_!)$ if $\tt{A}$ has a key-augmented representation, and $\tt{A}_I$ if $\tt{A}$ has a non-augmented representation.
In both cases, we persist $\tt{A}_E$ with the deferred merge rule
\[\tt{A}_E \quad\tt{<+}\quad \tt{A}_E.\]


\subsubsection{Garbage collection}
A garbage collection rule in $\mathfrak{P}_{GC}$
\begin{align*}
\langle \tt{A}_{TS}, \tt{C}_{1,TS}, \dots, \tt{C}_{n,TS} \rangle
\quad\tt{<+}\quad
\tt{GC}(\tt{A}_{TS}, \tt{C}_{1,TS}, \dots, \tt{C}_{n,TS}; \tt{B}_{1,TS}, \dots, \tt{B}_{m,TS}).
\end{align*}
is replaced in $\mathfrak{P}_{iGC}$ by the instantiated garbage collection rule
\begin{align*}
\langle \tt{A}_E, \tt{C}_{1,E}, \dots, \tt{C}_{n,E} \rangle
\quad\tt{<-}\quad
\tt{iGC}(\tt{A}_E, \tt{C}_{1,E}, \dots, \tt{C}_{n,E}; \tt{B}_{1,E}, \dots, \tt{B}_{m,E}).
\end{align*}

The choice of whether to use the non-augmented $\tt{A}_I$ or the key-augmented $(\tt{A}_I, \tt{A}_!)$ is dependent on the instantiated garbage collection rules that take $\tt{A}_E$ as input\footnote{
	The set $\tt{A}$ can be taken as input to multiple garbage collection rules since it appears on the RHS of at most one rule, but possibly on the LHS Of multiple rules in $\mathfrak{P}$.
}.
If any of these instantiated garbage collection rules require the use of $\tt{A}_!$, we will choose to use the key-augmented representation.

Also associated with each $\tt{iGC}$ is a partitioning of its output sets into $\mathbb{S}(\tt{iGC})$ and $\mathbb{W}(\tt{iGC})$, such that $\mathbb{S}(\tt{iGC}) \cap \mathbb{W}(\tt{iGC}) = \emptyset$ and $\mathbb{S}(\tt{iGC}) \cup \mathbb{W}(\tt{iGC}) = \{\tt{A}, \tt{C}_1, \dots, \tt{C}_n\}$,
and every set in $\mathbb{S}(\tt{iGC})$ has a key-augmented representation.
Intuitively, a set $\tt{C}$ in $\mathbb{S}(\tt{iGC})$ is one for which $\tt{iGC}$ is strongly consistent with $\tt{GC}$, in the sense that $\tt{iGC}$ guarantees that $\tt{C}_I = \tt{C}_\exists$, and $\tt{C}_! = \pi_!(\tt{C}_\exists \cup \tt{C}_\top)$;
on the other hand, $\tt{iGC}$ is weakly consistent with $\tt{GC}$ on a set $\tt{C}$ if it guarantees that $\tt{C}_\exists \subseteq \tt{C}_I \subseteq \tt{C}_\exists \cup \tt{C}_\top$
--- we formalize these notions in the sequel.
Since each set appears on the LHS in only a single garbage collection rule\footnote{
	Each set appears on the LHS of at most one garbage collection rule because it appears on the RHS of at most one rule in $\mathfrak{P}$.
}, we can define
$\mathbb{S}(\mathfrak{P}_{iGC}) = \bigcup_{\tt{iGC} \in \mathfrak{P}_{iGC}} \mathbb{S}(\tt{iGC})$
and
$\mathbb{W}(\mathfrak{P}_{iGC}) = \bigcup_{\tt{iGC} \in \mathfrak{P}_{iGC}} \mathbb{W}(\tt{iGC})$
as the union of the respective partitions over all garbage collection rules.
To simplify notation, we will use $\mathbb{S} := \mathbb{S}(\mathfrak{P}_{iGC})$ and $\mathbb{W} := \mathbb{W}(\mathfrak{P}_{iGC})$.


\subsubsection{Merge, Deferred Merge, Asynchronous Merge}
A rule $R$ in $\mathfrak{P}$,
\[\tt{B} \quad\tt{<op>}\quad \tt{f}\tt{(}\tt{A}_1,\dots,\tt{A}_n\tt{)}\]
where $\tt{<op>}$ is one of $\tt{<=}$, $\tt{<+}$ or $\tt{<}\sim$, is replaced in $\mathfrak{P}_{iGC}$ by
\[\tt{B}_I \quad\tt{<op>}\quad \tt{f}\tt{(}\tt{A}_{1,I},\dots,\tt{A}_{n,I}\tt{)}\]
if $\tt{B}$ has no primary keys, and otherwise
\[(\tt{B}_I, \tt{B}_!) \quad\tt{<op>}\quad (\tt{f}\tt{(}\tt{A}_{1,I},\dots,\tt{A}_{n,I}\tt{)}, \pi_!(\tt{f}\tt{(}\tt{A}_{1,I},\dots,\tt{A}_{n,I}\tt{)}).\]







%!TEX root = bloom_logical.tex

\section{Properties and Analysis of Logical Garbage Collection}
\label{sec:logical}

\subsection{Properties of GC rules}
For our rewrite to be sensible, we require that the GC rules have certain properties.
In particular, the GC rules must be safe -- they should only tombstone tuples that have no effect on future derivations.
We provide some intuition for this safety in the following examples.

\begin{example}[Safety for Projection]
Suppose we have the projection rule $\tt{X} ~\tt{<=}~ \pi_S(\tt{Y})$ in $\mathfrak{P}$, and correspondingly $\tt{X}_{TS} ~\tt{<=}~ (\pi_S(\tt{Y}_\exists \cup \tt{Y}_\top), \emptyset)$ in $\mathfrak{P}_{GC}$.
% , and $\tt{X}_I ~\tt{<=}~ \pi_S(\tt{Y}_I)$ in $\mathfrak{P}_{iGC}$.
% Our goal is to always maintain the relationships
% $\tt{X}_\exists \subseteq \tt{X}_I \subseteq \tt{X}_\exists \cup \tt{X}_\top = \tt{X}$ and
% $\tt{Y}_\exists \subseteq \tt{Y}_I \subseteq \tt{Y}_\exists \cup \tt{Y}_\top = \tt{Y}$.
Since we wish to eventually reclaim tombstoned tuples, these tuples in $\tt{Y}_\top$ should have no effect on the execution of the projection rule.
As explained in Section \ref{sec:prelims:exec}, executing the rule is equivalent to making the assignment $\tt{X}_{\exists,t,i+1} \cup \tt{X}_{\top,t,i+1} := \tt{X}_{\exists,t,i} \cup \tt{X}_{\top,t,i} \cup \pi_S(\tt{Y}_{\exists,t,i} \cup \tt{Y}_{\top,t,i})$.
For $\tt{Y}_\top$ to not affect this execution, it must be the case that $\tt{Y}_{\top,t,i} \subseteq \tt{X}_{\exists,t,i} \cup \tt{X}_{\top,t,i}$, so that $\tt{X}_{\exists,t,i} \cup \tt{X}_{\top,t,i} \cup \pi_S(\tt{Y}_{\exists,t,i} \cup \tt{Y}_{\top,t,i}) = \tt{X}_{\exists,t,i} \cup \tt{X}_{\top,t,i} \cup \pi_S(\tt{Y}_{\exists,t,i})$.
Thus, as long as $\tt{Y}_\top \subseteq \tt{X}_\exists \cup \tt{X}_\top$, it suffices to keep $\tt{Y}_\exists$ and we may safely discard $\tt{Y}_\top$.

As the execution of the program continues, $\tt{Y}_\exists$ may grow as tuples are merged into it, forming a larger $\widehat{\tt{Y}} \supseteq \tt{Y}_\exists$.
Despite this growth, the execution is still unaffected, i.e., $\tt{X}_{\exists,t,i} \cup \tt{X}_{\top,t,i} \cup \pi_S(\widehat{\tt{Y}} \cup \tt{Y}_{\top,t,i}) = \tt{X}_{\exists,t,i} \cup \tt{X}_{\top,t,i} \cup \pi_S(\widehat{\tt{Y}})$, since we have ensured that $\tt{Y}_\top \subseteq \tt{X}_\exists \cup \tt{X}_\top$.
\end{example}

\begin{example}[Safety for Set Difference]
Suppose we have the rule $\tt{Z} ~\tt{<=}~ \tt{X} - \tt{Y}$ in $\mathfrak{P}$, and correspondingly $\tt{Z}_{TS} ~\tt{<=}~ ((\tt{X}_\exists \cup \tt{X}_\top) - (\tt{Y}_\exists \cup \tt{Y}_\top), \emptyset)$ in $\mathfrak{P}_{GC}$.
% , and $\tt{Z}_I ~\tt{<=}~ \tt{X}_I - \tt{Y}_I$ in $\mathfrak{P}_{iGC}$.
Suppose also that $\tt{X}_E$ has a key-augmented representation in $\mathfrak{P}_{iGC}$ while $\tt{Y}_I$ has a non-augmented representation.
As in the previous example, we desire that the execution of the rule is unaffected by deletion of tombstones.
In particular, it suffices to have
$(\tt{X}_{\exists,t,i} \cup \tt{X}_{\top,t,i}) - (\tt{Y}_{\exists,t,i} \cup \tt{Y}_{\top,t,i}) = \tt{X}_{\exists,t,i} - \tt{Y}_{\exists,t,i}$.
We observe that if a tuple $x$ appears in both $\tt{X}$ and $\tt{Y}$, we can safely tombstone $x$ from $\tt{X}$ without affecting the set difference.
Once we have $x \in \tt{X}_\top$, it is not in $\tt{X}_\exists$, so we can also safely tombstone $x$ from $\tt{Y}$.
Hence, it suffices to ensure $\tt{Y}_\top \subseteq \tt{X}_\top \subseteq \tt{Y}_\top \cup \tt{Y}_\exists$.
We may then discard $\tt{X}_\top$ and $\tt{Y}_\top$ and keep only $\tt{X}_\exists$, $\tt{Y}_\exists$.
We also keep $\tt{X}_!$ in the key-augmented representation; its use will be demonstrated in the following paragraph.

As the execution of the program continues, both $\tt{X}_\exists$ and $\tt{Y}_\exists$ may grow as tuples are merged into them, former larger $\widehat{\tt{X}} \supseteq \tt{X}_\exists$ and $\widehat{\tt{Y}} \supseteq \tt{Y}_\exists$.
Since we use the key-augmented representation for $\tt{X}$, tuples that we add to $\tt{X}_\exists$ must be previously unseen tuples not in $\tt{X}_!$, which implies $\widehat{\tt{X}} \cap \tt{X}_\top = \emptyset$.
If we ensure that $\tt{Y}_\top \subseteq \tt{X}_\top \subseteq \tt{Y}_\top \cup \tt{Y}_\exists$, we can observe that
$
(\widehat{\tt{X}} \cup \tt{X}_\top) - (\widehat{\tt{Y}} \cup \tt{Y}_\top)
= \widehat{\tt{X}} - (\widehat{\tt{Y}} \cup \tt{Y}_\top)
= \widehat{\tt{X}} - \widehat{\tt{Y}}
$,
where the second equality holds because $\tt{Y}_\top \cap \widehat{\tt{X}} \subseteq \tt{X}_\top \cap \widehat{\tt{X}} = \emptyset$.
Thus, the execution of the set difference rule is unaffected by the deletion of tombstones.
\end{example}

\begin{invariant}[Merge GC Invariant]\label{inv:merge_gc}
Consider any merge or deferred merge rule 
\[R: \tt{B} ~\tt{<op>}~ \tt{f(}\tt{A}_1,\dots,\tt{A}_n, \tt{U}_1, \dots, \tt{U}_m\tt{)}\]
where $\tt{<op>}$ is either $\tt{<=}$ or $\tt{<+}$,
$\tt{A}_1,\dots,\tt{A}_n$ are non-augmented sets,
$\tt{U}_1,\dots,\tt{U}_m$ are key-augmented sets,
and RHS is not a channel.
We require that, 
$\forall \langle \widehat{\tt{A}}_1, \dots, \widehat{\tt{A}}_n\rangle \geq \langle \tt{A}_{1,\exists},\dots,\tt{A}_{n,\exists} \rangle,$
$\quad
\forall \langle \widehat{\tt{U}}_1 , \dots, \widehat{\tt{U}}_m \rangle \geq \langle \tt{U}_1, \dots, \tt{U}_m \rangle$
such that $\emptyset = \widehat{\tt{U}}_1 \cap \tt{U}_{1,\top} = \dots = \widehat{\tt{U}}_1 \cap \tt{U}_{m,\top}$:
\begin{align}
\tt{f(}\widehat{\tt{A}}_1,\dots,\widehat{\tt{A}}_n, \widehat{\tt{U}}_{1,\exists},\dots,\widehat{\tt{U}}_{n,\exists}\tt{)}
\cup \tt{B}_\exists\cup \tt{B}_\top
~=~
\tt{f(}
\widehat{\tt{A}}_1 \cup \tt{A}_{1,\top},\dots,\widehat{\tt{A}}_n \cup \tt{A}_{n,\top}, \widehat{\tt{U}}_1 \cup \tt{U}_{1,\top},\dots,\widehat{\tt{U}}_n \cup \tt{U}_{n,\top}\tt{)} 
\cup \tt{B}_\exists \cup \tt{B}_\top.
\label{eq:inv_merge_gc}
\end{align}
\end{invariant}

We are interested in keeping around only $\langle \tt{A}_{1,\exists},\dots,\tt{A}_{n,\exists} \rangle$ instead of $\langle \tt{A}_{1,\exists} \cup \tt{A}_{1,\top}, \dots, \tt{A}_{n,\exists}  \cup \tt{A}_{n,\top}\rangle$.
However, in the process of executing $\mathfrak{P}_{GC}$, we may find ourselves with a larger\footnote{
  The restrictions of Edelweiss ensures that lattices can only grow over time.
    In Dedalus$^+$ terminology, the program is `temporally inflationary'.
} lattice $\langle \widehat{\tt{A}}_1, \dots, \widehat{\tt{A}}_n\rangle \geq \langle \tt{A}_{1,\exists},\dots,\tt{A}_{n,\exists} \rangle$.
In such a case, we wish to ensure that the execution of $R$ in $\mathfrak{P}_{GC}$ (LHS of \eqref{eq:inv_merge_gc}) matches the corresponding execution of $R$ in $\mathfrak{P}$ (RHS of \eqref{eq:inv_merge_gc}) as if we had not reclaimed $\tt{A}_{1,\top}, \dots, \tt{A}_{n,\top}$.
As we will see in the examples below, the invariant often reduces to a simple intuitive assertion.

Note that the above invariant is a stronger assertion for $\tt{A}_1, \dots, \tt{A}_n$ than for $\tt{U}_1, \dots, \tt{U}_n$, since we place an extra restriction on $\widehat{\tt{U}}_1, \dots, \widehat{\tt{U}}_m$.
Despite that, we will sometimes choose to treat some tables $\tt{U}$ as if they do not have primary keys, and show that Invariant \ref{inv:merge_gc} holds for the larger range of $\widehat{\tt{U}}$.
This is because we will later place a stronger requirement (Property \ref{property:instantiated_gc_consistent}) on the instantiated garbage collection for tables with primary keys, and it may be more convenient to prove Property \ref{property:gc_safety} without primary keys than it is to show Property \ref{property:instantiated_gc_consistent} with primary keys.

\begin{property}[GC Rule Output Respect]
A garbage collection rule $\tt{GC}$ \emph{respects output relations} if it does not tombstone tuples from output relations.
That is, letting $\langle \tt{A}_{1,TS}^*, \dots, \tt{A}_{n,TS}^* \rangle = \tt{GC}(\tt{A}_{1,TS}, \dots, \tt{A}_{n,TS};$ $\tt{B}_{TS}, \tt{D}_{1,TS}, \dots, \tt{D}_{l,TS})$, then whenever $\tt{A}_j$ is an output relation and $\tt{A}_{n,\top} = \emptyset$, it must be the case that $\tt{A}_{j,\top}^* = \emptyset$
\end{property}

\begin{property}[Merge GC Safety]
\label{property:gc_safety}
Let $R: \tt{B} ~\tt{<op>}~ \tt{f(}\tt{A}_1,\dots,\tt{A}_n, \tt{U}_1, \dots, \tt{U}_m\tt{)}$ be a merge or deferred merge rule.
A garbage collection rule $\tt{GC}$ is safe for $R$ if it respects output relations, and maintains the Merge GC Invariant \ref{inv:merge_gc} for $R$.
That is, if
\begin{enumerate}
\item $\langle \tt{A}_{1,TS}, \dots, \tt{A}_{n,TS}, \tt{U}_{1,TS}, \dots, \tt{U}_{m,TS} \rangle$ and $\tt{B}_{TS}$ satisfies Merge GC Invariant \ref{inv:merge_gc} for $R$, and
\item $\tt{B}_\exists \cup \tt{B}_\top \supseteq \tt{f}\tt{(}\tt{A}_{1,\exists} \cup \tt{A}_{1,\top},\dots,\tt{A}_{n,\exists} \cup \tt{A}_{n,\top}, \tt{U}_{1,\exists} \cup \tt{U}_{1,\top},\dots,\tt{U}_{n,\exists} \cup \tt{U}_{n,\top}\tt{)}$,
\end{enumerate}
then $\langle \tt{A}_{1,TS}^*, \dots, \tt{A}_{n,TS}^*, \tt{U}_{1,TS}^*, \dots, \tt{U}_{m,TS}^* \rangle = \tt{GC}(\tt{A}_{1,TS}, \dots, \tt{A}_{n,TS},$ $\tt{U}_{1,TS}, \dots, \tt{U}_{m,TS};$ $\tt{B}_{TS}, \tt{D}_{1,TS}, \dots, \tt{D}_{l,TS})$ and $\tt{B}_{TS}$ also satisfies the Merge GC Invariant \ref{inv:merge_gc}.
\end{property}

\begin{property}[GC Rule Monotonicity]
% Suppose $\tt{A}_{1,TS}, \dots, \tt{A}_{n,TS}, \tt{B}_{1,TS}, \dots, \tt{B}_{m,TS}$ satisfies Merge GC Invariant \ref{inv:merge_gc}.
% Suppose that $\tt{A}'_{1,TS}, \dots, \tt{A}'_{n,TS}, \tt{B}'_{1,TS}, \dots, \tt{B}'_{m,TS}$ also satisfies Merge GC Invariant \ref{inv:merge_gc} and 
Suppose that $\tt{A}'_{i,TS} \geq \tt{A}_{i,TS}$ and $\tt{B}'_{j,TS} \geq \tt{B}_{j,TS}$.
Then, $\tt{GC}$ is \emph{monotone} if $\tt{GC}(\tt{A}_{1,TS}, \dots, \tt{A}_{n,TS}; \tt{B}_{1,TS}, \dots, \tt{B}_{m,TS}) \leq \tt{GC}(\tt{A}'_{1,TS}, \dots, \tt{A}'_{n,TS}; \tt{B}'_{1,TS}, \dots, \tt{B}'_{m,TS})$.
\end{property}

\begin{property}[GC Rule Conservation]
\label{property:gc_conservative}
For a garbage collection rule $\tt{GC}$, denote
$\langle \tt{A}_{1,TS}^*, \dots, \tt{A}_{n,TS}^* \rangle = \tt{GC}(\tt{A}_{1,TS}, \dots, \tt{A}_{n,TS}; \tt{B}_{1,TS}, \dots, \tt{B}_{m,TS})$.
Then, $\tt{GC}$ is \emph{conservative} if $\tt{A}_{j,\exists}^* \subseteq \tt{A}_{j,\exists}$ and $\tt{A}_{j,\exists}^* \cup \tt{A}_{j,\top}^* = \tt{A}_{j,\exists} \cup \tt{A}_{j,\top}$ for all $j=1,\dots,n$.
That is, $\tt{GC}$ only moves tuples from $\exists$ to $\top$.
\end{property}

\subsection{Examples of GC rules}
\label{sec:logical:example}
A number of example logical garbage collection rules are presented below.
Each of these rules are monotone, conservative, and safe for some rule $R$.
\begin{example}[Trivial GC]
$\tt{GCNone}(\tt{A}_{1,TS}, \dots, \tt{A}_{n,TS}; \tt{B}_{1,TS}, \dots, \tt{B}_{m,TS}) = \langle \tt{A}_{1,TS}, \dots, \tt{A}_{n,TS} \rangle$.
\end{example}
The trivial GC returns the tombstone lattices unchanged.
If $\tt{A}_1, \dots, \tt{A}_n$ are all output relations, then $\tt{GCNone}$ is the only garbage collection rule that respects output relations.

% \begin{example}[Indicator GC]
% Consider the indicator function $\tt{f}$ with $\tt{f}\tt{(}\tt{X}\tt{)} = \emptyset$ if $\tt{X} = \emptyset$ and $\{1\}$ otherwise, and $R: \tt{B} ~\tt{<=}~ \tt{f}\tt{(}\tt{X}\tt{)}$.
% Suppose that $\emptyset \subseteq \tt{X} \subseteq \{a,b,c\}$.
% Define $\tt{GCInd}_i(\tt{X}_{TS}; \tt{B}_{TS}) = (\tt{X}_\exists - \tt{X}_i^*, \tt{X}_\top \cup \tt{X}_i^*)$, for $i=1,2,3$ and
% \begin{align*}
% \tt{X}_1^* = \begin{cases}
% \emptyset & \text{if $|\tt{X}_\exists \cup \tt{X}_\top| \leq 1$} \\
% a & \text{if $\tt{X}_\exists \cup \tt{X}_\top = \{a, b\}$}\\
% c & \text{if $\tt{X}_\exists \cup \tt{X}_\top = \{a, c\}$}\\
% b & \text{if $\tt{X}_\exists \cup \tt{X}_\top = \{b, c\}$}\\
% a,b & \text{if $\tt{X}_\exists \cup \tt{X}_\top = \{a, b, c\}$}\\
% \end{cases}
% &&
% \tt{X}_2^* = \begin{cases}
% \emptyset & \text{if $|\tt{X}_\exists \cup \tt{X}_\top| \leq 1$} \\
% a & \text{if $\tt{X}_\exists \cup \tt{X}_\top = \{a, b\}$}\\
% c & \text{if $\tt{X}_\exists \cup \tt{X}_\top = \{a, c\}$}\\
% b & \text{if $\tt{X}_\exists \cup \tt{X}_\top = \{b, c\}$}\\
% a,b,c & \text{if $\tt{X}_\exists \cup \tt{X}_\top = \{a, b, c\}$}\\
% \end{cases}
% &&
% \tt{X}_3^* = \begin{cases}
% \emptyset & \text{if $|\tt{X}_\exists \cup \tt{X}_\top| \leq 1$} \\
% a & \text{if $\tt{X}_\exists \cup \tt{X}_\top = \{a, b\}$}\\
% a & \text{if $\tt{X}_\exists \cup \tt{X}_\top = \{a, c\}$}\\
% b & \text{if $\tt{X}_\exists \cup \tt{X}_\top = \{b, c\}$}\\
% a,b & \text{if $\tt{X}_\exists \cup \tt{X}_\top = \{a, b, c\}$}\\
% \end{cases}
% \end{align*}
% Then $\tt{GCInd}_1$ is safe for $R$ but not monotone, $\tt{GCInd}_2$ is monotone but not safe for $R$, and $\tt{GCInd}_3$ is monotone and safe for $R$.
% \end{example}
% $\tt{GCInd}_1$ is not monotone because $\tt{GCInd}_1((\{a,c\}, \emptyset); \tt{B}_{TS}) \not\leq \tt{GCInd}_1((\{a,b,c\}, \emptyset); \tt{B}_{TS}); \tt{B}_{TS})$.
% $\tt{GCInd}_2$ is not safe because garbage collection on $\{a, b, c\}$ causes 
% 
% GC on indicator: $\emptyset -> 0$, non-$\emptyset -> 1$.
% If GC(a,b) = a, GC(a,c) = c, GC(b,c) = b, GC(a,b,c) = a,b then satisfies GCI but not monotone.
% If GC(a,b) = a, GC(a,c) = c, GC(b,c) = b, GC(a,b,c) = a,b,c then not satisfies GCI but monotone.
% If GC(a,b) = a, GC(a,c) = a, GC(b,c) = b, GC(a,b,c) = a,b then satisfies GCI and monotone.

\begin{example}[Copy GC]
Suppose $\tt{A}$ appears as input on the RHS of rules $R_i: \tt{B}_i ~\tt{<op>}~ \tt{A}$, where $\tt{<op>}$ is either a merge or deferred merge.
Then $\tt{GCCopy}(\tt{A}_{TS}; \tt{B}_{1,TS}, \dots, \tt{B}_{m,TS}) = \left(\tt{A}_\exists - \tt{B}_\cap, \tt{A}_\top \cup \tt{B}_\cap\right)$,
where $\tt{B}_\cap = \tt{A}_\exists \cap \bigcap_{i=1}^m (\tt{B}_{i,\exists} \cup \tt{B}_{i,\top})$,
is a monotone, conservative garbage collection rule that is safe for $R_1, \dots, R_n$.
\end{example}
Clearly, $\tt{GCCopy}$ is conservative.
For the copy rule, the Merge GC Invariant \ref{inv:merge_gc} reduces to $\forall i, \tt{A}_\exists \cup \tt{B}_{i,\exists} \cup \tt{B}_{i,\top} = \tt{A}_\exists \cup \tt{A}_\top \cup \tt{B}_{i,\exists} \cup \tt{B}_{i,\top}$, which is the case iff $\tt{A}_\top \subseteq \tt{B}_{i,\exists} \cup \tt{B}_{i,\top}$, since $\tt{A}_\exists \cap \tt{A}_\top = \emptyset$.
That is, tombstoned tuples are those that have been copied to all $\tt{B}_i$'s.
Hence, if $\tt{A}_{TS}, \tt{B}_{i,TS}$ satisfies Invariant \ref{inv:merge_gc}, we have $\tt{A}_\top \subseteq \tt{B}_{i,\exists} \cup \tt{B}_{i,\top}$, and so $\tt{A}_\top \cup \tt{B}_\cap = \tt{A}_\top \cup \left(\tt{A}_\exists \cap \bigcap_{i=1}^m (\tt{B}_{i,\exists} \cup \tt{B}_{i,\top})\right) \subseteq \tt{B}_{i,\exists} \cup \tt{B}_{i,\top}$.
Therefore, $\tt{GCCopy}$ is safe for $R_i$.

Suppose $\tt{A}_\text{TS}' \geq \tt{A}_\text{TS}$, i.e., $\tt{A}_\top' \supseteq \tt{A}_\top$ and $\tt{A}_\top' \cup \tt{A}_\exists' \supseteq \tt{A}_\top \cup \tt{A}_\exists$.
Suppose also $\tt{A}_\text{TS}'$ satisfies the Merge GC Invariant \ref{inv:merge_gc} with respect to $\tt{B}_{i,\text{TS}}' \geq \tt{B}_{i,\text{TS}}$.
Then
\begin{align*}
\tt{A}_\top \cup \left(\tt{A}_\exists \cap \bigcap_{i=1}^m (\tt{B}_{i,\exists} \cup \tt{B}_{i,\top})\right)
&\subseteq \tt{A}_\top' \cup \left((\tt{A}_\exists' \cup \tt{A}_\top') \cap \bigcap_{i=1}^m (\tt{B}_{i,\exists}' \cup \tt{B}_{i,\top}')\right)
% \\
% &= \tt{A}_\top' \cup \left(\tt{A}_\exists' \cap \bigcap_{i=1}^m (\tt{B}_{i,\exists}' \cup \tt{B}_{i,\top}')\right) \cup \left(\tt{A}_\top' \cap \bigcap_{i=1}^m (\tt{B}_{i,\exists}' \cup \tt{B}_{i,\top}')\right)\\
&= \tt{A}_\top' \cup \left(\tt{A}_\exists' \cap \bigcap_{i=1}^m (\tt{B}_{i,\exists}' \cup \tt{B}_{i,\top}')\right).
\end{align*}
Furthermore, it follows from the GC Rule Conservation Property \ref{property:gc_conservative}, we also have that $(\tt{A}_\exists - \tt{B}_\cap) \cup (\tt{A}_\top \cup \tt{B}_\cap) \subseteq (\tt{A}_\exists' - \tt{B}_\cap') \cup (\tt{A}_\top' \cup \tt{B}_\cap')$.
Hence, $\tt{GCcopy}$ is monotone.

Note also that if the copies are done via a (non-deferred) merge, and if we reach a fixed point at $t$, $\iota$, then it must be that $\tt{B}_{i,\exists,t,\iota} \cup \tt{B}_{i,\top,t,\iota} \supseteq \tt{A}_{\exists,t,\iota} \cup \tt{A}_{\top,t,\iota}$, so $\tt{GCCopy}$ can tombstone all of $\tt{A}_\exists$.
~

% GC-Max: GC on A, B gives some tombstones larger than GC on any A' < A, B' < B, and is of largest cardinality among all such tombstones.
% Note: GC-Max is not unique; but since monotone, will still give a unique minimal model for each GC-Max rule.

\begin{example}[Project GC]
\label{ex:project_gc}
Consider the projection $R: \tt{B} ~\tt{<op>}~ \pi_S(\tt{A})$, where $\tt{<op>}$ is a merge or deferred merge.
Then $\tt{GCProject}(\tt{A}_{TS}; \tt{B}_{TS}) = (\tt{A}_\exists - \tt{A}^*, \tt{A}_\top \cup \tt{A}^*)$ where $\tt{A}^* = \{a \in \tt{A}_\exists \cup \tt{A}_\top ~:~ \pi_S(\{a\}) \subseteq \tt{B}_\exists \cup \tt{B}_\top\}$
is safe for $R$.
\end{example}
Simple relational algebraic calculations would show that the Merge GC Invariant \ref{inv:merge_gc} reduces to $\pi_S(\tt{A}_\top) \subseteq \tt{B}_\exists \cup \tt{B}_\top$.
By definition, $\pi_L(\tt{A}^*) \subseteq \tt{B}_\exists \cup \tt{B}_\top$.
Hence, if $\tt{A}_{TS}$ and $\tt{B}_{TS}$ satisfies the invariant, then $\pi_L(\tt{A}_\top \cup \tt{A}^*) = \pi_L(\tt{A}_\top) \cup \pi_L(\tt{A}^*) \subseteq \tt{B}_\exists \cup \tt{B}_\top$, so $\tt{GCProject}$ is safe for $R$. 

\begin{example}[Select GC]
Consider the selection $R: \tt{B} ~\tt{<op>}~ \sigma_P(\tt{A})$, where $\tt{<op>}$ is either a merge or deferred merge.
Then $\tt{GCSelect}(\tt{A}_{TS}; \tt{B}_{TS}) = (\tt{A}_\exists - \tt{A}^*, \tt{A}_\top \cup \tt{A}^*)$ where $\tt{A}^* = \sigma_{\neg P}(\tt{A}_\exists \cup \tt{A}_\top) \cup (\sigma_P(\tt{A}_\exists \cup \tt{A}_\top) \cap (\tt{B}_\exists \cup \tt{B}_\top))$ is safe for $R$.
\end{example}
Similar to Example \ref{ex:project_gc}, the Merge GC Invariant \ref{inv:merge_gc} for selection reduces to $\sigma_P(\tt{A}_\top) \subseteq \tt{B}_\exists \cup \tt{B}_\top$.
If $\tt{A}_{TS}$ and $\tt{B}_{TS}$ satisfy the invariant, then
\begin{align*}
\sigma_P(\tt{A}_\top \cup \tt{A}^*)
&= \sigma_P(\tt{A}_\top) \cup \sigma_P(\sigma_{\neg P}(\tt{A}_\exists \cup \tt{A}_\top) \cup (\sigma_P(\tt{A}_\exists \cup \tt{A}_\top) \cap (\tt{B}_\exists \cup \tt{B}_\top)))\\
&= \sigma_P(\tt{A}_\top) \cup (\sigma_P(\tt{A}_\exists \cup \tt{A}_\top) \cap (\tt{B}_\exists \cup \tt{B}_\top))\\
&= \sigma_P(\tt{A}_\top) \cup (\sigma_P(\tt{A}_\exists) \cap (\tt{B}_\exists \cup \tt{B}_\top))\\
&\subseteq \tt{B}_\exists \cup \tt{B}_\top,
\end{align*}
so $\tt{GCSelect}$ is safe for $R$.

\begin{example}[Linear GC]
Consider the rule $R: \tt{B} ~\tt{<op>}~ \tt{f}(\tt{A})$, where $\tt{f}$ is a linear function, i.e., $\tt{f}(\tt{A}) = \bigcup_{a\in \tt{A}} \tt{f}(\{a\})$.
Then $\tt{GCAll}(\tt{A}_{TS}; \tt{B}_{1,TS}, \dots, \tt{B}_{m,TS}) = (\emptyset, \tt{A}_\exists \cup \tt{A}_\top)$ is safe for $R$.
\end{example}
Since $\tt{f}$ is linear, Invariant \ref{inv:merge_gc} reduces to $\tt{f}\tt{(}\tt{A}_\top\tt{}) \subseteq \tt{B}_\exists \cup \tt{B}_\top$, but this is true under the condition $\tt{B}_\exists \cup \tt{B}_\top \supseteq \tt{f}\tt{(}\tt{A}_{\exists} \cup \tt{A}_{\top}\tt{)} = \tt{f}\tt{(}\tt{A}_\exists\tt{}) \cup \tt{f}\tt{(}\tt{A}_\top\tt{})$.
Thus, Invariant \ref{inv:merge_gc} is trivially true for linear $\tt{f}$.
Intuitively, a tuple $a \in \tt{A}_\exists$ is independent of all other tuples when $\tt{f}$ is linear, and hence, its complete effects will have been realized in $\tt{B}$ (either through a merge or deferred merge) by the point at which we perform garbage collection.
Therefore, we can always reclaim the entire set without impacting our results.

In particular, observe that the copy, select, and project rules of previous examples are all linear, and thus can be fully reclaimed at the end of each timestep.

\begin{example}[Join GC]
\label{ex:join_gc_no_manifests}
Consider the join $R: \tt{B} ~\tt{<op>}~ \tt{A} \bowtie_{\tt{A}.k_a = \tt{C}.k_c} \tt{C}$, where $\tt{<op>}$ is either a merge or deferred merge.
The only garbage collection rule that is safe for $R$ is the trivial $\tt{GCTrivial}(\tt{A}_{TS}, \tt{C}_{TS}; \tt{B}_{TS}) = \langle \tt{A}_{TS}, \tt{C}_{TS} \rangle$.
\end{example}
Without additional information, one cannot guarantee that a tuple in $\tt{A}$ will not match any future tuple in $\tt{C}$ and vice versa.
In some cases, however, one may be provided with the specific information that no more tuples matching the attribute will ever appear.

Formally, let $\mathcal{X}$ be the domain of possible tuples for a table $\tt{X}$, and denote the powerset of $\mathcal{X}$ as $\mathbb{P}(\mathcal{X})$.
Also let $\mathcal{K}$ be the domain of possible values of an attribute $\tt{X}.k$.
A manifest $p = (p.k, p.\tt{X}) \in \mathcal{K} \times \mathbb{P}(\mathcal{X})$ maps a key in $\mathcal{K}$ to a subset of $\mathbb{P}(\mathcal{X})$.
We denote by $\tt{X}$ by $\tt{P}_\tt{X}$ a table of manifests (with $k$ as its primary key) for $\tt{X}$, and say it is consistent with $\tt{X}$ if $\forall p \in \tt{P}_\tt{X}$, $p.\tt{X} \supseteq \{x \in \tt{X} ~:~ x.k = p.k\}$.
We will always assume that manifest tables are always consistent.

In the join rule above, we can reclaim from $\tt{A}$ given manifests on $\tt{C}$ and vice versa:

\begin{example}[Join with manifests]
Consider the join in Example \ref{ex:join_gc_no_manifests}, now augmented with manifest tables $\tt{P}_\tt{A}$ and $\tt{P}_\tt{C}$.
(If manifests are not available, we can always let $\tt{P} = \emptyset$ with no loss of generality.)
The garbage collection rule $\tt{GCJoin}(\tt{A}_{TS}, \tt{P}_{\tt{A},TS}, \tt{C}_{TS}, \tt{P}_{\tt{C},TS}; \tt{B}_{TS}) = \langle (\tt{A}_\exists - \tt{A}^*, \tt{A}_\top \cup \tt{A}^*), (\tt{P}_{\tt{A},\exists} - \tt{P}_\tt{A}^*, \tt{P}_{\tt{A},\top} \cup \tt{P}_\tt{A}^*), (\tt{C}_\exists - \tt{C}^*, \tt{C}_\top \cup \tt{C}^*), (\tt{P}_{\tt{C},\exists} - \tt{P}_\tt{C}^*, \tt{P}_{\tt{C},\top} \cup \tt{P}_\tt{C}^*) \rangle$, with
\begin{align*}
\tt{A}^* &= \left\{a \in \tt{A}_\exists \cup \tt{A}_\top ~:~ \exists p_c \in \tt{P}_{\tt{C},\exists} \cup \tt{P}_{\tt{C},\top}, \text{ such that } (a.k_a = p_c.k_c) \wedge \left(\{a\} \times p_c.\tt{C} \subseteq \tt{B}_\exists \cup \tt{B}_\top\right)\right\}\\
\tt{P}_\tt{A}^* &= \{p_a \in \tt{P}_{\tt{A},\exists} \cup \tt{P}_{\tt{A},\top} ~:~ \exists p_c \in \tt{P}_{\tt{C},\exists} \cup \tt{P}_{\tt{C},\top} \text{ such that } (p_a.k_a = p_c.k_c) \wedge \left(p_a.\tt{A} \times p_c.\tt{C} \subseteq \tt{B}_\exists \cup \tt{B}_\top\right)\}\\
\tt{C}^* &= \left\{c \in \tt{C}_\exists \cup \tt{C}_\top ~:~ \exists p_a \in \tt{P}_{\tt{A},\exists} \cup \tt{P}_{\tt{A},\top}, \text{ such that } (p_a.k_a = c.k_c) \wedge \left(p_a.\tt{A} \times \{c\} \subseteq \tt{B}_\exists \cup \tt{B}_\top\right)\right\}\\
\tt{P}_\tt{A}^* &= \{p_a \in \tt{P}_{\tt{A},\exists} \cup \tt{P}_{\tt{A},\top} ~:~ \exists p_c \in \tt{P}_{\tt{C},\exists} \cup \tt{P}_{\tt{C},\top} \text{ such that } (p_a.k_a = p_c.k_c) \wedge \left(p_a.\tt{A} \times p_c.\tt{C} \subseteq \tt{B}_\exists \cup \tt{B}_\top\right)\}
\end{align*}
is safe for the join.
\end{example}
In words, a tuple $a$ can be tombstoned if it matches a punctuation $p_c$ in $\tt{P}_\tt{C}$ \emph{and} the result of joining $\{a\}$ with $\tt{C}$ is already effected in $\tt{B}_{TS}$.
Hence, any $c \in \tt{C}$ either does not match $a$, or is already joined with $a$ in $\tt{B}$ and so will not affect any future join computation.
In addition, we tombstone a punctuation $p_a$ if it matches a punctuation $p_c$ and the join of their sets $p_a.\tt{A} \times p_c.\tt{C}$ is already effected in $\tt{B}_{TS}$.
It is a simple exercise to show that $\tt{GCJoin}$ tombstones a punctuation $p_a$ only when its set $p_a.\tt{A}$ is tombstoned (or does not exist yet), i.e., $(p_a \in \tt{P}_{\tt{A},\top}) \implies \tt{A}_\exists \cap p_a.\tt{A} = \emptyset \implies \sigma_{\tt{A}.k_a = p_a.k_a}(\tt{A}_\exists) = \emptyset$.

\begin{example}[DR+: Positive Difference Reclamation]
Consider the rule $R: \tt{B} ~\tt{<op>}~ \tt{A} - \tt{C}$, where $\tt{<op>}$ is a merge or deferred merge.
Then the rule $\tt{GCDR+}(\tt{A}_{TS}, \tt{C}_{TS}; \tt{B}_{TS}) = \langle (\tt{A}_\exists - \tt{A}^*, \tt{A}_\top \cup \tt{A}^*), \tt{C}_{TS}\rangle$
where $\tt{A}^* = (\tt{A}_\exists \cup \tt{A}_\top) \cap (\tt{C}_\exists \cup \tt{C}_\top)$
is safe for $R$.
\end{example}
For set differences, the Merge GC Invariant \ref{inv:merge_gc} reduces to
$\forall \widehat{\tt{A}} \supseteq \tt{A}_\exists, \forall \widehat{\tt{C}} \supseteq \tt{C}_\exists$: $\widehat{\tt{A}} - \widehat{\tt{C}} = (\widehat{\tt{A}} \cup \tt{A}_\top) - (\widehat{\tt{C}} \cup \tt{C}_\top)$.
In the case where $\widehat{\tt{A}} = \tt{A}_\exists$ and $\widehat{\tt{C}} = \tt{C}_\exists$, it implies that $\tt{A}_\top \subseteq \tt{C}_\exists \cup \tt{C}_\top$;
in the case where $\widehat{\tt{A}} = \tt{A}_\exists \cup \tt{C}_\top$ and $\widehat{\tt{C}} = \tt{A}_\exists \cup \tt{C}_\exists$, it implies that $\tt{C}_\top = \emptyset$.
Conversely $\tt{C}_\top = \emptyset$ and $\tt{A}_\top \subseteq \tt{C}_\exists \cup \tt{C}_\top$ suffices to satisfy the invariant.
One can now easily verify that $\tt{GCDR+}$ maintains the invariant.

\begin{example}[DR-: Negative Difference Reclamation]
Consider the rule $R: \tt{B} ~\tt{<op>}~ \tt{A} - \tt{C}$, where $\tt{<op>}$ is a merge or deferred merge, and $\tt{A}$ is a table with primary keys.
Then the rule $\tt{GCDR-}(\tt{A}_{TS}, \tt{C}_{TS}; \tt{B}_{TS}) = \langle (\tt{A}_\exists - \tt{A}^*, \tt{A}_\top \cup \tt{A}^*), (\tt{C}_\exists - \tt{C}^*, \tt{C}_\top \cup \tt{C}^*)\rangle$,
where $\tt{A}^* = (\tt{A}_\exists \cup \tt{A}_\top) \cap (\tt{C}_\exists \cup \tt{C}_\top)$ and $\tt{C}^* = (\tt{C}_\exists \cup \tt{C}_\top) \cap \tt{A}_\top$,
is safe for $R$.
\end{example}
Since primary keys are available for $\tt{A}$, the Merge GC Invariant \ref{inv:merge_gc} reduces to
$\forall \widehat{\tt{A}} \supseteq \tt{A}_\exists, \widehat{\tt{A}} \cap \tt{A}_\top = \emptyset, \forall \widehat{\tt{C}} \supseteq \tt{C}_\exists$: $\widehat{\tt{A}} - \widehat{\tt{C}} = (\widehat{\tt{A}} \cup \tt{A}_\top) - (\widehat{\tt{C}} \cup \tt{C}_\top)$.
In the case where $\widehat{\tt{A}} = \tt{A}_\exists$ and $\widehat{\tt{C}} = \tt{C}_\exists$, it implies that $\tt{A}_\top \subseteq \tt{C}_\exists \cup \tt{C}_\top$;
in the case where $\widehat{\tt{A}} = \tt{A}_\exists \cup (\tt{C}_\top - \tt{A}_\top)$ and $\widehat{\tt{C}} = \tt{A}_\exists \cup \tt{C}_\exists$, it implies that $\tt{C}_\top \subseteq \tt{A}_\top$.
Conversely $\tt{C}_\top \subseteq \tt{A}_\top \subseteq \tt{C}_\exists \cup \tt{C}_\top$ suffices to satisfy the invariant.
One can now easily verify that $\tt{GCDR-}$ maintains the invariant.





\subsection{Analysis of GC Rewrite}

\subsubsection{Correctness}
\label{sec:logical:analysis:correctness}

\begin{lemma}
\label{lem:unchangedexists}
For every set $\tt{A}$, we have that $\tt{A}_{\top,t,i} = \tt{A}_{\top,t,0}$.
\end{lemma}
\begin{proof}
The proof is straightforward: all non-GC rules are of the form $(\tt{A}_\exists, \tt{A}_\top) ~\tt{<op>}~ (\tt{X}, \emptyset)$, and all GC rules are deferred merges.
Thus, $\tt{A}_\top$ is unchanged during an instantaneous run.
\end{proof}
Due to Lemma \ref{lem:unchangedexists}, we can write $\tt{A}_{\top,t}$ in place of $\tt{A}_{\top,t,i}$.

\begin{lemma}
\label{lem:merge_gc_inv}
The rewritten GC program $\mathfrak{P}_{GC}$ maintains the Merge GC Invariant \ref{inv:merge_gc} for any rule $R$.
\end{lemma}
\begin{proof}
We prove the lemma for any given $R$ by an induction on $t$ and $i$.
That is, we show for every $t$ and $i$,
$\forall \langle \widehat{\tt{A}}_1, \dots, \widehat{\tt{A}}_n\rangle \geq \langle \tt{A}_{1,\exists,t,i},\dots,\tt{A}_{n,\exists,t,i} \rangle$,
$\forall \langle \widehat{\tt{U}}_1, \dots, \widehat{\tt{U}}_n\rangle \geq \langle \tt{U}_{1,\exists,t,i},\dots,\tt{U}_{n,\exists,t,i} \rangle$,
such that
$\emptyset = \widehat{\tt{U}}_1 \cap \tt{U}_{1,\top,t,i} = \dots =  \widehat{\tt{U}}_n \cap \tt{U}_{n,\top,t,i}$,
\begin{align}
&
\tt{B}_{\exists,t,i} \cup \tt{B}_{\top,t}
\cup
\tt{f(}
  \widehat{\tt{A}}_1,
    \dots,
    \widehat{\tt{A}}_n,
  \widehat{\tt{U}}_1,
    \dots,
    \widehat{\tt{U}}_n
\tt{)}
\nonumber\\
=&
\tt{B}_{\exists,t,i} \cup \tt{B}_{\top,t}
\cup
\tt{f(}
  \widehat{\tt{A}}_1 \cup \tt{A}_{1,\top,t},
  \dots,
  \widehat{\tt{A}}_n \cup \tt{A}_{n,\top,t},
  \widehat{\tt{U}}_1 \cup \tt{U}_{1,\top,t},
  \dots,
  \widehat{\tt{U}}_n \cup \tt{U}_{n,\top,t}
\tt{)}. \label{eq:mergegcinv_withtime}
\end{align}
At $t=0$, $i=0$, we have $\tt{A}_{k,\top,0} = \tt{U}_{j,\top,0} = \tt{B}_{\top,0} = \emptyset$ for all $j=1,\dots,m$ and $k=1,\dots,n$, so the statement \eqref{eq:mergegcinv_withtime} is trivially true.

Assume that \eqref{eq:mergegcinv_withtime} is true at some $t$ and $i-1$.
The assignment at $t$, $i$, is derived from applying a merge operator to the assignment at $t$, $i-1$, and may only grow $\tt{A}_{1,\exists}, \dots, \tt{A}_{n,\exists}$, $\tt{U}_{1,\exists}, \dots, \tt{U}_{m,\exists}$, and $\tt{B}_\exists$,
but keeps $\tt{A}_{1,\top}, \dots, \tt{A}_{n,\top}$, $\tt{U}_{1,\top}, \dots, \tt{U}_{m,\top}$, and $\tt{B}_\top$ unchanged.
This also implies that $\widehat{\tt{U}}_k \cap \tt{U}_{k,\top,t,i-1} = \emptyset \implies \widehat{\tt{U}}_k \cap \tt{U}_{k,\top,t,i-1} = \emptyset$.
Write $\widetilde{\tt{B}} = \tt{B}_{\exists,t,i} - \tt{B}_{\exists,t,i-1}$.
Then, $\forall \langle \widehat{\tt{A}}_1, \dots, \widehat{\tt{A}}_n\rangle \geq \langle \tt{A}_{1,\exists,t,i},\dots,\tt{A}_{n,\exists,t,i} \rangle \geq \langle \tt{A}_{1,\exists,t,i-1},\dots,\tt{A}_{n,\exists,t,i-1} \rangle$,
$\forall \langle \widehat{\tt{U}}_1, \dots, \widehat{\tt{U}}_n\rangle \geq \langle \tt{U}_{1,\exists,t,i},\dots,\tt{U}_{n,\exists,t,i} \rangle \geq \langle \tt{U}_{1,\exists,t,i-1},\dots,\tt{U}_{n,\exists,t,i-1} \rangle$,
such that
$\forall j=1,\dots,m$, $\widehat{\tt{U}}_j \cap \tt{U}_{1,\top,t} = \emptyset$,
\begin{align*}
&\tt{B}_{\exists,t,i} \cup \tt{B}_{\top,t}
\cup
\tt{f(}
  \widehat{\tt{A}}_1,
    \dots,
    \widehat{\tt{A}}_n,
  \widehat{\tt{U}}_1,
    \dots,
    \widehat{\tt{U}}_n
\tt{)}\\
=&
\widetilde{\tt{B}} \cup \tt{B}_{\exists,t,i-1} \cup \tt{B}_{\top,t}
\cup
\tt{f(}
  \widehat{\tt{A}}_1,
    \dots,
    \widehat{\tt{A}}_n,
  \widehat{\tt{U}}_1,
    \dots,
    \widehat{\tt{U}}_n
\tt{)}\\
=&
\widetilde{\tt{B}} \cup \tt{B}_{\exists,t,i-1} \cup \tt{B}_{\top,t}
\cup
\tt{f(}
  \widehat{\tt{A}}_1 \cup \tt{A}_{1,\top,t},
  \dots,
  \widehat{\tt{A}}_n \cup \tt{A}_{n,\top,t},
  \widehat{\tt{U}}_1 \cup \tt{U}_{1,\top,t},
  \dots,
  \widehat{\tt{U}}_n \cup \tt{U}_{n,\top,t}
\tt{)}\\
=&
\tt{B}_{\exists,t,i} \cup \tt{B}_{\top,t}
\cup
\tt{f(}
  \widehat{\tt{A}}_1 \cup \tt{A}_{1,\top,t},
  \dots,
  \widehat{\tt{A}}_n \cup \tt{A}_{n,\top,t},
  \widehat{\tt{U}}_1 \cup \tt{U}_{1,\top,t},
  \dots,
  \widehat{\tt{U}}_n \cup \tt{U}_{n,\top,t}
\tt{)}
\end{align*}
where the second equality follows from the inductive hypothesis.

Next, we show that if \eqref{eq:mergegcinv_withtime} is true at some fixed point $t-1$ and $i=\iota$, then it is also true at $t$, $i=0$.
By design / assumption, there is only one garbage collection rule $\tt{GC}$ in $\mathfrak{P}_{GC}$ that has any of $\tt{A}_{1,TS},\dots,\tt{A}_{n,TS}$ on its LHS.
Note that we can express $\forall j=1,\dots,n$,
\begin{align*}
(\tt{A}_{j,\exists,t,0}, \tt{A}_{j,\top,t})
= (\tt{A}_{j,\exists,t-1,\iota}, \tt{A}_{j,\top,t-1})
\cup (\tt{A}_{j,\exists}^*, \tt{A}_{j,\top}^*)
\cup \bigcup_{k=1}^l (g_k(\tt{C}_{k1,TS,t-1,\iota}, \dots, \tt{C}_{kn_k,TS,t-1,\iota}), \emptyset)
\end{align*}
where $\langle \tt{A}_{1,TS}^*, \dots, \tt{A}_{n,TS}^*\rangle = \tt{GC}(\tt{A}_{1,TS,t-1,\iota}, \dots, \tt{A}_{n,TS,t-1,\iota}; \tt{B}_{1,TS,t-1,\iota}, \dots, \tt{B}_{m,TS,t-1,\iota})$.
By the conservative property of $\tt{GC}$, $(\tt{A}_{j,\exists,t-1,\iota}, \tt{A}_{j,\top,t-1}) \leq (\tt{A}_{j,\exists}^*, \tt{A}_{j,\top}^*)$, so the above equation can be written as
\begin{align*}
(\tt{A}_{j,\exists,t,0}, \tt{A}_{j,\top,t})
= (\tt{A}_{j,\exists}^*, \tt{A}_{j,\top}^*)
\cup \bigcup_{k=1}^l (g_k(\tt{C}_{k1,TS,t-1,\iota}, \dots, \tt{C}_{kn_k,TS,t-1,\iota}), \emptyset)
\end{align*}
which implies $\tt{A}_{j,\exists,t,0} \supseteq \tt{A}_{j,\exists}^*$ and $\tt{A}_{j,\top,t} = \tt{A}_{j,\top}^*$.

The same holds for $\tt{U}_j$ for $j=1,\dots,m$, which further implies that $\widehat{\tt{U}}_j \cap \tt{U}_{j,\top}^* = \emptyset \implies \widehat{\tt{U}}_j \cap \tt{U}_{j,\top,t} = \emptyset$.

Also, by the monotonicity of $\tt{<+}$, we see that $\tt{B}_{\exists,t,0} \cup \tt{B}_{\top,t} \supseteq \tt{B}_{\exists,t-1,\iota} \cup \tt{B}_{\top,t-1}$.
Let $\widetilde{\tt{B}} = (\tt{B}_{\exists,t,0} \cup \tt{B}_{\top,t}) - (\tt{B}_{\exists,t-1,\iota} \cup \tt{B}_{\top,t-1})$.
Note that
$\tt{B}_{\exists,t,0} \cup \tt{B}_{\top,t} \supseteq \tt{f}\tt{(}\tt{A}_{\exists,1,t-1,\iota} \cup \tt{A}_{1,\top,t-1},\dots,\tt{A}_{n,\exists,t-1,\iota} \cup \tt{A}_{n,\top,t-1}, \tt{U}_{1,\exists,t-1,\iota} \cup \tt{U}_{1,\top,t-1},\dots,\tt{U}_{n,\exists,t-1,\iota} \cup \tt{U}_{n,\top,t-1}\tt{)}$ ---
if $R$ is a merge rule, it is true because $t,\iota$ is a fixed point;
if $R$ is a deferred merge rule, then the inclusion holds by definition of $\tt{B}_{TS,t,0}$.

Observe that $\forall \langle \widehat{\tt{A}}_1, \dots, \widehat{\tt{A}}_n \rangle$ $\geq$ $\langle \tt{A}_{1,\exists,t,0}, \dots, \tt{A}_{n,\exists,t,0} \rangle$ $\geq$ $\langle \tt{A}_{1,\exists}^*, \dots, \tt{A}_{n,\exists}^* \rangle$,
$\forall \langle \widehat{\tt{U}}_1, \dots, \widehat{\tt{U}}_n \rangle$ $\geq$ $\langle \tt{U}_{1,\exists,t,0}, \dots, \tt{U}_{n,\exists,t,0} \rangle$ $\geq$ $\langle \tt{U}_{1,\exists}^*, \dots, \tt{U}_{n,\exists}^* \rangle$,
such that $\emptyset = \widetilde{\tt{U}}_j \cap \tt{U}_{j,\top}^* = \widetilde{\tt{U}}_{j,\top,t}$ for all $j = 1,\dots,m$,
\begin{align*}
&
\tt{B}_{\exists,t,0} \cup \tt{B}_{\top,t}
\cup \tt{f}(\widehat{\tt{A}}_1, \dots, \widehat{\tt{A}}_n, \widehat{\tt{U}}_1, \dots, \widehat{\tt{U}}_m)
\\
=&~
\widetilde{\tt{B}} \cup \tt{B}_{\exists,t-1,\iota} \cup \tt{B}_{\top,t-1}
\cup \tt{f}(\widehat{\tt{A}}_1, \dots, \widehat{\tt{A}}_n, \widehat{\tt{U}}_1, \dots, \widehat{\tt{U}}_m)
\\
=&~
\widetilde{\tt{B}} \cup \tt{B}_{\exists,t-1,\iota} \cup \tt{B}_{\top,t-1}
\cup \tt{f}(
\widehat{\tt{A}}_1 \cup \tt{A}_{1,\top}^*, \dots, \widehat{\tt{A}}_n \cup \tt{A}_{n,\top}^*,
\widehat{\tt{U}}_1 \cup \tt{U}_{1,\top}^*, \dots, \widehat{\tt{U}}_n \cup \tt{U}_{n,\top}^*)
&\text{(\tt{GC} safety)}
\\
=&~
\widetilde{\tt{B}} \cup \tt{B}_{\exists,t-1,\iota} \cup \tt{B}_{\top,t-1}
\cup \tt{f}(
\widehat{\tt{A}}_1 \cup \tt{A}_{1,\top,t}, \dots, \widehat{\tt{A}}_n \cup \tt{A}_{n,\top,t},
\widehat{\tt{U}}_1 \cup \tt{U}_{1,\top,t}, \dots, \widehat{\tt{U}}_n \cup \tt{U}_{n,\top,t})
\\
=&~
\tt{B}_{\exists,t,0} \cup \tt{B}_{\top,t}
\cup \tt{f}(
\widehat{\tt{A}}_1 \cup \tt{A}_{1,\top,t}, \dots, \widehat{\tt{A}}_n \cup \tt{A}_{n,\top,t},
\widehat{\tt{U}}_1 \cup \tt{U}_{1,\top,t}, \dots, \widehat{\tt{U}}_n \cup \tt{U}_{n,\top,t})
\end{align*}
which establishes that the Merge GC Invariant \ref{inv:merge_gc} holds at $t$ and $i=0$.
\end{proof}

\begin{rmk}
Lemma \ref{lem:merge_gc_inv} is required for showing that the instantiated GC program $\mathfrak{P}_{iGC}$ presented later is correct, but not used in the below Theorem \ref{thm:equivalence_of_sets} establishing correctness of $\mathfrak{P}_{GC}$.
\end{rmk}




It is easy to see that $\mathfrak{P}$ and $\mathfrak{P}_{GC}$ have equivalent output sets.
That is, every run of $\mathfrak{P}$ has a one-to-one corresponding run of $\mathfrak{P}_{GC}$ that produces lattices such that $\tt{A}_\exists \cup \tt{A}_\top = \tt{A}$.
And in particular, for output sets where no garbage collection is performed, we get $\tt{A}_\exists = \tt{A}$.

\begin{thm}
\label{thm:equivalence_of_sets}
For every set $\tt{A}$, we have that $\tt{A}_{t,i} = \tt{A}_{\exists,t,i} \cup \tt{A}_{\top,t}$.
\end{thm}
\begin{proof}
We will again prove the theorem via induction.

Clearly this is true for $t=i=0$.

Now suppose that it is true at some $t$ and $i-1$, and we will show it holds at $t$ and $i$.
During the instantaneous run, the only rules that we are concerned with are merges where $\tt{A}$ appears on the LHS.
If the rule is of the form $\tt{A} ~\tt{<=}~ \overline{\tt{C}}$ with a channel on the RHS, then
\begin{align*}
\tt{A}_{t,i}
= \tt{A}_{t,i-1} \cup \overline{\tt{C}}_{t}
= \tt{A}_{\exists,t,i-1} \cup \tt{A}_{\top,t} \cup \overline{\tt{C}}_{t}
= \tt{A}_{\exists,t,i} \cup \tt{A}_{\top,t}.
\end{align*}
If the merge rule is of the form $\tt{A} ~\tt{<=}~ \tt{f}\tt{(}\tt{C}_1,\dots,\tt{C}_n\tt{)}$, then
\begin{align*}
\tt{A}_{t,i}
&= \tt{A}_{t,i-1} \cup \tt{f}\tt{(}\tt{C}_{1,t,i-1},\dots,\tt{C}_{n,t,i-1}\tt{)}\\
&= \tt{A}_{\exists,t,i-1} \cup \tt{A}_{\top,t} \cup \tt{f}\tt{(}\tt{C}_{1,\exists,t,i-1}\cup\tt{C}_{1,\top,t},\dots,\tt{C}_{n,\exists,t,i-1}\cup\tt{C}_{n,\top,t}\tt{)}\\
&= \tt{A}_{\exists,t,i} \cup \tt{A}_{\top,t},
\end{align*}
where we used the inductive hypothesis in the second equality.

Suppose $\mathfrak{P}$ reaches a fixed point for time $t$ at $i=\iota$.
We will next show that $\tt{A}_{t-1,\iota} = \tt{A}_{\exists,t-1,\iota} \cup \tt{A}_{\top,t-1}$ implies $\tt{A}_{t,0} = \tt{A}_{\exists,t,0} \cup \tt{A}_{\top,t}$.
We must first establish that the rewritten program will reach a fixed point for timestep $t$.

\begin{claim}
If $\mathfrak{P}$ reaches a fixed point at $t$, $\mathfrak{P}_{GC}$ also reaches a fixed point.
(This statement should be made more formal.)
\end{claim}
\begin{claimproof}
From the previous argument, we know that $\tt{A}_{\exists,t,j} \cup \tt{A}_{\top,t,i} = \tt{A}_{t,j}$ for all $j \geq \iota$.
But $\tt{A}_{t,j} = \tt{A}_{t,\iota}$ and $\tt{A}_{\top,t,i} = \tt{A}_{\top,t,0}$ are constant independent of $j$, which implies $\tt{A}_{\exists,t,j} = \tt{A}_{\exists,t,\iota}$.
Hence, $\mathfrak{P}_{GC}$ also reaches at fixed point at $t$, $i=\iota$.
\end{claimproof}

Finally, suppose we have $\tt{A}_{t-1,\iota} = \tt{A}_{\exists,t-1,\iota} \cup \tt{A}_{\top,t-1}$.
Suppose $\tt{A}$ appears in the LHS of $m$ deferred merge rules in $\mathfrak{P}$: $\tt{A} ~\tt{<+}~ \tt{f}_j\tt{(}\tt{C}_{j,1},\dots,\tt{C}_{j,n_j}\tt{)}$.
Additionally, in $\mathfrak{P}_{GC}$, $\tt{A}_{TS}$ appears in (at most) one GC rule $\langle \tt{A}_{TS}, \tt{D}_{1,TS}, \dots, \tt{D}_{k,TS} \rangle ~\tt{<+}~ \tt{GC}(\tt{A}_{TS}, \tt{D}_{1,TS}, \dots, \tt{D}_{k,TS}; \tt{B}_{1,TS}, \dots, \tt{B}_{k',TS})$.
Let $\langle \tt{A}_{TS}^*, \tt{D}_{1,TS}^*, \dots, \tt{D}_{k,TS}^* \rangle = \tt{GC}(\tt{A}_{TS,t-1,\iota}, \tt{D}_{1,TS,t-1,\iota}, \dots, \tt{D}_{k,TS,t-1,\iota}; \tt{B}_{1,TS,t-1,\iota}, \dots, \tt{B}_{k',TS,t-1,\iota})$, and note that $\tt{A}_{\exists,t-1,\iota} \cup \tt{A}_{\top,t-1} = \tt{A}_{\exists}^* \cup \tt{A}_{\top}^*$ by the conservation property of $\tt{GC}$.
Then,
\begin{align*}
&\tt{A}_{t,0}\\
=& \tt{A}_{t-1,\iota} \cup \bigcup_{j=1}^m \tt{f}_j\tt{(}\tt{C}_{j,1,t-1,\iota},\dots,\tt{C}_{j,n_j,t-1,\iota}\tt{)}\\
=& \tt{A}_{\exists,t-1,\iota} \cup \tt{A}_{\top,t-1}
\cup \bigcup_{j=1}^m \tt{f}_j\tt{(}\tt{C}_{j,1,\exists,t-1,\iota} \cup \tt{C}_{j,1,\top,t-1}, \dots, \tt{C}_{j,n_j,\exists,t-1,\iota} \cup \tt{C}_{j,n_j,\top,t-1}\tt{)}\\
=& \tt{A}_{\exists,t-1,\iota} \cup \tt{A}_{\top,t-1}
\cup \tt{A}_{\exists}^* \cup \tt{A}_{\top}^*
\cup \bigcup_{j=1}^m \tt{f}_j\tt{(}\tt{C}_{j,1,\exists,t-1,\iota} \cup \tt{C}_{j,1,\top,t-1}, \dots, \tt{C}_{j,n_j,\exists,t-1,\iota} \cup \tt{C}_{j,n_j,\top,t-1}\tt{)}\\
=& \tt{A}_{\exists,t,0} \cup \tt{A}_{\top,t}.
\end{align*}
Thus, we have established that $\tt{A}_{t-1,\iota} = \tt{A}_{\exists,t-1,\iota} \cup \tt{A}_{\top,t-1}$ implies $\tt{A}_{t,0} = \tt{A}_{\exists,t,0} \cup \tt{A}_{\top,t}$.
\end{proof}

For output sets, we never perform garbage collection, and hence always have equivalence of the sets in $\mathfrak{P}$ and $\mathfrak{P}_{GC}$.
The below corollary formally states this.
\begin{cor}
\label{cor:equivalence_of_output}
If a set $\tt{A}$ is an output set, it has $\tt{A}_{\top,t} = \emptyset$, and hence $\tt{A}_{t,i} = \tt{A}_{\exists,t,i}$.
\end{cor}



\subsubsection{Coordination-freeness}

\begin{thm}
\label{thm:coord-free}
$\mathfrak{P}_{GC}$ does not require any additional coordination over $\mathfrak{P}$.
If $\mathfrak{P}$ is confluent in all tables, then $\mathfrak{P}_{GC}$ is also confluent in all tables.
In particular, if all functions $\tt{f}$'s are monotone, then $\mathfrak{P}$ is expressed in monotone Bloom, $\mathfrak{P}_{GC}$ is expressed in monotone Bloom$^L$, and both $\mathfrak{P}$ and $\mathfrak{P}_{GC}$ are confluent.
\end{thm}
\begin{rmk}
This is a stronger but different statement than Theorem \ref{thm:equivalence_of_sets} and Corollary \ref{cor:equivalence_of_output}.
In particular, it says that every run of $\mathfrak{P}_{GC}$ produces the same $\tt{A}_\exists$ and $\tt{A}_\top$.
As we will see later, this is not the case for the instantiated program $\mathfrak{P}_{iGC}$.
\end{rmk}
\begin{proof}
Rules in $\mathfrak{P}$ either correspond to rules in $\mathfrak{P}$, or are new garbage collection rules.
In the first case, we replace the rule $R: \tt{B} ~\tt{<op>}~ \tt{f}(\tt{A}_1,\dots,\tt{A}_n\tt{)}$ with $(\tt{B}_\exists, \tt{B}_\top) ~\tt{<op>}~ (\tt{f}(\tt{A}_{1,\exists} \cup \tt{A}_{1,\top}, \dots, \tt{A}_{n,\exists} \cup \tt{A}_{n,\top} \tt{)}$, which requires a mapping $(\tt{X}_\exists, \tt{X}_\top) \mapsto \tt{X}_\exists \cup \tt{X}_\top$ and the projection $\tt{X} \mapsto (\tt{X}, \emptyset)$, both of which are monotone.
Hence, the corresponding rule in $\mathfrak{P}$ is monotone if $\tt{f}$ is monotone.
In the second case, our $\tt{GC}$ rules are monotone, and involve a deferred merge, and thus do not require additional coordination.

Suppose $\mathfrak{P}$ is confluent in all sets.
Theorem \ref{thm:equivalence_of_sets} implies $\tt{A}_{\exists} \cup \tt{A}_\top = \tt{A}$ is confluent in $\mathfrak{P}_{GC}$.
The only rules in $\mathfrak{P}_{GC}$ that promote tuples from $\exists$ to $\top$ are GC rules; since these rules are monotone, the tombstone lattices are also confluent.
\end{proof}

\begin{cor}
\label{cor:confluence_of_output}
If $\tt{A}$ is an output set, and $\mathfrak{P}$ is confluent, then every run of $\mathfrak{P}_{GC}$ produces $\tt{A}_\exists = \tt{A}$.
\end{cor}

% Our stated goal was to reconcile garbage collection with CALM consistency.
% The astute read might have noticed that we have sneaked in a non-monotone function in \eqref{eq:aexistmerge} to project $(\tt{A}_\exists, \tt{A}_\top)$ to $(\tt{A}_\exists, \emptyset)$, and thus $\mathfrak{P}_{GC}$ is not expressed in monotone Bloom$^L$ or stratified Bloom$^L$.
% Indeed, while Corollary \ref{cor:equivalence_of_output} and \ref{cor:confluence_of_output} state our output sets are confluent in $\mathfrak{P}_{GC}$, non-output sets \emph{may not} be confluent in $\mathfrak{P}_{GC}$.
% In particular, different runs (with different message-time assignments) can result in a different partitioning of a non-output $\tt{A}$ into $\tt{A}_\exists$ and $\tt{A}_\top$.

% However, a close examination of the analysis of \cite{marczak2012confluence} (and of our Bloom$^L$ formalism writeup) would reveal that while all lattices are confluent in any monotone Bloom$^L$ program, it is not necessarily the case that no lattices are confluent in a non-monotone Bloom$^L$ program.
% Our GC program $\mathfrak{P}_{GC}$ is exactly an example of a program confluent in some sets and expressed in non-monotone Bloom$^L$.

% In fact, both \cite{ameloot2013relational} and \cite{marczak2012confluence} defined ``confluence'' or ``eventual consistency'' in terms of output sets, and allowed for non-output sets that are possibly not eventually consistent.
% Dedalus$^+$ and monotone Bloom$^L$ restrict the programming language such that \emph{every} derived fact is provably eventually always true.
% We instead use the GC Invariant \ref{inv:merge_gc} to ensure that output sets are eventually always true (as long as $\mathfrak{P}$ is itself confluent), while allowing for non-output sets to be non-confluent.

% In the CALM analysis (refer to our transducer writeup), ``coordination'' is required if and only if a query can only be computed after establishing the completeness of its input, which is exactly the case for non-monotone queries.
% Although we have introduced a non-monotone projection, the GC invariance asserts that downstream computations are consistent regardless of future input.
% In other words, the GC invariance states that outputs can be computed without knowledge of input completeness!





%!TEX root = bloom_logical.tex

\section{Properties and Analysis of Instantiated Garbage Collection}
\label{sec:instantiated}



\subsection{Properties of Instantiated GC rules}
\begin{property}[Instantiated GC Conservation]\label{property:instantiated_gc_conservative}
Let $(\tt{A}_{1,E}^-, \dots, \tt{A}_{n,E}^-) = \tt{iGC}(\tt{A}_{1,E}, \dots, \tt{A}_{n,E}; \tt{B}_{1,E}, \dots, \tt{B}_{m,E})$.
The instantiated GC rule $\tt{iGC}$ is \emph{conservative} if $\forall \tt{A}_{1,E}, \dots, \tt{A}_{n,E}; \tt{B}_{1,E}, \dots, \tt{B}_{m,E}$, $\forall i=1,\dots,n$: $\tt{A}_{i,I}^- \subseteq \tt{A}_{i,I}$ and $\tt{A}_{i,!}^- = \pi_!(\tt{A}_{i,I})$.
\end{property}

\begin{property}[Instantiated Set Consistency]
A non-augmented set $\tt{A}_I$ in $\mathfrak{P}_{iGC}$ is \emph{weakly consistent} with $\tt{A}_{TS}$ in $\mathfrak{P}_{GC}$ if $\forall t, i$, $\tt{A}_{\exists,t,i} \subseteq \tt{A}_{I,t,i} \subseteq \tt{A}_{\exists,t,i} \cup \tt{A}_{\top,t}$.
% A non-augmented set $\tt{B}_I$ in $\mathfrak{P}_{iGC}$ is \emph{strongly consistent} with $\tt{B}_{TS}$ in $\mathfrak{P}_{GC}$ if $\forall t, i$, $\tt{B}_{\exists,t,i} = \tt{B}_{I,t,i}$.

A key-augmented set $\tt{B}_E$ in $\mathfrak{P}_{iGC}$ is \emph{weakly consistent} with $\tt{B}_{TS}$ in $\mathfrak{P}_{GC}$ if $\forall t, i$, $\tt{B}_{\exists,t,i} \subseteq \tt{B}_{I,t,i} \subseteq \tt{B}_{\exists,t,i} \cup \tt{B}_{\top,t}$, and $\tt{B}_{!,t,i} = \pi_!(\tt{B}_{\exists,t,i} \cup \tt{B}_{\top,t})$.
A key-augmented set $\tt{C}_E$ in $\mathfrak{P}_{iGC}$ is \emph{strongly consistent} with $\tt{C}_{TS}$ in $\mathfrak{P}_{GC}$ if $\forall t, i$, $\tt{C}_{\exists,t,i} = \tt{C}_{I,t,i}$, and $\tt{C}_{!,t,i} = \pi_!(\tt{C}_{\exists,t,i} \cup \tt{C}_{\top,t})$.
\end{property}

In general, it is not possible to have a strongly consistent non-augmented set, since the merge rule $\tt{A}_I ~\tt{<=}~ \tt{f}\tt{(}\dots\tt{)}$ could possibly merge some tuples from $\tt{A}_\top$ into $\tt{A}_I$.
For strongly consistent key-augmented sets, this is prevented since tombstoned tuples are filtered from merging into $\tt{A}_I$ by checking against $\tt{A}_!$.

\begin{property}[Instantiated GC Consistency]\label{property:instantiated_gc_consistent}
Consider a logical garbage collection rule $\tt{GC}$ that is safe for $R$, and a corresponding instantiated garbage collection rule $\tt{iGC}$.
We say that
$\tt{iGC}$ is \emph{weakly consistent} with $\tt{GC}$ on a set $\tt{A}$ if it preserves weak consistency of $\tt{A}_I$ (or $\tt{A}_E$) with $\tt{A}_{TS}$;
$\tt{iGC}$ is \emph{strongly consistent} with $\tt{GC}$ on a set $\tt{A}$ if it preserves strong consistency of $\tt{A}_I$ (or $\tt{A}_E$) with $\tt{A}_{TS}$.

Formally, let
\begin{align*}
&\langle
\tt{A}_{1,TS}^*, \dots, \tt{A}_{n_1,TS}^*,
\tt{B}_{1,TS}^*, \dots, \tt{B}_{n_1,TS}^*, 
\tt{C}_{1,TS}^*, \dots, \tt{C}_{n_3,TS}^*
\rangle \\
&= \tt{GC}(
\tt{A}_{1,TS}, \dots, \tt{A}_{n_1,TS},
\tt{B}_{1,TS}, \dots, \tt{B}_{n_2,TS},
\tt{C}_{1,TS}, \dots, \tt{C}_{n_3,TS};
\tt{D}_{1,TS}, \dots, \tt{D}_{n_4,TS},
\tt{E}_{1,TS}, \dots, \tt{E}_{n_5,TS}
),\\
&\langle
\tt{A}_{1,I}^-, \dots, \tt{A}_{n_1,I}^-,
\tt{B}_{1,I}^-, \dots, \tt{B}_{n_1,I}^-,
\tt{C}_{1,E}^-, \dots, \tt{C}_{n_3,E}^-
\rangle\\
&= \tt{iGC}(
\tt{A}_{1,I}, \dots, \tt{A}_{n_1,I},
\tt{B}_{1,I}, \dots, \tt{B}_{n_2,I},
\tt{C}_{1,E}, \dots, \tt{C}_{n_3,E};
\tt{D}_{1,E}, \dots, \tt{D}_{n_4,E},
\tt{E}_{1,I}, \dots, \tt{E}_{n_5,I}
).
\end{align*}
We say that $\tt{iGC}$ is \emph{weakly consistent} with $\tt{GC}$ on $\tt{A}_1, \dots, \tt{A}_{n_1}, \tt{B}_1, \dots, \tt{B}_{n_2}$, and \emph{strongly consistent} with $\tt{GC}$ on $\tt{C}_1, \dots, \tt{C}_{n_3}$, if under the conditions
\begin{enumerate}
\item $\forall i=1,\dots,n_1$, $\tt{A}_{i,\exists} \subseteq \tt{A}_{i,I} \subseteq \tt{A}_{i,\exists} \cup \tt{A}_{i,\top}$,
\item $\forall i=1,\dots,n_2$, $\tt{B}_{i,\exists} \subseteq \tt{B}_{i,I} \subseteq \tt{B}_{i,\exists} \cup \tt{B}_{i,\top}$, and $\tt{B}_{i,!} = \pi_!(\tt{B}_{i,\exists} \cup \tt{B}_{i,\top})$,
\item $\forall i=1,\dots,n_3$, $\tt{C}_{i,\exists} =         \tt{C}_{i,I}$,                                                   and $\tt{C}_{i,!} = \pi_!(\tt{C}_{i,\exists} \cup \tt{C}_{i,\top})$,
\item $\forall i=1,\dots,n_4$, $\tt{D}_{i,\exists} \subseteq \tt{D}_{i,I} \subseteq \tt{D}_{i,\exists} \cup \tt{D}_{i,\top}$,
\item $\forall i=1,\dots,n_5$, $\tt{E}_{i,\exists} \subseteq \tt{E}_{i,I} \subseteq \tt{E}_{i,\exists} \cup \tt{E}_{i,\top}$, and $\tt{E}_{i,!} = \pi_!(\tt{E}_{i,\exists} \cup \tt{E}_{i,\top})$,
\item $\tt{A}_{1,TS}, \dots, \tt{A}_{n_1,TS},$ $\tt{B}_{1,TS}, \dots, \tt{B}_{n_2,TS},$ $\tt{C}_{1,TS}, \dots, \tt{C}_{n_3,TS},$ $\tt{D}_{1,TS}, \dots, \tt{D}_{n_4,TS},$ $\tt{E}_{1,TS}, \dots, \tt{E}_{n_5,TS}$ satisfy the Merge GC Invariant \ref{inv:merge_gc} for $R$.
\end{enumerate}
the following are true:
\begin{enumerate}
\item $\forall i=1,\dots,n_1$, $\tt{A}_{i,I}^- \subseteq \tt{A}_{i,\top}^*$,
\item $\forall i=1,\dots,n_2$, $\tt{B}_{i,I}^- \subseteq \tt{B}_{i,\top}^*$,
\item $\forall i=1,\dots,n_3$, $\tt{C}_{i,I}^- =         \tt{C}_{i,\top}^*$,
\end{enumerate}
\end{property}



% \begin{property}[Instantiated GC Consistency]\label{property:instantiated_gc_consistent}
% Suppose $\tt{GC}$ is a logical garbage collection rule safe for $R$, and that the following are true:
% \begin{enumerate}
% \item $\tt{A}_{1,TS}, \dots, \tt{A}_{n,TS}$, $\tt{B}_{1,TS}, \dots, \tt{B}_{m,TS}$ satisfy the Merge GC Invariant \ref{inv:merge_gc} for $R$,
% \item $\forall i=1,\dots,n$, if $\tt{A}_i$ has no primary keys, then $\tt{A}_{i,\exists} \subseteq \tt{A}_{i,I} \subseteq \tt{A}_{i,\exists} \cup \tt{A}_{i,\top}$; otherwise, $\tt{A}_{i,\exists} = \tt{A}_{i,I}$ and $\tt{A}_{i,!} = \pi_!(\tt{A}_{i,\exists} \cup \tt{A}_{i,\top})$,
% \item $\forall i=1,\dots,m$, if $\tt{B}_i$ has no primary keys, then $\tt{B}_{i,\exists} \subseteq \tt{B}_{i,I} \subseteq \tt{B}_{i,\exists} \cup \tt{B}_{i,\top}$; otherwise, $\tt{B}_{i,\exists} = \tt{B}_{i,I}$ and $\tt{B}_{i,!} = \pi_!(\tt{B}_{i,\exists} \cup \tt{B}_{i,\top})$.
% \end{enumerate}
% Let $(\tt{A}_{1,TS}^*, \dots, \tt{A}_{n,TS}^*) = \tt{GC}(\tt{A}_{1,TS}, \dots, \tt{A}_{n,TS}; \tt{B}_{1,TS}, \dots, \tt{B}_{m,TS})$
% and $(\tt{A}_{1,E}^-, \dots, \tt{A}_{n,E}^-)$ $=$ $\tt{GCI}(\tt{A}_{1,E}, \dots, \tt{A}_{n,E};$ $\tt{B}_{1,E}, \dots, \tt{B}_{m,E})$.
% Then, $\tt{GCI}$ is \emph{consistent with} $\tt{GC}$ if $\forall i=1,\dots,n$: $\tt{A}_{i,I}^- \subseteq \tt{A}_{i,\top}^*$ if $\tt{A}_i$ does not have primary keys,
% and 
% $(\tt{A}_{i,!} - \tt{A}_{i,I}) \cup \tt{A}_{i,I}^- = \tt{A}_{i,\top}^*$ if $\tt{A}_i$ has primary keys.
% \end{property}
% In words, $\tt{GCI}$ is consistent with $\tt{GC}$ if the tuples that are deleted by $\tt{GCI}$ are a subset of the tuples tombstoned by $\tt{GC}$,
% and in the case where primary keys are available, both $\tt{GC}$ and $\tt{GCI}$ agree on the tombstoned tuples.

In other words, $\tt{iGC}$ deletes a subset of $\tt{A}$ and $\tt{B}$ tuples that $\tt{GC}$ tombstones, and $\tt{iGC}$ exactly agrees with $\tt{GC}$ on the deleted / tombstoned tuples in $\tt{C}$.
Note that the strong consistency of $\tt{iGC}$ with $\tt{GC}$ on $\tt{C}$ does not necessary imply the corresponding weak consistency, because the strong consistency holds under the strong assumption that $\tt{C}_{i,\exists} = \tt{C}_{i,I}$ instead of the weaker assumption that $\tt{C}_{i,\exists} \subseteq \tt{C}_{i,I} \subseteq \tt{C}_{i,\exists} \cup \tt{C}_{i,\top}$.
Conversely, the weak consistency of $\tt{iGC}$ with $\tt{GC}$ on $\tt{A}$ and $\tt{B}$ does not necessary imply the corresponding strong consistency, because weak consistency only provides the weaker result that $\tt{A}_{i,I}^- \subseteq \tt{A}_{i,\top}^*$ and $\tt{B}_{i,I}^- \subseteq \tt{B}_{i,\top}^*$ instead of the stronger requirement that $\tt{A}_{i,I}^- = \tt{A}_{i,\top}^*$ and $\tt{B}_{i,I}^- = \tt{B}_{i,\top}^*$.

We will require that any instantiated garbage collection rule $\tt{GCI}$ for $R$ be conservative and consistent with some logical garbage collection rule $\tt{GC}$, which is itself safe for $R$, conservative, and monotone\footnote{
  I believe that monotonicity is required only to show $\mathfrak{P}_{GC}$ is confluent in $\tt{A}_\top$, and $\mathfrak{P}_{iGC}$ is confluent in $\tt{A}_! - \tt{A}_I$.
}.



\subsection{Examples of Instantiated GC rules}
We now present some examples of instantiated GC rules that are consistent with the logical GC rules presented in Section \ref{sec:logical:example}.

\begin{example}[Copy GC]
$\tt{iGCCopy}(\tt{A}_I; \tt{B}_{1,I}, \dots, \tt{B}_{m,I}) = \tt{A}_I \cap \bigcap_{i=1}^m \tt{B}_{i,I}$ is weakly consistent with $\tt{GCCopy}$.
\end{example}
Suppose
$\tt{A}_\exists \subseteq \tt{A}_I \subseteq \tt{A}_\exists \cup \tt{A}_\top$
and for all $i = 1, \dots, m$:
$\tt{B}_{i,\exists} \subseteq \tt{B}_{i,I} \subseteq \tt{B}_{i,\exists} \cup \tt{B}_{i,\top}$.
Then,
\begin{align*}
\tt{A}_I \cap \bigcap_{i=1}^m \tt{B}_{i,I}
% &
~\subseteq~
(\tt{A}_\exists \cup \tt{A}_\top) \cap \bigcap_{i=1}^m (\tt{B}_{i,\exists} \cup \tt{B}_{i,\top})
% \\
% &
% ~=~
% \left(\tt{A}_\exists \cap \bigcap_{i=1}^m (\tt{B}_{i,\exists} \cup \tt{B}_{i,\top})\right)
% \cup
% \left(\tt{A}_\top \cap \bigcap_{i=1}^m (\tt{B}_{i,\exists} \cup \tt{B}_{i,\top})\right)\\
% &
~\subseteq~
\left(\tt{A}_\exists \cap \bigcap_{i=1}^m (\tt{B}_{i,\exists} \cup \tt{B}_{i,\top})\right)
\cup
\tt{A}_\top.
\end{align*}
Hence, $\tt{iGCCopy}$ is consistent with $\tt{GCCopy}$.

\begin{example}[Project GC]
$\tt{iGCProject}(\tt{A}_I; \tt{B}_I) = \{a \in \tt{A}_I ~:~ \pi_S(\{a\}) \subseteq \tt{B}_I\}$ is weakly consistent with $\tt{GCProject}$.
\end{example}

If $\tt{A}_I \subseteq \tt{A}_\exists \cup \tt{A}_\top$ and $\tt{B}_I \subseteq \tt{B}_\exists \cup \tt{B}_\top$, then
clearly $\tt{iGCProject}(\tt{A}_I; \tt{B}_I) = \{a \in \tt{A}_I ~:~ \pi_S(\{a\}) \subseteq \tt{B}_I\} \subseteq \{a \in \tt{A}_\exists \cup \tt{A}_\top ~:~ \pi_S(\{a\}) \subseteq \tt{B}_\exists \cup \tt{B}_\top\} \subseteq \tt{GCProject}(\tt{A}_{TS}; \tt{B}_{TS})$.

\begin{example}[Select GC]
$\tt{iGCSelect}(\tt{A}_I; \tt{B}_I) = \sigma_{\neg P}(\tt{A}_I) \cup (\sigma_P(\tt{A}_I) \cap \tt{B}_I)$ is weakly consistent with $\tt{GCSelect}$.
\end{example}
If $\tt{A}_I \subseteq \tt{A}_\exists \cup \tt{A}_\top$ and $\tt{B}_I \subseteq \tt{B}_\exists \cup \tt{B}_\top$, then
simple relational algebra demonstrates that
$\tt{iGCSelect}(\tt{A}_I; \tt{B}_I) = \sigma_{\neg P}(\tt{A}_I) \cup (\sigma_P(\tt{A}_I) \cap \tt{B}_I) \subseteq \sigma_{\neg P}(\tt{A}_\exists \cup \tt{A}_\top) \cup (\sigma_P(\tt{A}_\exists \cup \tt{A}_\top) \cap (\tt{B}_\exists \cup \tt{B}_\top)) = \tt{GCSelect}(\tt{A}_{TS}; \tt{B}_{TS})$.

\begin{example}[Linear GC]
$\tt{iGCAll}(\tt{A}_I; \tt{B}_{1,E}, \dots, \tt{B}_{m,E}) = \tt{A}_I$ is strongly consistent with $\tt{GCAll}$.
\end{example}

\begin{example}[Join with punctuations]
$\tt{iGCJoin}(\tt{A}_I, \tt{P}_{\tt{A},I}, \tt{C}_I, \tt{P}_{\tt{C},E}; \tt{B}_I) = \langle \tt{A}_I^-, \tt{P}_{\tt{A},I}^-, \tt{C}_I^-, \tt{P}_{\tt{C},I}^-\rangle$, where
\begin{align*}
\tt{A}_I^- &= \{a \in \tt{A}_I ~:~ \exists p_c \in \tt{P}_{\tt{C},I}, \text{ such that } (a.k_a = p_c.k_c) \wedge \left(\{a\} \times p_c.\tt{C} \subseteq \tt{B}_I\right)\}\\
\tt{P}_\tt{A}^- &= \{p_a \in \tt{P}_{\tt{A},I} ~:~ \exists p_c \in \tt{P}_{\tt{C},I} \text{ such that } (p_a.k_a = p_c.k_c) \wedge \left(p_a.\tt{A} \times p_c.\tt{C} \subseteq \tt{B}_I\right)\}\\
\tt{C}_I^- &= \{c \in \tt{C}_I ~:~ \exists p_a \in \tt{P}_{\tt{A},I}, \text{ such that } (p_a.k_a = c.k_c) \wedge \left(\{c\} \times p_a.\tt{A} \subseteq \tt{B}_I\right)\}\\
\tt{P}_\tt{C}^- &= \{p_c \in \tt{P}_{\tt{C},I} ~:~ \exists p_a \in \tt{P}_{\tt{A},I} \text{ such that } (p_a.k_a = p_c.k_c) \wedge \left(p_a.\tt{A} \times p_c.\tt{C} \subseteq \tt{B}_I\right)\}
\end{align*}
is weakly consistent with $\tt{GCJoin}$.
\end{example}
Proof of consistency of $\tt{iGCJoin}$ with $\tt{GCJoin}$ is straightforward\footnote{
  We treat $\tt{P}_\tt{A}$ and $\tt{P}_\tt{C}$ as if they are tables without primary keys.
} and thus omitted.
As noted before, $p_a \in \tt{P}_{\tt{A},!} - \tt{P}_{\tt{A},I} \implies p_a \in \tt{P}_{\tt{A},\top} \implies \sigma_{\tt{A}.k_a = p_a.k_a}(\tt{A}_\exists) = \emptyset$, so we can also tombstone tuples whose keys match $\tt{P}_{\tt{A},!} - \tt{P}_{\tt{A},I}$:

\begin{example}[Join with punctuations, version 2]
$\tt{iGCJoinV2}(\tt{A}_I, \tt{P}_{\tt{A},I}, \tt{C}_I, \tt{P}_{\tt{C},E}; \tt{B}_I) = \langle \tt{A}_I^-, \tt{P}_{\tt{A},E}^-, \tt{C}_I^-, \tt{P}_{\tt{C},E}^-\rangle$, where
\begin{align*}
\tt{A}_I^- &= \{a \in \tt{A}_I ~:~ \exists p_c \in \tt{P}_{\tt{C},I}, \text{ such that } (a.k_a = p_c.k_c) \wedge \left(\{a\} \times p_c.\tt{C} \subseteq \tt{B}_I\right)\} \cup \{a \in \tt{A}_I ~:~ a.k_a \in \tt{P}_{\tt{A},!} - \pi_!(\tt{P}_{\tt{A},I})\}\\
\tt{P}_\tt{A}^- &= \{p_a \in \tt{P}_{\tt{A},I} ~:~ \exists p_c \in \tt{P}_{\tt{C},I} \text{ such that } (p_a.k_a = p_c.k_c) \wedge \left(p_a.\tt{A} \times p_c.\tt{C} \subseteq \tt{B}_I\right)\}\\
\tt{C}_I^- &= \{c \in \tt{C}_I ~:~ \exists p_a \in \tt{P}_{\tt{A},I}, \text{ such that } (p_a.k_a = c.k_c) \wedge \left(\{c\} \times p_a.\tt{A} \subseteq \tt{B}_I\right)\} \cup \{c \in \tt{C}_I ~:~ c.k_c \in \tt{P}_{\tt{C},!} - \pi_!(\tt{P}_{\tt{C},I})\}\\
\tt{P}_\tt{C}^- &= \{p_c \in \tt{P}_{\tt{C},I} ~:~ \exists p_a \in \tt{P}_{\tt{A},I} \text{ such that } (p_a.k_a = p_c.k_c) \wedge \left(p_a.\tt{A} \times p_c.\tt{C} \subseteq \tt{B}_I\right)\}
\end{align*}
is weakly consistent with $\tt{GCJoin}$.
\end{example}


\begin{example}[DR+: Positive Difference Reclamation]
$\tt{iGCDR+}(\tt{A}_I, \tt{C}_I; \tt{B}_I) = \langle \tt{A}_I \cap \tt{C}, \tt{C}_I\rangle$ is weakly consistent with $\tt{GCDR+}$.
\end{example}

\begin{example}[DR-: Negative Difference Reclamation]
$\tt{iGCDR-}(\tt{A}_E, \tt{C}_I; \tt{B}_I) = \langle (\tt{A}_I \cap \tt{C}_I, \pi_!(\tt{A}_I \cap \tt{C}_I), \tt{C}_I \cap (\tt{A}_! - \tt{A}_I)\rangle$ is strongly consistent with $\tt{GCDR+}$.
\end{example}
Suppose $\tt{A}_{TS}$, $\tt{C}_{TS}$, $\tt{A}_E$, and $\tt{C}_I$ satisfy the conditions of Property \ref{property:instantiated_gc_consistent}.
In particular, $\tt{A}_{TS}$ and $\tt{C}_{TS}$ satisfies Invariant \ref{inv:merge_gc}, so as previously discussed, this implies $\tt{C}_\top \subseteq \tt{A}_\top \subseteq \tt{C}_\exists \cup \tt{C}_\exists$.
Thus,
\begin{align*}
(\tt{A}_! - \tt{A}_I) \cup (\tt{A}_I \cap \tt{C}_I)
= \tt{A}_\top \cup (\tt{A}_\exists \cap \tt{C}_\exists)
= \tt{A}_\top \cup (\tt{A}_\exists \cap (\tt{C}_\exists \cup \tt{C}_\top))
= \tt{A}_\top \cup ((\tt{A}_\exists \cup \tt{A}_\top) \cap (\tt{C}_\exists \cup \tt{C}_\top))
\end{align*}
which are exactly the tombstones of $\tt{GCDR-}$.


\subsection{Correctness}
\begin{thm}[Representation Correctness]
\label{thm:instantiated_correctness}
For any set $\tt{A}$, let $\tt{GC}$ be its logical GC rule in $\mathfrak{P}_{GC}$ and $\tt{iGC}$ be its instantiated GC rule in $\mathfrak{P}_{iGC}$.
Then, it is either the case that
\begin{enumerate}
\item $\tt{iGC}$ is weakly consistent with $\tt{GC}$ on $\tt{A}$ and $\tt{A}_I$ (or $\tt{A}_E$) is weakly consistent with $\tt{A}_{TS}$, i.e.,
\begin{align}
\forall t, i: \quad \tt{A}_{\exists,t,i} \subseteq \tt{A}_{I,t,i} \subseteq \tt{A}_{\exists,t,i} \cup \tt{A}_{\top,t,i} \label{eq:thm_instantiated_correctness_set}
\end{align}
\item $\tt{iGC}$ is strongly consistent with $\tt{GC}$ and $\tt{A}_E$ is strongly consistent with $\tt{A}_{TS}$, i.e.,
\begin{align}
\forall t, i: \quad \tt{A}_{\exists,t,i} = \tt{A}_{I,t,i} \label{eq:thm_instantiated_correctness_set_strong}
\end{align}
\end{enumerate}
In both cases, if $\tt{A}_E$ is key-augmented in $\mathfrak{P}_{iGC}$, then
\begin{align}
\forall t, i : \tt{A}_{!,t,i} = \pi_!(\tt{A}_{\exists,t,i} \cup \tt{A}_{\top,t,i}) \label{eq:thm_instantiated_correctness_key}.
\end{align}
\end{thm}
\begin{rmk}
An immediate consequence is that $\forall t, i: \tt{A}_{!,t,i} - \pi_!(\tt{A}_{I,t,i}) \subseteq \pi_!(\tt{A}_{\top,t,i})$ for weakly consistent $\tt{A}_E$, and $\forall t, i: \tt{A}_{!,t,i} - \pi_!(\tt{A}_{I,t,i}) = \pi_!(\tt{A}_{\top,t,i})$ for strongly consistent $\tt{A}_E$.
\end{rmk}
% \begin{thm}[Representation Correctness]
% \label{thm:instantiated_correctness}
% For any set $\tt{A}$ without primary keys in $\mathfrak{P}$ and its corresponding lattice $(\tt{A}_\exists,t, \tt{A}_\top)$ in $\mathfrak{P}_{GC}$ and set $\tt{A}_E$ in $\mathfrak{P}_{iGC}$,
% \begin{align}
% \tt{A}_{\exists,t,i} \subseteq &\tt{A}_{I,t,i} \subseteq \tt{A}_{\exists,t,i} \cup \tt{A}_{\top,t}.\label{eq:thm_instantiated_correctness_set}
% \end{align}
% If $\tt{A}$ has primary keys, then
% \begin{align}
% \tt{A}_{I,t,i} =& \tt{A}_\exists,\label{eq:thm_instantiated_correctness_set_key}\\
% \tt{A}_{!,t,i} =& \pi_!(\tt{A}_{\exists,t,i} \cup \tt{A}_{\top,t}).\label{eq:thm_instantiated_correctness_key}
% \end{align}
% As an immediate consequence of \eqref{eq:thm_instantiated_correctness_set_key} and \eqref{eq:thm_instantiated_correctness_key}, we get $\pi_!(\tt{A}_{\top,i}) = \tt{A}_{!,t,i} - \pi_!(\tt{A}_{I,t,i})$ because $\tt{A}_{\exists,t,i} \cap \tt{A}_{\top,t} = \emptyset$
% \end{thm}
\begin{proof}
We provide the proofs for weak consistency; the proofs for strong consistency follow the same approach, replacing inclusion $\subseteq$ with equality $=$ in the appropriate locations.
Our proof proceeds by induction on $t$ and $i$.

At $t=0$, $i=0$, we initialize $\tt{A}_{I,0,0} = \tt{A}_{0,0} = \tt{A}_{\exists,0,0} = \tt{A}_{\exists,0,0} \cup \tt{A}_{\top,0}$, so \eqref{eq:thm_instantiated_correctness_set} and \eqref{eq:thm_instantiated_correctness_key} hold trivially.

Assume that \eqref{eq:thm_instantiated_correctness_set} and \eqref{eq:thm_instantiated_correctness_key} hold at some $t$ and $i-1$.
The assignment at $t$, $i$ is obtained by executing some merge rule $R$ on the assignment at $t$, $i-1$.
If $\tt{A}$ does not appear on the LHS of $R$, then we are done because $\tt{A}_{E,t,i} = \tt{A}_{E,t,i-1}$ and $\tt{A}_{TS,t,i} = \tt{A}_{TS,t,i-1}$.
If $\tt{A}$ appears on the LHS of $R$ and the RHS is not a channel, then WLOG let $R: \tt{A} ~\tt{<=}~ \tt{f}\tt{(}\tt{C}_1, \dots, \tt{C}_n\tt{)}$.
\begin{align}
\tt{A}_{\exists,t,i}
=& (\tt{A}_{\exists,t,i-1} \cup \tt{f}\tt{(}\tt{C}_{1,\exists,t,i-1} \cup \tt{C}_{1,\top,t}, \dots, \tt{C}_{n,\exists,t,i-1} \cup \tt{C}_{n,\top,t}\tt{)}) - \tt{A}_{\top,t}\nonumber\\
=& (\tt{A}_{\exists,t,i-1} \cup \tt{f}\tt{(}\tt{C}_{1,I,t,i-1} \cup \tt{C}_{1,\top,t}, \dots, \tt{C}_{n,I,t,i-1} \cup \tt{C}_{n,\top,t}\tt{)} \cup \tt{A}_{\top,t}) - \tt{A}_{\top,t}\nonumber\\
=& (\tt{A}_{\exists,t,i-1} \cup \tt{f}\tt{(}\tt{C}_{1,I,t,i-1}, \dots, \tt{C}_{n,I,t,i-1}\tt{)} \cup \tt{A}_{\top,t}) - \tt{A}_{\top,t} & \text{(Lemma \ref{lem:merge_gc_inv})}\nonumber\\
=& (\tt{A}_{\exists,t,i-1} \cup \tt{f}\tt{(}\tt{C}_{1,I,t,i-1}, \dots, \tt{C}_{n,I,t,i-1}\tt{)}) - \tt{A}_{\top,t}.\label{eq:thm:ins_correct:instant}
\end{align}
If $\tt{A}$ is non-augmented, we apply the inductive hypothesis to \eqref{eq:thm:ins_correct:instant},
\begin{align*}
\tt{A}_{\exists,t,i}
&&\subseteq&& \tt{A}_{I,t,i-1} \cup \tt{f}\tt{(}\tt{C}_{1,I,t,i-1}, \dots, \tt{C}_{n,I,t,i-1}\tt{)}
&&=&& \tt{A}_{I,t,i}.
\end{align*}
If $\tt{A}$ is key-augmented, we apply the inductive hypothesis to \eqref{eq:thm:ins_correct:instant},
\begin{align*}
\tt{A}_{\exists,t,i}
&&\subseteq&& (\tt{A}_{I,t,i-1} \cup \tt{f}\tt{(}\tt{C}_{1,I,t,i-1}, \dots, \tt{C}_{n,I,t,i-1}\tt{)}) - (\tt{A}_{!,t,i-1} - \tt{A}_{I,t,i-1})
&&=&& \tt{A}_{I,t,i}.
\end{align*}
In both cases, we see that
\begin{align*}
\tt{A}_{I,t,i}
\subseteq& \tt{A}_{\exists,t,i-1} \cup \tt{A}_{\top,t} \cup \tt{f}\tt{(}\tt{C}_{1,I,t,i-1}, \dots, \tt{C}_{n,I,t,i-1}\tt{)}\\
=& \tt{A}_{\exists,t,i-1} \cup \tt{A}_{\top,t} \cup \tt{f}\tt{(}\tt{C}_{1,\exists,t,i-1} \cup \tt{C}_{1,\top,t}, \dots, \tt{C}_{n,\exists,t,i-1} \cup \tt{C}_{n,\top,t}\tt{)} & \text{(Lemma \ref{lem:merge_gc_inv})}\\
=& \tt{A}_{\exists,t,i} \cup \tt{A}_{\top,t}
\end{align*}
For key-augmented $\tt{A}$, we also have
\begin{align*}
\tt{A}_{!,t,i}
&= \tt{A}_{!,t,i-1} \cup \pi_!(\tt{f}\tt{(}\tt{C}_{1,I,t,i-1}, \dots, \tt{C}_{n,I,t,i-1}\tt{)})\\
&= \pi_!(\tt{A}_{\exists,t,i-1} \cup \tt{A}_{\top,t}) \cup \pi_!(\tt{f}\tt{(}\tt{C}_{1,I,t,i-1}, \dots, \tt{C}_{n,I,t,i-1}\tt{)})\\
&= \pi_!(\tt{A}_{\exists,t,i-1} \cup \tt{A}_{\top,t} \cup \tt{f}\tt{(}\tt{C}_{1,I,t,i-1}, \dots, \tt{C}_{n,I,t,i-1}\tt{)})\\
&= \pi_!(\tt{A}_{\exists,t,i-1} \cup \tt{A}_{\top,t} \cup \tt{f}\tt{(}\tt{C}_{1,\exists,t,i-1} \cup \tt{C}_{1,\top,t}, \dots, \tt{C}_{n,\exists,t,i-1} \cup \tt{C}_{n,\top,t}\tt{)}) & \text{(Lemma \ref{lem:merge_gc_inv})}\\
&= \pi_!(\tt{A}_{\exists,t,i} \cup \tt{A}_{\top,t}).
\end{align*}
Otherwise, $\tt{A}$ appears on the LHS of $R$ and the RHS of $R$ is a channel $\overline{\tt{C}}$.
The argument above goes through by replacing $\tt{f}$ with $\overline{\tt{C}}$.

Suppose $\mathfrak{P}_{GC}$ reaches a finite fixed point for time $t$ at $i=\iota$.
We first establish that $\mathfrak{P}_{iGC}$ also eventually reaches a fixed point for $t$.
\begin{claim}
If $\mathfrak{P}_{GC}$ reaches a finite fixed point at $t$, $\iota$, then $\mathfrak{P}_{iGC}$ also eventually reaches a fixed point.
\end{claim}
\begin{claimproof}
By the above argument, $\tt{A}_{I,t,j} \subseteq \tt{A}_{\exists,t,j} \cup \tt{A}_{\top,t} = \tt{A}_{\exists,t,\iota} \cup \tt{A}_{\top,t}$ for all $j \geq \iota$.
Since $\tt{<=}$ can only increase $\tt{A}_{I,t,j}$ and $\tt{A}_{I,t,j}$ is upper bounded, $\mathfrak{P}_{iGC}$ must also reach a fixed point for $t$ after a finite number of steps after $\iota$.
\end{claimproof}

Finally, suppose \eqref{eq:thm_instantiated_correctness_set} and \eqref{eq:thm_instantiated_correctness_key} hold at $t-1$, $\iota$ which is a fixed point of $\mathfrak{P}_{iGC}$ for time $t-1$.
Suppose $\tt{A}$ appears in the LHS of $m$ deferred merge rules in $\mathfrak{P}$: $\tt{A} ~\tt{<+}~ \tt{f}_j\tt{(}\tt{C}_{j,1},\dots,\tt{C}_{j,n_j}\tt{)}$.
Additionally, in $\mathfrak{P}_{GC}$, $\tt{A}_{TS}$ appears in (at most) one GC rule $\langle \tt{A}_{TS}, \tt{D}_{1,TS}, \dots, \tt{D}_{k,TS} \rangle$ $~\tt{<+}~$ $\tt{GC}(\tt{A}_{TS}, \tt{D}_{1,TS}, \dots, \tt{D}_{k,TS}; \tt{B}_{1,TS}, \dots, \tt{B}_{k',TS})$.
Correspondingly, $\tt{A}_E$ appears in (at most) one instantiated GC rule $\langle \tt{A}_E, \tt{D}_{1,E}, \dots, \tt{D}_{k,E} \rangle$ $~\tt{<-}~$ $\tt{GCI}(\tt{A}_{E},$ $\tt{D}_{1,E}, \dots, \tt{D}_{k,E}; \tt{B}_{1,E}, \dots, \tt{B}_{k',E})$.
Let $\langle \tt{A}_{TS}^*, \tt{D}_{1,TS}^*, \dots, \tt{D}_{k,TS}^* \rangle$ $= \tt{GC}(\tt{A}_{TS,t-1,\iota}, \tt{D}_{1,TS,t-1,\iota}, \dots, \tt{D}_{k,TS,t-1,\iota};$ $\tt{B}_{1,TS,t-1,\iota}, \dots, \tt{B}_{k',TS,t-1,\iota})$, and
$\langle \tt{A}_E^-, \tt{D}_{1,E}^-, \dots, \tt{D}_{k,E}^- \rangle = \tt{GCI}(\tt{A}_{E,t-1,\iota}, \tt{D}_{1,E,t-1,\iota}, \dots, \tt{D}_{k,E,t-1,\iota};$ $\tt{B}_{1,E,t-1,\iota},$ $\dots, \tt{B}_{k',E,t-1,\iota})$.
By the conservative property of $\tt{GC}$, we have that $\tt{A}_{\exists}^* \subseteq \tt{A}_{\exists,t-1,\iota}$, $\tt{A}_{\top}^* \supseteq \tt{A}_{\top,t-1}$, and $\tt{A}_{\exists}^* \cup \tt{A}_{\top}^* = \tt{A}_{\exists,t-1,\iota} \cup \tt{A}_{\top,t-1,}$.
Furthermore, we have that $\tt{A}_I^- \subseteq \tt{A}_{T}^*$ if $\tt{iGC}$ is weakly consistent with $\tt{GC}$ on $\tt{A}$.
Hence,
\begin{align}
&\tt{A}_{\exists,t,0}\nonumber\\
=& \left(\tt{A}_{\exists,t-1,\iota} \cup \tt{A}_\exists^* \cup \bigcup_{j=1}^m \tt{f}_j\tt{(}\tt{C}_{j,1,\exists,t-1,\iota} \cup \tt{C}_{j,1,\top,t-1}, \dots, \tt{C}_{j,n_j,\exists,t-1,\iota} \cup \tt{C}_{j,n_j,\top,t-1}\tt{)}\right)
- \tt{A}_{\top,t-1} - \tt{A}_\top^*\nonumber\\
=& \left(\tt{A}_{\exists,t-1,\iota} \cup \bigcup_{j=1}^m \tt{f}_j\tt{(}\tt{C}_{j,1,\exists,t-1,\iota} \cup \tt{C}_{j,1,\top,t-1}, \dots, \tt{C}_{j,n_j,\exists,t-1,\iota} \cup \tt{C}_{j,n_j,\top,t-1}\tt{)}\right)
- \tt{A}_{\top,t-1} - \tt{A}_\top^* & \text{(Property \ref{property:gc_conservative})}\nonumber\\
=& \left(\tt{A}_{\exists,t-1,\iota} \cup \tt{A}_{\top,t-1} \cup \bigcup_{j=1}^m \tt{f}_j\tt{(}\tt{C}_{j,1,\exists,t-1,\iota} \cup \tt{C}_{j,1,\top,t-1}, \dots, \tt{C}_{j,n_j,\exists,t-1,\iota} \cup \tt{C}_{j,n_j,\top,t-1}\tt{)}\right)
- \tt{A}_{\top,t-1} - \tt{A}_\top^* & \text{(Property \ref{property:gc_conservative})}\nonumber\\
=& \left(\tt{A}_{\exists,t-1,\iota} \cup \tt{A}_{\top,t-1} \cup \bigcup_{j=1}^m \tt{f}_j\tt{(}\tt{C}_{j,1,I,t-1,\iota}, \dots, \tt{C}_{j,n_j,I,t-1,\iota}\tt{)}\right)
- \tt{A}_{\top,t-1} - \tt{A}_\top^* & \text{(Lemma \ref{lem:merge_gc_inv})}\nonumber\\
=& \left(\tt{A}_{\exists,t-1,\iota} \cup \bigcup_{j=1}^m \tt{f}_j\tt{(}\tt{C}_{j,1,I,t-1,\iota}, \dots, \tt{C}_{j,n_j,I,t-1,\iota}\tt{)}\right)
- \tt{A}_{\top,t-1} - \tt{A}_\top^*. & \text{(Property \ref{property:gc_conservative})} \label{eq:thm:ins_correct:timestep}
\end{align}
If $\tt{A}$ is non-augmented, \eqref{eq:thm:ins_correct:timestep} becomes
\begin{align}
\tt{A}_{\exists,t,0}
=& \left(\tt{A}_{\exists,t-1,\iota} \cup \bigcup_{j=1}^m \tt{f}_j\tt{(}\tt{C}_{j,1,I,t-1,\iota}, \dots, \tt{C}_{j,n_j,I,t-1,\iota}\tt{)}\right)
- \tt{A}_\top^*. & \text{(Property \ref{property:gc_conservative})}\nonumber\\
\subseteq& \left(\tt{A}_{\exists,t-1,\iota} \cup \bigcup_{j=1}^m \tt{f}_j\tt{(}\tt{C}_{j,1,I,t-1,\iota}, \dots, \tt{C}_{j,n_j,I,t-1,\iota}\tt{)}\right)
- \tt{A}_I^- & \text{(Property \ref{property:instantiated_gc_consistent})} \nonumber\\% \label{eq:thm:inst_correct:timestep_0}\\
\subseteq& \left(\tt{A}_{I,t-1,\iota} \cup \bigcup_{j=1}^m \tt{f}_j\tt{(}\tt{C}_{j,1,I,t-1,\iota}, \dots, \tt{C}_{j,n_j,I,t-1,\iota}\tt{)}\right)
- \tt{A}_I^- & \text{(I.H.)} \nonumber\\% \label{eq:thm:inst_correct:timestep_1}\\
=& \tt{A}_{I,t,0} \nonumber
\end{align}
If $\tt{A}$ is key-augmented, \eqref{eq:thm:ins_correct:timestep} becomes
\begin{align*}
\tt{A}_{\exists,t,0}
\subseteq& \left(\tt{A}_{\exists,t-1,\iota} \cup \bigcup_{j=1}^m \tt{f}_j\tt{(}\tt{C}_{j,1,I,t-1,\iota}, \dots, \tt{C}_{j,n_j,I,t-1,\iota}\tt{)}\right)
- \tt{A}_{\top,t-1} - \tt{A}_I^- & \text{(Property \ref{property:instantiated_gc_consistent})} 
\\
\subseteq& \left(\tt{A}_{I,t-1,\iota} \cup \bigcup_{j=1}^m \tt{f}_j\tt{(}\tt{C}_{j,1,I,t-1,\iota}, \dots, \tt{C}_{j,n_j,I,t-1,\iota}\tt{)}\right)
-(\tt{A}_{!,t-1,\iota} - \tt{A}_{I,t-1,\iota}) - \tt{A}_I^- & \text{(I.H.)}\\
=& \tt{A}_{I,t,0}
\end{align*}
In both cases, we see that
\begin{align*}
&\tt{A}_{I,t,0}\\
\subseteq&~ \tt{A}_{I,t-1,\iota} \cup \bigcup_{j=1}^m \tt{f}_j\tt{(}\tt{C}_{j,1,I,t-1,\iota}, \dots, \tt{C}_{j,n_j,I,t-1,\iota}\tt{)} \\
\subseteq&~ \tt{A}_{\exists,t-1,\iota} \cup \tt{A}_{\top,t-1} \cup \bigcup_{j=1}^m \tt{f}_j\tt{(}\tt{C}_{j,1,I,t-1,\iota}, \dots, \tt{C}_{j,n_j,I,t-1,\iota}\tt{)} & \text{(I.H.)} \\
=&~ \tt{A}_{\exists,t-1,\iota} \cup \tt{A}_{\top,t-1,} \cup \bigcup_{j=1}^m \tt{f}_j\tt{(}\tt{C}_{j,1,\exists,t-1,\iota} \cup \tt{C}_{j,1,\top,t-1}, \dots, \tt{C}_{j,n_j,\exists,t-1,\iota} \cup \tt{C}_{j,n_j,\top,t-1}\tt{)} & \text{(Lemma \ref{lem:merge_gc_inv})} \\
=&~ \tt{A}_{\exists,t-1,\iota} \cup \tt{A}_{\top,t-1} \cup \tt{A}_\exists^* \cup \tt{A}_\top^* \cup \bigcup_{j=1}^m \tt{f}_j\tt{(}\tt{C}_{j,1,\exists,t-1,\iota} \cup \tt{C}_{j,1,\top,t-1}, \dots, \tt{C}_{j,n_j,\exists,t-1,\iota} \cup \tt{C}_{j,n_j,\top,t-1}\tt{)} & \text{(Property \ref{property:gc_conservative})}\\
=&~ \tt{A}_{\exists,t,0} \cup \tt{A}_{\top,t}
\end{align*}
Furthermore, for $\tt{A}_!$, we have
\begin{align*}
\tt{A}_{!,t,0}
&= \tt{A}_{!,t-1,\iota} \cup \pi_!\left(\bigcup_{j=1}^m \tt{f}_j\tt{(}\tt{C}_{j,1,I,t-1,\iota}, \dots, \tt{C}_{j,n_j,I,t-1,\iota}\tt{)}\right)\\
&= \pi_!\left(\tt{A}_{\exists,t-1,\iota} \cup \tt{A}_{\top,t-1} \cup \bigcup_{j=1}^m \tt{f}_j\tt{(}\tt{C}_{j,1,I,t-1,\iota}, \dots, \tt{C}_{j,n_j,I,t-1,\iota}\tt{)}\right) & \text{(I.H.)} \\
&= \pi_!(\tt{A}_{\exists,t,0} \cup \tt{A}_{\top,t}) & \text{(Lemma \ref{lem:merge_gc_inv})}
\end{align*}
where in the final equality we have used the same argument as above for sets without primary keys.
\end{proof}

\begin{cor}
\label{cor:instantiated_equivalence_of_output}
If $\tt{A}$ is an output set, it has $\tt{A}_I = \tt{A}_\exists = \tt{A}$.
\end{cor}

\begin{cor}
\label{cor:instantiated_confluence_of_output}
If $\mathfrak{P}$ is confluent and $\tt{A}$ is an output set, then $\mathfrak{P}_{iGC}$ is confluent in $\tt{A}_I$.
\end{cor}

\begin{cor}
\label{cor:instantiated_confluence_of_keys}
If $\mathfrak{P}$ is confluent and $\mathfrak{P}_{iGC}$ is strongly consistent with $\mathfrak{P}_{GC}$ on $\tt{A}$, then $\mathfrak{P}_{iGC}$ is confluent in $\tt{A}_I$ and $\tt{A}_!$.
\end{cor}

\subsubsection{Incomplete}
In the previous section, Theorem \ref{thm:instantiated_correctness} established the soundness of our garbage collection rewrite --- tuples are deleted only if they do not affect future computation.
Unfortunately, it is not generally possible to establish completeness, i.e., that any tuple that could be removed without affecting future computation will be removed.

There are a couple of possible reasons for incompleteness.
Firstly, there may not be a unique largest set of reclaimable tuples.
For example, consider the function $\tt{f}(\tt{A}) = \emptyset$ if $|\tt{A}| = 0$ and $\{1\}$ otherwise.
Given the input $\tt{A} = \{a,b\}$, either $a$ or $b$ can be reclaimed without affecting computation, but we cannot safely reclaim both tuples.

Secondly, the act of garbage collection itself reduces the information available for reasonable about safe reclamation.
Consider the program
\begin{align*}
\tt{A} &\quad\tt{<}\sim\quad \tt{chnA}\\
\tt{B} &\quad\tt{<}\sim\quad \tt{chnB}\\
\tt{C} &\quad\tt{<=}\quad \tt{A} \cap \tt{B}\\
\tt{D} &\quad\tt{<}\sim \tt{C}
\end{align*}
If a tuple $x$ arrives in both $\tt{A}$ and $\tt{B}$, it is derived in $\tt{C}$ and sent to $\tt{D}$.
A complete garbage collection must immediately reclaim $x$ from both $\tt{A}$ and $\tt{B}$, and eventually from $\tt{C}$ once its arrival at $\tt{D}$ has been determined.
Now if $x$ arrives in $\tt{A}$ again but not in $\tt{B}$, we cannot determine from $\tt{A}$, $\tt{B}$ and $\tt{C}$ that it is safe to delete.
While it is possible to deduce that $x$ in $\tt{A}$ is safe for deletion by examining $\tt{D}$, this would require multi-rule reasoning, and in the worst case, over the entire program $\tt{P}$.
Doing so is complex and beyond the scope of our current work.fi



\subsection{Coordination}
Unlike $\mathfrak{P}_{GC}$, the instantiated GC program $\tt{P}_{iGC}$ is in general \emph{not confluent} even if $\mathfrak{P}$ is.
This appears to run counter to our goal of reconciling CALM with garbage collection.
However, we highlight that $\mathfrak{P}_{iGC}$ is `confluent' in $\tt{A}_\exists$, in the sense that if a tuple $x$ is in $\tt{A}_\exists$, then all runs of $\mathfrak{P}_{iGC}$ will agree that $x \in \tt{A}_I$.
Thus, $\mathfrak{P}$ is `non-confluent' only with respect to tombstone sets $\tt{A}_\top$, i.e., separate runs only disagree on tuples that, by our invariants, do not affect the outcome!

In fact, the `confluence' of \cite{marczak2012confluence} was defined in terms of agreement on output sets.
Dedalus$^+$ (and monotone Bloom$^L$) are restrictions on the programming language to ensure that \emph{all sets and lattices} are temporally inflationary and confluent.
This does not exclude the possibility of other programs in non-monotone Bloom$^L$ being confluent in some output sets;
our program $\mathfrak{P}_{iGC}$ is exactly an example of such a program.
Instead of restricting the language, we restrict the possible GC functions to ensure that output sets always converge.

The CALM analysis of \cite{ameloot2014weaker} also explicitly defined `coordination-freeness' and `monotonicity of queries' with respect to output relations, and allowed for the possibility of divergent intermediate states.
(In particular, there are no restrictions placed on the nodes' memory relations.)
A close examination of the analyses (see our transducer writeup) reveals that `coordination' is required only to determine that one has received the complete input.
The GC invariant can be interpreted as an assertion that tombstoned tuples are independent of any future input, and thus do not affect the requirement (or lack thereof) of knowledge of input completeness for determining output relations.
Said differently, coordination (or knowledge of input completeness) is required only to converge on where $\tt{A}_I$ lies in the range of $\tt{A}_\exists \subseteq \tt{A}_I \subseteq \tt{A}_\exists \cup \tt{A}_\top$, but not for the confluence of the output relations that we care about.








%!TEX root = bloom_logical.tex

\section{Conclusion}
\label{sec:conclude}

In this work, we have introduced a formal and general framework for garbage collection for Bloom programs.
Garbage collection rules in our framework obey the intuitive restriction imposed by Invariant \ref{inv:merge_gc} that deleted tuples should have no effect on any possible future executions of its corresponding Bloom rule.
We have also argued that the garbage collection rules do not introduce the need for more coordination.
In addition, we demonstrated that monotone Bloom rules can also be automatically garbage collected in our framework.
This makes it possible to write coordination-free programs with automated garbage collection.

Some open questions remain.
Firstly, like Edelweiss \cite{conway2014edelweiss}, we have provided garbage collection for sets in Bloom, and it is unclear if the same approach extends to general join semilattices in Bloom$^L$.
In fact, the very notion of storage reclamation for join semilattices is less well-defined than sets, for which smaller cardinalities correspond to lower storage.
In contrast, there is no necessary correspondence between the storage required for a lattice element and its position within the lattice.

Secondly, we have left open the matter of completeness.
In general we believe that a maximal garbage collection would require reasoning about the entire distributed program state.
Nevertheless, there may be more restricted notions of completeness, possibly from a node-local perspective, that could be formalized and proven.
Being able to recognize completeness could help guide the programmer towards more storage-efficient implementations.


\bibliographystyle{abbrv}
\bibliography{calmbib}

\balancecolumns


\newpage\onecolumn
\appendix


%!TEX root = bloom_logical.tex

\section{Additional Examples}
\label{sec:examples}

\subsection{Examples of Logical GC Rules}
\label{sec:examples:logical}
\begin{example}[Trivial GC]
$\tt{GCNone}(\tt{A}_{1,TS},$ $\dots,$ $\tt{A}_{n,TS}; \tt{B}_{1,TS}, \dots, \tt{B}_{m,TS}) = \langle \tt{A}_{1,TS}, \dots, \tt{A}_{n,TS} \rangle$.
\end{example}
The trivial GC returns the tombstone lattices unchanged.
If $\tt{A}_1, \dots, \tt{A}_n$ are all output relations, then $\tt{GCNone}$ is the only garbage collection rule that respects output relations.

\begin{example}[Union GC]
\label{ex:union_gc}
Consider the union $R: \tt{B} ~\tt{<op>}~ \tt{A} \cup \tt{C}$, where $\tt{<op>}$ is a merge or deferred merge.
Then $\tt{GCUnion}(\tt{A}_{TS}, \tt{C}_{TS}; \tt{B}_{TS}) = \langle (\tt{A}_\exists - \tt{A}^*, \tt{A}_\top \cup \tt{A}^*), (\tt{C}_\exists - \tt{C}^*, \tt{C}_\top \cup \tt{C}^*) \rangle$ where $\tt{A}^* = \tt{A}_\exists \cap  (\tt{B}_\exists \cup \tt{B}_\top)$ and $\tt{C}^* = \tt{C}_\exists \cap  (\tt{B}_\exists \cup \tt{B}_\top)$ is safe for $R$.
\end{example}
Invariant \ref{inv:merge_gc} for the union rule reduces to $\tt{A}_\top \cup \tt{C}_\top \subseteq \tt{A}_\exists \cup \tt{C}_\exists \cup \tt{B}_\exists \cup \tt{B}_\top$.
Then,
\begin{align*}
\tt{A}_\top \cup \tt{A}^* \cup \tt{C}_\top \cup \tt{A}^*
&\quad\subseteq\quad \tt{A}_\exists \cup \tt{C}_\exists \cup \tt{B}_\exists \cup \tt{B}_\top\\
&\quad=\quad (\tt{A}_\exists \cap (\tt{B}_\exists \cup \tt{B}_\top)^c) \cup (\tt{C}_\exists \cap (\tt{B}_\exists \cup \tt{B}_\top)^c) \cup \tt{B}_\exists \cup \tt{B}_\top\\
&\quad=\quad (\tt{A}_\exists - \tt{A}^*) \cup (\tt{C}_\exists - \tt{A}^*) \cup \tt{B}_\exists \cup \tt{B}_\top\\
\end{align*}
so $\tt{GCUnion}$ is safe for $R$.

\begin{example}[Intersection GC]
\label{ex:intersection_gc}
Consider the union $R: \tt{B} ~\tt{<op>}~ \tt{A} \cap \tt{C}$, where $\tt{<op>}$ is a merge or deferred merge.
Then $\tt{GCIntersect}(\tt{A}_{TS}, \tt{C}_{TS}; \tt{B}_{TS}) = \langle (\tt{A}_\exists - \tt{A}^*, \tt{A}_\top \cup \tt{A}^*), (\tt{C}_\exists - \tt{C}^*, \tt{C}_\top \cup \tt{C}^*) \rangle$ where $\tt{A}^* = \tt{A}_\exists \cap  (\tt{B}_\exists \cup \tt{B}_\top)$ and $\tt{C}^* = \tt{C}_\exists \cap  (\tt{B}_\exists \cup \tt{B}_\top)$ is safe for $R$.
\end{example}
Note that $\tt{GCIntersect} = \tt{GCUnion}$, both of which tombstones tuples from $\tt{A}_\exists$ and $\tt{C}_\exists$ after they have appeared in $\tt{B}_\exists \cup \tt{B}_\top$.
In this case, Invariant \ref{inv:merge_gc} reduces to $\tt{A}_\top \in \tt{B}_\exists \cup \tt{B}_\top$ and $\tt{C}_\top \in \tt{B}_\exists \cup \tt{B}_\top$.
Then $\tt{A}_\top \cup \tt{A}^* \subseteq \tt{B}_\exists \cup \tt{B}_\top$ and $\tt{C}_\top \cup \tt{C}^* \subseteq \tt{B}_\exists \cup \tt{B}_\top$, so $\tt{GCIntersect}$ is safe for $R$.

\begin{example}[Select GC]
Consider the selection $R: \tt{B} ~\tt{<op>}~ \sigma_P(\tt{A})$, where $\tt{<op>}$ is either a merge or deferred merge.
Then $\tt{GCSelect}(\tt{A}_{TS}; \tt{B}_{TS}) = (\tt{A}_\exists - \tt{A}^*, \tt{A}_\top \cup \tt{A}^*)$ where $\tt{A}^* = \sigma_{\neg P}(\tt{A}_\exists \cup \tt{A}_\top) \cup (\sigma_P(\tt{A}_\exists \cup \tt{A}_\top) \cap (\tt{B}_\exists \cup \tt{B}_\top))$ is safe for $R$.
\end{example}
Similar to Example \ref{ex:project_gc}, the Merge GC Invariant \ref{inv:merge_gc} for selection reduces to $\sigma_P(\tt{A}_\top) \subseteq \tt{B}_\exists \cup \tt{B}_\top$.
If $\tt{A}_{TS}$ and $\tt{B}_{TS}$ satisfy the invariant, then
\begin{align*}
\sigma_P(\tt{A}_\top \cup \tt{A}^*)
&= \sigma_P(\tt{A}_\top) \cup \sigma_P(\sigma_{\neg P}(\tt{A}_\exists \cup \tt{A}_\top) \cup (\sigma_P(\tt{A}_\exists \cup \tt{A}_\top) \cap (\tt{B}_\exists \cup \tt{B}_\top)))\\
&= \sigma_P(\tt{A}_\top) \cup (\sigma_P(\tt{A}_\exists \cup \tt{A}_\top) \cap (\tt{B}_\exists \cup \tt{B}_\top))\\
&= \sigma_P(\tt{A}_\top) \cup (\sigma_P(\tt{A}_\exists) \cap (\tt{B}_\exists \cup \tt{B}_\top))\\
&\subseteq \tt{B}_\exists \cup \tt{B}_\top,
\end{align*}
so $\tt{GCSelect}$ is safe for $R$.

\begin{example}[Linear GC]
Consider the rule $R: \tt{B} ~\tt{<op>}~ \tt{f}(\tt{A})$, where $\tt{f}$ is a linear function, i.e., $\tt{f}(\tt{A}) = \bigcup_{a\in \tt{A}} \tt{f}(\{a\})$.
Then $\tt{GCAll}(\tt{A}_{TS}; \tt{B}_{1,TS}, \dots, \tt{B}_{m,TS})$ $=$ $(\emptyset,$ $\tt{A}_\exists \cup \tt{A}_\top)$ is safe for $R$.
\end{example}
Since $\tt{f}$ is linear, Invariant \ref{inv:merge_gc} reduces to $\tt{f}\tt{(}\tt{A}_\top\tt{}) \subseteq \tt{B}_\exists \cup \tt{B}_\top$, but this is true under the condition $\tt{B}_\exists \cup \tt{B}_\top \supseteq \tt{f}\tt{(}\tt{A}_{\exists} \cup \tt{A}_{\top}\tt{)} = \tt{f}\tt{(}\tt{A}_\exists\tt{}) \cup \tt{f}\tt{(}\tt{A}_\top\tt{})$.
Thus, Invariant \ref{inv:merge_gc} is trivially true for linear $\tt{f}$.
Intuitively, a tuple $a \in \tt{A}_\exists$ is independent of all other tuples when $\tt{f}$ is linear, and hence, its complete effects will have been realized in $\tt{B}$ (either through a merge or deferred merge) by the point at which we perform garbage collection.
Therefore, we can always reclaim the entire set without impacting our results.

In particular, observe that the copy, union, select, and project rules of previous examples are all linear, and thus can be fully reclaimed at the end of each timestep.

\begin{example}[Join GC]
\label{ex:join_gc_no_manifests}
Consider the join $R: \tt{B} ~\tt{<op>}~ \tt{A} \bowtie_{\tt{A}.k_a = \tt{C}.k_c} \tt{C}$, where $\tt{<op>}$ is either a merge or deferred merge.
The only garbage collection rule that is safe for $R$ is the trivial $\tt{GCTrivial}(\tt{A}_{TS}, \tt{C}_{TS}; \tt{B}_{TS}) = \langle \tt{A}_{TS}, \tt{C}_{TS} \rangle$.
\end{example}
Without additional information, one cannot guarantee that a tuple in $\tt{A}$ will not match any future tuple in $\tt{C}$ and vice versa.
In some cases, however, one may be provided with the specific information that no more tuples matching the attribute will ever appear.

Formally, let $\mathcal{X}$ be the domain of possible tuples for a table $\tt{X}$, and denote the powerset of $\mathcal{X}$ as $\mathbb{P}(\mathcal{X})$.
Also let $\mathcal{K}$ be the domain of possible values of an attribute $\tt{X}.k$.
A manifest $p = (p.k, p.\tt{X}) \in \mathcal{K} \times \mathbb{P}(\mathcal{X})$ maps a key in $\mathcal{K}$ to a subset of $\mathbb{P}(\mathcal{X})$.
We denote by $\tt{X}$ by $\tt{P}_\tt{X}$ a table of manifests (with $k$ as its primary key) for $\tt{X}$, and say it is consistent with $\tt{X}$ if $\forall p \in \tt{P}_\tt{X}$, $p.\tt{X} \supseteq \{x \in \tt{X} ~:~ x.k = p.k\}$.
We will always assume that manifest tables are always consistent.

In the join rule above, we can reclaim from $\tt{A}$ given manifests on $\tt{C}$ and vice versa:

\begin{example}[Join with manifests]
Consider the join in Example \ref{ex:join_gc_no_manifests}, now augmented with manifest tables $\tt{P}_\tt{A}$ and $\tt{P}_\tt{C}$.
(If manifests are not available, we can always let $\tt{P} = \emptyset$ with no loss of generality.)
The garbage collection rule $\tt{GCJoin}(\tt{A}_{TS},$ $\tt{P}_{\tt{A},TS},$ $\tt{C}_{TS},$ $\tt{P}_{\tt{C},TS};$ $\tt{B}_{TS})$ $=$ $\langle (\tt{A}_\exists - \tt{A}^*, \tt{A}_\top \cup \tt{A}^*), (\tt{P}_{\tt{A},\exists} - \tt{P}_\tt{A}^*, \tt{P}_{\tt{A},\top} \cup \tt{P}_\tt{A}^*), (\tt{C}_\exists - \tt{C}^*, \tt{C}_\top \cup \tt{C}^*), (\tt{P}_{\tt{C},\exists} - \tt{P}_\tt{C}^*, \tt{P}_{\tt{C},\top} \cup \tt{P}_\tt{C}^*) \rangle$, with
\begin{align*}
\tt{A}^* &= \{a \in \tt{A}_\exists \cup \tt{A}_\top ~:~ \exists p_c \in \tt{P}_{\tt{C},\exists} \cup \tt{P}_{\tt{C},\top},
% \\&
\text{ such that } (a.k_a = p_c.k_c) \wedge \left(\{a\} \times p_c.\tt{C} \subseteq \tt{B}_\exists \cup \tt{B}_\top\right)\}\\
\tt{P}_\tt{A}^* &= \{p_a \in \tt{P}_{\tt{A},\exists} \cup \tt{P}_{\tt{A},\top} ~:~ \exists p_c \in \tt{P}_{\tt{C},\exists} \cup \tt{P}_{\tt{C},\top},
% \\&
\text{ such that } (p_a.k_a = p_c.k_c) \wedge \left(p_a.\tt{A} \times p_c.\tt{C} \subseteq \tt{B}_\exists \cup \tt{B}_\top\right)\}\\
\tt{C}^* &= \{c \in \tt{C}_\exists \cup \tt{C}_\top ~:~ \exists p_a \in \tt{P}_{\tt{A},\exists} \cup \tt{P}_{\tt{A},\top},
% \\&
\text{ such that } (p_a.k_a = c.k_c) \wedge \left(p_a.\tt{A} \times \{c\} \subseteq \tt{B}_\exists \cup \tt{B}_\top\right)\}\\
\tt{P}_\tt{A}^* &= \{p_a \in \tt{P}_{\tt{A},\exists} \cup \tt{P}_{\tt{A},\top} ~:~ \exists p_c \in \tt{P}_{\tt{C},\exists} \cup \tt{P}_{\tt{C},\top}
% \\&
\text{ such that } (p_a.k_a = p_c.k_c) \wedge \left(p_a.\tt{A} \times p_c.\tt{C} \subseteq \tt{B}_\exists \cup \tt{B}_\top\right)\}
\end{align*}
is safe for the join.
\end{example}
In words, a tuple $a$ can be tombstoned if it matches a punctuation $p_c$ in $\tt{P}_\tt{C}$ \emph{and} the result of joining $\{a\}$ with $\tt{C}$ is already effected in $\tt{B}_{TS}$.
Hence, any $c \in \tt{C}$ either does not match $a$, or is already joined with $a$ in $\tt{B}$ and so will not affect any future join computation.
In addition, we tombstone a punctuation $p_a$ if it matches a punctuation $p_c$ and the join of their sets $p_a.\tt{A} \times p_c.\tt{C}$ is already effected in $\tt{B}_{TS}$.
It is a simple exercise to show that $\tt{GCJoin}$ tombstones a punctuation $p_a$ only when its set $p_a.\tt{A}$ is tombstoned (or does not exist yet), i.e., $(p_a \in \tt{P}_{\tt{A},\top}) \implies \tt{A}_\exists \cap p_a.\tt{A} = \emptyset \implies \sigma_{\tt{A}.k_a = p_a.k_a}(\tt{A}_\exists) = \emptyset$.

\begin{example}[DR+: Positive Difference]
Consider the rule $R: \tt{B} ~\tt{<op>}~ \tt{A} - \tt{C}$, where $\tt{<op>}$ is a merge or deferred merge.
Then the rule $\tt{GCDR+}(\tt{A}_{TS}, \tt{C}_{TS};$ $\tt{B}_{TS}) = \langle (\tt{A}_\exists - \tt{A}^*, \tt{A}_\top \cup \tt{A}^*), \tt{C}_{TS}\rangle$
where $\tt{A}^* = (\tt{A}_\exists \cup \tt{A}_\top) \cap (\tt{C}_\exists \cup \tt{C}_\top)$
is safe for $R$.
\end{example}
For set differences, the Merge GC Invariant \ref{inv:merge_gc} reduces to
$\forall \widehat{\tt{A}} \supseteq \tt{A}_\exists, \forall \widehat{\tt{C}} \supseteq \tt{C}_\exists$: $\widehat{\tt{A}} - \widehat{\tt{C}} = (\widehat{\tt{A}} \cup \tt{A}_\top) - (\widehat{\tt{C}} \cup \tt{C}_\top)$.
In the case where $\widehat{\tt{A}} = \tt{A}_\exists$ and $\widehat{\tt{C}} = \tt{C}_\exists$, it implies that $\tt{A}_\top \subseteq \tt{C}_\exists \cup \tt{C}_\top$;
in the case where $\widehat{\tt{A}} = \tt{A}_\exists \cup \tt{C}_\top$ and $\widehat{\tt{C}} = \tt{A}_\exists \cup \tt{C}_\exists$, it implies that $\tt{C}_\top = \emptyset$.
Conversely $\tt{C}_\top = \emptyset$ and $\tt{A}_\top \subseteq \tt{C}_\exists \cup \tt{C}_\top$ suffices to satisfy the invariant.
One can now easily verify that $\tt{GCDR+}$ maintains the invariant.


\subsection{Examples of Instantiated GC Rules}
\label{sec:examples:instantiated}

\begin{example}[Select GC]
$\tt{iGCSelect}(\tt{A}_I; \tt{B}_I) = \sigma_{\neg P}(\tt{A}_I) \cup (\sigma_P(\tt{A}_I) \cap \tt{B}_I)$ with $\mathbb{W}(\tt{iGCProject}) = \{\tt{A}\}$ is weakly consistent with $\tt{GCSelect}$ on $\tt{A}$.
\end{example}
If $\tt{A}_I \subseteq \tt{A}_\exists \cup \tt{A}_\top$ and $\tt{B}_I \subseteq \tt{B}_\exists \cup \tt{B}_\top$, then
simple relational algebra demonstrates that
$\tt{iGCSelect}(\tt{A}_I; \tt{B}_I) = \sigma_{\neg P}(\tt{A}_I) \cup (\sigma_P(\tt{A}_I) \cap \tt{B}_I) \subseteq \sigma_{\neg P}(\tt{A}_\exists \cup \tt{A}_\top) \cup (\sigma_P(\tt{A}_\exists \cup \tt{A}_\top) \cap (\tt{B}_\exists \cup \tt{B}_\top)) = \tt{GCSelect}(\tt{A}_{TS}; \tt{B}_{TS})$.

\begin{example}[Linear GC]
$\tt{iGCAll}(\tt{A}_I;$ $\tt{B}_{1,E},$ $\dots,$ $\tt{B}_{m,E})$ $=$ $\tt{A}_I$ with $\mathbb{W}(\tt{iGCProject}) = \{\tt{A}\}$ is weakly consistent with $\tt{GCAll}$ on $\tt{A}$.
\end{example}

\begin{example}[Join with punctuations]
$\tt{iGCJoin}(\tt{A}_I,$ $\tt{P}_{\tt{A},I},$ $\tt{C}_I,$ $\tt{P}_{\tt{C},E};$ $\tt{B}_I)$ $=$ $\langle \tt{A}_I^-, \tt{P}_{\tt{A},I}^-, \tt{C}_I^-, \tt{P}_{\tt{C},I}^-\rangle$, where
\begin{align*}
\tt{A}_I^- &= \{a \in \tt{A}_I ~:~ \exists p_c \in \tt{P}_{\tt{C},I},
% \\&
\text{ such that } (a.k_a = p_c.k_c) \wedge \left(\{a\} \times p_c.\tt{C} \subseteq \tt{B}_I\right)\}\\
\tt{P}_\tt{A}^- &= \{p_a \in \tt{P}_{\tt{A},I} ~:~ \exists p_c \in \tt{P}_{\tt{C},I},
% \\&
\text{ such that } (p_a.k_a = p_c.k_c) \wedge \left(p_a.\tt{A} \times p_c.\tt{C} \subseteq \tt{B}_I\right)\}\\
\tt{C}_I^- &= \{c \in \tt{C}_I ~:~ \exists p_a \in \tt{P}_{\tt{A},I},
% \\&
\text{ such that } (p_a.k_a = c.k_c) \wedge \left(\{c\} \times p_a.\tt{A} \subseteq \tt{B}_I\right)\}\\
\tt{P}_\tt{C}^- &= \{p_c \in \tt{P}_{\tt{C},I} ~:~ \exists p_a \in \tt{P}_{\tt{A},I},
% \\&
\text{ such that } (p_a.k_a = p_c.k_c) \wedge \left(p_a.\tt{A} \times p_c.\tt{C} \subseteq \tt{B}_I\right)\}
\end{align*}
with $\mathbb{W}(\tt{iGCJoin}) = \{\tt{A}, \tt{P}_\tt{A}, \tt{C}, \tt{P}_\tt{A}\}$ is weakly consistent with $\tt{GCJoin}$ on $\tt{A}$, $\tt{P}_\tt{A}$, $\tt{C}$ and $\tt{P}_\tt{C}$.
\end{example}
Proof of consistency of $\tt{iGCJoin}$ with $\tt{GCJoin}$ is straightforward and thus omitted.
As noted before, $p_a \in \tt{P}_{\tt{A},!} - \tt{P}_{\tt{A},I} \implies p_a \in \tt{P}_{\tt{A},\top} \implies \sigma_{\tt{A}.k_a = p_a.k_a}(\tt{A}_\exists) = \emptyset$, so we can also tombstone tuples whose keys match $\tt{P}_{\tt{A},!} - \tt{P}_{\tt{A},I}$:

\begin{example}[Join with punctuations, version 2]
$\tt{iGCJoinV2}(\tt{A}_I, \tt{P}_{\tt{A},I}, \tt{C}_I, \tt{P}_{\tt{C},E}; \tt{B}_I) = \langle \tt{A}_I^-, \tt{P}_{\tt{A},E}^-, \tt{C}_I^-, \tt{P}_{\tt{C},E}^-\rangle$, where
\begin{align*}
\tt{A}_I^- &= \{a \in \tt{A}_I ~:~ \exists p_c \in \tt{P}_{\tt{C},I}, \text{ such that } (a.k_a = p_c.k_c)
% \\&
\wedge \left(\{a\} \times p_c.\tt{C} \subseteq \tt{B}_I\right)\} \cup \{a \in \tt{A}_I ~:~ a.k_a \in \tt{P}_{\tt{A},!} - \pi_!(\tt{P}_{\tt{A},I})\}\\
\tt{P}_\tt{A}^- &= \{p_a \in \tt{P}_{\tt{A},I} ~:~ \exists p_c \in \tt{P}_{\tt{C},I}, \text{ such that } (p_a.k_a = p_c.k_c)
% \\&
\wedge \left(p_a.\tt{A} \times p_c.\tt{C} \subseteq \tt{B}_I\right)\}\\
\tt{C}_I^- &= \{c \in \tt{C}_I ~:~ \exists p_a \in \tt{P}_{\tt{A},I}, \text{ such that } (p_a.k_a = c.k_c)
% \\&
\wedge \left(\{c\} \times p_a.\tt{A} \subseteq \tt{B}_I\right)\} \cup \{c \in \tt{C}_I ~:~ c.k_c \in \tt{P}_{\tt{C},!} - \pi_!(\tt{P}_{\tt{C},I})\}\\
\tt{P}_\tt{C}^- &= \{p_c \in \tt{P}_{\tt{C},I} ~:~ \exists p_a \in \tt{P}_{\tt{A},I}, \text{ such that } (p_a.k_a = p_c.k_c)
% \\&
\wedge \left(p_a.\tt{A} \times p_c.\tt{C} \subseteq \tt{B}_I\right)\}
\end{align*}
with $\mathbb{W}(\tt{iGCJoin}) = \{\tt{A}, \tt{P}_\tt{A}, \tt{C}, \tt{P}_\tt{A}\}$ is weakly consistent with $\tt{GCJoin}$ on $\tt{A}$, $\tt{P}_\tt{A}$, $\tt{C}$ and $\tt{P}_\tt{C}$.
\end{example}


\begin{example}[DR+: Positive Difference]
$\tt{iGCDR+}(\tt{A}_I, \tt{C}_I; \tt{B}_I) = \langle \tt{A}_I \cap \tt{C}, \tt{C}_I\rangle$ with $\mathbb{W}(\tt{iGCDR+}) = \{\tt{A}, \tt{C}\}$ is weakly consistent with $\tt{GCDR+}$ on both $\tt{A}$ and $\tt{C}$.
\end{example}




%!TEX root = bloom_logical.tex

\section{Proofs}
\label{sec:proofs}


\subsection{Proofs for Section \ref{sec:logical}}
\subsubsection{Lemma \ref{lem:merge_gc_inv}}
\begin{lemma}[Lemma \ref{lem:merge_gc_inv} restated]
\lemmergegcinv
\end{lemma}

\begin{proof}
We prove the lemma for any given $R$ by an induction on $t$ and $i$.
That is, we show for every $t$ and $i$,
$\forall \langle \widehat{\tt{A}}_1, \dots, \widehat{\tt{A}}_n\rangle \geq \langle \tt{A}_{1,\exists,t,i},\dots,\tt{A}_{n,\exists,t,i} \rangle$,
$\forall \langle \widehat{\tt{U}}_1, \dots, \widehat{\tt{U}}_n\rangle \geq \langle \tt{U}_{1,\exists,t,i},\dots,\tt{U}_{n,\exists,t,i} \rangle$,
such that
$\emptyset = \widehat{\tt{U}}_1 \cap \tt{U}_{1,\top,t,i} = \dots =  \widehat{\tt{U}}_n \cap \tt{U}_{n,\top,t,i}$,
\begin{align}
&
\tt{B}_{\exists,t,i} \cup \tt{B}_{\top,t}
\cup
\tt{f(}
  \widehat{\tt{A}}_1,
    \dots,
    \widehat{\tt{A}}_n,
  \widehat{\tt{U}}_1,
    \dots,
    \widehat{\tt{U}}_n
\tt{)}
\nonumber\\
=~&
\tt{B}_{\exists,t,i} \cup \tt{B}_{\top,t}
\cup
\tt{f(}
  \widehat{\tt{A}}_1 \cup \tt{A}_{1,\top,t},
  \dots,
  \widehat{\tt{A}}_n \cup \tt{A}_{n,\top,t},
  \widehat{\tt{U}}_1 \cup \tt{U}_{1,\top,t},
  \dots,
  \widehat{\tt{U}}_n \cup \tt{U}_{n,\top,t}
\tt{)}. \label{eq:mergegcinv_withtime}
\end{align}
At $t=0$, $i=0$, we have $\tt{A}_{k,\top,0} = \tt{U}_{j,\top,0} = \tt{B}_{\top,0} = \emptyset$ for all $j=1,\dots,m$ and $k=1,\dots,n$, so the statement \eqref{eq:mergegcinv_withtime} is trivially true.

Assume that \eqref{eq:mergegcinv_withtime} is true at some $t$ and $i-1$.
The assignment at $t$, $i$, is derived from applying a merge operator to the assignment at $t$, $i-1$, and may only grow $\tt{A}_{1,\exists}, \dots, \tt{A}_{n,\exists}$, $\tt{U}_{1,\exists}, \dots, \tt{U}_{m,\exists}$, and $\tt{B}_\exists$,
but keeps $\tt{A}_{1,\top}, \dots, \tt{A}_{n,\top}$, $\tt{U}_{1,\top}, \dots, \tt{U}_{m,\top}$, and $\tt{B}_\top$ unchanged.
This also implies that $\widehat{\tt{U}}_k \cap \tt{U}_{k,\top,t,i-1} = \emptyset \implies \widehat{\tt{U}}_k \cap \tt{U}_{k,\top,t,i-1} = \emptyset$.
Write $\widetilde{\tt{B}} = \tt{B}_{\exists,t,i} - \tt{B}_{\exists,t,i-1}$.
Then, $\forall \langle \widehat{\tt{A}}_1, \dots, \widehat{\tt{A}}_n\rangle \geq \langle \tt{A}_{1,\exists,t,i},\dots,\tt{A}_{n,\exists,t,i} \rangle \geq \langle \tt{A}_{1,\exists,t,i-1},\dots,\tt{A}_{n,\exists,t,i-1} \rangle$,
$\forall \langle \widehat{\tt{U}}_1, \dots, \widehat{\tt{U}}_n\rangle \geq \langle \tt{U}_{1,\exists,t,i},\dots,\tt{U}_{n,\exists,t,i} \rangle \geq \langle \tt{U}_{1,\exists,t,i-1},\dots,\tt{U}_{n,\exists,t,i-1} \rangle$,
such that
$\forall j=1,\dots,m$, $\widehat{\tt{U}}_j \cap \tt{U}_{1,\top,t} = \emptyset$,
\begin{align*}
&\tt{B}_{\exists,t,i} \cup \tt{B}_{\top,t}
\cup
\tt{f(}
  \widehat{\tt{A}}_1,
    \dots,
    \widehat{\tt{A}}_n,
  \widehat{\tt{U}}_1,
    \dots,
    \widehat{\tt{U}}_n
\tt{)}\\
=~&
\widetilde{\tt{B}} \cup \tt{B}_{\exists,t,i-1} \cup \tt{B}_{\top,t}
\cup
\tt{f(}
  \widehat{\tt{A}}_1,
    \dots,
    \widehat{\tt{A}}_n,
  \widehat{\tt{U}}_1,
    \dots,
    \widehat{\tt{U}}_n
\tt{)}\\
=~&
\widetilde{\tt{B}} \cup \tt{B}_{\exists,t,i-1} \cup \tt{B}_{\top,t}
\cup
\tt{f(}
  \widehat{\tt{A}}_1 \cup \tt{A}_{1,\top,t},
  \dots,
  \widehat{\tt{A}}_n \cup \tt{A}_{n,\top,t},
  \widehat{\tt{U}}_1 \cup \tt{U}_{1,\top,t},
  \dots,
  \widehat{\tt{U}}_n \cup \tt{U}_{n,\top,t}
\tt{)}\\
=~&
\tt{B}_{\exists,t,i} \cup \tt{B}_{\top,t}
\cup
\tt{f(}
  \widehat{\tt{A}}_1 \cup \tt{A}_{1,\top,t},
  \dots,
  \widehat{\tt{A}}_n \cup \tt{A}_{n,\top,t},
  \widehat{\tt{U}}_1 \cup \tt{U}_{1,\top,t},
  \dots,
  \widehat{\tt{U}}_n \cup \tt{U}_{n,\top,t}
\tt{)}
\end{align*}
where the second equality follows from the inductive hypothesis.

Next, we show that if \eqref{eq:mergegcinv_withtime} is true at some fixed point $t-1$ and $i=\iota$, then it is also true at $t$, $i=0$.
By design / assumption, there is only one garbage collection rule $\tt{GC}$ in $\mathfrak{P}_{GC}$ that has any of $\tt{A}_{1,TS},\dots,\tt{A}_{n,TS}$ on its LHS.
Note that we can express $\forall j=1,\dots,n$,
\begin{align*}
(\tt{A}_{j,\exists,t,0}, \tt{A}_{j,\top,t})
= (\tt{A}_{j,\exists,t-1,\iota}, \tt{A}_{j,\top,t-1})
\cup (\tt{A}_{j,\exists}^*, \tt{A}_{j,\top}^*)
\cup \bigcup_{k=1}^l (g_k(\tt{C}_{k1,TS,t-1,\iota}, \dots, \tt{C}_{kn_k,TS,t-1,\iota}), \emptyset)
\end{align*}
where $\langle \tt{A}_{1,TS}^*, \dots, \tt{A}_{n,TS}^*\rangle = \tt{GC}(\tt{A}_{1,TS,t-1,\iota}, \dots, \tt{A}_{n,TS,t-1,\iota}; \tt{B}_{1,TS,t-1,\iota}, \dots, \tt{B}_{m,TS,t-1,\iota})$.
By the conservative property of $\tt{GC}$, $(\tt{A}_{j,\exists,t-1,\iota}, \tt{A}_{j,\top,t-1}) \leq (\tt{A}_{j,\exists}^*, \tt{A}_{j,\top}^*)$, so the above equation can be written as
\begin{align*}
(\tt{A}_{j,\exists,t,0}, \tt{A}_{j,\top,t})
= (\tt{A}_{j,\exists}^*, \tt{A}_{j,\top}^*)
\cup \bigcup_{k=1}^l (g_k(\tt{C}_{k1,TS,t-1,\iota}, \dots, \tt{C}_{kn_k,TS,t-1,\iota}), \emptyset)
\end{align*}
which implies $\tt{A}_{j,\exists,t,0} \supseteq \tt{A}_{j,\exists}^*$ and $\tt{A}_{j,\top,t} = \tt{A}_{j,\top}^*$.

The same holds for $\tt{U}_j$ for $j=1,\dots,m$, which further implies that $\widehat{\tt{U}}_j \cap \tt{U}_{j,\top}^* = \emptyset \implies \widehat{\tt{U}}_j \cap \tt{U}_{j,\top,t} = \emptyset$.

Also, by the monotonicity of $\tt{<+}$, we see that $\tt{B}_{\exists,t,0} \cup \tt{B}_{\top,t} \supseteq \tt{B}_{\exists,t-1,\iota} \cup \tt{B}_{\top,t-1}$.
Let $\widetilde{\tt{B}} = (\tt{B}_{\exists,t,0} \cup \tt{B}_{\top,t}) - (\tt{B}_{\exists,t-1,\iota} \cup \tt{B}_{\top,t-1})$.
Note that
$\tt{B}_{\exists,t,0} \cup \tt{B}_{\top,t} \supseteq \tt{f}\tt{(}\tt{A}_{\exists,1,t-1,\iota} \cup \tt{A}_{1,\top,t-1},\dots,\tt{A}_{n,\exists,t-1,\iota} \cup \tt{A}_{n,\top,t-1}, \tt{U}_{1,\exists,t-1,\iota} \cup \tt{U}_{1,\top,t-1},\dots,\tt{U}_{n,\exists,t-1,\iota} \cup \tt{U}_{n,\top,t-1}\tt{)}$ ---
if $R$ is a merge rule, it is true because $t,\iota$ is a fixed point;
if $R$ is a deferred merge rule, then the inclusion holds by definition of $\tt{B}_{TS,t,0}$.

Observe that $\forall \langle \widehat{\tt{A}}_1, \dots, \widehat{\tt{A}}_n \rangle$ $\geq$ $\langle \tt{A}_{1,\exists,t,0}, \dots, \tt{A}_{n,\exists,t,0} \rangle$ $\geq$ $\langle \tt{A}_{1,\exists}^*, \dots, \tt{A}_{n,\exists}^* \rangle$,
$\forall \langle \widehat{\tt{U}}_1, \dots, \widehat{\tt{U}}_n \rangle$ $\geq$ $\langle \tt{U}_{1,\exists,t,0}, \dots, \tt{U}_{n,\exists,t,0} \rangle$ $\geq$ $\langle \tt{U}_{1,\exists}^*, \dots, \tt{U}_{n,\exists}^* \rangle$,
such that $\emptyset = \widetilde{\tt{U}}_j \cap \tt{U}_{j,\top}^* = \widetilde{\tt{U}}_{j,\top,t}$ for all $j = 1,\dots,m$,
\begin{align*}
&
\tt{B}_{\exists,t,0} \cup \tt{B}_{\top,t}
\cup \tt{f}(\widehat{\tt{A}}_1, \dots, \widehat{\tt{A}}_n, \widehat{\tt{U}}_1, \dots, \widehat{\tt{U}}_m)
\\
=&~
\widetilde{\tt{B}} \cup \tt{B}_{\exists,t-1,\iota} \cup \tt{B}_{\top,t-1}
\cup \tt{f}(\widehat{\tt{A}}_1, \dots, \widehat{\tt{A}}_n, \widehat{\tt{U}}_1, \dots, \widehat{\tt{U}}_m)
\\
=&~
\widetilde{\tt{B}} \cup \tt{B}_{\exists,t-1,\iota} \cup \tt{B}_{\top,t-1}
\cup \tt{f}(
\widehat{\tt{A}}_1 \cup \tt{A}_{1,\top}^*, \dots, \widehat{\tt{A}}_n \cup \tt{A}_{n,\top}^*,
\widehat{\tt{U}}_1 \cup \tt{U}_{1,\top}^*, \dots, \widehat{\tt{U}}_n \cup \tt{U}_{n,\top}^*)
&\text{(\tt{GC} safety)}
\\
=&~
\widetilde{\tt{B}} \cup \tt{B}_{\exists,t-1,\iota} \cup \tt{B}_{\top,t-1}
\cup \tt{f}(
\widehat{\tt{A}}_1 \cup \tt{A}_{1,\top,t}, \dots, \widehat{\tt{A}}_n \cup \tt{A}_{n,\top,t},
\widehat{\tt{U}}_1 \cup \tt{U}_{1,\top,t}, \dots, \widehat{\tt{U}}_n \cup \tt{U}_{n,\top,t})
\\
=&~
\tt{B}_{\exists,t,0} \cup \tt{B}_{\top,t}
\cup \tt{f}(
\widehat{\tt{A}}_1 \cup \tt{A}_{1,\top,t}, \dots, \widehat{\tt{A}}_n \cup \tt{A}_{n,\top,t},
\widehat{\tt{U}}_1 \cup \tt{U}_{1,\top,t}, \dots, \widehat{\tt{U}}_n \cup \tt{U}_{n,\top,t})
\end{align*}
which establishes that the Merge GC Invariant \ref{inv:merge_gc} holds at $t$ and $i=0$.
\end{proof}



\subsubsection{Theorem \ref{thm:equivalence_of_sets}}
\begin{thm}[Theorem \ref{thm:equivalence_of_sets} restated]
\thmequivalenceofsets
\end{thm}

\begin{proof}
We will again prove the theorem via induction.

Clearly this is true for $t=i=0$.

Now suppose that it is true at some $t$ and $i-1$, and we will show it holds at $t$ and $i$.
During the instantaneous run, the only rules that we are concerned with are merges where $\tt{A}$ appears on the LHS.
If the rule is of the form $\tt{A} ~\tt{<=}~ \overline{\tt{C}}$ with a channel on the RHS, then
\begin{align*}
\tt{A}_{t,i}
= \tt{A}_{t,i-1} \cup \overline{\tt{C}}_{t}
= \tt{A}_{\exists,t,i-1} \cup \tt{A}_{\top,t} \cup \overline{\tt{C}}_{t}
= \tt{A}_{\exists,t,i} \cup \tt{A}_{\top,t}.
\end{align*}
If the merge rule is of the form $\tt{A} ~\tt{<=}~ \tt{f}\tt{(}\tt{C}_1,\dots,\tt{C}_n\tt{)}$, then
\begin{align*}
\tt{A}_{t,i}
&= \tt{A}_{t,i-1} \cup \tt{f}\tt{(}\tt{C}_{1,t,i-1},\dots,\tt{C}_{n,t,i-1}\tt{)}\\
&= \tt{A}_{\exists,t,i-1} \cup \tt{A}_{\top,t} \cup \tt{f}\tt{(}\tt{C}_{1,\exists,t,i-1}\cup\tt{C}_{1,\top,t},\dots,\tt{C}_{n,\exists,t,i-1}\cup\tt{C}_{n,\top,t}\tt{)}\\
&= \tt{A}_{\exists,t,i} \cup \tt{A}_{\top,t},
\end{align*}
where we used the inductive hypothesis in the second equality.

Suppose $\mathfrak{P}$ reaches a fixed point for time $t$ at $i=\iota$.
We will next show that $\tt{A}_{t-1,\iota} = \tt{A}_{\exists,t-1,\iota} \cup \tt{A}_{\top,t-1}$ implies $\tt{A}_{t,0} = \tt{A}_{\exists,t,0} \cup \tt{A}_{\top,t}$.
We must first establish that the rewritten program will reach a fixed point for timestep $t$.

\begin{claim}
If $\mathfrak{P}$ reaches a fixed point at $t$, $\mathfrak{P}_{GC}$ also reaches a fixed point.
(This statement should be made more formal.)
\end{claim}
\begin{claimproof}
From the previous argument, we know that $\tt{A}_{\exists,t,j} \cup \tt{A}_{\top,t,i} = \tt{A}_{t,j}$ for all $j \geq \iota$.
But $\tt{A}_{t,j} = \tt{A}_{t,\iota}$ and $\tt{A}_{\top,t,i} = \tt{A}_{\top,t,0}$ are constant independent of $j$, which implies $\tt{A}_{\exists,t,j} = \tt{A}_{\exists,t,\iota}$.
Hence, $\mathfrak{P}_{GC}$ also reaches at fixed point at $t$, $i=\iota$.
\end{claimproof}

Finally, suppose we have $\tt{A}_{t-1,\iota} = \tt{A}_{\exists,t-1,\iota} \cup \tt{A}_{\top,t-1}$.
Suppose $\tt{A}$ appears in the LHS of $m$ deferred merge rules in $\mathfrak{P}$: $\tt{A}$ $\tt{<+}$ $\tt{f}_j\tt{(}\tt{C}_{j,1},\dots,\tt{C}_{j,n_j}\tt{)}$.
Additionally, in $\mathfrak{P}_{GC}$, $\tt{A}_{TS}$ appears in (at most) one GC rule $\langle \tt{A}_{TS}, \tt{D}_{1,TS}, \dots, \tt{D}_{k,TS} \rangle$ $\tt{<+}$ $\tt{GC}(\tt{A}_{TS}, \tt{D}_{1,TS}, \dots,$ $\tt{D}_{k,TS};$ $\tt{B}_{1,TS}, \dots, \tt{B}_{k',TS})$.
Let $\langle \tt{A}_{TS}^*, \tt{D}_{1,TS}^*, \dots, \tt{D}_{k,TS}^* \rangle = \tt{GC}(\tt{A}_{TS,t-1,\iota}, \tt{D}_{1,TS,t-1,\iota}, \dots, \tt{D}_{k,TS,t-1,\iota};$ $\tt{B}_{1,TS,t-1,\iota}, \dots, \tt{B}_{k',TS,t-1,\iota})$, and note that $\tt{A}_{\exists,t-1,\iota} \cup \tt{A}_{\top,t-1} = \tt{A}_{\exists}^* \cup \tt{A}_{\top}^*$ by the conservation property of $\tt{GC}$.
Then,
\begin{align*}
\tt{A}_{t,0}
=~& \tt{A}_{t-1,\iota} \cup \bigcup_{j=1}^m \tt{f}_j\tt{(}\tt{C}_{j,1,t-1,\iota},\dots,\tt{C}_{j,n_j,t-1,\iota}\tt{)}\\
=~& \tt{A}_{\exists,t-1,\iota} \cup \tt{A}_{\top,t-1}
\cup \bigcup_{j=1}^m \tt{f}_j\tt{(}\tt{C}_{j,1,\exists,t-1,\iota} \cup \tt{C}_{j,1,\top,t-1}, \dots, \tt{C}_{j,n_j,\exists,t-1,\iota} \cup \tt{C}_{j,n_j,\top,t-1}\tt{)}\\
=~& \tt{A}_{\exists,t-1,\iota} \cup \tt{A}_{\top,t-1}
\cup \tt{A}_{\exists}^* \cup \tt{A}_{\top}^*
\cup \bigcup_{j=1}^m \tt{f}_j\tt{(}\tt{C}_{j,1,\exists,t-1,\iota} \cup \tt{C}_{j,1,\top,t-1}, \dots, \tt{C}_{j,n_j,\exists,t-1,\iota} \cup \tt{C}_{j,n_j,\top,t-1}\tt{)}\\
=~& \tt{A}_{\exists,t,0} \cup \tt{A}_{\top,t}.
\end{align*}
Thus, we have established that $\tt{A}_{t-1,\iota} = \tt{A}_{\exists,t-1,\iota} \cup \tt{A}_{\top,t-1}$ implies $\tt{A}_{t,0} = \tt{A}_{\exists,t,0} \cup \tt{A}_{\top,t}$.
\end{proof}



\subsubsection{Theorem \ref{thm:coord-free}}
\begin{thm}[Theorem \ref{thm:coord-free} restated]
\thmcoordfree
\end{thm}

\begin{proof}
Rules in $\mathfrak{P}$ either correspond to rules in $\mathfrak{P}$, or are new garbage collection rules.
In the first case, we replace the rule $R: \tt{B} ~\tt{<op>}~ \tt{f}(\tt{A}_1,\dots,\tt{A}_n\tt{)}$ with $(\tt{B}_\exists, \tt{B}_\top) ~\tt{<op>}~ (\tt{f}(\tt{A}_{1,\exists} \cup \tt{A}_{1,\top}, \dots, \tt{A}_{n,\exists} \cup \tt{A}_{n,\top} \tt{)}$, which requires a mapping $(\tt{X}_\exists, \tt{X}_\top) \mapsto \tt{X}_\exists \cup \tt{X}_\top$ and the projection $\tt{X} \mapsto (\tt{X}, \emptyset)$, both of which are monotone.
Hence, the corresponding rule in $\mathfrak{P}$ is monotone if $\tt{f}$ is monotone.
In the second case, our $\tt{GC}$ rules are monotone, and involve a deferred merge, and thus do not require additional coordination.

Suppose $\mathfrak{P}$ is confluent in all sets.
Theorem \ref{thm:equivalence_of_sets} implies $\tt{A}_{\exists} \cup \tt{A}_\top = \tt{A}$ is confluent in $\mathfrak{P}_{GC}$.
The only rules in $\mathfrak{P}_{GC}$ that promote tuples from $\exists$ to $\top$ are GC rules; since these rules are monotone, the tombstone lattices are also confluent.
\end{proof}


\subsection{Proofs for Section \ref{sec:instantiated}}

\subsubsection{Theorem \ref{thm:instantiated_correctness}}
\begin{thm}[Theorem \ref{thm:instantiated_correctness} restated]
\thminstantiatedcorrectness
\end{thm}

\begin{proof}
Our proof proceeds by induction on $t$ and $i$.

At $t=0$, $i=0$, we initialize $\tt{A}_{I,0,0} = \tt{A}_{0,0} = \tt{A}_{\exists,0,0} = \tt{A}_{\exists,0,0} \cup \tt{A}_{\top,0}$, so \eqref{eq:thm_instantiated_correctness_set} and \eqref{eq:thm_instantiated_correctness_key} hold trivially.

Assume that \eqref{eq:thm_instantiated_correctness_set} and \eqref{eq:thm_instantiated_correctness_key} hold at some $t$ and $i-1$.
The assignment at $t$, $i$ is obtained by executing some merge rule $R$ on the assignment at $t$, $i-1$.
If $\tt{A}$ does not appear on the LHS of $R$, then we are done because $\tt{A}_{E,t,i} = \tt{A}_{E,t,i-1}$ and $\tt{A}_{TS,t,i} = \tt{A}_{TS,t,i-1}$.
If $\tt{A}$ appears on the LHS of $R$ and the RHS is not a channel, then WLOG let $R: \tt{A} ~\tt{<=}~ \tt{f}\tt{(}\tt{C}_1, \dots, \tt{C}_n\tt{)}$.
\begin{align}
\tt{A}_{\exists,t,i}
=& (\tt{A}_{\exists,t,i-1} \cup \tt{f}\tt{(}\tt{C}_{1,\exists,t,i-1} \cup \tt{C}_{1,\top,t}, \dots, \tt{C}_{n,\exists,t,i-1} \cup \tt{C}_{n,\top,t}\tt{)}) - \tt{A}_{\top,t}\nonumber\\
=& (\tt{A}_{\exists,t,i-1} \cup \tt{f}\tt{(}\tt{C}_{1,I,t,i-1} \cup \tt{C}_{1,\top,t}, \dots, \tt{C}_{n,I,t,i-1} \cup \tt{C}_{n,\top,t}\tt{)} \cup \tt{A}_{\top,t}) - \tt{A}_{\top,t}\nonumber\\
=& (\tt{A}_{\exists,t,i-1} \cup \tt{f}\tt{(}\tt{C}_{1,I,t,i-1}, \dots, \tt{C}_{n,I,t,i-1}\tt{)} \cup \tt{A}_{\top,t}) - \tt{A}_{\top,t} & \text{(Lemma \ref{lem:merge_gc_inv})} \label{eq:thm:ins_correct:invariant1}\\
=& (\tt{A}_{\exists,t,i-1} \cup \tt{f}\tt{(}\tt{C}_{1,I,t,i-1}, \dots, \tt{C}_{n,I,t,i-1}\tt{)}) - \tt{A}_{\top,t}.\label{eq:thm:ins_correct:instant}
\end{align}
We were able to apply Lemma \ref{lem:merge_gc_inv} in \eqref{eq:thm:ins_correct:invariant1} because by the inductive hypothesis, $\tt{C}_{j,I,t,i-1} \supseteq \tt{C}_{j,\exists,t,i-1}$, and if $\tt{C}_j \in \mathbb{S}$ then $\tt{C}_{j,I,t,i-1} \cap \tt{C}_{j,\top,t} = \tt{C}_{j,\exists,t,i-1} \cap \tt{C}_{j,\top,t} = \emptyset$.

If $\tt{A}$ is non-augmented, we apply the inductive hypothesis to \eqref{eq:thm:ins_correct:instant},
\begin{align*}
\tt{A}_{\exists,t,i}
&&\subseteq&& \tt{A}_{\exists,t,i-1} \cup \tt{f}\tt{(}\tt{C}_{1,I,t,i-1}, \dots, \tt{C}_{n,I,t,i-1}\tt{)}
&&\subseteq&& \tt{A}_{I,t,i-1} \cup \tt{f}\tt{(}\tt{C}_{1,I,t,i-1}, \dots, \tt{C}_{n,I,t,i-1}\tt{)}
&&=&& \tt{A}_{I,t,i}.
\end{align*}

If $\tt{A}$ is key-augmented and $\tt{A} \in \mathbb{W}$, we apply the inductive hypothesis to \eqref{eq:thm:ins_correct:instant},
\begin{align*}
\tt{A}_{\exists,t,i}
&&\subseteq&& (\tt{A}_{I,t,i-1} \cup \tt{f}\tt{(}\tt{C}_{1,I,t,i-1}, \dots, \tt{C}_{n,I,t,i-1}\tt{)}) - (\tt{A}_{!,t,i-1} - \tt{A}_{I,t,i-1})
&&=&& \tt{A}_{I,t,i}.
\end{align*}

If $\tt{A}$ is key-augmented and $\tt{A} \in \mathbb{S}$, we apply the inductive hypothesis to \eqref{eq:thm:ins_correct:instant},
\begin{align*}
\tt{A}_{\exists,t,i}
&&=&& (\tt{A}_{I,t,i-1} \cup \tt{f}\tt{(}\tt{C}_{1,I,t,i-1}, \dots, \tt{C}_{n,I,t,i-1}\tt{)}) - (\tt{A}_{!,t,i-1} - \tt{A}_{I,t,i-1})
&&=&& \tt{A}_{I,t,i}.
\end{align*}

In all cases, we see that
\begin{align*}
\tt{A}_{I,t,i}
\subseteq& \tt{A}_{I,t,i-1} \cup \tt{f}\tt{(}\tt{C}_{1,I,t,i-1}, \dots, \tt{C}_{n,I,t,i-1}\tt{)}\\
\subseteq& \tt{A}_{\exists,t,i-1} \cup \tt{A}_{\top,t} \cup \tt{f}\tt{(}\tt{C}_{1,I,t,i-1}, \dots, \tt{C}_{n,I,t,i-1}\tt{)} & \text{(I.H.)}\\
=& \tt{A}_{\exists,t,i-1} \cup \tt{A}_{\top,t} \cup \tt{f}\tt{(}\tt{C}_{1,\exists,t,i-1} \cup \tt{C}_{1,\top,t}, \dots, \tt{C}_{n,\exists,t,i-1} \cup \tt{C}_{n,\top,t}\tt{)} & \text{(Lemma \ref{lem:merge_gc_inv})}\\
=& \tt{A}_{\exists,t,i} \cup \tt{A}_{\top,t}
\end{align*}
For key-augmented $\tt{A}$, we also have
\begin{align*}
\tt{A}_{!,t,i}
&= \tt{A}_{!,t,i-1} \cup \pi_!(\tt{f}\tt{(}\tt{C}_{1,I,t,i-1}, \dots, \tt{C}_{n,I,t,i-1}\tt{)})\\
&= \pi_!(\tt{A}_{\exists,t,i-1} \cup \tt{A}_{\top,t}) \cup \pi_!(\tt{f}\tt{(}\tt{C}_{1,I,t,i-1}, \dots, \tt{C}_{n,I,t,i-1}\tt{)})\\
&= \pi_!(\tt{A}_{\exists,t,i-1} \cup \tt{A}_{\top,t} \cup \tt{f}\tt{(}\tt{C}_{1,I,t,i-1}, \dots, \tt{C}_{n,I,t,i-1}\tt{)})\\
&= \pi_!(\tt{A}_{\exists,t,i-1} \cup \tt{A}_{\top,t} \cup \tt{f}\tt{(}\tt{C}_{1,\exists,t,i-1} \cup \tt{C}_{1,\top,t}, \dots, \tt{C}_{n,\exists,t,i-1} \cup \tt{C}_{n,\top,t}\tt{)}) & \text{(Lemma \ref{lem:merge_gc_inv})}\\
&= \pi_!(\tt{A}_{\exists,t,i} \cup \tt{A}_{\top,t}).
\end{align*}
Otherwise, $\tt{A}$ appears on the LHS of $R$ and the RHS of $R$ is a channel $\overline{\tt{C}}$.
The argument above goes through by replacing $\tt{f}$ with $\overline{\tt{C}}$.

Suppose $\mathfrak{P}_{GC}$ reaches a finite fixed point for time $t$ at $i=\iota$.
We first establish that $\mathfrak{P}_{iGC}$ also eventually reaches a fixed point for $t$.
\begin{claim}
If $\mathfrak{P}_{GC}$ reaches a finite fixed point at $t$, $\iota$, then $\mathfrak{P}_{iGC}$ also eventually reaches a fixed point.
\end{claim}
\begin{claimproof}
By the above argument, $\tt{A}_{I,t,j} \subseteq \tt{A}_{\exists,t,j} \cup \tt{A}_{\top,t} = \tt{A}_{\exists,t,\iota} \cup \tt{A}_{\top,t}$ for all $j \geq \iota$.
Since $\tt{<=}$ can only increase $\tt{A}_{I,t,j}$ and $\tt{A}_{I,t,j}$ is upper bounded, $\mathfrak{P}_{iGC}$ must also reach a fixed point for $t$ after a finite number of steps after $\iota$.
\end{claimproof}

Finally, suppose \eqref{eq:thm_instantiated_correctness_set} and \eqref{eq:thm_instantiated_correctness_key} hold at $t-1$, $\iota$ which is a fixed point of $\mathfrak{P}_{iGC}$ for time $t-1$.
Suppose $\tt{A}$ appears in the LHS of $m$ deferred merge rules in $\mathfrak{P}$: $\tt{A} ~\tt{<+}~ \tt{f}_j\tt{(}\tt{C}_{j,1},\dots,\tt{C}_{j,n_j}\tt{)}$.
Additionally, in $\mathfrak{P}_{GC}$, $\tt{A}_{TS}$ appears in (at most) one GC rule $\langle \tt{A}_{TS}, \tt{D}_{1,TS}, \dots, \tt{D}_{k,TS} \rangle$ $~\tt{<+}~$ $\tt{GC}(\tt{A}_{TS}, \tt{D}_{1,TS}, \dots, \tt{D}_{k,TS}; \tt{B}_{1,TS}, \dots, \tt{B}_{k',TS})$.
Correspondingly, $\tt{A}_E$ appears in (at most) one instantiated GC rule $\langle \tt{A}_E, \tt{D}_{1,E}, \dots, \tt{D}_{k,E} \rangle$ $~\tt{<-}~$ $\tt{GCI}(\tt{A}_{E},$ $\tt{D}_{1,E}, \dots, \tt{D}_{k,E}; \tt{B}_{1,E}, \dots, \tt{B}_{k',E})$.
Let $\langle \tt{A}_{TS}^*, \tt{D}_{1,TS}^*, \dots, \tt{D}_{k,TS}^* \rangle$ $= \tt{GC}(\tt{A}_{TS,t-1,\iota}, \tt{D}_{1,TS,t-1,\iota}, \dots, \tt{D}_{k,TS,t-1,\iota};$ $\tt{B}_{1,TS,t-1,\iota}, \dots, \tt{B}_{k',TS,t-1,\iota})$, and
$\langle \tt{A}_E^-, \tt{D}_{1,E}^-, \dots, \tt{D}_{k,E}^- \rangle$ $=$ $\tt{GCI}(\tt{A}_{E,t-1,\iota}, \tt{D}_{1,E,t-1,\iota}, \dots, \tt{D}_{k,E,t-1,\iota};$ $\tt{B}_{1,E,t-1,\iota},$ $\dots, \tt{B}_{k',E,t-1,\iota})$.
By the conservative property of $\tt{GC}$, we have that $\tt{A}_{\exists}^* \subseteq \tt{A}_{\exists,t-1,\iota}$, $\tt{A}_{\top}^* \supseteq \tt{A}_{\top,t-1}$, and $\tt{A}_{\exists}^* \cup \tt{A}_{\top}^* = \tt{A}_{\exists,t-1,\iota} \cup \tt{A}_{\top,t-1,}$.
Furthermore, we have that $\tt{A}_I^- \subseteq \tt{A}_{T}^*$ if $\tt{iGC}$ is weakly consistent with $\tt{GC}$ on $\tt{A}$.
Hence,
\begin{align}
&\tt{A}_{\exists,t,0}\nonumber\\
=& \left(\tt{A}_{\exists,t-1,\iota} \cup \tt{A}_\exists^* \cup \bigcup_{j=1}^m \tt{f}_j\tt{(}\tt{C}_{j,1,\exists,t-1,\iota} \cup \tt{C}_{j,1,\top,t-1}, \dots, \tt{C}_{j,n_j,\exists,t-1,\iota} \cup \tt{C}_{j,n_j,\top,t-1}\tt{)}\right)
- \tt{A}_{\top,t-1} - \tt{A}_\top^*\nonumber\\
=& \left(\tt{A}_{\exists,t-1,\iota} \cup \bigcup_{j=1}^m \tt{f}_j\tt{(}\tt{C}_{j,1,\exists,t-1,\iota} \cup \tt{C}_{j,1,\top,t-1}, \dots, \tt{C}_{j,n_j,\exists,t-1,\iota} \cup \tt{C}_{j,n_j,\top,t-1}\tt{)}\right)
- \tt{A}_{\top,t-1} - \tt{A}_\top^* & \text{(Property \ref{property:gc_conservative})}\nonumber\\
=& \left(\tt{A}_{\exists,t-1,\iota} \cup \tt{A}_{\top,t-1} \cup \bigcup_{j=1}^m \tt{f}_j\tt{(}\tt{C}_{j,1,\exists,t-1,\iota} \cup \tt{C}_{j,1,\top,t-1}, \dots, \tt{C}_{j,n_j,\exists,t-1,\iota} \cup \tt{C}_{j,n_j,\top,t-1}\tt{)}\right)
- \tt{A}_{\top,t-1} - \tt{A}_\top^* & \text{(Property \ref{property:gc_conservative})}\nonumber\\
=& \left(\tt{A}_{\exists,t-1,\iota} \cup \tt{A}_{\top,t-1} \cup \bigcup_{j=1}^m \tt{f}_j\tt{(}\tt{C}_{j,1,I,t-1,\iota}, \dots, \tt{C}_{j,n_j,I,t-1,\iota}\tt{)}\right)
- \tt{A}_{\top,t-1} - \tt{A}_\top^* & \text{(Lemma \ref{lem:merge_gc_inv})}\nonumber\\
=& \left(\tt{A}_{\exists,t-1,\iota} \cup \bigcup_{j=1}^m \tt{f}_j\tt{(}\tt{C}_{j,1,I,t-1,\iota}, \dots, \tt{C}_{j,n_j,I,t-1,\iota}\tt{)}\right)
- \tt{A}_{\top,t-1} - \tt{A}_\top^*. & \text{(Property \ref{property:gc_conservative})} \label{eq:thm:ins_correct:timestep}
\end{align}
If $\tt{A}$ is non-augmented, \eqref{eq:thm:ins_correct:timestep} becomes
\begin{align}
\tt{A}_{\exists,t,0}
=& \left(\tt{A}_{\exists,t-1,\iota} \cup \bigcup_{j=1}^m \tt{f}_j\tt{(}\tt{C}_{j,1,I,t-1,\iota}, \dots, \tt{C}_{j,n_j,I,t-1,\iota}\tt{)}\right)
- \tt{A}_\top^*. & \text{(Property \ref{property:gc_conservative})}\nonumber\\
\subseteq& \left(\tt{A}_{\exists,t-1,\iota} \cup \bigcup_{j=1}^m \tt{f}_j\tt{(}\tt{C}_{j,1,I,t-1,\iota}, \dots, \tt{C}_{j,n_j,I,t-1,\iota}\tt{)}\right)
- \tt{A}_I^- & \text{(Property \ref{property:instantiated_gc_consistent})} \nonumber\\% \label{eq:thm:inst_correct:timestep_0}\\
\subseteq& \left(\tt{A}_{I,t-1,\iota} \cup \bigcup_{j=1}^m \tt{f}_j\tt{(}\tt{C}_{j,1,I,t-1,\iota}, \dots, \tt{C}_{j,n_j,I,t-1,\iota}\tt{)}\right)
- \tt{A}_I^- & \text{(I.H.)} \nonumber\\% \label{eq:thm:inst_correct:timestep_1}\\
=& \tt{A}_{I,t,0} \nonumber
\end{align}
If $\tt{A}$ is key-augmented and $\tt{A} \in \mathbb{W}$, \eqref{eq:thm:ins_correct:timestep} becomes
\begin{align*}
\tt{A}_{\exists,t,0}
\subseteq& \left(\tt{A}_{\exists,t-1,\iota} \cup \bigcup_{j=1}^m \tt{f}_j\tt{(}\tt{C}_{j,1,I,t-1,\iota}, \dots, \tt{C}_{j,n_j,I,t-1,\iota}\tt{)}\right)
- \tt{A}_{\top,t-1} - \tt{A}_I^- & \text{(Property \ref{property:instantiated_gc_consistent})} 
\\
\subseteq& \left(\tt{A}_{I,t-1,\iota} \cup \bigcup_{j=1}^m \tt{f}_j\tt{(}\tt{C}_{j,1,I,t-1,\iota}, \dots, \tt{C}_{j,n_j,I,t-1,\iota}\tt{)}\right)
-(\tt{A}_{!,t-1,\iota} - \tt{A}_{I,t-1,\iota}) - \tt{A}_I^- & \text{(I.H.)}\\
=& \tt{A}_{I,t,0}
\end{align*}
If $\tt{A}$ is key-augmented and $\tt{A} \in \mathbb{S}$, \eqref{eq:thm:ins_correct:timestep} becomes
\begin{align*}
&\tt{A}_{\exists,t,0}\\
=& \left(\tt{A}_{\exists,t-1,\iota} \cup \bigcup_{j=1}^m \tt{f}_j\tt{(}\tt{C}_{j,1,I,t-1,\iota}, \dots, \tt{C}_{j,n_j,I,t-1,\iota}\tt{)}\right)
- \tt{A}_{\top,t-1} - (\tt{A}_{!,t-1,\iota} - \tt{A}_{I,t-1,\iota}) - \tt{A}_I^- & \text{(Property \ref{property:instantiated_gc_consistent})} 
\\
=& \left(\tt{A}_{I,t-1,\iota} \cup \bigcup_{j=1}^m \tt{f}_j\tt{(}\tt{C}_{j,1,I,t-1,\iota}, \dots, \tt{C}_{j,n_j,I,t-1,\iota}\tt{)}\right)
-(\tt{A}_{!,t-1,\iota} - \tt{A}_{I,t-1,\iota}) - \tt{A}_I^- & \text{(I.H.)}\\
=& \tt{A}_{I,t,0}
\end{align*}
In all cases, we see that
\begin{align*}
&\tt{A}_{I,t,0}\\
\subseteq&~ \tt{A}_{I,t-1,\iota} \cup \bigcup_{j=1}^m \tt{f}_j\tt{(}\tt{C}_{j,1,I,t-1,\iota}, \dots, \tt{C}_{j,n_j,I,t-1,\iota}\tt{)} \\
\subseteq&~ \tt{A}_{\exists,t-1,\iota} \cup \tt{A}_{\top,t-1} \cup \bigcup_{j=1}^m \tt{f}_j\tt{(}\tt{C}_{j,1,I,t-1,\iota}, \dots, \tt{C}_{j,n_j,I,t-1,\iota}\tt{)} & \text{(I.H.)} \\
=&~ \tt{A}_{\exists,t-1,\iota} \cup \tt{A}_{\top,t-1,} \cup \bigcup_{j=1}^m \tt{f}_j\tt{(}\tt{C}_{j,1,\exists,t-1,\iota} \cup \tt{C}_{j,1,\top,t-1}, \dots, \tt{C}_{j,n_j,\exists,t-1,\iota} \cup \tt{C}_{j,n_j,\top,t-1}\tt{)} & \text{(Lemma \ref{lem:merge_gc_inv})} \\
=&~ \tt{A}_{\exists,t-1,\iota} \cup \tt{A}_{\top,t-1} \cup \tt{A}_\exists^* \cup \tt{A}_\top^* \cup \bigcup_{j=1}^m \tt{f}_j\tt{(}\tt{C}_{j,1,\exists,t-1,\iota} \cup \tt{C}_{j,1,\top,t-1}, \dots, \tt{C}_{j,n_j,\exists,t-1,\iota} \cup \tt{C}_{j,n_j,\top,t-1}\tt{)} & \text{(Property \ref{property:gc_conservative})}\\
=&~ \tt{A}_{\exists,t,0} \cup \tt{A}_{\top,t}
\end{align*}
Furthermore, for $\tt{A}_!$, we have
\begin{align*}
\tt{A}_{!,t,0}
&= \tt{A}_{!,t-1,\iota} \cup \pi_!\left(\bigcup_{j=1}^m \tt{f}_j\tt{(}\tt{C}_{j,1,I,t-1,\iota}, \dots, \tt{C}_{j,n_j,I,t-1,\iota}\tt{)}\right)\\
&= \pi_!\left(\tt{A}_{\exists,t-1,\iota} \cup \tt{A}_{\top,t-1} \cup \bigcup_{j=1}^m \tt{f}_j\tt{(}\tt{C}_{j,1,I,t-1,\iota}, \dots, \tt{C}_{j,n_j,I,t-1,\iota}\tt{)}\right) & \text{(I.H.)} \\
&= \pi_!(\tt{A}_{\exists,t,0} \cup \tt{A}_{\top,t}) & \text{(Lemma \ref{lem:merge_gc_inv})}
\end{align*}
where in the final equality we have used the same argument as above for sets without primary keys.
\end{proof}


%!TEX root = bloom_logical.tex

\section{Garbage Collection for Asynchronous Rules}
\label{sec:async_gc}
Consider an asynchronous merge rule on a sender node $v$
\begin{align*}
\overline{\tt{B}} \quad\tt{<}\sim\quad \tt{f}\tt{(}\tt{A}_1,\dots,\tt{A}_n\tt{)}
\end{align*}
and the corresponding merge rule on the receiver node $v'$
\begin{align*}
\tt{B} \quad\tt{<=}\quad \overline{\tt{B}}.
\end{align*}
We would like to garbage collect from $\tt{A}_1, \dots, \tt{A}_n$ and also to reduce the amount of communication needed.
Unfortunately, the asynchronous merge differs from a merge $\tt{B} ~\tt{<=}~ \tt{f}\tt{(}\tt{A}_1,\dots,\tt{A}_n\tt{)}$ or a deferred merge $\tt{B} ~\tt{<+}~ \tt{f}\tt{(}\tt{A}_1,\dots,\tt{A}_n\tt{)}$ in an important way:
the sender node has no knowledge of the contents of $\tt{B}$.

To fix this, \cite{conway2014edelweiss} introduced the following additional rules.
On the receiver side, we acknowledge all received messages with
\begin{align*}
\overline{\underline{\tt{B}}} \quad\tt{<}\sim\quad \overline{\tt{B}},
\end{align*}
which is stored on the sender side
\begin{align*}
\underline{\tt{B}} \quad\tt{<=}\quad \overline{\underline{\tt{B}}}.
\end{align*}

Thus, $\underline{\tt{B}}$ contains a subset of $\tt{B}$'s contents.
\begin{lemma}
\label{lem:async_merge_contains}
For any $(v,t) <_\tau (v',t')$, $I^{\Pi_\tau}_{v,t}(\underline{\tt{B}}@(v,t)) \subseteq I^{\Pi_\tau}_{v',t'}(\tt{B}@(v',t'))$.
\end{lemma}
\begin{proof}(Sketch)
For any $x \in I^{\Pi_\tau}_{v,t}(\underline{\tt{B}}@(v,t)) = \bigcup_{s=0}^t I^{\Pi_\tau}_{v,s}(\overline{\underline{\tt{B}}}@(v,s))$, it must be the case that $\exists s \leq t$: $x \in I^{\Pi_\tau}_{v,s}(\overline{\underline{\tt{B}}}@(v,s))$, i.e., $x$ was received via $\overline{\underline{\tt{B}}}$ at some earlier time $(v,s)$.
That in turn implies
$\exists t''$ such that $(v',t'') <_\tau (v,s)$ and $x \in I^{\Pi_\tau}_{v',t''}(\tt{B}@(v',t''))$, i.e., $x$ was sent from $v'$ at time $(v',t'')$.
Now $(v',t'') <_\tau (v,s) <_\tau (v,t) <_\tau (v',t')$ and $\tt{B}$ is persistent implies that $x \in I^{\Pi_\tau}_{v',t'}(\tt{B}@(v',t'))$.
\end{proof}

\subsection{Logical Garbage Collection}
Armed with (partial) knowledge of $\tt{B}$, we can now introduce logical garbage collection for the asynchronous merge rule.
On the sender node $v$ in $\mathfrak{P}_{GC}$, we have rules
\begin{align}
(\overline{\tt{B}},\emptyset) &\quad\tt{<}\sim\quad (\tt{f}\tt{(}\tt{A}_{1,\exists} \cup \tt{A}_{1,\top}, \dots, \tt{A}_{n,\exists} \cup \tt{A}_{n,\top}\tt{)}, \emptyset)\label{eq:async_logical_send}\\
(\underline{\tt{B}}_\exists, \underline{\tt{B}}_\top) &\quad\tt{<=}\quad (\overline{\underline{\tt{B}}}, \emptyset),\nonumber
\end{align}
and on the receiver node $v'$,
\begin{align*}
(\tt{B}_\exists, \tt{B}_\top) &\quad\tt{<=}\quad (\overline{\tt{B}}, \emptyset)\\
(\overline{\underline{\tt{B}}}, \emptyset) &\quad\tt{<}\sim\quad (\overline{\tt{B}}, \emptyset).
\end{align*}
Note that we can replace \eqref{eq:async_logical_send} with $(\overline{\tt{B}},\emptyset) ~\tt{<}\sim~ (\tt{f}\tt{(}\tt{A}_{1,\exists} \cup \tt{A}_{1,\top}, \dots, \tt{A}_{n,\exists} \cup \tt{A}_{n,\top}\tt{)}  - (\underline{\tt{B}}_\exists \cup \underline{\tt{B}}_\top), \emptyset)$ without affecting the overall program logic.
We term this trick ARM, avoidance of redundant messages.

In addition, we add the garbage collection rule
\begin{align}
\langle \tt{A}_{1,TS}, \dots, \tt{A}_{n,TS} \rangle
\quad\tt{<+}\quad
\tt{GC}(\tt{A}_{1,TS}, \dots, \tt{A}_{n,TS}; \underline{\tt{B}}_{TS})
\end{align}
which depends on the $v$-accessible $\underline{\tt{B}}_{TS}$ instead of $\tt{B}_{TS}$.

Analogously to Merge GC Invariant \ref{inv:merge_gc}, we will require that the Asynchronous Merge GC Invariant \ref{inv:async_merge_gc} is maintained by all rules, including the above GC rule.

\begin{invariant}[Asynchronous Merge GC Invariant]\label{inv:async_merge_gc}
Consider any asynchronous merge rule 
\[R: \tt{B} ~\tt{<}\sim~ \tt{f(}\tt{A}_1,\dots,\tt{A}_n, \tt{U}_1, \dots, \tt{U}_m\tt{)}\]
where
$\tt{A}_1,\dots,\tt{A}_n$ are tables without keys,
$\tt{U}_1,\dots,\tt{U}_m$ are tables with keys,
and RHS is not a channel.
We require that, 
$\forall \langle \widehat{\tt{A}}_1, \dots, \widehat{\tt{A}}_n\rangle \geq \langle \tt{A}_{1,\exists},\dots,\tt{A}_{n,\exists} \rangle,
\quad
\forall \langle \widehat{\tt{U}}_1 , \dots, \widehat{\tt{U}}_m \rangle \geq \langle \tt{U}_{1,\exists}, \dots, \tt{U}_{m,\exists} \rangle$
such that $\emptyset = \widehat{\tt{U}}_1 \cap \tt{U}_{1,\top} = \dots = \widehat{\tt{U}}_1 \cap \tt{U}_{m,\top}$:
\begin{align}
\tt{f(}\widehat{\tt{A}}_1,\dots,\widehat{\tt{A}}_n, \widehat{\tt{U}}_{1,\exists},\dots,\widehat{\tt{U}}_{n,\exists}\tt{)}
\cup \underline{\tt{B}}_\exists\cup \underline{\tt{B}}_\top
~=~
\tt{f(}
\widehat{\tt{A}}_1 \cup \tt{A}_{1,\top},\dots,\widehat{\tt{A}}_n \cup \tt{A}_{n,\top}, \widehat{\tt{U}}_1 \cup \tt{U}_{1,\top},\dots,\widehat{\tt{U}}_n \cup \tt{U}_{n,\top}\tt{)} 
\cup \underline{\tt{B}}_\exists \cup \underline{\tt{B}}_\top.
\label{eq:inv_async_merge_gc}
\end{align}
\end{invariant}
If \eqref{eq:inv_async_merge_gc} holds for $\underline{\tt{B}}_{TS}@(v,t)$, then by Lemma \ref{lem:async_merge_contains}, it also holds for $\tt{B}_{TS}@(v',t')$ if $(v,t) <_\tau (v',t')$.

Correspondingly, we say a GC rule is \emph{safe for} an asynchronous rule $R$ if it maintains the Asynchronous Merge Invariant \ref{inv:async_merge_gc} with respect to $R$.

\begin{property}[Asynchronous Merge GC Safety]
A garbage collection rule $\tt{GC}$ is safe for an asynchronous merge rule $R: \tt{B} ~\tt{<}\sim~ \tt{f(}\tt{A}_1,\dots,\tt{A}_n, \tt{U}_1, \dots, \tt{U}_m\tt{)}$ if it respects output, and maintains the Asynchronous Merge GC Invariant \ref{inv:async_merge_gc} for $R$.
That is, if $\langle \tt{A}_{1,TS}, \dots, \tt{A}_{n,TS}, \tt{U}_{1,TS}, \dots, \tt{U}_{m,TS} \rangle$ and $\underline{\tt{B}}_{TS}$ satisfies Asynchronous Merge GC Invariant \ref{inv:async_merge_gc}, then
$\langle \tt{A}_{1,TS}^*, \dots, \tt{A}_{n,TS}^*,$ $\tt{U}_{1,TS}^*, \dots, \tt{U}_{m,TS}^* \rangle$ $=$ $\tt{GC}(\tt{A}_{1,TS}, \dots, \tt{A}_{n,TS},$ $\tt{U}_{1,TS}, \dots, \tt{U}_{m,TS}; \underline{\tt{B}}_{TS})$ and $\underline{\tt{B}}_{TS}$ satisfies Asynchronous Merge GC Invariant \ref{inv:async_merge_gc}.
\end{property}

Unlike with the Merge GC Safety Property \ref{property:gc_safety}, we require that $\tt{GC}$ maintains Invariant \ref{inv:async_merge_gc} without the condition that 
$\tt{B} \supseteq \tt{f}\tt{(}\tt{A}_{1,\exists} \cup \tt{A}_{1,\top},\dots,\tt{A}_{n,\exists} \cup \tt{A}_{n,\top}\tt{)}$,
since there is no guarantee that this condition is true at the time that the GC rule is executed, nor that it will ever be true if an unreliable messaging protocol is used.
As a simple example, consider the asynchronous copy rule $\tt{B} \tt{<}\sim \tt{A}$.
If we apply $\tt{GCAll}$ for the linear copy function, we could tombstone all tuples in $\tt{A}$ and never send them again.
However, the message could be dropped and thus never be seen by $\tt{B}$.


We will require that the garbage collection rule $\tt{GC}$ that is added to $\mathfrak{P}_{GC}$ for asynchronous merge rule $R$ be safe for $R$, monotone, and conservative.

It is not difficult to see that $\mathfrak{P}_{GC}$ with garbage collection for asynchronous merges maintains both Merge GC Invariant \ref{inv:merge_gc} and Asynchronous Merge GC Invariant \ref{inv:async_merge_gc}.
We omit the proof as it largely follows the structure of the proof of Lemma \ref{lem:merge_gc_inv}.

\begin{lemma}
\label{lem:async_merge_gc_inv}
The rewritten GC program $\mathfrak{P}_{GC}$ maintains the Merge GC Invariant \ref{inv:merge_gc} and the Asynchronous Merge GC Invariant \ref{inv:async_merge_gc}.
\end{lemma}

Furthermore, Theorem \ref{thm:equivalence_of_sets}, Corollary \ref{cor:equivalence_of_output}, Theorem \ref{thm:coord-free}, and Corollary \ref{cor:confluence_of_output} hold with little modification of the proofs.


\subsection{Instantiated Garbage Collection}
On the sender node $v$ in $\mathfrak{P}_{iGC}$, we have rules
\begin{align*}
\overline{\tt{B}}_I &\quad\tt{<}\sim\quad \tt{f}\tt{(}\tt{A}_{1,I}, \dots, \tt{A}_{n,I}\tt{)} - \underline{\tt{B}}_I\\
\underline{\tt{B}}_I &\quad\tt{<=}\quad \overline{\underline{\tt{B}}}_I\\
\langle \tt{A}_{1,E}, \dots, \tt{A}_{n,E} \rangle
&\quad\tt{<-}\quad
\tt{GCI}(\tt{A}_{1,E}, \dots, \tt{A}_{n,E}; \underline{\tt{B}}_{E})
\end{align*}
and on the receiver node $v'$,
\begin{align*}
\tt{B}_I &\quad\tt{<=}\quad \overline{\tt{B}}_I & \text{(or $(\tt{B}_I, \tt{B}_!) \quad\tt{<=}\quad (\overline{\tt{B}}_I, \pi_!(\overline{\tt{B}}_I$))}\\
\overline{\underline{\tt{B}}}_I &\quad\tt{<}\sim\quad \overline{\tt{B}}_I.
\end{align*}

As before, we require that $\tt{GCI}$ is conservative and consistent with some $\tt{GC}$ which is itself safe for $\tt{R}$, monotone, and conservative.

The proof of Theorem \ref{thm:instantiated_correctness} is largely unchanged, except with some careful handling of the differences between $\overline{\tt{B}}_I$ and $\underline{\tt{B}}$ resulting from the instantiated garbage collection and ARM.
The implications of Corollaries \ref{cor:instantiated_equivalence_of_output}, \ref{cor:instantiated_confluence_of_output}, \ref{cor:instantiated_confluence_of_keys} follow from Theorem \ref{thm:instantiated_correctness} without change.



\subsection{Garbage Collecting $\underline{\tt{B}}$}
We assume WLOG in this subsection that the RHS of an asynchronous rule is a table $\tt{A}$ with primary keys.
If not, we can replace $\overline{\tt{B}} ~\tt{<}\sim~ \tt{f}\tt{(}\tt{A}_1, \dots, \tt{A}_n\tt{)}$ with rules $\overline{\tt{B}} ~\tt{<}\sim~ \tt{A}$ and $\tt{A} ~\tt{<=}~ \tt{f}\tt{(}\tt{A}_1, \dots, \tt{A}_n\tt{)}$, where we attach unique message IDs to tuples merged into $\tt{A}$.

Then, we can apply DR- to the RHS $\tt{A}_I - \underline{\tt{B}}_I$ to garbage collect from both $\tt{A}_E$ and $\underline{\tt{B}}_E$, by replacing the garbage collection rule with
\[
\langle \tt{A}_E, \underline{\tt{B}}_E \rangle \quad\tt{<-}\quad \tt{iGCDR-}(\tt{A}_E, \underline{\tt{B}}_E; \underline{\tt{B}}_E).
\]



\end{document}