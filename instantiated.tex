%!TEX root = bloom_logical.tex

\section{Properties and Analysis of Instantiated Garbage Collection}
\label{sec:instantiated}



\setcounter{subsection}{-1}
Logical GC rules define the tombstoned tuples that can be safely deleted without affecting derivations.
Thus, for $\mathfrak{P}_{iGC}$ to produce the correct output, we will require that instantiated GC rules are `consistent' with their logical GC counterparts, in the sense that instantiated rules delete only the tuples that the logical garbage collection rule deems safe for deletion.
We expand on this intuition with a couple of illustrative examples below.

\subsection{Intuition}
\begin{example}[Consistency for Projection]~

Consider the projection $R: \tt{B} ~\tt{<op>}~ \pi_S(\tt{A})$, where $\tt{<op>}$ is a merge or deferred merge.
Further suppose that $\mathfrak{P}_{GC}$ and $\mathfrak{P}_{iGC}$ are at a `weakly consistent' state with respect to $\tt{A}$ and $\tt{B}$, i.e.,
$\tt{A}_\exists \subseteq \tt{A}_I \subseteq \tt{A}_\exists \cup \tt{A}_\top$ and also
$\tt{B}_\exists \subseteq \tt{B}_I \subseteq \tt{B}_\exists \cup \tt{B}_\top$.

The logical garbage collection rule $\tt{GCProject}$ for projection tells us that the  tuples that are safe for deletion are those in $\tt{A}_\exists \cup \tt{A}_\top$ whose projection have appeared in $\tt{B}_\exists \cup \tt{B}_\top$, i.e., $\tt{A}_\top \cup \tt{A}^*$, where $\tt{A}^* = \{a \in \tt{A}_\exists \cup \tt{A}_\top ~:~ \pi_S(\{a\}) \subseteq \tt{B}_\exists \cup \tt{B}_\top\}$.
Our instantiated garbage collection rule must identify safe-for-deletion tuples using only $\tt{A}_I$ and $\tt{B}_I$; the only such tuples that can be guaranteed safe are the tuples in $\tt{A}_I$ whose projection have appeared in $\tt{B}_I$, i.e., $\tt{A}_I^- = \{a \in \tt{A}_I ~:~ \pi_S(\{a\}) \subseteq \tt{B}_I\} \subseteq \tt{A}_\top \cup \tt{A}^*$.

Note that this choice $\tt{A}^-$ of tuples to delete maintains weak consistency of $\tt{A}$, such that $\tt{A}_\exists - \tt{A}^* \subseteq \tt{A}_I - \tt{A}^- \subseteq (\tt{A}_\exists - \tt{A}^*) \cup (\tt{A}_\top \cup \tt{A}^*)$.
\end{example}

\begin{example}[Consistency for DR-]
Suppose we are in the setting of Example \ref{ex:logical:dr-}, and that $\mathfrak{P}_{GC}$ and $\mathfrak{P}_{iGC}$ are at a weakly consistent state for $\tt{B}$ and $\tt{C}$, and a strongly consistent state for $\tt{A}$, i.e.,
$\tt{A}_I = \tt{A}_\exists$,
$\tt{B}_\exists \subseteq \tt{B}_I \subseteq \tt{B}_\exists \cup \tt{B}_\top$, and
$\tt{C}_\exists \subseteq \tt{C}_I \subseteq \tt{C}_\exists \cup \tt{C}_\top$.
By Lemma \ref{lem:merge_gc_inv}, $\tt{A}_{TS}$, $\tt{B}_{TS}$, and $\tt{C}_{TS}$ satisfy the Invariant \ref{inv:merge_gc}, so $\tt{C}_\top \subseteq \tt{A}_\top \subseteq \tt{C}_\exists \cup \tt{C}_\top$.

The logical garbage collection rule $\tt{GCDR-}$ for negative set difference reclamation tells us that the tuples that are safe for deletion are
$\tt{A}_\top \cup \tt{A}^* := \tt{A}_\top \cup [(\tt{A}_\exists \cup \tt{A}_\top) \cap (\tt{C}_\exists \cup \tt{C}_\top)]$
and
$\tt{C}_\top \cup \tt{C}^* := \tt{C}_\top \cup [(\tt{C}_\exists \cup \tt{C}_\top) \cap \tt{A}_\top]$.
The instantiated garbage collection rule must identify the correct tuples to delete: since $\tt{A} \in \mathbb{S}$ and $\tt{C} \in \mathbb{W}$, we must delete all of $\tt{A}_\top$ but can delete a subset of $\tt{C}_\top$.
Furthermore, this determination must be made using only $\tt{A}_I$, $\tt{A}_!$, and $\tt{C}_I$.
Given our above assumptions, we can see (Example \ref{ex:instantiated:dr-}) that 
$(\tt{A}_I \cap \tt{C}_I) \cup (\tt{A}_! - \tt{A}_I) = \tt{A}_\top \cup \tt{A}^*$
and $\tt{C}_I \cap (\tt{A}_! - \tt{A}_I) \subseteq \tt{C}_\top \cup \tt{C}^*$.

As with the previous example, the choice of tuples to delete maintains strong consistency of $\tt{A}$ and the weak consistency of $\tt{C}$.
\end{example}

\subsection{Properties of Instantiated GC rules}
We formalize the above intuitions in Property \ref{property:instantiated_gc_consistent}.
To do this, we will need to define set consistency.

\begin{property}[Instantiated Set Consistency]~

A non-augmented set $\tt{A}_I$ in $\mathfrak{P}_{iGC}$ is \emph{weakly consistent} with $\tt{A}_{TS}$ in $\mathfrak{P}_{GC}$ if $\forall t, i$, $\tt{A}_{\exists,t,i} \subseteq \tt{A}_{I,t,i} \subseteq \tt{A}_{\exists,t,i} \cup \tt{A}_{\top,t}$.
% A non-augmented set $\tt{B}_I$ in $\mathfrak{P}_{iGC}$ is \emph{strongly consistent} with $\tt{B}_{TS}$ in $\mathfrak{P}_{GC}$ if $\forall t, i$, $\tt{B}_{\exists,t,i} = \tt{B}_{I,t,i}$.

A key-augmented set $\tt{B}_E$ in $\mathfrak{P}_{iGC}$ is \emph{weakly consistent} with $\tt{B}_{TS}$ in $\mathfrak{P}_{GC}$ if $\forall t, i$, $\tt{B}_{\exists,t,i} \subseteq \tt{B}_{I,t,i} \subseteq \tt{B}_{\exists,t,i} \cup \tt{B}_{\top,t}$, and $\tt{B}_{!,t,i} = \pi_!(\tt{B}_{\exists,t,i} \cup \tt{B}_{\top,t})$.
A key-augmented set $\tt{C}_E$ in $\mathfrak{P}_{iGC}$ is \emph{strongly consistent} with $\tt{C}_{TS}$ in $\mathfrak{P}_{GC}$ if $\forall t, i$, $\tt{C}_{\exists,t,i} = \tt{C}_{I,t,i}$, and $\tt{C}_{!,t,i} = \pi_!(\tt{C}_{\exists,t,i} \cup \tt{C}_{\top,t})$.
\end{property}

It is not possible to have a strongly consistent non-augmented set, since the merge rule $\tt{A}_I ~\tt{<=}~ \tt{f}\tt{(}\dots\tt{)}$ could possibly merge some tuples from $\tt{A}_\top$ into $\tt{A}_I$.
For strongly consistent key-augmented sets, this is prevented since tombstoned tuples are filtered from merging into $\tt{A}_I$ by checking against $\tt{A}_!$.


\begin{property}[Instantiated GC Consistency]\label{property:instantiated_gc_consistent}~

Consider a logical garbage collection rule $\tt{GC}$ that is safe for $R$, and a corresponding instantiated garbage collection rule $\tt{iGC}$.
We say that
$\tt{iGC}$ is \emph{weakly consistent}   with $\tt{GC}$ on a set $\tt{A} \in \mathbb{W}(\tt{iGC})$ if it preserves weak consistency   of $\tt{A}_I$                 with $\tt{A}_{TS}$;
$\tt{iGC}$ is \emph{strongly consistent} with $\tt{GC}$ on a set $\tt{C} \in \mathbb{S}(\tt{iGC})$ if it preserves strong consistency of $\tt{C}_I$ (or $\tt{C}_E$) with $\tt{C}_{TS}$.

Formally, let
\begin{align*}
&\langle
\tt{A}_{1,TS}^*, \dots, \tt{A}_{n_1,TS}^*,
\tt{B}_{1,TS}^*, \dots, \tt{B}_{n_1,TS}^*, 
\tt{C}_{1,TS}^*, \dots, \tt{C}_{n_3,TS}^*
\rangle \\
&= \tt{GC}(
\tt{A}_{1,TS}, \dots, \tt{A}_{n_1,TS},
\tt{B}_{1,TS}, \dots, \tt{B}_{n_2,TS},
\tt{C}_{1,TS}, \dots, \tt{C}_{n_3,TS};
\\&\qquad
\tt{D}_{1,TS}, \dots, \tt{D}_{n_4,TS},
\tt{E}_{1,TS}, \dots, \tt{E}_{n_5,TS}
),\\
&\langle
\tt{A}_{1,I}^-, \dots, \tt{A}_{n_1,I}^-,
\tt{B}_{1,E}^-, \dots, \tt{B}_{n_1,E}^-,
\tt{C}_{1,E}^-, \dots, \tt{C}_{n_3,E}^-
\rangle\\
&= \tt{iGC}(
\tt{A}_{1,I}, \dots, \tt{A}_{n_1,I},
\tt{B}_{1,E}, \dots, \tt{B}_{n_2,E},
\tt{C}_{1,E}, \dots, \tt{C}_{n_3,E};
\\&\qquad
\tt{D}_{1,I}, \dots, \tt{D}_{n_4,I},
\tt{E}_{1,E}, \dots, \tt{E}_{n_5,E}
)
\end{align*}
where $\tt{A}_1, \dots, \tt{A}_{n_1}, \tt{B}_1, \dots, \tt{B}_{n_2} \in \mathbb{W}(\tt{iGC})$ and $\tt{C}_1, \dots, \tt{C}_{n_3} \in \mathbb{S}(\tt{iGC})$.

We say that $\tt{iGC}$ is \emph{weakly consistent} with $\tt{GC}$ on $\tt{A}_1,$ $\dots,$ $\tt{A}_{n_1},$ $\tt{B}_1,$ $\dots,$ $\tt{B}_{n_2}$, and \emph{strongly consistent} with $\tt{GC}$ on $\tt{C}_1,$ $\dots,$ $\tt{C}_{n_3}$, if under the conditions
\begin{enumerate}
\item $\forall i=1,\dots,n_1$, $\tt{A}_{i,\exists} \subseteq \tt{A}_{i,I} \subseteq \tt{A}_{i,\exists} \cup \tt{A}_{i,\top}$,
\item $\forall i=1,\dots,n_2$, $\tt{B}_{i,\exists} \subseteq \tt{B}_{i,I} \subseteq \tt{B}_{i,\exists} \cup \tt{B}_{i,\top}$, and $\tt{B}_{i,!} = \pi_!(\tt{B}_{i,\exists} \cup \tt{B}_{i,\top})$,
\item $\forall i=1,\dots,n_3$, $\tt{C}_{i,\exists} =         \tt{C}_{i,I}$,                                                   and $\tt{C}_{i,!} = \pi_!(\tt{C}_{i,\exists} \cup \tt{C}_{i,\top})$,
\item $\forall i=1,\dots,n_4$, $\tt{D}_{i,\exists} \subseteq \tt{D}_{i,I} \subseteq \tt{D}_{i,\exists} \cup \tt{D}_{i,\top}$,
\item $\forall i=1,\dots,n_5$, $\tt{E}_{i,\exists} \subseteq \tt{E}_{i,I} \subseteq \tt{E}_{i,\exists} \cup \tt{E}_{i,\top}$, and $\tt{E}_{i,!} = \pi_!(\tt{E}_{i,\exists} \cup \tt{E}_{i,\top})$,
\item $\tt{A}_{1,TS}, \dots, \tt{A}_{n_1,TS},$ $\tt{B}_{1,TS}, \dots, \tt{B}_{n_2,TS},$ $\tt{C}_{1,TS}, \dots, \tt{C}_{n_3,TS},$ $\tt{D}_{1,TS}, \dots, \tt{D}_{n_4,TS},$ $\tt{E}_{1,TS}, \dots, \tt{E}_{n_5,TS}$ satisfy the Merge GC Invariant \ref{inv:merge_gc} for $R$.
\end{enumerate}
the following are true:
\begin{enumerate}
\item $\forall i=1,\dots,n_1$, $\tt{A}_{i,I}^- \subseteq \tt{A}_{i,\top}^*$,
\item $\forall i=1,\dots,n_2$, $\tt{B}_{i,I}^- \subseteq \tt{B}_{i,\top}^*$,
\item $\forall i=1,\dots,n_3$, $\tt{C}_{i,I}^- \cup (\tt{C}_{i,!} - \tt{C}_{i,I}) =         \tt{C}_{i,\top}^*$,
\end{enumerate}
\end{property}

% \begin{property}[Instantiated GC Consistency]\label{property:instantiated_gc_consistent}
% Consider a logical garbage collection rule $\tt{GC}$ that is safe for $R$, and a corresponding instantiated garbage collection rule $\tt{iGC}$.
% We say that
% $\tt{iGC}$ is \emph{weakly consistent} with $\tt{GC}$ on a set $\tt{A}$ if it preserves weak consistency of $\tt{A}_I$ (or $\tt{A}_E$) with $\tt{A}_{TS}$;
% $\tt{iGC}$ is \emph{strongly consistent} with $\tt{GC}$ on a set $\tt{A}$ if it preserves strong consistency of $\tt{A}_I$ (or $\tt{A}_E$) with $\tt{A}_{TS}$.

% Formally, let
% \begin{align*}
% &\langle
% \tt{A}_{1,TS}^*, \dots, \tt{A}_{n_1,TS}^*,
% \tt{B}_{1,TS}^*, \dots, \tt{B}_{n_1,TS}^*, 
% \tt{C}_{1,TS}^*, \dots, \tt{C}_{n_3,TS}^*
% \rangle \\
% &= \tt{GC}(
% \tt{A}_{1,TS}, \dots, \tt{A}_{n_1,TS},
% \tt{B}_{1,TS}, \dots, \tt{B}_{n_2,TS},
% \tt{C}_{1,TS}, \dots, \tt{C}_{n_3,TS};
% \tt{D}_{1,TS}, \dots, \tt{D}_{n_4,TS},
% \tt{E}_{1,TS}, \dots, \tt{E}_{n_5,TS}
% ),\\
% &\langle
% \tt{A}_{1,I}^-, \dots, \tt{A}_{n_1,I}^-,
% \tt{B}_{1,I}^-, \dots, \tt{B}_{n_1,I}^-,
% \tt{C}_{1,E}^-, \dots, \tt{C}_{n_3,E}^-
% \rangle\\
% &= \tt{iGC}(
% \tt{A}_{1,I}, \dots, \tt{A}_{n_1,I},
% \tt{B}_{1,I}, \dots, \tt{B}_{n_2,I},
% \tt{C}_{1,E}, \dots, \tt{C}_{n_3,E};
% \tt{D}_{1,E}, \dots, \tt{D}_{n_4,E},
% \tt{E}_{1,I}, \dots, \tt{E}_{n_5,I}
% ).
% \end{align*}
% We say that $\tt{iGC}$ is \emph{weakly consistent} with $\tt{GC}$ on $\tt{A}_1, \dots, \tt{A}_{n_1}, \tt{B}_1, \dots, \tt{B}_{n_2}$, and \emph{strongly consistent} with $\tt{GC}$ on $\tt{C}_1, \dots, \tt{C}_{n_3}$, if under the conditions
% \begin{enumerate}
% \item $\forall i=1,\dots,n_1$, $\tt{A}_{i,\exists} \subseteq \tt{A}_{i,I} \subseteq \tt{A}_{i,\exists} \cup \tt{A}_{i,\top}$,
% \item $\forall i=1,\dots,n_2$, $\tt{B}_{i,\exists} \subseteq \tt{B}_{i,I} \subseteq \tt{B}_{i,\exists} \cup \tt{B}_{i,\top}$, and $\tt{B}_{i,!} = \pi_!(\tt{B}_{i,\exists} \cup \tt{B}_{i,\top})$,
% \item $\forall i=1,\dots,n_3$, $\tt{C}_{i,\exists} =         \tt{C}_{i,I}$,                                                   and $\tt{C}_{i,!} = \pi_!(\tt{C}_{i,\exists} \cup \tt{C}_{i,\top})$,
% \item $\forall i=1,\dots,n_4$, $\tt{D}_{i,\exists} \subseteq \tt{D}_{i,I} \subseteq \tt{D}_{i,\exists} \cup \tt{D}_{i,\top}$,
% \item $\forall i=1,\dots,n_5$, $\tt{E}_{i,\exists} \subseteq \tt{E}_{i,I} \subseteq \tt{E}_{i,\exists} \cup \tt{E}_{i,\top}$, and $\tt{E}_{i,!} = \pi_!(\tt{E}_{i,\exists} \cup \tt{E}_{i,\top})$,
% \item $\tt{A}_{1,TS}, \dots, \tt{A}_{n_1,TS},$ $\tt{B}_{1,TS}, \dots, \tt{B}_{n_2,TS},$ $\tt{C}_{1,TS}, \dots, \tt{C}_{n_3,TS},$ $\tt{D}_{1,TS}, \dots, \tt{D}_{n_4,TS},$ $\tt{E}_{1,TS}, \dots, \tt{E}_{n_5,TS}$ satisfy the Merge GC Invariant \ref{inv:merge_gc} for $R$.
% \end{enumerate}
% the following are true:
% \begin{enumerate}
% \item $\forall i=1,\dots,n_1$, $\tt{A}_{i,I}^- \subseteq \tt{A}_{i,\top}^*$,
% \item $\forall i=1,\dots,n_2$, $\tt{B}_{i,I}^- \subseteq \tt{B}_{i,\top}^*$,
% \item $\forall i=1,\dots,n_3$, $\tt{C}_{i,I}^- =         \tt{C}_{i,\top}^*$,
% \end{enumerate}
% \end{property}



% \begin{property}[Instantiated GC Consistency]\label{property:instantiated_gc_consistent}
% Suppose $\tt{GC}$ is a logical garbage collection rule safe for $R$, and that the following are true:
% \begin{enumerate}
% \item $\tt{A}_{1,TS}, \dots, \tt{A}_{n,TS}$, $\tt{B}_{1,TS}, \dots, \tt{B}_{m,TS}$ satisfy the Merge GC Invariant \ref{inv:merge_gc} for $R$,
% \item $\forall i=1,\dots,n$, if $\tt{A}_i$ has no primary keys, then $\tt{A}_{i,\exists} \subseteq \tt{A}_{i,I} \subseteq \tt{A}_{i,\exists} \cup \tt{A}_{i,\top}$; otherwise, $\tt{A}_{i,\exists} = \tt{A}_{i,I}$ and $\tt{A}_{i,!} = \pi_!(\tt{A}_{i,\exists} \cup \tt{A}_{i,\top})$,
% \item $\forall i=1,\dots,m$, if $\tt{B}_i$ has no primary keys, then $\tt{B}_{i,\exists} \subseteq \tt{B}_{i,I} \subseteq \tt{B}_{i,\exists} \cup \tt{B}_{i,\top}$; otherwise, $\tt{B}_{i,\exists} = \tt{B}_{i,I}$ and $\tt{B}_{i,!} = \pi_!(\tt{B}_{i,\exists} \cup \tt{B}_{i,\top})$.
% \end{enumerate}
% Let $(\tt{A}_{1,TS}^*, \dots, \tt{A}_{n,TS}^*) = \tt{GC}(\tt{A}_{1,TS}, \dots, \tt{A}_{n,TS}; \tt{B}_{1,TS}, \dots, \tt{B}_{m,TS})$
% and $(\tt{A}_{1,E}^-, \dots, \tt{A}_{n,E}^-)$ $=$ $\tt{GCI}(\tt{A}_{1,E}, \dots, \tt{A}_{n,E};$ $\tt{B}_{1,E}, \dots, \tt{B}_{m,E})$.
% Then, $\tt{GCI}$ is \emph{consistent with} $\tt{GC}$ if $\forall i=1,\dots,n$: $\tt{A}_{i,I}^- \subseteq \tt{A}_{i,\top}^*$ if $\tt{A}_i$ does not have primary keys,
% and 
% $(\tt{A}_{i,!} - \tt{A}_{i,I}) \cup \tt{A}_{i,I}^- = \tt{A}_{i,\top}^*$ if $\tt{A}_i$ has primary keys.
% \end{property}
% In words, $\tt{GCI}$ is consistent with $\tt{GC}$ if the tuples that are deleted by $\tt{GCI}$ are a subset of the tuples tombstoned by $\tt{GC}$,
% and in the case where primary keys are available, both $\tt{GC}$ and $\tt{GCI}$ agree on the tombstoned tuples.

In other words, $\tt{iGC}$ deletes a subset of $\tt{A}$ and $\tt{B}$ tuples that $\tt{GC}$ tombstones, and $\tt{iGC}$ exactly agrees with $\tt{GC}$ on the deleted / tombstoned tuples in $\tt{C}$.
Note that the strong consistency of $\tt{iGC}$ with $\tt{GC}$ on $\tt{C}$ does not necessary imply the corresponding weak consistency, because the strong consistency holds under the strong assumption that $\tt{C}_{i,\exists} = \tt{C}_{i,I}$ instead of the weaker assumption that $\tt{C}_{i,\exists} \subseteq \tt{C}_{i,I} \subseteq \tt{C}_{i,\exists} \cup \tt{C}_{i,\top}$.
Conversely, the weak consistency of $\tt{iGC}$ with $\tt{GC}$ on $\tt{A}$ and $\tt{B}$ does not necessary imply the corresponding strong consistency, because weak consistency only provides the weaker result that $\tt{A}_{i,I}^- \subseteq \tt{A}_{i,\top}^*$ and $\tt{B}_{i,I}^- \subseteq \tt{B}_{i,\top}^*$ instead of the stronger requirement that $\tt{A}_{i,I}^- = \tt{A}_{i,\top}^*$ and $\tt{B}_{i,I}^- = \tt{B}_{i,\top}^*$.


A natural property for instantiated garbage collection rules to have is to not delete any non-existent tuple.

\begin{property}[Instantiated GC Conservation]\label{property:instantiated_gc_conservative}
Let $(\tt{A}_{1,E}^-, \dots, \tt{A}_{n,E}^-) = \tt{iGC}(\tt{A}_{1,E}, \dots, \tt{A}_{n,E}; \tt{B}_{1,E}, \dots, \tt{B}_{m,E})$.
The instantiated GC rule $\tt{iGC}$ is \emph{conservative} if $\forall \tt{A}_{1,E},$ $\dots,$ $\tt{A}_{n,E};$ $\tt{B}_{1,E},$ $\dots,$ $\tt{B}_{m,E}$, $\forall i=1,\dots,n$: $\tt{A}_{i,I}^- \subseteq \tt{A}_{i,I}$ and $\tt{A}_{i,!}^- = \pi_!(\tt{A}_{i,I})$.
\end{property}

We will require that any instantiated garbage collection rule $\tt{GCI}$ for $R$ be conservative and consistent with some logical garbage collection rule $\tt{GC}$, which is itself safe for $R$, conservative, and monotone\footnote{
  I believe that monotonicity is required only to show $\mathfrak{P}_{GC}$ is confluent in $\tt{A}_\top$, and $\mathfrak{P}_{iGC}$ is confluent in $\tt{A}_! - \tt{A}_I$.
}.



\subsection{Examples of Instantiated GC rules}
We now present some examples of instantiated GC rules that are consistent with the logical GC rules presented in Section \ref{sec:logical:example}.

\begin{example}[Copy GC]
$\tt{iGCCopy}(\tt{A}_I;$ $\tt{B}_{1,I},$ $\dots,$ $\tt{B}_{m,I}) = \tt{A}_I \cap \bigcap_{i=1}^m \tt{B}_{i,I}$ with $\mathbb{W}(\tt{iGCCopy}) = \{\tt{A}\}$ is weakly consistent with $\tt{GCCopy}$ on $\tt{A}$.
\end{example}
Suppose
$\tt{A}_\exists \subseteq \tt{A}_I \subseteq \tt{A}_\exists \cup \tt{A}_\top$
and for all $i = 1, \dots, m$:
$\tt{B}_{i,\exists} \subseteq \tt{B}_{i,I} \subseteq \tt{B}_{i,\exists} \cup \tt{B}_{i,\top}$.
Then,
\begin{align*}
\tt{A}_I \cap \bigcap_{i=1}^m \tt{B}_{i,I}
&\subseteq
(\tt{A}_\exists \cup \tt{A}_\top) \cap \bigcap_{i=1}^m (\tt{B}_{i,\exists} \cup \tt{B}_{i,\top})\\
% \\
% &
% ~=~
% \left(\tt{A}_\exists \cap \bigcap_{i=1}^m (\tt{B}_{i,\exists} \cup \tt{B}_{i,\top})\right)
% \cup
% \left(\tt{A}_\top \cap \bigcap_{i=1}^m (\tt{B}_{i,\exists} \cup \tt{B}_{i,\top})\right)\\
% &
&\subseteq
\left(\tt{A}_\exists \cap \bigcap_{i=1}^m (\tt{B}_{i,\exists} \cup \tt{B}_{i,\top})\right)
\cup
\tt{A}_\top.
\end{align*}
Hence, $\tt{iGCCopy}$ is weakly consistent with $\tt{GCCopy}$ on $\tt{A}$.

\begin{example}[Union and Intersection GC]
~

$\tt{iGCUnion}(\tt{A}_I, \tt{C}_I; \tt{B}_I)$ $=$ $\tt{iGCIntersect}(\tt{A}_I, \tt{C}_I; \tt{B}_I)$ $=$ $\langle \tt{A}_I \cap \tt{B}_I, \tt{C}_I \cap \tt{B}_I \rangle$ is weakly consistent with $\tt{GCIntersect}$ and $\tt{GCUnion}$ on $\tt{A}$ and $\tt{C}$.
\end{example}
The proof of consistency is essentially the same as for $\tt{iGCCopy}$.

\begin{example}[Project GC]
$\tt{iGCProject}(\tt{A}_I;$ $\tt{B}_I)$ $=$ $\{a \in \tt{A}_I ~:~ \pi_S(\{a\}) \subseteq \tt{B}_I\}$ with $\mathbb{W}(\tt{iGCProject}) = \{\tt{A}\}$ is weakly consistent with $\tt{GCProject}$ on $\tt{A}$.
\end{example}

If $\tt{A}_I \subseteq \tt{A}_\exists \cup \tt{A}_\top$ and $\tt{B}_I \subseteq \tt{B}_\exists \cup \tt{B}_\top$, then
clearly $\tt{iGCProject}(\tt{A}_I;$ $\tt{B}_I) = \{a \in \tt{A}_I ~:~ \pi_S(\{a\}) \subseteq \tt{B}_I\} \subseteq \{a \in \tt{A}_\exists \cup \tt{A}_\top ~:~ \pi_S(\{a\}) \subseteq \tt{B}_\exists \cup \tt{B}_\top\} \subseteq \tt{GCProject}(\tt{A}_{TS}; \tt{B}_{TS})$.

\begin{example}[DR-: Negative Difference]
\label{ex:instantiated:dr-}
~

$\tt{iGCDR-}(\tt{A}_E, \tt{C}_I; \tt{B}_I) = \langle (\tt{A}_I \cap \tt{C}_I, \pi_!(\tt{A}_I \cap \tt{C}_I), \tt{C}_I \cap (\tt{A}_! - \tt{A}_I)\rangle$ with $\mathbb{W}(\tt{iGCDR-}) = \{\tt{A}\}$ and $\mathbb{S}(\tt{iGCDR-}) = \{\tt{C}\}$ is strongly consistent with $\tt{GCDR-}$ on $\tt{A}$ and weakly consistent with $\tt{GCDR-}$ on $\tt{C}$.
\end{example}
Suppose $\tt{A}_{TS}$, $\tt{C}_{TS}$, $\tt{A}_E$, and $\tt{C}_I$ satisfy the conditions of Property \ref{property:instantiated_gc_consistent}.
In particular, $\tt{A}_{TS}$ and $\tt{C}_{TS}$ satisfies Invariant \ref{inv:merge_gc}, so as previously discussed, this implies $\tt{C}_\top \subseteq \tt{A}_\top \subseteq \tt{C}_\exists \cup \tt{C}_\exists$.
Thus,
\begin{align*}
(\tt{A}_! - \tt{A}_I) \cup (\tt{A}_I \cap \tt{C}_I)
&= \tt{A}_\top \cup (\tt{A}_\exists \cap \tt{C}_\exists)\\
&= \tt{A}_\top \cup (\tt{A}_\exists \cap (\tt{C}_\exists \cup \tt{C}_\top))\\
&= \tt{A}_\top \cup ((\tt{A}_\exists \cup \tt{A}_\top) \cap (\tt{C}_\exists \cup \tt{C}_\top))
\end{align*}
which are exactly the tombstones of $\tt{GCDR-}$ for $\tt{A}_{TS}$, and
\begin{align*}
\tt{C}_I \cap (\tt{A}_! - \tt{A}_I) \subseteq (\tt{C}_\exists \cup \tt{C}_\top) \cap \tt{A}_\top \subseteq \tt{C}_\top \cup [(\tt{C}_\exists \cup \tt{C}_\top) \cap \tt{A}_\top]
\end{align*}
which are a subset of the tombstones of $\tt{GCDR-}$ for $\tt{C}_{TS}$.


\subsection{Correctness}
We will now show that $\mathfrak{P}_{iGC}$ produces the same output as $\mathfrak{P}$.

\newcommand{\thminstantiatedcorrectness}{
  For any set $\tt{A}$, let $\tt{GC}$ be its logical GC rule in $\mathfrak{P}_{GC}$ and $\tt{iGC}$ be its instantiated GC rule in $\mathfrak{P}_{iGC}$.
Then, it is either the case that
\begin{enumerate}
\item $\tt{A} \in \mathbb{W}(\tt{iGC})$, $\tt{iGC}$ is weakly consistent with $\tt{GC}$ on $\tt{A}$, and $\tt{A}_I$ (or $\tt{A}_E$) is weakly consistent with $\tt{A}_{TS}$, i.e.,
\begin{align}
\forall t, i: \quad \tt{A}_{\exists,t,i} \subseteq \tt{A}_{I,t,i} \subseteq \tt{A}_{\exists,t,i} \cup \tt{A}_{\top,t,i} \label{eq:thm_instantiated_correctness_set}
\end{align}
\item $\tt{A} \in \mathbb{S}(\tt{iGC})$, $\tt{iGC}$ is strongly consistent with $\tt{GC}$ on $\tt{A}$, and $\tt{A}_E$ is strongly consistent with $\tt{A}_{TS}$, i.e.,
\begin{align}
\forall t, i: \quad \tt{A}_{\exists,t,i} = \tt{A}_{I,t,i} \label{eq:thm_instantiated_correctness_set_strong}
\end{align}
\end{enumerate}
In both cases, if $\tt{A}_E$ is key-augmented in $\mathfrak{P}_{iGC}$, then
\begin{align}
\forall t, i : \tt{A}_{!,t,i} = \pi_!(\tt{A}_{\exists,t,i} \cup \tt{A}_{\top,t,i}) \label{eq:thm_instantiated_correctness_key}.
\end{align}
}
\begin{thm}[Representation Correctness]
\label{thm:instantiated_correctness}
\thminstantiatedcorrectness
\end{thm}
\begin{rmk}
An immediate consequence is that $\forall t, i: \tt{A}_{!,t,i} - \pi_!(\tt{A}_{I,t,i}) \subseteq \pi_!(\tt{A}_{\top,t,i})$ for weakly consistent $\tt{A}_E$, and $\forall t, i: \tt{A}_{!,t,i} - \pi_!(\tt{A}_{I,t,i}) = \pi_!(\tt{A}_{\top,t,i})$ for strongly consistent $\tt{A}_E$.
\end{rmk}
% \begin{thm}[Representation Correctness]
% \label{thm:instantiated_correctness}
% For any set $\tt{A}$ without primary keys in $\mathfrak{P}$ and its corresponding lattice $(\tt{A}_\exists,t, \tt{A}_\top)$ in $\mathfrak{P}_{GC}$ and set $\tt{A}_E$ in $\mathfrak{P}_{iGC}$,
% \begin{align}
% \tt{A}_{\exists,t,i} \subseteq &\tt{A}_{I,t,i} \subseteq \tt{A}_{\exists,t,i} \cup \tt{A}_{\top,t}.\label{eq:thm_instantiated_correctness_set}
% \end{align}
% If $\tt{A}$ has primary keys, then
% \begin{align}
% \tt{A}_{I,t,i} =& \tt{A}_\exists,\label{eq:thm_instantiated_correctness_set_key}\\
% \tt{A}_{!,t,i} =& \pi_!(\tt{A}_{\exists,t,i} \cup \tt{A}_{\top,t}).\label{eq:thm_instantiated_correctness_key}
% \end{align}
% As an immediate consequence of \eqref{eq:thm_instantiated_correctness_set_key} and \eqref{eq:thm_instantiated_correctness_key}, we get $\pi_!(\tt{A}_{\top,i}) = \tt{A}_{!,t,i} - \pi_!(\tt{A}_{I,t,i})$ because $\tt{A}_{\exists,t,i} \cap \tt{A}_{\top,t} = \emptyset$
% \end{thm}

\begin{cor}
\label{cor:instantiated_equivalence_of_output}
If $\tt{A}$ is an output set, it has $\tt{A}_I = \tt{A}_\exists = \tt{A}$.
\end{cor}

\begin{cor}
\label{cor:instantiated_confluence_of_output}
If $\mathfrak{P}$ is confluent and $\tt{A}$ is an output set, then $\mathfrak{P}_{iGC}$ is confluent in $\tt{A}_I$.
\end{cor}

\begin{cor}
\label{cor:instantiated_confluence_of_keys}
If $\mathfrak{P}$ is confluent and $\mathfrak{P}_{iGC}$ is strongly consistent with $\mathfrak{P}_{GC}$ on $\tt{A}$, then $\mathfrak{P}_{iGC}$ is confluent in $\tt{A}_I$ and $\tt{A}_!$.
\end{cor}

\subsubsection{Incomplete}
In the previous section, Theorem \ref{thm:instantiated_correctness} established the soundness of our garbage collection rewrite --- tuples are deleted only if they do not affect future computation.
Unfortunately, it is not generally possible to establish completeness, i.e., that any tuple that could be removed without affecting future computation will be removed.

There are a couple of possible reasons for incompleteness.
Firstly, there may not be a unique largest set of reclaimable tuples.
For example, consider the function $\tt{f}(\tt{A}) = \emptyset$ if $|\tt{A}| = 0$ and $\{1\}$ otherwise.
Given the input $\tt{A} = \{a,b\}$, either $a$ or $b$ can be reclaimed without affecting computation, but we cannot safely reclaim both tuples.

Secondly, the act of garbage collection itself reduces the information available for reasonable about safe reclamation.
Consider the program
\begin{align*}
\tt{A} &\quad\tt{<}\sim\quad \tt{chnA}\\
\tt{B} &\quad\tt{<}\sim\quad \tt{chnB}\\
\tt{C} &\quad\tt{<=}\quad \tt{A} \cap \tt{B}\\
\tt{D} &\quad\tt{<}\sim \tt{C}
\end{align*}
If a tuple $x$ arrives in both $\tt{A}$ and $\tt{B}$, it is derived in $\tt{C}$ and sent to $\tt{D}$.
A complete garbage collection must immediately reclaim $x$ from both $\tt{A}$ and $\tt{B}$, and eventually from $\tt{C}$ once its arrival at $\tt{D}$ has been determined.
Now if $x$ arrives in $\tt{A}$ again but not in $\tt{B}$, we cannot determine from $\tt{A}$, $\tt{B}$ and $\tt{C}$ that it is safe to delete.
While it is possible to deduce that $x$ in $\tt{A}$ is safe for deletion by examining $\tt{D}$, this would require multi-rule reasoning, and in the worst case, over the entire program $\tt{P}$.
Doing so is complex and beyond the scope of our current work.



\subsection{Coordination}
Unlike $\mathfrak{P}_{GC}$, the instantiated GC program $\tt{P}_{iGC}$ is in general \emph{not confluent on all sets} even if $\mathfrak{P}$ is.
This appears to run counter to our goal of reconciling CALM with garbage collection.
However, we highlight that $\mathfrak{P}_{iGC}$ is `confluent' in $\tt{A}_\exists$, in the sense that if a tuple $x$ is in $\tt{A}_\exists$, then all runs of $\mathfrak{P}_{iGC}$ will agree that $x \in \tt{A}_I$.
Thus, $\mathfrak{P}_{iGC}$ is `non-confluent'\footnote{
	This is particularly true for weakly consistent sets; for strongly consistent sets, we also have confluence in $\tt{A}_I$ by Corollary \ref{cor:instantiated_confluence_of_keys}.
}
only with respect to tombstone sets $\tt{A}_\top$, i.e., separate runs only disagree on tuples that, by our invariants, do not affect the outcome.
In particular, for output sets, we have $\tt{A}_\exists = \tt{A}_I = \tt{A}$, and hence $\mathfrak{P}_{iGC}$ is confluent on the output that we care about.
We further elaborate on the coordination of $\mathfrak{P}_{iGC}$ using the analysis frameworks of \cite{marczak2012confluence} and \cite{ameloot2013relational}.


\subsubsection{Dedalus$^+$ confluence analysis of \cite{marczak2012confluence}}
In \cite{marczak2012confluence}, `confluence' of a program $\mathfrak{Q}$ is defined as the agreement of `all runs' on `eventually always true facts'.
Here, a `run' is an execution\footnote{
	To be precise, \cite{marczak2012confluence} defined confluence in terms of `models' and not `runs'.
    However, as we show in our discussion of Bloom$^L$, the model-theoretic and operational fixed-point approach are equivalent.
} of $\mathfrak{Q}$ obeying some causally plausible message-time assignment, and an `eventually true fact' is a tuple that is true at a node $v$ for all times greater than some $(v,t)$.
The eventually true facts are defined to be $\mathfrak{Q}$'s output.

In order to achieve confluence, \cite{marczak2012confluence} defined a sublanguage, Dedalus$^+$, for which all sets (or relations) are temporally inflationary, i.e., any derived fact is immediately eventually-always-true.
It is then straightforward to show that all relations of any program $\mathfrak{Q}$ expressed in Dedalus$^+$ are eventually-always-true and that $\mathfrak{Q}$ is confluent.

Our instantiated garbage collection program $\mathfrak{Q}$ is not expressible in Dedalus$^+$, nor is it confluent by the definition of \cite{marczak2012confluence}.
However, we could refine the definition to say that a program $\mathfrak{Q}$ is confluent on a tuple (or fact) $x$ if all runs agree on whether the tuple is eventually-always-true.
Then, our correctness proofs of Theorem \ref{thm:instantiated_correctness} and Corollary \ref{cor:instantiated_equivalence_of_output} imply that $\mathfrak{P}_{iGC}$ is confluent on all tuples of $\tt{A}_\exists$ and $\tt{A}^c = (\tt{A}_\exists \cup \tt{A}_\top)^c$.
In other words, no coordination is needed to control the message ordering as long as we do not care about tuples of $\tt{A}_\top$.
For the sets that have been denoted as output, we guarantee that $\tt{A}_\top = \emptyset$, so every run agrees on the tuples that are present (and that are absent) from $\tt{A}$.

One can therefore see that expression in Dedalus$^+$ is a sufficient but not necessary\footnote{
	A simple example of a non-Dedalus$^+$ but confluent program is $\tt{A} ~\tt{<=}~ \mathcal{E}$; $\tt{B} ~\tt{<=}~ \tt{A}$; $\tt{A} ~\tt{<-}~ \tt{B}$.
} condition for confluence.
The use of deferred deletes does not immediately condemn the program to be non-confluent.
While \cite{marczak2012confluence} had used a syntactic restriction of the language to achieve confluence, we instead use a semantic restriction on the possible garbage collection rules, via the imposition of Properties \ref{property:gc_safety} and \ref{property:instantiated_gc_consistent}.

% In fact, the `confluence' of \cite{marczak2012confluence} was defined in terms of agreement on output sets.
% Dedalus$^+$ (and monotone Bloom$^L$) are restrictions on the programming language to ensure that \emph{all sets and lattices} are temporally inflationary and confluent.
% It is a sufficient but not necessary condition for confluence, and does not exclude the possibility of other programs in non-monotone Bloom$^L$ being confluent in some output sets;
% our program $\mathfrak{P}_{iGC}$ is exactly an example of such a program.
% Instead of restricting the language, we restrict the possible GC functions to ensure that output sets always converge.



\subsubsection{Transducer network CALM analysis of \cite{ameloot2013relational}}
A transducer network as defined in \cite{ameloot2013relational} is a distributed set of compute nodes endowed with message buffers and memory.
At each time step, one node in the transducer network computes queries on their memory and messages in its buffer, sending messages to its neighbors, updating its memory, and emitting an output.
The output of the transducer network is the union of outputs of all nodes across all time steps.

Under this model, \cite{ameloot2013relational} proves the CALM theorem: a query is monotone if and only if there is a transducer network that can compute the query as its output under any message ordering, and the transducer network already produces the correct output without communication.
Note that this is statement only of the monotonicity of the \emph{output}, and in particular there are no restrictions placed on the memory relations, which may grow or shrink without affecting the validity of the CALM theorem.
In fact, this is exactly the situation with $\mathfrak{P}_{iGC}$ --- output sets grow monotonically, but other non-output sets (which are essentially memory relations of the transducer network) may be reclaimed so long as the derivation of outputs are not affected.


We also extended the analysis of \cite{ameloot2013relational} to an open-world model where inputs may arrive at arbitrarily late times, and show that a query is monotone if and only if there is a transducer network that can compute it under any message ordering and input arrival times.
Stated in this way, coordination is only need to determine that the completeness of the input, i.e., that all input has arrived and been received by the node.
The GC Invariant \ref{inv:merge_gc} can be interpreted as an assertion that tombstoned tuples are independent of any future input;
thus, no additional coordination is required to determine input completeness before garbage collecting tombstoned tuples.


