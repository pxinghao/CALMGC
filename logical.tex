%!TEX root = bloom_logical.tex

\section{Properties and Analysis of Logical Garbage Collection}
\label{sec:logical}
For our rewrite to be sensible, we require that the GC rules have certain properties.
In particular, the GC rules must be safe -- they should only tombstone tuples that have no effect on future derivations.
We provide some intuition for this safety in the following examples.

\setcounter{subsection}{-1}
\subsection{Intuition}
\begin{example}[Safety for Projection]
Suppose we have the projection rule $\tt{X} ~\tt{<=}~ \pi_S(\tt{Y})$ in $\mathfrak{P}$, and correspondingly $\tt{X}_{TS} ~\tt{<=}~ (\pi_S(\tt{Y}_\exists \cup \tt{Y}_\top), \emptyset)$ in $\mathfrak{P}_{GC}$.
% , and $\tt{X}_I ~\tt{<=}~ \pi_S(\tt{Y}_I)$ in $\mathfrak{P}_{iGC}$.
% Our goal is to always maintain the relationships
% $\tt{X}_\exists \subseteq \tt{X}_I \subseteq \tt{X}_\exists \cup \tt{X}_\top = \tt{X}$ and
% $\tt{Y}_\exists \subseteq \tt{Y}_I \subseteq \tt{Y}_\exists \cup \tt{Y}_\top = \tt{Y}$.
Since we wish to eventually reclaim tombstoned tuples, these tuples in $\tt{Y}_\top$ should have no effect on the execution of the projection rule.
As explained in Section \ref{sec:prelims:exec}, executing the rule is equivalent to making the assignment $\tt{X}_{\exists,t,i+1} \cup \tt{X}_{\top,t,i+1} := \tt{X}_{\exists,t,i} \cup \tt{X}_{\top,t,i} \cup \pi_S(\tt{Y}_{\exists,t,i} \cup \tt{Y}_{\top,t,i})$.
For $\tt{Y}_\top$ to not affect this execution, it must be the case that $\tt{Y}_{\top,t,i} \subseteq \tt{X}_{\exists,t,i} \cup \tt{X}_{\top,t,i}$, so that $\tt{X}_{\exists,t,i} \cup \tt{X}_{\top,t,i} \cup \pi_S(\tt{Y}_{\exists,t,i} \cup \tt{Y}_{\top,t,i}) = \tt{X}_{\exists,t,i} \cup \tt{X}_{\top,t,i} \cup \pi_S(\tt{Y}_{\exists,t,i})$.
Thus, as long as $\tt{Y}_\top \subseteq \tt{X}_\exists \cup \tt{X}_\top$, it suffices to keep $\tt{Y}_\exists$ and we may safely discard $\tt{Y}_\top$.

As the execution of the program continues, $\tt{Y}_\exists$ may grow as tuples are merged into it, forming a larger $\widehat{\tt{Y}} \supseteq \tt{Y}_\exists$.
Despite this growth, the execution is still unaffected, i.e., $\tt{X}_{\exists,t,i} \cup \tt{X}_{\top,t,i} \cup \pi_S(\widehat{\tt{Y}} \cup \tt{Y}_{\top,t,i}) = \tt{X}_{\exists,t,i} \cup \tt{X}_{\top,t,i} \cup \pi_S(\widehat{\tt{Y}})$, since we have ensured that $\tt{Y}_\top \subseteq \tt{X}_\exists \cup \tt{X}_\top$.
\end{example}

\begin{example}[Safety for Set Difference]
Suppose we have the rule $\tt{Z} ~\tt{<=}~ \tt{X} - \tt{Y}$ in $\mathfrak{P}$, and correspondingly $\tt{Z}_{TS} ~\tt{<=}~ ((\tt{X}_\exists \cup \tt{X}_\top) - (\tt{Y}_\exists \cup \tt{Y}_\top), \emptyset)$ in $\mathfrak{P}_{GC}$.
% , and $\tt{Z}_I ~\tt{<=}~ \tt{X}_I - \tt{Y}_I$ in $\mathfrak{P}_{iGC}$.
Suppose also that $\tt{X}_E$ has a key-augmented representation in $\mathfrak{P}_{iGC}$ while $\tt{Y}_I$ has a non-augmented representation.
As in the previous example, we desire that the execution of the rule is unaffected by deletion of tombstones.
In particular, it suffices to have
$(\tt{X}_{\exists,t,i} \cup \tt{X}_{\top,t,i}) - (\tt{Y}_{\exists,t,i} \cup \tt{Y}_{\top,t,i}) = \tt{X}_{\exists,t,i} - \tt{Y}_{\exists,t,i}$.
We observe that if a tuple $x$ appears in both $\tt{X}$ and $\tt{Y}$, we can safely tombstone $x$ from $\tt{X}$ without affecting the set difference.
Once we have $x \in \tt{X}_\top$, it is not in $\tt{X}_\exists$, so we can also safely tombstone $x$ from $\tt{Y}$.
Hence, it suffices to ensure $\tt{Y}_\top \subseteq \tt{X}_\top \subseteq \tt{Y}_\top \cup \tt{Y}_\exists$.
We may then discard $\tt{X}_\top$ and $\tt{Y}_\top$ and keep only $\tt{X}_\exists$, $\tt{Y}_\exists$.
We also keep $\tt{X}_!$ in the key-augmented representation; its use will be demonstrated in the following paragraph.

As the execution of the program continues, both $\tt{X}_\exists$ and $\tt{Y}_\exists$ may grow as tuples are merged into them, former larger $\widehat{\tt{X}} \supseteq \tt{X}_\exists$ and $\widehat{\tt{Y}} \supseteq \tt{Y}_\exists$.
Since we use the key-augmented representation for $\tt{X}$, tuples that we add to $\tt{X}_\exists$ must be previously unseen tuples not in $\tt{X}_!$, which implies $\widehat{\tt{X}} \cap \tt{X}_\top = \emptyset$.
If we ensure that $\tt{Y}_\top \subseteq \tt{X}_\top \subseteq \tt{Y}_\top \cup \tt{Y}_\exists$, we can observe that
$
(\widehat{\tt{X}} \cup \tt{X}_\top) - (\widehat{\tt{Y}} \cup \tt{Y}_\top)
= \widehat{\tt{X}} - (\widehat{\tt{Y}} \cup \tt{Y}_\top)
= \widehat{\tt{X}} - \widehat{\tt{Y}}
$,
where the second equality holds because $\tt{Y}_\top \cap \widehat{\tt{X}} \subseteq \tt{X}_\top \cap \widehat{\tt{X}} = \emptyset$.
Thus, the execution of the set difference rule is unaffected by the deletion of tombstones.
\end{example}


\subsection{Properties of Logical GC rules}
We now formalize these intuitions about the safety of logical garbage collection rules with the below invariant.


\begin{invariant}[Merge GC Invariant]\label{inv:merge_gc}
Consider any merge or deferred merge rule 
\[R: \tt{B} ~\tt{<op>}~ \tt{f(}\tt{A}_1,\dots,\tt{A}_n, \tt{U}_1, \dots, \tt{U}_m\tt{)}\]
where $\tt{<op>}$ is either $\tt{<=}$ or $\tt{<+}$,
the sets $\tt{A}_1,\dots,\tt{A}_n \in \mathbb{W}$,
the sets $\tt{U}_1,\dots,\tt{U}_m \in \mathbb{S}$,
and RHS is not a channel.
We require that, 
$\forall \langle \widehat{\tt{A}}_1, \dots, \widehat{\tt{A}}_n\rangle \geq \langle \tt{A}_{1,\exists},\dots,\tt{A}_{n,\exists} \rangle,$
$\quad
\forall \langle \widehat{\tt{U}}_1 , \dots, \widehat{\tt{U}}_m \rangle \geq \langle \tt{U}_1, \dots, \tt{U}_m \rangle$
such that $\emptyset = \widehat{\tt{U}}_1 \cap \tt{U}_{1,\top} = \dots = \widehat{\tt{U}}_1 \cap \tt{U}_{m,\top}$:
\begin{align}
\tt{f(}\widehat{\tt{A}}_1,\dots,\widehat{\tt{A}}_n, \widehat{\tt{U}}_{1,\exists},\dots,\widehat{\tt{U}}_{n,\exists}\tt{)}
\cup \tt{B}_\exists\cup \tt{B}_\top
~=~
\tt{f(}
\widehat{\tt{A}}_1 \cup \tt{A}_{1,\top},\dots,\widehat{\tt{A}}_n \cup \tt{A}_{n,\top}, \widehat{\tt{U}}_1 \cup \tt{U}_{1,\top},\dots,\widehat{\tt{U}}_n \cup \tt{U}_{n,\top}\tt{)} 
\cup \tt{B}_\exists \cup \tt{B}_\top.
\label{eq:inv_merge_gc}
\end{align}
\end{invariant}

We are interested in keeping around only $\langle \tt{A}_{1,\exists},\dots,\tt{A}_{n,\exists} \rangle$ instead of $\langle \tt{A}_{1,\exists} \cup \tt{A}_{1,\top}, \dots, \tt{A}_{n,\exists}  \cup \tt{A}_{n,\top}\rangle$.
However, in the process of executing $\mathfrak{P}_{GC}$, we may find ourselves with a larger\footnote{
  The restrictions of Edelweiss ensures that lattices can only grow over time.
    In Dedalus$^+$ terminology, the program is `temporally inflationary'.
} lattice $\langle \widehat{\tt{A}}_1, \dots, \widehat{\tt{A}}_n\rangle \geq \langle \tt{A}_{1,\exists},\dots,\tt{A}_{n,\exists} \rangle$.
In such a case, we wish to ensure that the execution of $R$ in $\mathfrak{P}_{GC}$ (LHS of \eqref{eq:inv_merge_gc}) matches the corresponding execution of $R$ in $\mathfrak{P}$ (RHS of \eqref{eq:inv_merge_gc}) as if we had not reclaimed $\tt{A}_{1,\top}, \dots, \tt{A}_{n,\top}$.
As we will see in the examples below, the invariant often reduces to a simple intuitive assertion.

A logical garbage collection rule is `safe' if it does not tombstone any tuple from output relations, and maintains the Merge GC Invariant.

\begin{property}[GC Rule Output Respect]
A garbage collection rule $\tt{GC}$ \emph{respects output relations} if it does not tombstone tuples from output relations.
That is, letting $\langle \tt{A}_{1,TS}^*, \dots, \tt{A}_{n,TS}^* \rangle = \tt{GC}(\tt{A}_{1,TS}, \dots, \tt{A}_{n,TS};$ $\tt{B}_{TS}, \tt{D}_{1,TS}, \dots, \tt{D}_{l,TS})$, then whenever $\tt{A}_j$ is an output relation and $\tt{A}_{n,\top} = \emptyset$, it must be the case that $\tt{A}_{j,\top}^* = \emptyset$
\end{property}

\begin{property}[Merge GC Safety]
\label{property:gc_safety}
Let $R: \tt{B} ~\tt{<op>}~ \tt{f(}\tt{A}_1,\dots,\tt{A}_n, \tt{U}_1, \dots, \tt{U}_m\tt{)}$ be a merge or deferred merge rule, where $\tt{A}_1, \dots, \tt{A}_n \in \mathbb{W}$ and $\tt{U}_1, \dots, \tt{U}_m \in \mathbb{S}$.
A garbage collection rule $\tt{GC}$ is safe for $R$ if it respects output relations, and maintains the Merge GC Invariant \ref{inv:merge_gc} for $R$.
That is, if
\begin{enumerate}
\item $\langle \tt{A}_{1,TS}, \dots, \tt{A}_{n,TS}, \tt{U}_{1,TS}, \dots, \tt{U}_{m,TS} \rangle$ and $\tt{B}_{TS}$ satisfies Merge GC Invariant \ref{inv:merge_gc} for $R$, and
\item $\tt{B}_\exists \cup \tt{B}_\top \supseteq \tt{f}\tt{(}\tt{A}_{1,\exists} \cup \tt{A}_{1,\top},\dots,\tt{A}_{n,\exists} \cup \tt{A}_{n,\top}, \tt{U}_{1,\exists} \cup \tt{U}_{1,\top},\dots,\tt{U}_{n,\exists} \cup \tt{U}_{n,\top}\tt{)}$,
\end{enumerate}
then $\langle \tt{A}_{1,TS}^*, \dots, \tt{A}_{n,TS}^*, \tt{U}_{1,TS}^*, \dots, \tt{U}_{m,TS}^* \rangle = \tt{GC}(\tt{A}_{1,TS}, \dots, \tt{A}_{n,TS},$ $\tt{U}_{1,TS}, \dots, \tt{U}_{m,TS};$ $\tt{B}_{TS}, \tt{D}_{1,TS}, \dots, \tt{D}_{l,TS})$ and $\tt{B}_{TS}$ also satisfies the Merge GC Invariant \ref{inv:merge_gc}.
\end{property}

We will also require $\mathfrak{P}_{GC}$ to be monotone so that it can be executed coordination-free whenever $\mathfrak{P}$ is itself monotone.

\begin{property}[GC Rule Monotonicity]
% Suppose $\tt{A}_{1,TS}, \dots, \tt{A}_{n,TS}, \tt{B}_{1,TS}, \dots, \tt{B}_{m,TS}$ satisfies Merge GC Invariant \ref{inv:merge_gc}.
% Suppose that $\tt{A}'_{1,TS}, \dots, \tt{A}'_{n,TS}, \tt{B}'_{1,TS}, \dots, \tt{B}'_{m,TS}$ also satisfies Merge GC Invariant \ref{inv:merge_gc} and 
Suppose that $\tt{A}'_{i,TS} \geq \tt{A}_{i,TS}$ and $\tt{B}'_{j,TS} \geq \tt{B}_{j,TS}$.
Then, $\tt{GC}$ is \emph{monotone} if $\tt{GC}(\tt{A}_{1,TS}, \dots, \tt{A}_{n,TS}; \tt{B}_{1,TS}, \dots, \tt{B}_{m,TS}) \leq \tt{GC}(\tt{A}'_{1,TS}, \dots, \tt{A}'_{n,TS}; \tt{B}'_{1,TS}, \dots, \tt{B}'_{m,TS})$.
\end{property}

Lastly, it is natural to expect that logical garbage collection rules neither creates new tuples nor loses them.

\begin{property}[GC Rule Conservation]
\label{property:gc_conservative}
For a garbage collection rule $\tt{GC}$, denote
$\langle \tt{A}_{1,TS}^*, \dots, \tt{A}_{n,TS}^* \rangle = \tt{GC}(\tt{A}_{1,TS}, \dots, \tt{A}_{n,TS}; \tt{B}_{1,TS}, \dots, \tt{B}_{m,TS})$.
Then, $\tt{GC}$ is \emph{conservative} if $\tt{A}_{j,\exists}^* \subseteq \tt{A}_{j,\exists}$ and $\tt{A}_{j,\exists}^* \cup \tt{A}_{j,\top}^* = \tt{A}_{j,\exists} \cup \tt{A}_{j,\top}$ for all $j=1,\dots,n$.
That is, $\tt{GC}$ only moves tuples from $\exists$ to $\top$.
\end{property}

We will require that all logical garbage collection rules be conservative, monotone, safe, and respects outputs.

\subsection{Examples of Logical GC rules}
\label{sec:logical:example}
A number of example logical garbage collection rules are presented below.
Each of these rules are monotone, conservative, and safe for some rule $R$.
\begin{example}[Trivial GC]
$\tt{GCNone}(\tt{A}_{1,TS}, \dots, \tt{A}_{n,TS}; \tt{B}_{1,TS}, \dots, \tt{B}_{m,TS}) = \langle \tt{A}_{1,TS}, \dots, \tt{A}_{n,TS} \rangle$.
\end{example}
The trivial GC returns the tombstone lattices unchanged.
If $\tt{A}_1, \dots, \tt{A}_n$ are all output relations, then $\tt{GCNone}$ is the only garbage collection rule that respects output relations.

% \begin{example}[Indicator GC]
% Consider the indicator function $\tt{f}$ with $\tt{f}\tt{(}\tt{X}\tt{)} = \emptyset$ if $\tt{X} = \emptyset$ and $\{1\}$ otherwise, and $R: \tt{B} ~\tt{<=}~ \tt{f}\tt{(}\tt{X}\tt{)}$.
% Suppose that $\emptyset \subseteq \tt{X} \subseteq \{a,b,c\}$.
% Define $\tt{GCInd}_i(\tt{X}_{TS}; \tt{B}_{TS}) = (\tt{X}_\exists - \tt{X}_i^*, \tt{X}_\top \cup \tt{X}_i^*)$, for $i=1,2,3$ and
% \begin{align*}
% \tt{X}_1^* = \begin{cases}
% \emptyset & \text{if $|\tt{X}_\exists \cup \tt{X}_\top| \leq 1$} \\
% a & \text{if $\tt{X}_\exists \cup \tt{X}_\top = \{a, b\}$}\\
% c & \text{if $\tt{X}_\exists \cup \tt{X}_\top = \{a, c\}$}\\
% b & \text{if $\tt{X}_\exists \cup \tt{X}_\top = \{b, c\}$}\\
% a,b & \text{if $\tt{X}_\exists \cup \tt{X}_\top = \{a, b, c\}$}\\
% \end{cases}
% &&
% \tt{X}_2^* = \begin{cases}
% \emptyset & \text{if $|\tt{X}_\exists \cup \tt{X}_\top| \leq 1$} \\
% a & \text{if $\tt{X}_\exists \cup \tt{X}_\top = \{a, b\}$}\\
% c & \text{if $\tt{X}_\exists \cup \tt{X}_\top = \{a, c\}$}\\
% b & \text{if $\tt{X}_\exists \cup \tt{X}_\top = \{b, c\}$}\\
% a,b,c & \text{if $\tt{X}_\exists \cup \tt{X}_\top = \{a, b, c\}$}\\
% \end{cases}
% &&
% \tt{X}_3^* = \begin{cases}
% \emptyset & \text{if $|\tt{X}_\exists \cup \tt{X}_\top| \leq 1$} \\
% a & \text{if $\tt{X}_\exists \cup \tt{X}_\top = \{a, b\}$}\\
% a & \text{if $\tt{X}_\exists \cup \tt{X}_\top = \{a, c\}$}\\
% b & \text{if $\tt{X}_\exists \cup \tt{X}_\top = \{b, c\}$}\\
% a,b & \text{if $\tt{X}_\exists \cup \tt{X}_\top = \{a, b, c\}$}\\
% \end{cases}
% \end{align*}
% Then $\tt{GCInd}_1$ is safe for $R$ but not monotone, $\tt{GCInd}_2$ is monotone but not safe for $R$, and $\tt{GCInd}_3$ is monotone and safe for $R$.
% \end{example}
% $\tt{GCInd}_1$ is not monotone because $\tt{GCInd}_1((\{a,c\}, \emptyset); \tt{B}_{TS}) \not\leq \tt{GCInd}_1((\{a,b,c\}, \emptyset); \tt{B}_{TS}); \tt{B}_{TS})$.
% $\tt{GCInd}_2$ is not safe because garbage collection on $\{a, b, c\}$ causes 
% 
% GC on indicator: $\emptyset -> 0$, non-$\emptyset -> 1$.
% If GC(a,b) = a, GC(a,c) = c, GC(b,c) = b, GC(a,b,c) = a,b then satisfies GCI but not monotone.
% If GC(a,b) = a, GC(a,c) = c, GC(b,c) = b, GC(a,b,c) = a,b,c then not satisfies GCI but monotone.
% If GC(a,b) = a, GC(a,c) = a, GC(b,c) = b, GC(a,b,c) = a,b then satisfies GCI and monotone.

\begin{example}[Copy GC]
Suppose $\tt{A}$ appears as input on the RHS of rules $R_i: \tt{B}_i ~\tt{<op>}~ \tt{A}$, where $\tt{<op>}$ is either a merge or deferred merge.
Then $\tt{GCCopy}(\tt{A}_{TS}; \tt{B}_{1,TS}, \dots, \tt{B}_{m,TS}) = \left(\tt{A}_\exists - \tt{B}_\cap, \tt{A}_\top \cup \tt{B}_\cap\right)$,
where $\tt{B}_\cap = \tt{A}_\exists \cap \bigcap_{i=1}^m (\tt{B}_{i,\exists} \cup \tt{B}_{i,\top})$,
is a monotone, conservative garbage collection rule that is safe for $R_1, \dots, R_n$.
\end{example}
Clearly, $\tt{GCCopy}$ is conservative.
For the copy rule, the Merge GC Invariant \ref{inv:merge_gc} reduces to $\forall i, \tt{A}_\exists \cup \tt{B}_{i,\exists} \cup \tt{B}_{i,\top} = \tt{A}_\exists \cup \tt{A}_\top \cup \tt{B}_{i,\exists} \cup \tt{B}_{i,\top}$, which is the case iff $\tt{A}_\top \subseteq \tt{B}_{i,\exists} \cup \tt{B}_{i,\top}$, since $\tt{A}_\exists \cap \tt{A}_\top = \emptyset$.
That is, tombstoned tuples are those that have been copied to all $\tt{B}_i$'s.
Hence, if $\tt{A}_{TS}, \tt{B}_{i,TS}$ satisfies Invariant \ref{inv:merge_gc}, we have $\tt{A}_\top \subseteq \tt{B}_{i,\exists} \cup \tt{B}_{i,\top}$, and so $\tt{A}_\top \cup \tt{B}_\cap = \tt{A}_\top \cup \left(\tt{A}_\exists \cap \bigcap_{i=1}^m (\tt{B}_{i,\exists} \cup \tt{B}_{i,\top})\right) \subseteq \tt{B}_{i,\exists} \cup \tt{B}_{i,\top}$.
Therefore, $\tt{GCCopy}$ is safe for $R_i$.

Suppose $\tt{A}_\text{TS}' \geq \tt{A}_\text{TS}$, i.e., $\tt{A}_\top' \supseteq \tt{A}_\top$ and $\tt{A}_\top' \cup \tt{A}_\exists' \supseteq \tt{A}_\top \cup \tt{A}_\exists$.
Suppose also $\tt{A}_\text{TS}'$ satisfies the Merge GC Invariant \ref{inv:merge_gc} with respect to $\tt{B}_{i,\text{TS}}' \geq \tt{B}_{i,\text{TS}}$.
Then
\begin{align*}
\tt{A}_\top \cup \left(\tt{A}_\exists \cap \bigcap_{i=1}^m (\tt{B}_{i,\exists} \cup \tt{B}_{i,\top})\right)
&\subseteq \tt{A}_\top' \cup \left((\tt{A}_\exists' \cup \tt{A}_\top') \cap \bigcap_{i=1}^m (\tt{B}_{i,\exists}' \cup \tt{B}_{i,\top}')\right)
% \\
% &= \tt{A}_\top' \cup \left(\tt{A}_\exists' \cap \bigcap_{i=1}^m (\tt{B}_{i,\exists}' \cup \tt{B}_{i,\top}')\right) \cup \left(\tt{A}_\top' \cap \bigcap_{i=1}^m (\tt{B}_{i,\exists}' \cup \tt{B}_{i,\top}')\right)\\
&= \tt{A}_\top' \cup \left(\tt{A}_\exists' \cap \bigcap_{i=1}^m (\tt{B}_{i,\exists}' \cup \tt{B}_{i,\top}')\right).
\end{align*}
Furthermore, it follows from the GC Rule Conservation Property \ref{property:gc_conservative}, we also have that $(\tt{A}_\exists - \tt{B}_\cap) \cup (\tt{A}_\top \cup \tt{B}_\cap) \subseteq (\tt{A}_\exists' - \tt{B}_\cap') \cup (\tt{A}_\top' \cup \tt{B}_\cap')$.
Hence, $\tt{GCcopy}$ is monotone.

~

% GC-Max: GC on A, B gives some tombstones larger than GC on any A' < A, B' < B, and is of largest cardinality among all such tombstones.
% Note: GC-Max is not unique; but since monotone, will still give a unique minimal model for each GC-Max rule.

\begin{example}[Union GC]
\label{ex:union_gc}
Consider the union $R: \tt{B} ~\tt{<op>}~ \tt{A} \cup \tt{C}$, where $\tt{<op>}$ is a merge or deferred merge.
Then $\tt{GCUnion}(\tt{A}_{TS}, \tt{C}_{TS}; \tt{B}_{TS}) = \langle (\tt{A}_\exists - \tt{A}^*, \tt{A}_\top \cup \tt{A}^*), (\tt{C}_\exists - \tt{C}^*, \tt{C}_\top \cup \tt{C}^*) \rangle$ where $\tt{A}^* = \tt{A}_\exists \cap  (\tt{B}_\exists \cup \tt{B}_\top)$ and $\tt{C}^* = \tt{C}_\exists \cap  (\tt{B}_\exists \cup \tt{B}_\top)$ is safe for $R$.
\end{example}
Invariant \ref{inv:merge_gc} for the union rule reduces to $\tt{A}_\top \cup \tt{C}_\top \subseteq \tt{A}_\exists \cup \tt{C}_\exists \cup \tt{B}_\exists \cup \tt{B}_\top$.
Then,
\begin{align*}
\tt{A}_\top \cup \tt{A}^* \cup \tt{C}_\top \cup \tt{A}^*
&\quad\subseteq\quad \tt{A}_\exists \cup \tt{C}_\exists \cup \tt{B}_\exists \cup \tt{B}_\top\\
&\quad=\quad (\tt{A}_\exists \cap (\tt{B}_\exists \cup \tt{B}_\top)^c) \cup (\tt{C}_\exists \cap (\tt{B}_\exists \cup \tt{B}_\top)^c) \cup \tt{B}_\exists \cup \tt{B}_\top\\
&\quad=\quad (\tt{A}_\exists - \tt{A}^*) \cup (\tt{C}_\exists - \tt{A}^*) \cup \tt{B}_\exists \cup \tt{B}_\top\\
\end{align*}
so $\tt{GCUnion}$ is safe for $R$.

\begin{example}[Intersection GC]
\label{ex:intersection_gc}
Consider the union $R: \tt{B} ~\tt{<op>}~ \tt{A} \cap \tt{C}$, where $\tt{<op>}$ is a merge or deferred merge.
Then $\tt{GCIntersect}(\tt{A}_{TS}, \tt{C}_{TS}; \tt{B}_{TS}) = \langle (\tt{A}_\exists - \tt{A}^*, \tt{A}_\top \cup \tt{A}^*), (\tt{C}_\exists - \tt{C}^*, \tt{C}_\top \cup \tt{C}^*) \rangle$ where $\tt{A}^* = \tt{A}_\exists \cap  (\tt{B}_\exists \cup \tt{B}_\top)$ and $\tt{C}^* = \tt{C}_\exists \cap  (\tt{B}_\exists \cup \tt{B}_\top)$ is safe for $R$.
\end{example}
Note that $\tt{GCIntersect} = \tt{GCUnion}$, both of which tombstones tuples from $\tt{A}_\exists$ and $\tt{C}_\exists$ after they have appeared in $\tt{B}_\exists \cup \tt{B}_\top$.
In this case, Invariant \ref{inv:merge_gc} reduces to $\tt{A}_\top \in \tt{B}_\exists \cup \tt{B}_\top$ and $\tt{C}_\top \in \tt{B}_\exists \cup \tt{B}_\top$.
Then $\tt{A}_\top \cup \tt{A}^* \subseteq \tt{B}_\exists \cup \tt{B}_\top$ and $\tt{C}_\top \cup \tt{C}^* \subseteq \tt{B}_\exists \cup \tt{B}_\top$, so $\tt{GCIntersect}$ is safe for $R$.

\begin{example}[Project GC]
\label{ex:project_gc}
Consider the projection $R: \tt{B} ~\tt{<op>}~ \pi_S(\tt{A})$, where $\tt{<op>}$ is a merge or deferred merge.
Then $\tt{GCProject}(\tt{A}_{TS}; \tt{B}_{TS}) = (\tt{A}_\exists - \tt{A}^*, \tt{A}_\top \cup \tt{A}^*)$ where $\tt{A}^* = \{a \in \tt{A}_\exists \cup \tt{A}_\top ~:~ \pi_S(\{a\}) \subseteq \tt{B}_\exists \cup \tt{B}_\top\}$
is safe for $R$.
\end{example}
Simple relational algebraic calculations would show that the Merge GC Invariant \ref{inv:merge_gc} reduces to $\pi_S(\tt{A}_\top) \subseteq \tt{B}_\exists \cup \tt{B}_\top$.
By definition, $\pi_L(\tt{A}^*) \subseteq \tt{B}_\exists \cup \tt{B}_\top$.
Hence, if $\tt{A}_{TS}$ and $\tt{B}_{TS}$ satisfies the invariant, then $\pi_L(\tt{A}_\top \cup \tt{A}^*) = \pi_L(\tt{A}_\top) \cup \pi_L(\tt{A}^*) \subseteq \tt{B}_\exists \cup \tt{B}_\top$, so $\tt{GCProject}$ is safe for $R$. 

\begin{example}[Select GC]
Consider the selection $R: \tt{B} ~\tt{<op>}~ \sigma_P(\tt{A})$, where $\tt{<op>}$ is either a merge or deferred merge.
Then $\tt{GCSelect}(\tt{A}_{TS}; \tt{B}_{TS}) = (\tt{A}_\exists - \tt{A}^*, \tt{A}_\top \cup \tt{A}^*)$ where $\tt{A}^* = \sigma_{\neg P}(\tt{A}_\exists \cup \tt{A}_\top) \cup (\sigma_P(\tt{A}_\exists \cup \tt{A}_\top) \cap (\tt{B}_\exists \cup \tt{B}_\top))$ is safe for $R$.
\end{example}
Similar to Example \ref{ex:project_gc}, the Merge GC Invariant \ref{inv:merge_gc} for selection reduces to $\sigma_P(\tt{A}_\top) \subseteq \tt{B}_\exists \cup \tt{B}_\top$.
If $\tt{A}_{TS}$ and $\tt{B}_{TS}$ satisfy the invariant, then
\begin{align*}
\sigma_P(\tt{A}_\top \cup \tt{A}^*)
&= \sigma_P(\tt{A}_\top) \cup \sigma_P(\sigma_{\neg P}(\tt{A}_\exists \cup \tt{A}_\top) \cup (\sigma_P(\tt{A}_\exists \cup \tt{A}_\top) \cap (\tt{B}_\exists \cup \tt{B}_\top)))\\
&= \sigma_P(\tt{A}_\top) \cup (\sigma_P(\tt{A}_\exists \cup \tt{A}_\top) \cap (\tt{B}_\exists \cup \tt{B}_\top))\\
&= \sigma_P(\tt{A}_\top) \cup (\sigma_P(\tt{A}_\exists) \cap (\tt{B}_\exists \cup \tt{B}_\top))\\
&\subseteq \tt{B}_\exists \cup \tt{B}_\top,
\end{align*}
so $\tt{GCSelect}$ is safe for $R$.

\begin{example}[Linear GC]
Consider the rule $R: \tt{B} ~\tt{<op>}~ \tt{f}(\tt{A})$, where $\tt{f}$ is a linear function, i.e., $\tt{f}(\tt{A}) = \bigcup_{a\in \tt{A}} \tt{f}(\{a\})$.
Then $\tt{GCAll}(\tt{A}_{TS}; \tt{B}_{1,TS}, \dots, \tt{B}_{m,TS}) = (\emptyset, \tt{A}_\exists \cup \tt{A}_\top)$ is safe for $R$.
\end{example}
Since $\tt{f}$ is linear, Invariant \ref{inv:merge_gc} reduces to $\tt{f}\tt{(}\tt{A}_\top\tt{}) \subseteq \tt{B}_\exists \cup \tt{B}_\top$, but this is true under the condition $\tt{B}_\exists \cup \tt{B}_\top \supseteq \tt{f}\tt{(}\tt{A}_{\exists} \cup \tt{A}_{\top}\tt{)} = \tt{f}\tt{(}\tt{A}_\exists\tt{}) \cup \tt{f}\tt{(}\tt{A}_\top\tt{})$.
Thus, Invariant \ref{inv:merge_gc} is trivially true for linear $\tt{f}$.
Intuitively, a tuple $a \in \tt{A}_\exists$ is independent of all other tuples when $\tt{f}$ is linear, and hence, its complete effects will have been realized in $\tt{B}$ (either through a merge or deferred merge) by the point at which we perform garbage collection.
Therefore, we can always reclaim the entire set without impacting our results.

In particular, observe that the copy, union, select, and project rules of previous examples are all linear, and thus can be fully reclaimed at the end of each timestep.

\begin{example}[Join GC]
\label{ex:join_gc_no_manifests}
Consider the join $R: \tt{B} ~\tt{<op>}~ \tt{A} \bowtie_{\tt{A}.k_a = \tt{C}.k_c} \tt{C}$, where $\tt{<op>}$ is either a merge or deferred merge.
The only garbage collection rule that is safe for $R$ is the trivial $\tt{GCTrivial}(\tt{A}_{TS}, \tt{C}_{TS}; \tt{B}_{TS}) = \langle \tt{A}_{TS}, \tt{C}_{TS} \rangle$.
\end{example}
Without additional information, one cannot guarantee that a tuple in $\tt{A}$ will not match any future tuple in $\tt{C}$ and vice versa.
In some cases, however, one may be provided with the specific information that no more tuples matching the attribute will ever appear.

Formally, let $\mathcal{X}$ be the domain of possible tuples for a table $\tt{X}$, and denote the powerset of $\mathcal{X}$ as $\mathbb{P}(\mathcal{X})$.
Also let $\mathcal{K}$ be the domain of possible values of an attribute $\tt{X}.k$.
A manifest $p = (p.k, p.\tt{X}) \in \mathcal{K} \times \mathbb{P}(\mathcal{X})$ maps a key in $\mathcal{K}$ to a subset of $\mathbb{P}(\mathcal{X})$.
We denote by $\tt{X}$ by $\tt{P}_\tt{X}$ a table of manifests (with $k$ as its primary key) for $\tt{X}$, and say it is consistent with $\tt{X}$ if $\forall p \in \tt{P}_\tt{X}$, $p.\tt{X} \supseteq \{x \in \tt{X} ~:~ x.k = p.k\}$.
We will always assume that manifest tables are always consistent.

In the join rule above, we can reclaim from $\tt{A}$ given manifests on $\tt{C}$ and vice versa:

\begin{example}[Join with manifests]
Consider the join in Example \ref{ex:join_gc_no_manifests}, now augmented with manifest tables $\tt{P}_\tt{A}$ and $\tt{P}_\tt{C}$.
(If manifests are not available, we can always let $\tt{P} = \emptyset$ with no loss of generality.)
The garbage collection rule $\tt{GCJoin}(\tt{A}_{TS}, \tt{P}_{\tt{A},TS}, \tt{C}_{TS}, \tt{P}_{\tt{C},TS}; \tt{B}_{TS}) = \langle (\tt{A}_\exists - \tt{A}^*, \tt{A}_\top \cup \tt{A}^*), (\tt{P}_{\tt{A},\exists} - \tt{P}_\tt{A}^*, \tt{P}_{\tt{A},\top} \cup \tt{P}_\tt{A}^*), (\tt{C}_\exists - \tt{C}^*, \tt{C}_\top \cup \tt{C}^*), (\tt{P}_{\tt{C},\exists} - \tt{P}_\tt{C}^*, \tt{P}_{\tt{C},\top} \cup \tt{P}_\tt{C}^*) \rangle$, with
\begin{align*}
\tt{A}^* &= \left\{a \in \tt{A}_\exists \cup \tt{A}_\top ~:~ \exists p_c \in \tt{P}_{\tt{C},\exists} \cup \tt{P}_{\tt{C},\top}, \text{ such that } (a.k_a = p_c.k_c) \wedge \left(\{a\} \times p_c.\tt{C} \subseteq \tt{B}_\exists \cup \tt{B}_\top\right)\right\}\\
\tt{P}_\tt{A}^* &= \{p_a \in \tt{P}_{\tt{A},\exists} \cup \tt{P}_{\tt{A},\top} ~:~ \exists p_c \in \tt{P}_{\tt{C},\exists} \cup \tt{P}_{\tt{C},\top} \text{ such that } (p_a.k_a = p_c.k_c) \wedge \left(p_a.\tt{A} \times p_c.\tt{C} \subseteq \tt{B}_\exists \cup \tt{B}_\top\right)\}\\
\tt{C}^* &= \left\{c \in \tt{C}_\exists \cup \tt{C}_\top ~:~ \exists p_a \in \tt{P}_{\tt{A},\exists} \cup \tt{P}_{\tt{A},\top}, \text{ such that } (p_a.k_a = c.k_c) \wedge \left(p_a.\tt{A} \times \{c\} \subseteq \tt{B}_\exists \cup \tt{B}_\top\right)\right\}\\
\tt{P}_\tt{A}^* &= \{p_a \in \tt{P}_{\tt{A},\exists} \cup \tt{P}_{\tt{A},\top} ~:~ \exists p_c \in \tt{P}_{\tt{C},\exists} \cup \tt{P}_{\tt{C},\top} \text{ such that } (p_a.k_a = p_c.k_c) \wedge \left(p_a.\tt{A} \times p_c.\tt{C} \subseteq \tt{B}_\exists \cup \tt{B}_\top\right)\}
\end{align*}
is safe for the join.
\end{example}
In words, a tuple $a$ can be tombstoned if it matches a punctuation $p_c$ in $\tt{P}_\tt{C}$ \emph{and} the result of joining $\{a\}$ with $\tt{C}$ is already effected in $\tt{B}_{TS}$.
Hence, any $c \in \tt{C}$ either does not match $a$, or is already joined with $a$ in $\tt{B}$ and so will not affect any future join computation.
In addition, we tombstone a punctuation $p_a$ if it matches a punctuation $p_c$ and the join of their sets $p_a.\tt{A} \times p_c.\tt{C}$ is already effected in $\tt{B}_{TS}$.
It is a simple exercise to show that $\tt{GCJoin}$ tombstones a punctuation $p_a$ only when its set $p_a.\tt{A}$ is tombstoned (or does not exist yet), i.e., $(p_a \in \tt{P}_{\tt{A},\top}) \implies \tt{A}_\exists \cap p_a.\tt{A} = \emptyset \implies \sigma_{\tt{A}.k_a = p_a.k_a}(\tt{A}_\exists) = \emptyset$.

\begin{example}[DR+: Positive Difference Reclamation]
Consider the rule $R: \tt{B} ~\tt{<op>}~ \tt{A} - \tt{C}$, where $\tt{<op>}$ is a merge or deferred merge.
Then the rule $\tt{GCDR+}(\tt{A}_{TS}, \tt{C}_{TS}; \tt{B}_{TS}) = \langle (\tt{A}_\exists - \tt{A}^*, \tt{A}_\top \cup \tt{A}^*), \tt{C}_{TS}\rangle$
where $\tt{A}^* = (\tt{A}_\exists \cup \tt{A}_\top) \cap (\tt{C}_\exists \cup \tt{C}_\top)$
is safe for $R$.
\end{example}
For set differences, the Merge GC Invariant \ref{inv:merge_gc} reduces to
$\forall \widehat{\tt{A}} \supseteq \tt{A}_\exists, \forall \widehat{\tt{C}} \supseteq \tt{C}_\exists$: $\widehat{\tt{A}} - \widehat{\tt{C}} = (\widehat{\tt{A}} \cup \tt{A}_\top) - (\widehat{\tt{C}} \cup \tt{C}_\top)$.
In the case where $\widehat{\tt{A}} = \tt{A}_\exists$ and $\widehat{\tt{C}} = \tt{C}_\exists$, it implies that $\tt{A}_\top \subseteq \tt{C}_\exists \cup \tt{C}_\top$;
in the case where $\widehat{\tt{A}} = \tt{A}_\exists \cup \tt{C}_\top$ and $\widehat{\tt{C}} = \tt{A}_\exists \cup \tt{C}_\exists$, it implies that $\tt{C}_\top = \emptyset$.
Conversely $\tt{C}_\top = \emptyset$ and $\tt{A}_\top \subseteq \tt{C}_\exists \cup \tt{C}_\top$ suffices to satisfy the invariant.
One can now easily verify that $\tt{GCDR+}$ maintains the invariant.

\begin{example}[DR-: Negative Difference Reclamation]
\label{ex:logical:dr-}
Consider the rule $R: \tt{B} ~\tt{<op>}~ \tt{A} - \tt{C}$, where $\tt{<op>}$ is a merge or deferred merge, and $\tt{A}$ is a table with primary keys.
Then the rule $\tt{GCDR-}(\tt{A}_{TS}, \tt{C}_{TS}; \tt{B}_{TS}) = \langle (\tt{A}_\exists - \tt{A}^*, \tt{A}_\top \cup \tt{A}^*), (\tt{C}_\exists - \tt{C}^*, \tt{C}_\top \cup \tt{C}^*)\rangle$,
where $\tt{A}^* = (\tt{A}_\exists \cup \tt{A}_\top) \cap (\tt{C}_\exists \cup \tt{C}_\top)$ and $\tt{C}^* = (\tt{C}_\exists \cup \tt{C}_\top) \cap \tt{A}_\top$,
is safe for $R$.
\end{example}
Since primary keys are available for $\tt{A}$, the Merge GC Invariant \ref{inv:merge_gc} reduces to
$\forall \widehat{\tt{A}} \supseteq \tt{A}_\exists, \widehat{\tt{A}} \cap \tt{A}_\top = \emptyset, \forall \widehat{\tt{C}} \supseteq \tt{C}_\exists$: $\widehat{\tt{A}} - \widehat{\tt{C}} = (\widehat{\tt{A}} \cup \tt{A}_\top) - (\widehat{\tt{C}} \cup \tt{C}_\top)$.
In the case where $\widehat{\tt{A}} = \tt{A}_\exists$ and $\widehat{\tt{C}} = \tt{C}_\exists$, it implies that $\tt{A}_\top \subseteq \tt{C}_\exists \cup \tt{C}_\top$;
in the case where $\widehat{\tt{A}} = \tt{A}_\exists \cup (\tt{C}_\top - \tt{A}_\top)$ and $\widehat{\tt{C}} = \tt{A}_\exists \cup \tt{C}_\exists$, it implies that $\tt{C}_\top \subseteq \tt{A}_\top$.
Conversely $\tt{C}_\top \subseteq \tt{A}_\top \subseteq \tt{C}_\exists \cup \tt{C}_\top$ suffices to satisfy the invariant.
One can now easily verify that $\tt{GCDR-}$ maintains the invariant.





\subsection{Analysis of Logical GC Rewrite}
We now analyze the logical GC rewrite $\mathfrak{P}_{GC}$, showing that it produces the same output as $\mathfrak{P}$, and that $\mathfrak{P}_{GC}$ is coordination-free whenever $\mathfrak{P}$ is.

\subsubsection{Correctness}
\label{sec:logical:analysis:correctness}

\begin{lemma}
\label{lem:unchangedexists}
For every set $\tt{A}$, we have that $\tt{A}_{\top,t,i} = \tt{A}_{\top,t,0}$.
\end{lemma}
\begin{proof}
The proof is straightforward: all non-GC rules are of the form $(\tt{A}_\exists, \tt{A}_\top) ~\tt{<op>}~ (\tt{X}, \emptyset)$, and all GC rules are deferred merges.
Thus, $\tt{A}_\top$ is unchanged during an instantaneous run.
\end{proof}
Due to Lemma \ref{lem:unchangedexists}, we can write $\tt{A}_{\top,t}$ in place of $\tt{A}_{\top,t,i}$.

\begin{lemma}
\label{lem:merge_gc_inv}
The rewritten GC program $\mathfrak{P}_{GC}$ maintains the Merge GC Invariant \ref{inv:merge_gc} for any rule $R$.
\end{lemma}
\begin{proof}
We prove the lemma for any given $R$ by an induction on $t$ and $i$.
That is, we show for every $t$ and $i$,
$\forall \langle \widehat{\tt{A}}_1, \dots, \widehat{\tt{A}}_n\rangle \geq \langle \tt{A}_{1,\exists,t,i},\dots,\tt{A}_{n,\exists,t,i} \rangle$,
$\forall \langle \widehat{\tt{U}}_1, \dots, \widehat{\tt{U}}_n\rangle \geq \langle \tt{U}_{1,\exists,t,i},\dots,\tt{U}_{n,\exists,t,i} \rangle$,
such that
$\emptyset = \widehat{\tt{U}}_1 \cap \tt{U}_{1,\top,t,i} = \dots =  \widehat{\tt{U}}_n \cap \tt{U}_{n,\top,t,i}$,
\begin{align}
&
\tt{B}_{\exists,t,i} \cup \tt{B}_{\top,t}
\cup
\tt{f(}
  \widehat{\tt{A}}_1,
    \dots,
    \widehat{\tt{A}}_n,
  \widehat{\tt{U}}_1,
    \dots,
    \widehat{\tt{U}}_n
\tt{)}
\nonumber\\
=&
\tt{B}_{\exists,t,i} \cup \tt{B}_{\top,t}
\cup
\tt{f(}
  \widehat{\tt{A}}_1 \cup \tt{A}_{1,\top,t},
  \dots,
  \widehat{\tt{A}}_n \cup \tt{A}_{n,\top,t},
  \widehat{\tt{U}}_1 \cup \tt{U}_{1,\top,t},
  \dots,
  \widehat{\tt{U}}_n \cup \tt{U}_{n,\top,t}
\tt{)}. \label{eq:mergegcinv_withtime}
\end{align}
At $t=0$, $i=0$, we have $\tt{A}_{k,\top,0} = \tt{U}_{j,\top,0} = \tt{B}_{\top,0} = \emptyset$ for all $j=1,\dots,m$ and $k=1,\dots,n$, so the statement \eqref{eq:mergegcinv_withtime} is trivially true.

Assume that \eqref{eq:mergegcinv_withtime} is true at some $t$ and $i-1$.
The assignment at $t$, $i$, is derived from applying a merge operator to the assignment at $t$, $i-1$, and may only grow $\tt{A}_{1,\exists}, \dots, \tt{A}_{n,\exists}$, $\tt{U}_{1,\exists}, \dots, \tt{U}_{m,\exists}$, and $\tt{B}_\exists$,
but keeps $\tt{A}_{1,\top}, \dots, \tt{A}_{n,\top}$, $\tt{U}_{1,\top}, \dots, \tt{U}_{m,\top}$, and $\tt{B}_\top$ unchanged.
This also implies that $\widehat{\tt{U}}_k \cap \tt{U}_{k,\top,t,i-1} = \emptyset \implies \widehat{\tt{U}}_k \cap \tt{U}_{k,\top,t,i-1} = \emptyset$.
Write $\widetilde{\tt{B}} = \tt{B}_{\exists,t,i} - \tt{B}_{\exists,t,i-1}$.
Then, $\forall \langle \widehat{\tt{A}}_1, \dots, \widehat{\tt{A}}_n\rangle \geq \langle \tt{A}_{1,\exists,t,i},\dots,\tt{A}_{n,\exists,t,i} \rangle \geq \langle \tt{A}_{1,\exists,t,i-1},\dots,\tt{A}_{n,\exists,t,i-1} \rangle$,
$\forall \langle \widehat{\tt{U}}_1, \dots, \widehat{\tt{U}}_n\rangle \geq \langle \tt{U}_{1,\exists,t,i},\dots,\tt{U}_{n,\exists,t,i} \rangle \geq \langle \tt{U}_{1,\exists,t,i-1},\dots,\tt{U}_{n,\exists,t,i-1} \rangle$,
such that
$\forall j=1,\dots,m$, $\widehat{\tt{U}}_j \cap \tt{U}_{1,\top,t} = \emptyset$,
\begin{align*}
&\tt{B}_{\exists,t,i} \cup \tt{B}_{\top,t}
\cup
\tt{f(}
  \widehat{\tt{A}}_1,
    \dots,
    \widehat{\tt{A}}_n,
  \widehat{\tt{U}}_1,
    \dots,
    \widehat{\tt{U}}_n
\tt{)}\\
=&
\widetilde{\tt{B}} \cup \tt{B}_{\exists,t,i-1} \cup \tt{B}_{\top,t}
\cup
\tt{f(}
  \widehat{\tt{A}}_1,
    \dots,
    \widehat{\tt{A}}_n,
  \widehat{\tt{U}}_1,
    \dots,
    \widehat{\tt{U}}_n
\tt{)}\\
=&
\widetilde{\tt{B}} \cup \tt{B}_{\exists,t,i-1} \cup \tt{B}_{\top,t}
\cup
\tt{f(}
  \widehat{\tt{A}}_1 \cup \tt{A}_{1,\top,t},
  \dots,
  \widehat{\tt{A}}_n \cup \tt{A}_{n,\top,t},
  \widehat{\tt{U}}_1 \cup \tt{U}_{1,\top,t},
  \dots,
  \widehat{\tt{U}}_n \cup \tt{U}_{n,\top,t}
\tt{)}\\
=&
\tt{B}_{\exists,t,i} \cup \tt{B}_{\top,t}
\cup
\tt{f(}
  \widehat{\tt{A}}_1 \cup \tt{A}_{1,\top,t},
  \dots,
  \widehat{\tt{A}}_n \cup \tt{A}_{n,\top,t},
  \widehat{\tt{U}}_1 \cup \tt{U}_{1,\top,t},
  \dots,
  \widehat{\tt{U}}_n \cup \tt{U}_{n,\top,t}
\tt{)}
\end{align*}
where the second equality follows from the inductive hypothesis.

Next, we show that if \eqref{eq:mergegcinv_withtime} is true at some fixed point $t-1$ and $i=\iota$, then it is also true at $t$, $i=0$.
By design / assumption, there is only one garbage collection rule $\tt{GC}$ in $\mathfrak{P}_{GC}$ that has any of $\tt{A}_{1,TS},\dots,\tt{A}_{n,TS}$ on its LHS.
Note that we can express $\forall j=1,\dots,n$,
\begin{align*}
(\tt{A}_{j,\exists,t,0}, \tt{A}_{j,\top,t})
= (\tt{A}_{j,\exists,t-1,\iota}, \tt{A}_{j,\top,t-1})
\cup (\tt{A}_{j,\exists}^*, \tt{A}_{j,\top}^*)
\cup \bigcup_{k=1}^l (g_k(\tt{C}_{k1,TS,t-1,\iota}, \dots, \tt{C}_{kn_k,TS,t-1,\iota}), \emptyset)
\end{align*}
where $\langle \tt{A}_{1,TS}^*, \dots, \tt{A}_{n,TS}^*\rangle = \tt{GC}(\tt{A}_{1,TS,t-1,\iota}, \dots, \tt{A}_{n,TS,t-1,\iota}; \tt{B}_{1,TS,t-1,\iota}, \dots, \tt{B}_{m,TS,t-1,\iota})$.
By the conservative property of $\tt{GC}$, $(\tt{A}_{j,\exists,t-1,\iota}, \tt{A}_{j,\top,t-1}) \leq (\tt{A}_{j,\exists}^*, \tt{A}_{j,\top}^*)$, so the above equation can be written as
\begin{align*}
(\tt{A}_{j,\exists,t,0}, \tt{A}_{j,\top,t})
= (\tt{A}_{j,\exists}^*, \tt{A}_{j,\top}^*)
\cup \bigcup_{k=1}^l (g_k(\tt{C}_{k1,TS,t-1,\iota}, \dots, \tt{C}_{kn_k,TS,t-1,\iota}), \emptyset)
\end{align*}
which implies $\tt{A}_{j,\exists,t,0} \supseteq \tt{A}_{j,\exists}^*$ and $\tt{A}_{j,\top,t} = \tt{A}_{j,\top}^*$.

The same holds for $\tt{U}_j$ for $j=1,\dots,m$, which further implies that $\widehat{\tt{U}}_j \cap \tt{U}_{j,\top}^* = \emptyset \implies \widehat{\tt{U}}_j \cap \tt{U}_{j,\top,t} = \emptyset$.

Also, by the monotonicity of $\tt{<+}$, we see that $\tt{B}_{\exists,t,0} \cup \tt{B}_{\top,t} \supseteq \tt{B}_{\exists,t-1,\iota} \cup \tt{B}_{\top,t-1}$.
Let $\widetilde{\tt{B}} = (\tt{B}_{\exists,t,0} \cup \tt{B}_{\top,t}) - (\tt{B}_{\exists,t-1,\iota} \cup \tt{B}_{\top,t-1})$.
Note that
$\tt{B}_{\exists,t,0} \cup \tt{B}_{\top,t} \supseteq \tt{f}\tt{(}\tt{A}_{\exists,1,t-1,\iota} \cup \tt{A}_{1,\top,t-1},\dots,\tt{A}_{n,\exists,t-1,\iota} \cup \tt{A}_{n,\top,t-1}, \tt{U}_{1,\exists,t-1,\iota} \cup \tt{U}_{1,\top,t-1},\dots,\tt{U}_{n,\exists,t-1,\iota} \cup \tt{U}_{n,\top,t-1}\tt{)}$ ---
if $R$ is a merge rule, it is true because $t,\iota$ is a fixed point;
if $R$ is a deferred merge rule, then the inclusion holds by definition of $\tt{B}_{TS,t,0}$.

Observe that $\forall \langle \widehat{\tt{A}}_1, \dots, \widehat{\tt{A}}_n \rangle$ $\geq$ $\langle \tt{A}_{1,\exists,t,0}, \dots, \tt{A}_{n,\exists,t,0} \rangle$ $\geq$ $\langle \tt{A}_{1,\exists}^*, \dots, \tt{A}_{n,\exists}^* \rangle$,
$\forall \langle \widehat{\tt{U}}_1, \dots, \widehat{\tt{U}}_n \rangle$ $\geq$ $\langle \tt{U}_{1,\exists,t,0}, \dots, \tt{U}_{n,\exists,t,0} \rangle$ $\geq$ $\langle \tt{U}_{1,\exists}^*, \dots, \tt{U}_{n,\exists}^* \rangle$,
such that $\emptyset = \widetilde{\tt{U}}_j \cap \tt{U}_{j,\top}^* = \widetilde{\tt{U}}_{j,\top,t}$ for all $j = 1,\dots,m$,
\begin{align*}
&
\tt{B}_{\exists,t,0} \cup \tt{B}_{\top,t}
\cup \tt{f}(\widehat{\tt{A}}_1, \dots, \widehat{\tt{A}}_n, \widehat{\tt{U}}_1, \dots, \widehat{\tt{U}}_m)
\\
=&~
\widetilde{\tt{B}} \cup \tt{B}_{\exists,t-1,\iota} \cup \tt{B}_{\top,t-1}
\cup \tt{f}(\widehat{\tt{A}}_1, \dots, \widehat{\tt{A}}_n, \widehat{\tt{U}}_1, \dots, \widehat{\tt{U}}_m)
\\
=&~
\widetilde{\tt{B}} \cup \tt{B}_{\exists,t-1,\iota} \cup \tt{B}_{\top,t-1}
\cup \tt{f}(
\widehat{\tt{A}}_1 \cup \tt{A}_{1,\top}^*, \dots, \widehat{\tt{A}}_n \cup \tt{A}_{n,\top}^*,
\widehat{\tt{U}}_1 \cup \tt{U}_{1,\top}^*, \dots, \widehat{\tt{U}}_n \cup \tt{U}_{n,\top}^*)
&\text{(\tt{GC} safety)}
\\
=&~
\widetilde{\tt{B}} \cup \tt{B}_{\exists,t-1,\iota} \cup \tt{B}_{\top,t-1}
\cup \tt{f}(
\widehat{\tt{A}}_1 \cup \tt{A}_{1,\top,t}, \dots, \widehat{\tt{A}}_n \cup \tt{A}_{n,\top,t},
\widehat{\tt{U}}_1 \cup \tt{U}_{1,\top,t}, \dots, \widehat{\tt{U}}_n \cup \tt{U}_{n,\top,t})
\\
=&~
\tt{B}_{\exists,t,0} \cup \tt{B}_{\top,t}
\cup \tt{f}(
\widehat{\tt{A}}_1 \cup \tt{A}_{1,\top,t}, \dots, \widehat{\tt{A}}_n \cup \tt{A}_{n,\top,t},
\widehat{\tt{U}}_1 \cup \tt{U}_{1,\top,t}, \dots, \widehat{\tt{U}}_n \cup \tt{U}_{n,\top,t})
\end{align*}
which establishes that the Merge GC Invariant \ref{inv:merge_gc} holds at $t$ and $i=0$.
\end{proof}

\begin{rmk}
Lemma \ref{lem:merge_gc_inv} is required for showing that the instantiated GC program $\mathfrak{P}_{iGC}$ presented later is correct, but not used in the below Theorem \ref{thm:equivalence_of_sets} establishing correctness of $\mathfrak{P}_{GC}$.
\end{rmk}




It is easy to see that $\mathfrak{P}$ and $\mathfrak{P}_{GC}$ have equivalent output sets.
That is, every run of $\mathfrak{P}$ has a one-to-one corresponding run of $\mathfrak{P}_{GC}$ that produces lattices such that $\tt{A}_\exists \cup \tt{A}_\top = \tt{A}$.
And in particular, for output sets where no garbage collection is performed, we get $\tt{A}_\exists = \tt{A}$.

\begin{thm}
\label{thm:equivalence_of_sets}
For every set $\tt{A}$, we have that $\tt{A}_{t,i} = \tt{A}_{\exists,t,i} \cup \tt{A}_{\top,t}$.
\end{thm}
\begin{proof}
We will again prove the theorem via induction.

Clearly this is true for $t=i=0$.

Now suppose that it is true at some $t$ and $i-1$, and we will show it holds at $t$ and $i$.
During the instantaneous run, the only rules that we are concerned with are merges where $\tt{A}$ appears on the LHS.
If the rule is of the form $\tt{A} ~\tt{<=}~ \overline{\tt{C}}$ with a channel on the RHS, then
\begin{align*}
\tt{A}_{t,i}
= \tt{A}_{t,i-1} \cup \overline{\tt{C}}_{t}
= \tt{A}_{\exists,t,i-1} \cup \tt{A}_{\top,t} \cup \overline{\tt{C}}_{t}
= \tt{A}_{\exists,t,i} \cup \tt{A}_{\top,t}.
\end{align*}
If the merge rule is of the form $\tt{A} ~\tt{<=}~ \tt{f}\tt{(}\tt{C}_1,\dots,\tt{C}_n\tt{)}$, then
\begin{align*}
\tt{A}_{t,i}
&= \tt{A}_{t,i-1} \cup \tt{f}\tt{(}\tt{C}_{1,t,i-1},\dots,\tt{C}_{n,t,i-1}\tt{)}\\
&= \tt{A}_{\exists,t,i-1} \cup \tt{A}_{\top,t} \cup \tt{f}\tt{(}\tt{C}_{1,\exists,t,i-1}\cup\tt{C}_{1,\top,t},\dots,\tt{C}_{n,\exists,t,i-1}\cup\tt{C}_{n,\top,t}\tt{)}\\
&= \tt{A}_{\exists,t,i} \cup \tt{A}_{\top,t},
\end{align*}
where we used the inductive hypothesis in the second equality.

Suppose $\mathfrak{P}$ reaches a fixed point for time $t$ at $i=\iota$.
We will next show that $\tt{A}_{t-1,\iota} = \tt{A}_{\exists,t-1,\iota} \cup \tt{A}_{\top,t-1}$ implies $\tt{A}_{t,0} = \tt{A}_{\exists,t,0} \cup \tt{A}_{\top,t}$.
We must first establish that the rewritten program will reach a fixed point for timestep $t$.

\begin{claim}
If $\mathfrak{P}$ reaches a fixed point at $t$, $\mathfrak{P}_{GC}$ also reaches a fixed point.
(This statement should be made more formal.)
\end{claim}
\begin{claimproof}
From the previous argument, we know that $\tt{A}_{\exists,t,j} \cup \tt{A}_{\top,t,i} = \tt{A}_{t,j}$ for all $j \geq \iota$.
But $\tt{A}_{t,j} = \tt{A}_{t,\iota}$ and $\tt{A}_{\top,t,i} = \tt{A}_{\top,t,0}$ are constant independent of $j$, which implies $\tt{A}_{\exists,t,j} = \tt{A}_{\exists,t,\iota}$.
Hence, $\mathfrak{P}_{GC}$ also reaches at fixed point at $t$, $i=\iota$.
\end{claimproof}

Finally, suppose we have $\tt{A}_{t-1,\iota} = \tt{A}_{\exists,t-1,\iota} \cup \tt{A}_{\top,t-1}$.
Suppose $\tt{A}$ appears in the LHS of $m$ deferred merge rules in $\mathfrak{P}$: $\tt{A} ~\tt{<+}~ \tt{f}_j\tt{(}\tt{C}_{j,1},\dots,\tt{C}_{j,n_j}\tt{)}$.
Additionally, in $\mathfrak{P}_{GC}$, $\tt{A}_{TS}$ appears in (at most) one GC rule $\langle \tt{A}_{TS}, \tt{D}_{1,TS}, \dots, \tt{D}_{k,TS} \rangle ~\tt{<+}~ \tt{GC}(\tt{A}_{TS}, \tt{D}_{1,TS}, \dots, \tt{D}_{k,TS}; \tt{B}_{1,TS}, \dots, \tt{B}_{k',TS})$.
Let $\langle \tt{A}_{TS}^*, \tt{D}_{1,TS}^*, \dots, \tt{D}_{k,TS}^* \rangle = \tt{GC}(\tt{A}_{TS,t-1,\iota}, \tt{D}_{1,TS,t-1,\iota}, \dots, \tt{D}_{k,TS,t-1,\iota}; \tt{B}_{1,TS,t-1,\iota}, \dots, \tt{B}_{k',TS,t-1,\iota})$, and note that $\tt{A}_{\exists,t-1,\iota} \cup \tt{A}_{\top,t-1} = \tt{A}_{\exists}^* \cup \tt{A}_{\top}^*$ by the conservation property of $\tt{GC}$.
Then,
\begin{align*}
\tt{A}_{t,0}
=& \tt{A}_{t-1,\iota} \cup \bigcup_{j=1}^m \tt{f}_j\tt{(}\tt{C}_{j,1,t-1,\iota},\dots,\tt{C}_{j,n_j,t-1,\iota}\tt{)}\\
=& \tt{A}_{\exists,t-1,\iota} \cup \tt{A}_{\top,t-1}
\cup \bigcup_{j=1}^m \tt{f}_j\tt{(}\tt{C}_{j,1,\exists,t-1,\iota} \cup \tt{C}_{j,1,\top,t-1}, \dots, \tt{C}_{j,n_j,\exists,t-1,\iota} \cup \tt{C}_{j,n_j,\top,t-1}\tt{)}\\
=& \tt{A}_{\exists,t-1,\iota} \cup \tt{A}_{\top,t-1}
\cup \tt{A}_{\exists}^* \cup \tt{A}_{\top}^*
\cup \bigcup_{j=1}^m \tt{f}_j\tt{(}\tt{C}_{j,1,\exists,t-1,\iota} \cup \tt{C}_{j,1,\top,t-1}, \dots, \tt{C}_{j,n_j,\exists,t-1,\iota} \cup \tt{C}_{j,n_j,\top,t-1}\tt{)}\\
=& \tt{A}_{\exists,t,0} \cup \tt{A}_{\top,t}.
\end{align*}
Thus, we have established that $\tt{A}_{t-1,\iota} = \tt{A}_{\exists,t-1,\iota} \cup \tt{A}_{\top,t-1}$ implies $\tt{A}_{t,0} = \tt{A}_{\exists,t,0} \cup \tt{A}_{\top,t}$.
\end{proof}

For output sets, we never perform garbage collection, and hence always have equivalence of the sets in $\mathfrak{P}$ and $\mathfrak{P}_{GC}$.
The below corollary formally states this.
\begin{cor}
\label{cor:equivalence_of_output}
If a set $\tt{A}$ is an output set, it has $\tt{A}_{\top,t} = \emptyset$, and hence $\tt{A}_{t,i} = \tt{A}_{\exists,t,i}$.
\end{cor}



\subsubsection{Coordination-freeness}

\begin{thm}
\label{thm:coord-free}
$\mathfrak{P}_{GC}$ does not require any additional coordination over $\mathfrak{P}$.
If $\mathfrak{P}$ is confluent in all tables, then $\mathfrak{P}_{GC}$ is also confluent in all tables.
In particular, if all functions $\tt{f}$'s are monotone, then $\mathfrak{P}$ is expressed in monotone Bloom, $\mathfrak{P}_{GC}$ is expressed in monotone Bloom$^L$, and both $\mathfrak{P}$ and $\mathfrak{P}_{GC}$ are confluent.
\end{thm}
\begin{rmk}
This is a stronger but different statement than Theorem \ref{thm:equivalence_of_sets} and Corollary \ref{cor:equivalence_of_output}.
In particular, it says that every run of $\mathfrak{P}_{GC}$ produces the same $\tt{A}_\exists$ and $\tt{A}_\top$.
As we will see later, this is not the case for the instantiated program $\mathfrak{P}_{iGC}$.
\end{rmk}
\begin{proof}
Rules in $\mathfrak{P}$ either correspond to rules in $\mathfrak{P}$, or are new garbage collection rules.
In the first case, we replace the rule $R: \tt{B} ~\tt{<op>}~ \tt{f}(\tt{A}_1,\dots,\tt{A}_n\tt{)}$ with $(\tt{B}_\exists, \tt{B}_\top) ~\tt{<op>}~ (\tt{f}(\tt{A}_{1,\exists} \cup \tt{A}_{1,\top}, \dots, \tt{A}_{n,\exists} \cup \tt{A}_{n,\top} \tt{)}$, which requires a mapping $(\tt{X}_\exists, \tt{X}_\top) \mapsto \tt{X}_\exists \cup \tt{X}_\top$ and the projection $\tt{X} \mapsto (\tt{X}, \emptyset)$, both of which are monotone.
Hence, the corresponding rule in $\mathfrak{P}$ is monotone if $\tt{f}$ is monotone.
In the second case, our $\tt{GC}$ rules are monotone, and involve a deferred merge, and thus do not require additional coordination.

Suppose $\mathfrak{P}$ is confluent in all sets.
Theorem \ref{thm:equivalence_of_sets} implies $\tt{A}_{\exists} \cup \tt{A}_\top = \tt{A}$ is confluent in $\mathfrak{P}_{GC}$.
The only rules in $\mathfrak{P}_{GC}$ that promote tuples from $\exists$ to $\top$ are GC rules; since these rules are monotone, the tombstone lattices are also confluent.
\end{proof}

\begin{cor}
\label{cor:confluence_of_output}
If $\tt{A}$ is an output set, and $\mathfrak{P}$ is confluent, then every run of $\mathfrak{P}_{GC}$ produces $\tt{A}_\exists = \tt{A}$.
\end{cor}

% Our stated goal was to reconcile garbage collection with CALM consistency.
% The astute read might have noticed that we have sneaked in a non-monotone function in \eqref{eq:aexistmerge} to project $(\tt{A}_\exists, \tt{A}_\top)$ to $(\tt{A}_\exists, \emptyset)$, and thus $\mathfrak{P}_{GC}$ is not expressed in monotone Bloom$^L$ or stratified Bloom$^L$.
% Indeed, while Corollary \ref{cor:equivalence_of_output} and \ref{cor:confluence_of_output} state our output sets are confluent in $\mathfrak{P}_{GC}$, non-output sets \emph{may not} be confluent in $\mathfrak{P}_{GC}$.
% In particular, different runs (with different message-time assignments) can result in a different partitioning of a non-output $\tt{A}$ into $\tt{A}_\exists$ and $\tt{A}_\top$.

% However, a close examination of the analysis of \cite{marczak2012confluence} (and of our Bloom$^L$ formalism writeup) would reveal that while all lattices are confluent in any monotone Bloom$^L$ program, it is not necessarily the case that no lattices are confluent in a non-monotone Bloom$^L$ program.
% Our GC program $\mathfrak{P}_{GC}$ is exactly an example of a program confluent in some sets and expressed in non-monotone Bloom$^L$.

% In fact, both \cite{ameloot2013relational} and \cite{marczak2012confluence} defined ``confluence'' or ``eventual consistency'' in terms of output sets, and allowed for non-output sets that are possibly not eventually consistent.
% Dedalus$^+$ and monotone Bloom$^L$ restrict the programming language such that \emph{every} derived fact is provably eventually always true.
% We instead use the GC Invariant \ref{inv:merge_gc} to ensure that output sets are eventually always true (as long as $\mathfrak{P}$ is itself confluent), while allowing for non-output sets to be non-confluent.

% In the CALM analysis (refer to our transducer writeup), ``coordination'' is required if and only if a query can only be computed after establishing the completeness of its input, which is exactly the case for non-monotone queries.
% Although we have introduced a non-monotone projection, the GC invariance asserts that downstream computations are consistent regardless of future input.
% In other words, the GC invariance states that outputs can be computed without knowledge of input completeness!


