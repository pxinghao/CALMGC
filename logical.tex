%!TEX root = bloom_logical.tex

\section{Properties and Analysis of Logical Garbage Collection}
\label{sec:logical}
For our rewrite to be sensible, we require that the GC rules have certain properties.
In particular, the GC rules must be safe -- they should only tombstone tuples that have no effect on future derivations.
We provide some intuition for this safety in the following examples.

\setcounter{subsection}{-1}
\subsection{Intuition}
\begin{example}[Safety for Projection]~

Suppose we have the projection rule $\tt{X} ~\tt{<=}~ \pi_S(\tt{Y})$ in $\mathfrak{P}$, and correspondingly $\tt{X}_{TS} ~\tt{<=}~ (\pi_S(\tt{Y}_\exists \cup \tt{Y}_\top), \emptyset)$ in $\mathfrak{P}_{GC}$.
% , and $\tt{X}_I ~\tt{<=}~ \pi_S(\tt{Y}_I)$ in $\mathfrak{P}_{iGC}$.
% Our goal is to always maintain the relationships
% $\tt{X}_\exists \subseteq \tt{X}_I \subseteq \tt{X}_\exists \cup \tt{X}_\top = \tt{X}$ and
% $\tt{Y}_\exists \subseteq \tt{Y}_I \subseteq \tt{Y}_\exists \cup \tt{Y}_\top = \tt{Y}$.
Since we wish to eventually reclaim tombstoned tuples, these tuples in $\tt{Y}_\top$ should have no effect on the execution of the projection rule.
As explained in Section \ref{sec:prelims:exec}, executing the rule is equivalent to making the assignment $\tt{X}_{\exists,t,i+1} \cup \tt{X}_{\top,t,i+1} := \tt{X}_{\exists,t,i} \cup \tt{X}_{\top,t,i} \cup \pi_S(\tt{Y}_{\exists,t,i} \cup \tt{Y}_{\top,t,i})$.
For $\tt{Y}_\top$ to not affect this execution, it must be the case that $\tt{Y}_{\top,t,i} \subseteq \tt{X}_{\exists,t,i} \cup \tt{X}_{\top,t,i}$, so that $\tt{X}_{\exists,t,i} \cup \tt{X}_{\top,t,i} \cup \pi_S(\tt{Y}_{\exists,t,i} \cup \tt{Y}_{\top,t,i}) = \tt{X}_{\exists,t,i} \cup \tt{X}_{\top,t,i} \cup \pi_S(\tt{Y}_{\exists,t,i})$.
Thus, as long as $\tt{Y}_\top \subseteq \tt{X}_\exists \cup \tt{X}_\top$, it suffices to keep $\tt{Y}_\exists$ and we may safely discard $\tt{Y}_\top$.

As the execution of the program continues, $\tt{Y}_\exists$ may grow as tuples are merged into it, forming a larger $\widehat{\tt{Y}} \supseteq \tt{Y}_\exists$.
Despite this growth, the execution is still unaffected, i.e., $\tt{X}_{\exists,t,i} \cup \tt{X}_{\top,t,i} \cup \pi_S(\widehat{\tt{Y}} \cup \tt{Y}_{\top,t,i}) = \tt{X}_{\exists,t,i} \cup \tt{X}_{\top,t,i} \cup \pi_S(\widehat{\tt{Y}})$, since we have ensured that $\tt{Y}_\top \subseteq \tt{X}_\exists \cup \tt{X}_\top$.
\end{example}

\begin{example}[Safety for Set Difference]
~

Suppose we have the rule $\tt{Z} ~\tt{<=}~ \tt{X} - \tt{Y}$ in $\mathfrak{P}$, and correspondingly $\tt{Z}_{TS} ~\tt{<=}~ ((\tt{X}_\exists \cup \tt{X}_\top) - (\tt{Y}_\exists \cup \tt{Y}_\top), \emptyset)$ in $\mathfrak{P}_{GC}$.
% , and $\tt{Z}_I ~\tt{<=}~ \tt{X}_I - \tt{Y}_I$ in $\mathfrak{P}_{iGC}$.
Suppose also that $\tt{X}_E$ has a key-augmented representation in $\mathfrak{P}_{iGC}$ while $\tt{Y}_I$ has a non-augmented representation.
As in the previous example, we desire that the execution of the rule is unaffected by deletion of tombstones.
In particular, it suffices to have
$(\tt{X}_{\exists,t,i} \cup \tt{X}_{\top,t,i}) - (\tt{Y}_{\exists,t,i} \cup \tt{Y}_{\top,t,i}) = \tt{X}_{\exists,t,i} - \tt{Y}_{\exists,t,i}$.
We observe that if a tuple $x$ appears in both $\tt{X}$ and $\tt{Y}$, we can safely tombstone $x$ from $\tt{X}$ without affecting the set difference.
Once we have $x \in \tt{X}_\top$, it is not in $\tt{X}_\exists$, so we can also safely tombstone $x$ from $\tt{Y}$.
Hence, it suffices to ensure $\tt{Y}_\top \subseteq \tt{X}_\top \subseteq \tt{Y}_\top \cup \tt{Y}_\exists$.
We may then discard $\tt{X}_\top$ and $\tt{Y}_\top$ and keep only $\tt{X}_\exists$, $\tt{Y}_\exists$.
We also keep $\tt{X}_!$ in the key-augmented representation; its use will be demonstrated in the following paragraph.

As the execution of the program continues, both $\tt{X}_\exists$ and $\tt{Y}_\exists$ may grow as tuples are merged into them, former larger $\widehat{\tt{X}} \supseteq \tt{X}_\exists$ and $\widehat{\tt{Y}} \supseteq \tt{Y}_\exists$.
Since we use the key-augmented representation for $\tt{X}$, tuples that we add to $\tt{X}_\exists$ must be previously unseen tuples not in $\tt{X}_!$, which implies $\widehat{\tt{X}} \cap \tt{X}_\top = \emptyset$.
If we ensure that $\tt{Y}_\top \subseteq \tt{X}_\top \subseteq \tt{Y}_\top \cup \tt{Y}_\exists$, we can observe that
$
(\widehat{\tt{X}} \cup \tt{X}_\top) - (\widehat{\tt{Y}} \cup \tt{Y}_\top)
= \widehat{\tt{X}} - (\widehat{\tt{Y}} \cup \tt{Y}_\top)
= \widehat{\tt{X}} - \widehat{\tt{Y}}
$,
where the second equality holds because $\tt{Y}_\top \cap \widehat{\tt{X}} \subseteq \tt{X}_\top \cap \widehat{\tt{X}} = \emptyset$.
Thus, the execution of the set difference rule is unaffected by the deletion of tombstones.
\end{example}


\subsection{Properties of Logical GC rules}
We now formalize these intuitions about the safety of logical garbage collection rules with the below invariant.


\begin{invariant}[Merge GC Invariant]\label{inv:merge_gc}
Consider any merge or deferred merge rule 
\[R: \tt{B} ~\tt{<op>}~ \tt{f(}\tt{A}_1,\dots,\tt{A}_n, \tt{U}_1, \dots, \tt{U}_m\tt{)}\]
where $\tt{<op>}$ is either $\tt{<=}$ or $\tt{<+}$,
the sets $\tt{A}_1,\dots,\tt{A}_n \in \mathbb{W}$,
the sets $\tt{U}_1,\dots,\tt{U}_m \in \mathbb{S}$,
and RHS is not a channel.
We require that, 
$\forall \langle \widehat{\tt{A}}_1, \dots, \widehat{\tt{A}}_n\rangle \geq \langle \tt{A}_{1,\exists},\dots,\tt{A}_{n,\exists} \rangle,$
$\quad
\forall \langle \widehat{\tt{U}}_1 , \dots, \widehat{\tt{U}}_m \rangle$ $\geq$ $\langle \tt{U}_1, \dots, \tt{U}_m \rangle$
such that $\emptyset = \widehat{\tt{U}}_1 \cap \tt{U}_{1,\top} = \dots = \widehat{\tt{U}}_1 \cap \tt{U}_{m,\top}$:
\begin{align}
&\tt{f(}\widehat{\tt{A}}_1,\dots,\widehat{\tt{A}}_n, \widehat{\tt{U}}_{1,\exists},\dots,\widehat{\tt{U}}_{n,\exists}\tt{)}
\cup \tt{B}_\exists\cup \tt{B}_\top\nonumber\\
&~=~
\tt{f(}
\widehat{\tt{A}}_1 \cup \tt{A}_{1,\top},\dots,\widehat{\tt{A}}_n \cup \tt{A}_{n,\top}, \widehat{\tt{U}}_1 \cup \tt{U}_{1,\top},\dots,\widehat{\tt{U}}_n \cup \tt{U}_{n,\top}\tt{)} \nonumber\\
&\qquad\cup \tt{B}_\exists \cup \tt{B}_\top.
\label{eq:inv_merge_gc}
\end{align}
\end{invariant}

We are interested in keeping around only $\langle \tt{A}_{1,\exists},\dots,\tt{A}_{n,\exists} \rangle$ instead of $\langle \tt{A}_{1,\exists} \cup \tt{A}_{1,\top}, \dots, \tt{A}_{n,\exists}  \cup \tt{A}_{n,\top}\rangle$.
However, in the process of executing $\mathfrak{P}_{GC}$, we may find ourselves with a larger\footnote{
  The restrictions of Edelweiss ensures that lattices can only grow over time.
    In Dedalus$^+$ terminology, the program is `temporally inflationary'.
} lattice $\langle \widehat{\tt{A}}_1, \dots, \widehat{\tt{A}}_n\rangle \geq \langle \tt{A}_{1,\exists},\dots,\tt{A}_{n,\exists} \rangle$.
In such a case, we wish to ensure that the execution of $R$ in $\mathfrak{P}_{GC}$ (LHS of \eqref{eq:inv_merge_gc}) matches the corresponding execution of $R$ in $\mathfrak{P}$ (RHS of \eqref{eq:inv_merge_gc}) as if we had not reclaimed $\tt{A}_{1,\top}, \dots, \tt{A}_{n,\top}$.
As we will see in the examples below, the invariant often reduces to a simple intuitive assertion.

A logical garbage collection rule is `safe' if it does not tombstone any tuple from output relations, and maintains the Merge GC Invariant.

\begin{property}[GC Rule Output Respect]
A garbage collection rule $\tt{GC}$ \emph{respects output relations} if it does not tombstone tuples from output relations.
That is, letting $\langle \tt{A}_{1,TS}^*,$ $\dots,$ $\tt{A}_{n,TS}^* \rangle$ $=$ $\tt{GC}(\tt{A}_{1,TS}, \dots, \tt{A}_{n,TS};$ $\tt{B}_{TS},$ $\tt{D}_{1,TS}, \dots, \tt{D}_{l,TS})$, then whenever $\tt{A}_j$ is an output relation and $\tt{A}_{n,\top} = \emptyset$, it must be the case that $\tt{A}_{j,\top}^* = \emptyset$.
\end{property}

\begin{property}[Merge GC Safety]
\label{property:gc_safety}
Let $R:$ $\tt{B}$ $\tt{<op>}$ $\tt{f(}\tt{A}_1,\dots,\tt{A}_n, \tt{U}_1, \dots, \tt{U}_m\tt{)}$ be a merge or deferred merge rule, where $\tt{A}_1, \dots, \tt{A}_n \in \mathbb{W}$ and $\tt{U}_1, \dots, \tt{U}_m \in \mathbb{S}$.
A garbage collection rule $\tt{GC}$ is safe for $R$ if it respects output relations, and maintains the Merge GC Invariant \ref{inv:merge_gc} for $R$.
That is, if
\begin{enumerate}
\item $\langle \tt{A}_{1,TS}, \dots, \tt{A}_{n,TS}, \tt{U}_{1,TS}, \dots, \tt{U}_{m,TS} \rangle$ and $\tt{B}_{TS}$ satisfies Merge GC Invariant \ref{inv:merge_gc} for $R$, and
\item $\tt{B}_\exists \cup \tt{B}_\top \supseteq \tt{f}\tt{(}\tt{A}_{1,\exists} \cup \tt{A}_{1,\top},\dots,\tt{A}_{n,\exists} \cup \tt{A}_{n,\top}, \tt{U}_{1,\exists} \cup \tt{U}_{1,\top},\dots,\tt{U}_{n,\exists} \cup \tt{U}_{n,\top}\tt{)}$,
\end{enumerate}
then $\langle \tt{A}_{1,TS}^*, \dots, \tt{A}_{n,TS}^*, \tt{U}_{1,TS}^*, \dots, \tt{U}_{m,TS}^* \rangle$ $=$ $\tt{GC}(\tt{A}_{1,TS},$ $\dots,$ $\tt{A}_{n,TS},$ $\tt{U}_{1,TS}, \dots, \tt{U}_{m,TS};$ $\tt{B}_{TS}, \tt{D}_{1,TS}, \dots, \tt{D}_{l,TS})$ and $\tt{B}_{TS}$ also satisfies the Merge GC Invariant \ref{inv:merge_gc}.
\end{property}

We will also require $\mathfrak{P}_{GC}$ to be monotone so that it can be executed coordination-free whenever $\mathfrak{P}$ is itself monotone.

\begin{property}[GC Rule Monotonicity]
% Suppose $\tt{A}_{1,TS}, \dots, \tt{A}_{n,TS}, \tt{B}_{1,TS}, \dots, \tt{B}_{m,TS}$ satisfies Merge GC Invariant \ref{inv:merge_gc}.
% Suppose that $\tt{A}'_{1,TS}, \dots, \tt{A}'_{n,TS}, \tt{B}'_{1,TS}, \dots, \tt{B}'_{m,TS}$ also satisfies Merge GC Invariant \ref{inv:merge_gc} and 
Suppose that $\tt{A}'_{i,TS} \geq \tt{A}_{i,TS}$ and $\tt{B}'_{j,TS} \geq \tt{B}_{j,TS}$.
Then, $\tt{GC}$ is \emph{monotone} if $\tt{GC}(\tt{A}_{1,TS}, \dots, \tt{A}_{n,TS}; \tt{B}_{1,TS}, \dots, \tt{B}_{m,TS})$ $\leq$ $\tt{GC}(\tt{A}'_{1,TS},$ $\dots,$ $\tt{A}'_{n,TS};$ $\tt{B}'_{1,TS}, \dots, \tt{B}'_{m,TS})$.
\end{property}

Lastly, it is natural to expect that logical garbage collection rules neither creates new tuples nor loses them.

\begin{property}[GC Rule Conservation]
\label{property:gc_conservative}
For a garbage collection rule $\tt{GC}$, denote
$\langle \tt{A}_{1,TS}^*, \dots, \tt{A}_{n,TS}^* \rangle$ $=$ $\tt{GC}(\tt{A}_{1,TS},$ $\dots,$ $\tt{A}_{n,TS};$ $\tt{B}_{1,TS}, \dots, \tt{B}_{m,TS})$.
Then, $\tt{GC}$ is \emph{conservative} if $\tt{A}_{j,\exists}^* \subseteq \tt{A}_{j,\exists}$ and $\tt{A}_{j,\exists}^* \cup \tt{A}_{j,\top}^* = \tt{A}_{j,\exists} \cup \tt{A}_{j,\top}$ for all $j=1,\dots,n$.
That is, $\tt{GC}$ only moves tuples from $\exists$ to $\top$.
\end{property}

We will require that all logical garbage collection rules be conservative, monotone, safe, and respects outputs.

\subsection{Examples of Logical GC rules}
\label{sec:logical:example}
A number of example logical garbage collection rules are presented below, with additional examples in Appendix \ref{sec:examples}.
Each of these rules are monotone, conservative, and safe for some rule $R$.

% \begin{example}[Indicator GC]
% Consider the indicator function $\tt{f}$ with $\tt{f}\tt{(}\tt{X}\tt{)} = \emptyset$ if $\tt{X} = \emptyset$ and $\{1\}$ otherwise, and $R: \tt{B} ~\tt{<=}~ \tt{f}\tt{(}\tt{X}\tt{)}$.
% Suppose that $\emptyset \subseteq \tt{X} \subseteq \{a,b,c\}$.
% Define $\tt{GCInd}_i(\tt{X}_{TS}; \tt{B}_{TS}) = (\tt{X}_\exists - \tt{X}_i^*, \tt{X}_\top \cup \tt{X}_i^*)$, for $i=1,2,3$ and
% \begin{align*}
% \tt{X}_1^* = \begin{cases}
% \emptyset & \text{if $|\tt{X}_\exists \cup \tt{X}_\top| \leq 1$} \\
% a & \text{if $\tt{X}_\exists \cup \tt{X}_\top = \{a, b\}$}\\
% c & \text{if $\tt{X}_\exists \cup \tt{X}_\top = \{a, c\}$}\\
% b & \text{if $\tt{X}_\exists \cup \tt{X}_\top = \{b, c\}$}\\
% a,b & \text{if $\tt{X}_\exists \cup \tt{X}_\top = \{a, b, c\}$}\\
% \end{cases}
% &&
% \tt{X}_2^* = \begin{cases}
% \emptyset & \text{if $|\tt{X}_\exists \cup \tt{X}_\top| \leq 1$} \\
% a & \text{if $\tt{X}_\exists \cup \tt{X}_\top = \{a, b\}$}\\
% c & \text{if $\tt{X}_\exists \cup \tt{X}_\top = \{a, c\}$}\\
% b & \text{if $\tt{X}_\exists \cup \tt{X}_\top = \{b, c\}$}\\
% a,b,c & \text{if $\tt{X}_\exists \cup \tt{X}_\top = \{a, b, c\}$}\\
% \end{cases}
% &&
% \tt{X}_3^* = \begin{cases}
% \emptyset & \text{if $|\tt{X}_\exists \cup \tt{X}_\top| \leq 1$} \\
% a & \text{if $\tt{X}_\exists \cup \tt{X}_\top = \{a, b\}$}\\
% a & \text{if $\tt{X}_\exists \cup \tt{X}_\top = \{a, c\}$}\\
% b & \text{if $\tt{X}_\exists \cup \tt{X}_\top = \{b, c\}$}\\
% a,b & \text{if $\tt{X}_\exists \cup \tt{X}_\top = \{a, b, c\}$}\\
% \end{cases}
% \end{align*}
% Then $\tt{GCInd}_1$ is safe for $R$ but not monotone, $\tt{GCInd}_2$ is monotone but not safe for $R$, and $\tt{GCInd}_3$ is monotone and safe for $R$.
% \end{example}
% $\tt{GCInd}_1$ is not monotone because $\tt{GCInd}_1((\{a,c\}, \emptyset); \tt{B}_{TS}) \not\leq \tt{GCInd}_1((\{a,b,c\}, \emptyset); \tt{B}_{TS}); \tt{B}_{TS})$.
% $\tt{GCInd}_2$ is not safe because garbage collection on $\{a, b, c\}$ causes 
% 
% GC on indicator: $\emptyset -> 0$, non-$\emptyset -> 1$.
% If GC(a,b) = a, GC(a,c) = c, GC(b,c) = b, GC(a,b,c) = a,b then satisfies GCI but not monotone.
% If GC(a,b) = a, GC(a,c) = c, GC(b,c) = b, GC(a,b,c) = a,b,c then not satisfies GCI but monotone.
% If GC(a,b) = a, GC(a,c) = a, GC(b,c) = b, GC(a,b,c) = a,b then satisfies GCI and monotone.

\begin{example}[Copy GC]
Suppose $\tt{A}$ appears as input on the RHS of rules $R_i: \tt{B}_i ~\tt{<op>}~ \tt{A}$, where $\tt{<op>}$ is either a merge or deferred merge.
Then $\tt{GCCopy}(\tt{A}_{TS};$ $\tt{B}_{1,TS},$ $\dots,$ $\tt{B}_{m,TS}) = \left(\tt{A}_\exists - \tt{B}_\cap, \tt{A}_\top \cup \tt{B}_\cap\right)$,
where $\tt{B}_\cap = \tt{A}_\exists \cap \bigcap_{i=1}^m (\tt{B}_{i,\exists} \cup \tt{B}_{i,\top})$,
is a monotone, conservative garbage collection rule that is safe for $R_1, \dots, R_n$.
\end{example}
Clearly, $\tt{GCCopy}$ is conservative.
For the copy rule, the Merge GC Invariant \ref{inv:merge_gc} reduces to $\forall i, \tt{A}_\exists \cup \tt{B}_{i,\exists} \cup \tt{B}_{i,\top} = \tt{A}_\exists \cup \tt{A}_\top \cup \tt{B}_{i,\exists} \cup \tt{B}_{i,\top}$, which is the case iff $\tt{A}_\top \subseteq \tt{B}_{i,\exists} \cup \tt{B}_{i,\top}$, since $\tt{A}_\exists \cap \tt{A}_\top = \emptyset$.
That is, tombstoned tuples are those that have been copied to all $\tt{B}_i$'s.
Hence, if $\tt{A}_{TS}, \tt{B}_{i,TS}$ satisfies Invariant \ref{inv:merge_gc}, we have $\tt{A}_\top \subseteq \tt{B}_{i,\exists} \cup \tt{B}_{i,\top}$, and so $\tt{A}_\top \cup \tt{B}_\cap = \tt{A}_\top \cup \left(\tt{A}_\exists \cap \bigcap_{i=1}^m (\tt{B}_{i,\exists} \cup \tt{B}_{i,\top})\right) \subseteq \tt{B}_{i,\exists} \cup \tt{B}_{i,\top}$.
Therefore, $\tt{GCCopy}$ is safe for $R_i$.

Suppose $\tt{A}_\text{TS}' \geq \tt{A}_\text{TS}$, i.e., $\tt{A}_\top' \supseteq \tt{A}_\top$ and $\tt{A}_\top' \cup \tt{A}_\exists' \supseteq \tt{A}_\top \cup \tt{A}_\exists$.
Suppose also $\tt{A}_\text{TS}'$ satisfies the Merge GC Invariant \ref{inv:merge_gc} with respect to $\tt{B}_{i,\text{TS}}' \geq \tt{B}_{i,\text{TS}}$.
Then
\begin{align*}
&\tt{A}_\top \cup \left(\tt{A}_\exists \cap \bigcap_{i=1}^m (\tt{B}_{i,\exists} \cup \tt{B}_{i,\top})\right)\\
&\subseteq \tt{A}_\top' \cup \left((\tt{A}_\exists' \cup \tt{A}_\top') \cap \bigcap_{i=1}^m (\tt{B}_{i,\exists}' \cup \tt{B}_{i,\top}')\right)\\
% \\
% &= \tt{A}_\top' \cup \left(\tt{A}_\exists' \cap \bigcap_{i=1}^m (\tt{B}_{i,\exists}' \cup \tt{B}_{i,\top}')\right) \cup \left(\tt{A}_\top' \cap \bigcap_{i=1}^m (\tt{B}_{i,\exists}' \cup \tt{B}_{i,\top}')\right)\\
&= \tt{A}_\top' \cup \left(\tt{A}_\exists' \cap \bigcap_{i=1}^m (\tt{B}_{i,\exists}' \cup \tt{B}_{i,\top}')\right).
\end{align*}
Furthermore, it follows from the GC Rule Conservation Property \ref{property:gc_conservative}, we also have that $(\tt{A}_\exists - \tt{B}_\cap) \cup (\tt{A}_\top \cup \tt{B}_\cap) \subseteq (\tt{A}_\exists' - \tt{B}_\cap') \cup (\tt{A}_\top' \cup \tt{B}_\cap')$.
Hence, $\tt{GCcopy}$ is monotone.

~

% GC-Max: GC on A, B gives some tombstones larger than GC on any A' < A, B' < B, and is of largest cardinality among all such tombstones.
% Note: GC-Max is not unique; but since monotone, will still give a unique minimal model for each GC-Max rule.

\begin{example}[Project GC]
\label{ex:project_gc}
Consider the projection $R: \tt{B} ~\tt{<op>}~ \pi_S(\tt{A})$, where $\tt{<op>}$ is a merge or deferred merge.
Then $\tt{GCProject}(\tt{A}_{TS}; \tt{B}_{TS}) = (\tt{A}_\exists - \tt{A}^*, \tt{A}_\top \cup \tt{A}^*)$ where $\tt{A}^* = \{a \in \tt{A}_\exists \cup \tt{A}_\top ~:~ \pi_S(\{a\}) \subseteq \tt{B}_\exists \cup \tt{B}_\top\}$
is safe for $R$.
\end{example}
Simple relational algebraic calculations would show that the Merge GC Invariant \ref{inv:merge_gc} reduces to $\pi_S(\tt{A}_\top) \subseteq \tt{B}_\exists \cup \tt{B}_\top$.
By definition, $\pi_L(\tt{A}^*) \subseteq \tt{B}_\exists \cup \tt{B}_\top$.
Hence, if $\tt{A}_{TS}$ and $\tt{B}_{TS}$ satisfies the invariant, then $\pi_L(\tt{A}_\top \cup \tt{A}^*) = \pi_L(\tt{A}_\top) \cup \pi_L(\tt{A}^*) \subseteq \tt{B}_\exists \cup \tt{B}_\top$, so $\tt{GCProject}$ is safe for $R$. 

\begin{example}[DR-: Negative Difference]
\label{ex:logical:dr-}
Consider the rule $R: \tt{B} ~\tt{<op>}~ \tt{A} - \tt{C}$, where $\tt{<op>}$ is a merge or deferred merge, and $\tt{A}$ is a table with primary keys.
Then the rule $\tt{GCDR-}(\tt{A}_{TS}, \tt{C}_{TS}; \tt{B}_{TS}) = \langle (\tt{A}_\exists - \tt{A}^*, \tt{A}_\top \cup \tt{A}^*), (\tt{C}_\exists - \tt{C}^*, \tt{C}_\top \cup \tt{C}^*)\rangle$,
where $\tt{A}^* = (\tt{A}_\exists \cup \tt{A}_\top) \cap (\tt{C}_\exists \cup \tt{C}_\top)$ and $\tt{C}^* = (\tt{C}_\exists \cup \tt{C}_\top) \cap \tt{A}_\top$,
is safe for $R$.
\end{example}
Since primary keys are available for $\tt{A}$, the Merge GC Invariant \ref{inv:merge_gc} reduces to
$\forall \widehat{\tt{A}} \supseteq \tt{A}_\exists, \widehat{\tt{A}} \cap \tt{A}_\top = \emptyset, \forall \widehat{\tt{C}} \supseteq \tt{C}_\exists$: $\widehat{\tt{A}} - \widehat{\tt{C}} = (\widehat{\tt{A}} \cup \tt{A}_\top) - (\widehat{\tt{C}} \cup \tt{C}_\top)$.
In the case where $\widehat{\tt{A}} = \tt{A}_\exists$ and $\widehat{\tt{C}} = \tt{C}_\exists$, it implies that $\tt{A}_\top \subseteq \tt{C}_\exists \cup \tt{C}_\top$;
in the case where $\widehat{\tt{A}} = \tt{A}_\exists \cup (\tt{C}_\top - \tt{A}_\top)$ and $\widehat{\tt{C}} = \tt{A}_\exists \cup \tt{C}_\exists$, it implies that $\tt{C}_\top \subseteq \tt{A}_\top$.
Conversely $\tt{C}_\top \subseteq \tt{A}_\top \subseteq \tt{C}_\exists \cup \tt{C}_\top$ suffices to satisfy the invariant.
One can now easily verify that $\tt{GCDR-}$ maintains the invariant.





\subsection{Analysis of Logical GC Rewrite}
We now analyze the logical GC rewrite $\mathfrak{P}_{GC}$, showing that it produces the same output as $\mathfrak{P}$, and that $\mathfrak{P}_{GC}$ is coordination-free whenever $\mathfrak{P}$ is.

\subsubsection{Correctness}
\label{sec:logical:analysis:correctness}

\begin{lemma}
\label{lem:unchangedexists}
For every set $\tt{A}$, we have that $\tt{A}_{\top,t,i} = \tt{A}_{\top,t,0}$.
\end{lemma}
\begin{proof}
The proof is straightforward: all non-GC rules are of the form $(\tt{A}_\exists, \tt{A}_\top) ~\tt{<op>}~ (\tt{X}, \emptyset)$, and all GC rules are deferred merges.
Thus, $\tt{A}_\top$ is unchanged during an instantaneous run.
\end{proof}
Due to Lemma \ref{lem:unchangedexists}, we can write $\tt{A}_{\top,t}$ in place of $\tt{A}_{\top,t,i}$.

\newcommand{\lemmergegcinv}{
The rewritten GC program $\mathfrak{P}_{GC}$ maintains the Merge GC Invariant \ref{inv:merge_gc} for any rule $R$.
}
\begin{lemma}
\label{lem:merge_gc_inv}
\lemmergegcinv
\end{lemma}

\begin{rmk}
Lemma \ref{lem:merge_gc_inv} is required for showing that the instantiated GC program $\mathfrak{P}_{iGC}$ presented later is correct, but not used in the below Theorem \ref{thm:equivalence_of_sets} establishing correctness of $\mathfrak{P}_{GC}$.
\end{rmk}




It is easy to see that $\mathfrak{P}$ and $\mathfrak{P}_{GC}$ have equivalent output sets.
That is, every run of $\mathfrak{P}$ has a one-to-one corresponding run of $\mathfrak{P}_{GC}$ that produces lattices such that $\tt{A}_\exists \cup \tt{A}_\top = \tt{A}$.
And in particular, for output sets where no garbage collection is performed, we get $\tt{A}_\exists = \tt{A}$.

\newcommand{\thmequivalenceofsets}{For every set $\tt{A}$, we have that $\tt{A}_{t,i} = \tt{A}_{\exists,t,i} \cup \tt{A}_{\top,t}$.}
\begin{thm}
\label{thm:equivalence_of_sets}
\thmequivalenceofsets
\end{thm}


For output sets, we never perform garbage collection, and hence always have equivalence of the sets in $\mathfrak{P}$ and $\mathfrak{P}_{GC}$.
The below corollary formally states this.
\begin{cor}
\label{cor:equivalence_of_output}
If a set $\tt{A}$ is an output set, it has $\tt{A}_{\top,t} = \emptyset$, and hence $\tt{A}_{t,i} = \tt{A}_{\exists,t,i}$.
\end{cor}



\subsubsection{Coordination-freeness}

\newcommand{\thmcoordfree}{
$\mathfrak{P}_{GC}$ does not require any additional coordination over $\mathfrak{P}$.
If $\mathfrak{P}$ is confluent in all tables, then $\mathfrak{P}_{GC}$ is also confluent in all tables.
In particular, if all functions $\tt{f}$'s are monotone, then $\mathfrak{P}$ is expressed in monotone Bloom, $\mathfrak{P}_{GC}$ is expressed in monotone Bloom$^L$, and both $\mathfrak{P}$ and $\mathfrak{P}_{GC}$ are confluent.}
\begin{thm}
\label{thm:coord-free}
\thmcoordfree
\end{thm}
\begin{rmk}
This is a stronger but different statement than Theorem \ref{thm:equivalence_of_sets} and Corollary \ref{cor:equivalence_of_output}.
In particular, it says that every run of $\mathfrak{P}_{GC}$ produces the same $\tt{A}_\exists$ and $\tt{A}_\top$.
As we will see later, this is not the case for the instantiated program $\mathfrak{P}_{iGC}$.
\end{rmk}

\begin{cor}
\label{cor:confluence_of_output}
If $\tt{A}$ is an output set, and $\mathfrak{P}$ is confluent, then every run of $\mathfrak{P}_{GC}$ produces $\tt{A}_\exists = \tt{A}$.
\end{cor}

% Our stated goal was to reconcile garbage collection with CALM consistency.
% The astute read might have noticed that we have sneaked in a non-monotone function in \eqref{eq:aexistmerge} to project $(\tt{A}_\exists, \tt{A}_\top)$ to $(\tt{A}_\exists, \emptyset)$, and thus $\mathfrak{P}_{GC}$ is not expressed in monotone Bloom$^L$ or stratified Bloom$^L$.
% Indeed, while Corollary \ref{cor:equivalence_of_output} and \ref{cor:confluence_of_output} state our output sets are confluent in $\mathfrak{P}_{GC}$, non-output sets \emph{may not} be confluent in $\mathfrak{P}_{GC}$.
% In particular, different runs (with different message-time assignments) can result in a different partitioning of a non-output $\tt{A}$ into $\tt{A}_\exists$ and $\tt{A}_\top$.

% However, a close examination of the analysis of \cite{marczak2012confluence} (and of our Bloom$^L$ formalism writeup) would reveal that while all lattices are confluent in any monotone Bloom$^L$ program, it is not necessarily the case that no lattices are confluent in a non-monotone Bloom$^L$ program.
% Our GC program $\mathfrak{P}_{GC}$ is exactly an example of a program confluent in some sets and expressed in non-monotone Bloom$^L$.

% In fact, both \cite{ameloot2013relational} and \cite{marczak2012confluence} defined ``confluence'' or ``eventual consistency'' in terms of output sets, and allowed for non-output sets that are possibly not eventually consistent.
% Dedalus$^+$ and monotone Bloom$^L$ restrict the programming language such that \emph{every} derived fact is provably eventually always true.
% We instead use the GC Invariant \ref{inv:merge_gc} to ensure that output sets are eventually always true (as long as $\mathfrak{P}$ is itself confluent), while allowing for non-output sets to be non-confluent.

% In the CALM analysis (refer to our transducer writeup), ``coordination'' is required if and only if a query can only be computed after establishing the completeness of its input, which is exactly the case for non-monotone queries.
% Although we have introduced a non-monotone projection, the GC invariance asserts that downstream computations are consistent regardless of future input.
% In other words, the GC invariance states that outputs can be computed without knowledge of input completeness!


