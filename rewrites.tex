%!TEX root = bloom_logical.tex

\section{Rewrites}
\label{sec:rewrites}

\setcounter{subsection}{-1}
\subsection{Simplifying copy rewrite}
We make the following simplifying assumption about $\mathfrak{P}$.
\begin{assumption}
\label{ass:copy}
Let $\tt{A}$ be a table or channel in $\mathfrak{P}$ appearing in multiple rules
\begin{align*}
R_1: && \tt{B}_1 &\quad\tt{<op>}\quad \tt{f}_1\tt{(}\tt{A},\tt{C}_{11},\dots,\tt{C}_{1k_1}\tt{)}\\
&&&\quad\vdots\\
R_n: && \tt{B}_n &\quad\tt{<op>}\quad \tt{f}_n\tt{(}\tt{A},\tt{C}_{n1},\dots,\tt{C}_{nk_n}\tt{)}
\end{align*}
with $n \geq 2$.
Then it must be the case that $R_i: \tt{B}_i ~\tt{<=}~ \tt{A}$ is an identity copy.
\end{assumption}
Assumption \ref{ass:copy} states that any table $\tt{A}$ can either appear as input on the RHS of a single rule, or otherwise each of the rules it appears in must be a copy rule.

The assumption can be made without loss of generality, as any $\mathfrak{P}$ can always be rewritten into a program $\mathfrak{P}_0$ that satisfies Assumption \ref{ass:copy}.
Suppose $\tt{A}$ is a table in $\mathfrak{P}$ appearing on the RHS of multiple rules $R_1,\dots,R_n$.
We can define $\tt{A}_i$ in $\mathfrak{P}_0$ for each $R_i$, and replace $\tt{A}$ with $\tt{A}_i$ in $R_i$, and add the rules $\tt{A}_i ~\tt{<=}~ \tt{A}$.
The resultant $\mathfrak{P}_0$ is essentially equivalent to $\mathfrak{P}$, in the sense that any run of $\mathfrak{P}$ corresponds to a run of $\mathfrak{P}_0$, and vice versa.

In practice, the above rewrite should not increase storage significantly.
One only needs to keep a single additional bit for each tuple and $R_i$ to record whether the tuple is in $\tt{A}_i$.

This assumption is really only required so that we can specify garbage collection for each individual rule $R_i$ without having to reason about the logic of composing the garbage collection together.

We also point out that the rules in Assumption \ref{ass:copy} can be combined into a single rule using the product lattice notation:
\begin{align*}
R: && \langle \tt{B}_1, \dots, \tt{B}_n \rangle &\quad\tt{<=}\quad \langle \underbrace{\tt{A}, \dots, \tt{A}}_{\text{$n$ times}} \rangle
\end{align*}
so we can make the following assumption instead:
\begin{assumption}
\label{ass:single_rule}
Every table or channel in $\mathfrak{P}$ appears in the RHS of at most one rule.
\end{assumption}


\subsection{Logical Garbage Collection Rewrite}
The idea of the GC rewrite is to lift the program's state into tombstone lattices.
Within the timestep, we perform operations on the lifted representation instead of the original state.
Garbage collection occurs between timesteps, marking tuples for reclamation.
In this section, our storage reclamation is only logical, i.e., tuples are never actually deleted.
This leads to a program with no additional coordination required.
We then present a second rewrite where the reclamation is instantiated; in the second rewrite, we guarantee confluence of output sets, but not of intermediate sets from which storage is reclaimed.

For clarity, we also omit garbage collection for asynchronous rules.
Doing so will allow us to develop simpler analyses that illuminate the overall approach.
We will then revisit garbage collection for asynchronous rules in later sections, building on the approach in this section.

We begin with a program $\mathfrak{P}$ with rules $R_1, \dots, R_n$, and rewrite it into a GC program $\mathfrak{P}_{GC}$.
We assume that all tables in $\mathfrak{P}$ are persistent (with implicit rules $\tt{A} ~\tt{<+}~ \tt{A}$), but will make any persistence in $\mathfrak{P}$ explicit.



\subsubsection{State}
Let $\tt{A}$ be a table in $\mathfrak{P}$.
We introduce the table $\tt{A}_\exists$ and tombstone set $\tt{A}_{TS} = (\tt{A}_\exists, \tt{A}_\top)$ in $\mathfrak{P}_{GC}$.
Intuitively, $\tt{A}_\exists$ should reflect facts in $\tt{A}$ which are important for future derivations and thus not garbage collected.
$\tt{A}_{TS}$ contains facts of the current timestep, but with tombstones marking tuples that could be deleted without affecting future derivations.
We also add rule
\begin{align}
(\tt{A}_\exists, \tt{A}_\top) &\quad\tt{<+}\quad (\tt{A}_\exists, \tt{A}_\top)\label{eq:aexistmerge}
\end{align}
All non-GC rules in $\mathfrak{P}_{GC}$ will be of the form $(\tt{A}_\exists, \tt{A}_\top) ~\tt{<op>}~ (\tt{X}, \emptyset)$.


\subsubsection{Merges, Deferred Merges}
We next consider rules in the original program $\mathfrak{P}$.
For every rule
\begin{align}
\tt{B} \quad\tt{<op>}\quad \tt{f(}\tt{A}_1,\dots,\tt{A}_n\tt{)}
\end{align}
in $\mathfrak{P}$, where $\tt{<op>}$ is a merge $\tt{<=}$ or deferred merge $\tt{<+}$, we add the below rule to $\mathfrak{P}_{GC}$:
\begin{align*}
(\tt{B}_\exists, \tt{B}_\top) &\quad\tt{<op>}\quad (\tt{f(}\tt{A}_{1,\exists} \cup \tt{A}_{1,\top},\dots,\tt{A}_{n,\exists} \cup \tt{A}_{n,\top}\tt{)}, \emptyset)
\end{align*}

\subsubsection{Asynchronous merges}
Asynchronous merges are messaging rules between nodes.
On the sender side, in $\mathfrak{P}$, we have the rule 
\begin{align}
\overline{\tt{B}} \quad\tt{<}\sim\quad \tt{f(}\tt{A}_1,\dots,\tt{A}_n\tt{)}
\end{align}
and on the receiver side we have
\begin{align}
\tt{B} \quad\tt{<=}\quad \overline{\tt{B}}
\end{align}
These are respectively replaced in $\mathfrak{P}_{GC}$ by
\begin{align}
(\overline{\tt{B}}, \emptyset) &\quad\tt{<}\sim\quad (\tt{f(}\tt{A}_{1,\exists} \cup \tt{A}_{1,\top},\dots,\tt{A}_{n,\exists} \cup \tt{A}_{n,\top}\tt{)}, \emptyset)
\end{align}
and
\begin{align}
(\tt{B}_\exists, \tt{B}_\exists) &\quad\tt{<=}\quad (\overline{\tt{B}}, \emptyset)
\end{align}
For now, we do not perform garbage collection for asynchronous merge rules.
That is, we treat all sets $\tt{A}_1, \dots, \tt{A}_n$ that appear as input on the RHS of asynchronous rules as `output' sets from which we never tombstone any tuple.
We defer the garbage collection for asynchronous rules to Section \ref{sec:async_gc}.

\subsubsection{Garbage collection}
The above rewrite rules do not perform any garbage collection; in fact, it is easy to see that they are equivalent to $\mathfrak{P}$.
We will now add rules that mark tuples for reclamation.

We say that a GC rule is safe for a rule $R_i$ if it only tombstones tuples that are guaranteed to have no effect on the future derivations of $R_i$.
This will be made formal in the section below.

Suppose that a table $\tt{A}$ is the input to a single rule $R: \tt{B} ~\tt{<op>}~ \tt{f}\tt{(}\tt{A},\tt{C}_1,\dots,\tt{C}_m\tt{)}$, where $\tt{<op>}$ is either a merge $\tt{<=}$ or deferred merge $\tt{<+}$, and $R$ is not the implicit persistence rule $\tt{A} ~\tt{<+} \tt{A}$.
Suppose also that $\tt{GC}$ is a garbage collection rule that is safe for $R$.
We will then add the following GC rule to $\mathfrak{P}_{GC}$:
\begin{align}
\langle \tt{A}_{TS}, \tt{C}_{1,TS}, \dots, \tt{C}_{m,TS} \rangle
\quad\tt{<+}\quad
\tt{GC}(\tt{A}_{TS}, \tt{C}_{1,TS}, \dots, \tt{C}_{m,TS}; \tt{B}_{TS}).
\end{align}
By Assumption \ref{ass:copy}, $\tt{C}_1,\dots,\tt{C}_m$ also do not appear in the input to any other rule of $\mathfrak{P}$.

If the table $\tt{A}$ is the input to multiple rules $R_1, \dots, R_m$, then it must be the case that $R_i: \tt{B}_i ~\tt{<=}~ \tt{A}$ is an identity copy.
In that case, we add the garbage collection rule
\begin{align}
(\tt{A}_\exists, \tt{A}_\top) \quad\tt{<+}\quad \tt{GCCopy}(\tt{A}_{TS}; \tt{B}_{1,TS}, \dots, \tt{B}_{m,TS})
\end{align}
where $\tt{GCCopy}(\tt{A}_{TS}; \tt{B}_{1,TS}, \dots, \tt{B}_{m,TS}) = \left(\tt{A}_\exists - \bigcap_{i=1}^m (\tt{B}_{i,\exists} \cup \tt{B}_{i,\top}), \tt{A}_\top \cup \left(\tt{A}_\exists \cap \bigcap_{i=1}^m (\tt{B}_{i,\exists} \cup \tt{B}_{i,\top})\right)\right)$
We will show $\tt{GCCopy}$ is safe for each $R_i$.

We highlight that we only add at most one GC rule for each $\tt{A}$ in $\mathfrak{P}_{GC}$\footnote{
While it is possible to have multiple GC rules performing reclamation, we would require an additional restriction that the GC rules maintain the GC Invariant \ref{inv:merge_gc} under merges.
(Maybe we can present this in an appendix?)
}, i.e. each $\tt{A}$ only appears on the LHS of one GC rule.


% \subsubsection{Example}
% Here we show an example Bloom program $\mathfrak{P}$ which we rewrite to a GC-enabled program $\mathfrak{P}_{GC}$.
% \begin{align}
% \tt{A} &\quad\tt{<=}\quad \tt{chn\_in}\\
% \tt{B} &\quad\tt{<=}\quad \pi_{x,y}\tt{(A)}\\
% \tt{C} &\quad\tt{<+}\quad \pi_z\tt{(A)}
% % \tt{chn\_out} &\quad\tt{<}\sim\quad \sigma_{x=1}\tt{(B)}\\
% \end{align}
% is rewritten as
% \begin{align}
% (\tt{A}_\exists, \tt{A}_\top) &\quad\tt{<+}\quad (\tt{A}_\exists, \tt{A}_\top)\\
% (\tt{B}_\exists, \tt{B}_\top) &\quad\tt{<+}\quad (\tt{B}_\exists, \tt{A}_\top)\\
% (\tt{C}_\exists, \tt{C}_\top) &\quad\tt{<+}\quad (\tt{C}_\exists, \tt{A}_\top)\\
% \nonumber\\
% (\tt{A}_\exists, \tt{A}_\top) &\quad\tt{<=}\quad (\tt{chn\_in},\emptyset)\\
% % \tt{chn\_ack\_in} &\quad\tt{<}\sim\quad \tt{chn\_in}\\
% \nonumber\\
% (\tt{B}_\exists, \tt{B}_\top) &\quad\tt{<=}\quad (\pi_{x,y}\tt{(}\tt{A}_\exists \cup \tt{A}_\top\tt{)}, \emptyset)\\
% \nonumber\\
% (\tt{C}_\exists, \tt{C}_\top) &\quad\tt{<+}\quad (\pi_z\tt{(}\tt{A}_\exists \cup \tt{A}_\top\tt{)}, \emptyset)\\
% \nonumber\\
% % \tt{chn\_out} &\quad\tt{<}\sim\quad \sigma_{x=1}\tt{(}\tt{B}_\exists^{now} \cup \tt{B}_\top^{now}\tt{)}\\
% % \tt{chn\_approx\_out} &\quad\tt{<=}\quad \tt{chn\_ack\_out}\\
% % \nonumber\\
% (\tt{A}_\exists, \tt{A}_\top)
% &\quad\tt{<=}\quad
% (\emptyset, (\tt{A}_\exists \cup \tt{A}_\top) \bowtie_{x,y} (\tt{B}_\exists \cup \tt{B}_\top)
% \cap
% ((\tt{A}_\exists \cup \tt{A}_\top) \bowtie_{z} (\tt{C}_\exists \cup \tt{C}_\top))\\
% % (\tt{B}_\exists^{now}, \tt{B}_\top^{now})
% % &\quad\tt{<=}\quad (\emptyset, (\tt{B}_\exists^{now} \cup \tt{B}_\top^{now}) \bowtie \tt{chn\_approx\_out})\\
% \end{align}


\subsection{Instantiated Garbage Collection Rewrite}

\subsubsection{State}
For every set $\tt{A}$ in $\mathfrak{P}$, we introduce a corresponding $\tt{A}_I$ in $\mathfrak{P}_{iGC}$ holding the instantiated tuples.
If $\tt{A}$ has primary keys $\tt{A}_!$, we may choose to keep $\tt{A}_!$ together with $\tt{A}_I$ in a key-augmented lattice $(\tt{A}_I, \tt{A}_!)$ in $\mathfrak{P}_{iGC}$;
otherwise we simply use the non-augmented representation $\tt{A}_I$ without keys.
This choice is dependent on the instantiated garbage collection rules, which we describe in the following subsection.
For convenience, we will write $\tt{A}_E$ to mean $(\tt{A}_I, \tt{A}_!)$ if $\tt{A}$ has a key-augmented representation, and $\tt{A}_I$ if $\tt{A}$ has a non-augmented representation.
In both cases, we persist $\tt{A}_E$ with the deferred merge rule
\[\tt{A}_E \quad\tt{<+}\quad \tt{A}_E.\]


\subsubsection{Garbage collection}
A garbage collection rule in $\mathfrak{P}_{GC}$
\begin{align*}
\langle \tt{A}_{TS}, \tt{C}_{1,TS}, \dots, \tt{C}_{n,TS} \rangle
\quad\tt{<+}\quad
\tt{GC}(\tt{A}_{TS}, \tt{C}_{1,TS}, \dots, \tt{C}_{n,TS}; \tt{B}_{1,TS}, \dots, \tt{B}_{m,TS}).
\end{align*}
is replaced in $\mathfrak{P}_{iGC}$ by the instantiated garbage collection rule
\begin{align*}
\langle \tt{A}_E, \tt{C}_{1,E}, \dots, \tt{C}_{n,E} \rangle
\quad\tt{<-}\quad
\tt{iGC}(\tt{A}_E, \tt{C}_{1,E}, \dots, \tt{C}_{n,E}; \tt{B}_{1,E}, \dots, \tt{B}_{m,E}).
\end{align*}

The choice of whether to use the non-augmented $\tt{A}_I$ or the key-augmented $(\tt{A}_I, \tt{A}_!)$ is dependent on the instantiated garbage collection rules that take $\tt{A}_E$ as input\footnote{
	The set $\tt{A}$ can be taken as input to multiple garbage collection rules since it appears on the RHS of at most one rule, but possibly on the LHS Of multiple rules in $\mathfrak{P}$.
}.
If any of these instantiated garbage collection rules require the use of $\tt{A}_!$, we will choose to use the key-augmented representation.

Also associated with each $\tt{iGC}$ is a partitioning of its output sets into $\mathbb{S}(\tt{iGC})$ and $\mathbb{W}(\tt{iGC})$, such that $\mathbb{S}(\tt{iGC}) \cap \mathbb{W}(\tt{iGC}) = \emptyset$ and $\mathbb{S}(\tt{iGC}) \cup \mathbb{W}(\tt{iGC}) = \{\tt{A}, \tt{C}_1, \dots, \tt{C}_n\}$,
and every set in $\mathbb{S}(\tt{iGC})$ has a key-augmented representation.
Intuitively, a set $\tt{C}$ in $\mathbb{S}(\tt{iGC})$ is one for which $\tt{iGC}$ is strongly consistent with $\tt{GC}$, in the sense that $\tt{iGC}$ guarantees that $\tt{C}_I = \tt{C}_\exists$, and $\tt{C}_! = \pi_!(\tt{C}_\exists \cup \tt{C}_\top)$;
on the other hand, $\tt{iGC}$ is weakly consistent with $\tt{GC}$ on a set $\tt{C}$ if it guarantees that $\tt{C}_\exists \subseteq \tt{C}_I \subseteq \tt{C}_\exists \cup \tt{C}_\top$
--- we formalize these notions in the sequel.
Since each set appears on the LHS in only a single garbage collection rule\footnote{
	Each set appears on the LHS of at most one garbage collection rule because it appears on the RHS of at most one rule in $\mathfrak{P}$.
}, we can define
$\mathbb{S}(\mathfrak{P}_{iGC}) = \bigcup_{\tt{iGC} \in \mathfrak{P}_{iGC}} \mathbb{S}(\tt{iGC})$
and
$\mathbb{W}(\mathfrak{P}_{iGC}) = \bigcup_{\tt{iGC} \in \mathfrak{P}_{iGC}} \mathbb{W}(\tt{iGC})$
as the union of the respective partitions over all garbage collection rules.
To simplify notation, we will use $\mathbb{S} := \mathbb{S}(\mathfrak{P}_{iGC})$ and $\mathbb{W} := \mathbb{W}(\mathfrak{P}_{iGC})$.


\subsubsection{Merge, Deferred Merge, Asynchronous Merge}
A rule $R$ in $\mathfrak{P}$,
\[\tt{B} \quad\tt{<op>}\quad \tt{f}\tt{(}\tt{A}_1,\dots,\tt{A}_n\tt{)}\]
where $\tt{<op>}$ is one of $\tt{<=}$, $\tt{<+}$ or $\tt{<}\sim$, is replaced in $\mathfrak{P}_{iGC}$ by
\[\tt{B}_I \quad\tt{<op>}\quad \tt{f}\tt{(}\tt{A}_{1,I},\dots,\tt{A}_{n,I}\tt{)}\]
if $\tt{B}$ has no primary keys, and otherwise
\[(\tt{B}_I, \tt{B}_!) \quad\tt{<op>}\quad (\tt{f}\tt{(}\tt{A}_{1,I},\dots,\tt{A}_{n,I}\tt{)}, \pi_!(\tt{f}\tt{(}\tt{A}_{1,I},\dots,\tt{A}_{n,I}\tt{)}).\]




